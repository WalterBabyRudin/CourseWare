%% ----------------------------------------------------------------------------
%% Book Parameters - Included in body and cover
%% ----------------------------------------------------------------------------
\usepackage{tikz}% Plotting and formatting
\usepackage{pgf}% Fancy math
\usepackage{xifthen}% add ifthenelse and condtionals

%\def\uselayouts{}% Enable to display page layout info at end of document.

% At 215 pages, the spine is wide enough to hold a publishers logo
% and a nice sized title. At 315 pages the spine can hold a subtitle.
% The logo is not available below 215 pages, and below 150 pages the
% title size is very reduced. A blank spine is used when the book has
% fewer than 123 pages.
%
% Note that some printers require specific page count multiples:
% -- Createspace: page count MUST be divisible by 2.
% -- Ingram: page count MUST be divisible by 2.
% -- Blurb : page count MUST be divisible by 6.
% Add blank pages as needed in final PDF generations!
\pgfmathsetmacro\TotalPageCount{315}% Must be manually entered
\pgfmathsetmacro\PaperWidthPt{6in}%
\pgfmathsetmacro\PaperHeightPt{9in}%

%\def\ShowCoverStatistics{}% Enable to show some numbers on back cover.

\def\TheMainTitle{Introduction to Topology}
\def\TheSubTitle{MAT4002 Notebook}
\def\TheSubTitleA{Hey!}
\def\TheSubTitleB{Why Not Read This One?}
\def\TheBookSeries{}

\def\TheAuthor{\quad}%Prof. Tom Luo \\  Prof. Ruoyu Sun}
\def\TheAuthorA{Wilson J.}
\def\TheAuthorB{Schmeggles}
\def\TheAuthorLNF{Schmeggles, Wilson, J.}

\def\TheCopyrightYear{2018}
\def\TheEdition{First Edition}

\def\ThePublisherName{The First Edition}% Just the short form name
\def\ThePublisherLineA{MathPi}% Just one part of the name
\def\ThePublisherLineB{Club}% Just one part of the name
\def\ThePublisher{\ThePublisherName}% Add Ltd, Inc, LLC etc here
\def\ThePublisherAddrA{1234 Xyzzy Plugh Ln}
\def\ThePublisherAddrB{North Wyodaka, QM 45678}
\def\ThePublisherCity{Wyodaka}
\def\ThePublisherState{QM}

\def\ThePrinter{CreateSpace, An Amazon.com Company}

\def\TheSubjectArea{Yada, Yada, Yada}

\def\TheKeywords{Yadas, Books, Odd, Wierd}

%% ----------------------------------------------------------------------------
% NOTE Each version of a book usually requires its own ISBN. You can
% obtain blocks of 10 ISBNs for $250 from https://www.myidentifiers.com/
% You must know you Imprint Name (publisher). 
% Dashes MUST be placed in the correct postions in the ISBNs.
\def\PrintISBN{978-4-4444444-4-6}% 
\def\PrintISBNShort{4-4444444-4-9}%
%\def\HardcoverISBN{978-4-4444444-4-6}%
%\def\HardcoverISBNShort{4-4444444-4-9}%  
\def\EbookISBN{978-5-5555555-5-7}%
\def\EbookISBNShort{5-5555555-5-9}%

%% ----------------------------------------------------------------------------
% The ISBN barcode is created with the GhostScript barcode.ps generator:
% see Terry Burton's PostScript % based barcode generator, at: 
%       https://github.com/bwipp/postscriptbarcode.
% Use 99.99 if your price exceeds $100 (good luck selling that puppy).
% Comment out these entries if you don't want a price in the barcode.
\def\PrintPrice{9.95}% Price in dollars and cents for barcode.
%\def\HardcoverPrice{19.95}% Price in dollars and cents for barcode.

%% ----------------------------------------------------------------------------
% NOTE that P-CIP (Publisher's Catalog in Publication) data block must be
% aquired from http://www.quality-books.com/pcip.htm. This is placed on the
% copyright page, and is required by all libraries and many booksellers.
% The P-CIP costs $100 and cannot be obtained until much of a first draft
% is available as an ASCII text file. 
%
% Only US Publishers (not independent or self), can obtain an LC-CIP
% (Library of Congress Catalog in Publication) data block directly from 
% the Library of Congress.
\def\TheCIPType{Publisher's Cataloging-in-Publication Data}
%\def\TheCIPType{Library of Congress Cataloging-in-Publication Data}

% Library of Congress Catalog Subject Headings
\def\TheCIPSubjectHeadings{1. Get from QualityBooks}

% Dewey Decimal System classification number
\def\TheDDSN{333.33-dc33}

% LCCN: Library of Congress classification number (not a Control Number or PCN)
\def\TheLCCN{QP333.K33 2017}

%% ----------------------------------------------------------------------------
% The PCN (also called an LCCN) are acquired from the Library of Congress.
% This is done by contacting: http://www.loc.gov/publish/pcn/newaccount.html
% after you have purchased an ISBN (13 digit) from Bowker, preferably for both
% your print and eBook versions.

% PCN: Library of Congress’ Preassigned Control Number (PCN)
% Also called : Library of Congress Control Number
% Also called : Library of Congress Card Number
%
% This takes about a day to get assigned and is free.
\def\TheLCPCN{2017xxxxxx}

%% ----------------------------------------------------------------------------
\def\TheCopyrightKeywords{1. History; 2. Nonsense; 3. Made up stuff}

\def\ShortDescription{% Normally restricted to 300 t0 350 characters depending on publisher.
\textsf{\textbf{World War II and the Manhattan Project} -- a group of Hungarian scientists that included émigré physicist Leó Szilárd attempted to alert Washington.}}

\def\BackDescription{% Typically max 2000-4000 characters
\textsf{\textbf{Course Synopsis} -- 
\begin{itemize}
\item
Elementary Theory of Linear Transformations:
Basis, Dimension,
Linear Transformation, Matrix Representation,
Quotient Spaces, First Isomorphism Theorem,
Dual Space, Annihilators,
Adjoint Map;
\item
Eigenvectors and Eigenvalues: the Characteristic Polynomial,
Minimal Polynomial,
Cayley-Hamiton Theorem,
Primary Decomposition Theorem,
Spectral Decomposition Theorem,
Jordan Normal Form,
Normal Operator;
\item
Introduction to Tensor Product:
Basis for Tensor Product, Universal Property,
Tensor Product for Linear Transformations,
Multi-linear Tensor Product, Exterior Power
\end{itemize}
%\textbf{Grade Descriptor of F} -- Unsatisfactory performance on a number of learning outcomes, or failure to meet specified assessment requirements. In other words, you perform too bad on final!
}
}


\def\AuthorBio{% 
\textsf{\textbf{WalterBabyRudin} is a person who writes this book using \LaTeX. He is interested in Mathematics. Recently he is working on Information Theory and Graph Theory. You can contact with him on these fields. But he is very carelessness. If you find some typos in this book, don't hesitate to ask him directly. Hope you enjoy the journey to Math!}}

%% ----------------------------------------------------------------------------
%% Computed parameters for cover and jacket design 

% CreateSpace and Ingram, use different paper stock so the spine width must
% be adjusted to refect that. 
% For example Createspace, White paper: multiply page count by 0.002252
%
\pgfmathsetmacro\SinglePageThicknessPt{0.002252in}% Createspace, White B&W Paper
%\pgfmathsetmacro\SinglePageThicknessPt{0.002143in}% Blurb, Standard and Econony B&W Paper
%\pgfmathsetmacro\SinglePageThicknessPt{0.002602in}% Blurb, Standard and Economy Color Paper
%\pgfmathsetmacro\SinglePageThicknessPt{0.002110in}% Ingram, Standard White B&W Paper (50lb)
%\pgfmathsetmacro\SinglePageThicknessPt{0.002110in}% Ingram, Standard Color Paper (50lb)
%\pgfmathsetmacro\SinglePageThicknessPt{0.002720in}% Ingram, Premium Color Paper (70lb)

\pgfmathsetmacro\HorBleedPt{0.125in}%
\pgfmathsetmacro\VerBleedPt{0.125in}%
\pgfmathsetmacro\FoldVariancePt{0.0625in}%

%% Every book will vary slightly when bound. Allow for 0.0625" variance on either
%% side of the fold lines for your cover. For example, if your spine width is 1",
%% your text should be no wider than 0.875". Because of this variance, avoid hard
%% edges or lines that end on the fold line.

%% Cover Width calculation at 6" x 9" cover with 60 B&W pages on white paper:
%%    0.125" + 6" + (60 * 0.002252)" + 6" + .125" = 12.385"
%% Cover Height calculation: 6" x 9": 0.125" + 9" + .125" = 9.25"

\pgfmathsetmacro\SpineWidthPt{\SinglePageThicknessPt*\TotalPageCount}%
\pgfmathsetmacro\CoverWidthPt{\PaperWidthPt+\HorBleedPt}%
\pgfmathsetmacro\JacketWidthPt{\CoverWidthPt+\SpineWidthPt+\CoverWidthPt}%
\pgfmathsetmacro\CoverHeightPt{\VerBleedPt+\PaperHeightPt+\VerBleedPt}%

\pgfmathsetmacro\PtsPerInch{72.27}% Slightly more than 72 - odd.

\pgfmathsetmacro\SpineWidth{\SpineWidthPt    / \PtsPerInch}%
\pgfmathsetmacro\PaperWidth{\PaperWidthPt    / \PtsPerInch}%
\pgfmathsetmacro\CoverWidth{\CoverWidthPt    / \PtsPerInch}%
\pgfmathsetmacro\JacketWidth{\JacketWidthPt  / \PtsPerInch}%

\pgfmathsetmacro\PaperHeight{\PaperHeightPt  / \PtsPerInch}%
\pgfmathsetmacro\CoverHeight{\CoverHeightPt  / \PtsPerInch}%

\pgfmathsetmacro\HorBleed{\HorBleedPt        / \PtsPerInch}%
\pgfmathsetmacro\VerBleed{\VerBleedPt        / \PtsPerInch}%
\pgfmathsetmacro\FoldVariance{\FoldVariancePt/ \PtsPerInch}%
 
%% ----------------------------------------------------------------------------
\renewcommand{\bf}[1]{\textbf{#1}}% Legacy \bf support.

\newcommand{\SetBool}[2]{%
  \ifthenelse{\equal{#2}{true}\OR\equal{#2}{on}\OR\equal{#2}{yes}}%
  {\setboolean{#1}{true}}{\setboolean{#1}{false}}}%

\newcommand{\Boolean}[2]{\newboolean{#1}%
  \ifthenelse{\isempty{#2}}{}{\SetBool{#1}{#2}}}
  
%% End of Book Parameters.
%% ----------------------------------------------------------------------------
