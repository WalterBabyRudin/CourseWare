
\chapter{Week1}

\section{Monday}\index{Monday_lecture}

\subsection{Introduction to Convex Optimization}
The basic optimization model is as follows:
\begin{equation}\label{Eq:cvxopt}
\begin{array}{ll}
\min&\quad f(x):\mathbb{R}^n\to\mathbb{R}\\
\mbox{s.t.}&\quad x\in X\subseteq \mathbb{R}^n
\end{array}
\end{equation}
For instance, the constraint $X$ can be union of some inequality constraints:
\[
X=\bigg\{
x\in\mathbb{R}^n
\bigg|
f_i(x)\le b_i,\quad i=1,\dots,m
\bigg\}
\]
We only consider \emph{convex} problems in this course,
which means that in (\ref{Eq:cvxopt}):
\begin{itemize}
\item
the objective function $f$ is convex;
\item
the constraint set $X$ is a convex set.
\end{itemize}
The goal of optimization is to find $x^*\in X$ such that 
\[
f(x^*)\le f(x),\quad \forall x\in X.
\]
or determine whether such an $x^*$ exists or not.

\begin{example}
The least squares problem is a convex problem:
\[
\begin{array}{ll}
\min&\quad\frac{1}{2}\|\bm{Ax}-\bm b\|_2^2\\
\mbox{s.t.}&\quad x\in\mathbb{R}^n
\end{array}
\]
The optimal solution can be compactly written as
\[
\bm x^* = \bm A^{\mbox{*}}\bm b
\]
where $\bm A^{\mbox{*}}$ denotes the pseudo inverse of $\bm A$.
\end{example}
\begin{example}
The linear programming is also a convex problem:
 \[
\begin{array}{ll}
\min&\quad\bm c\trans\bm x\\
\mbox{s.t.}&\quad \bm a_i\trans\bm x\le b_i,\quad i=1,2,\dots,m.
\end{array}
\]
The linear programming is a special case of the conic programming.
We will cover conic programming in this course.
\end{example}


\begin{definition}[Line Segments]
The line segment between $\bm x_1,\bm x_2\in\mathbb{R}^n$ is defined as
\[
\{
\theta\bm x_1 + (1-\theta)\bm x_2\mid 
\theta\in[0,1]
\}
=
\{
\bm x_2 + \theta(\bm x_1-\bm x_2)\mid \theta\in[0,1]
\}
\]
\end{definition}

\begin{definition}[Convex Set in $\mathbb{R}^n$]
The set $X\subseteq\mathbb{R}^n$ is convex if for any $x_1,x_2\in X$,
\[
\theta x_1 + (1-\theta)x_2\in X,\qquad
\forall\theta\in[0,1].
\]
\end{definition}
Here are two examples for convex sets over matrices:
\[
\mathcal{S}^n = \{\bm W\in\mathbb{R}^n\times\mathbb{R}^n\mid \bm W=\bm W\trans\},
\quad
\mathcal{S}^n_+ = \{\bm W\in\mathcal{S}^n\mid \bm W\succeq0\},
\]
\begin{remark}
Sometimes the convex set could be a collection of matrices, but we can vectorize a matrix into a vector.
\end{remark}

\begin{definition}[Convex Function]
A function $f:\mathbb{R}^n\to\mathbb{R}$ is \emph{convex} if
\begin{itemize}
\item
$\text{dorm}(f)$ is a convex set, and
\item
$\forall x,y\in\text{dorm}(f)$, 
it holds that for $\theta\in[0,1]$,
\[
f(
\theta x + (1-\theta) y
)
\le
\theta f(x) + (1-\theta) f(y),
\]
which means that the secant line betweeen any two points is above the function.
\end{itemize}
\end{definition}
\begin{proposition}
$f(x) = \max\limits_{i}~f_i(x)$ is convex if $f_i(x)$ is convex for each $i$.
\end{proposition}
\begin{proof}
For any $\theta\in[0,1]$, the following inequality holds:
\begin{align*}
f(\theta x + (1-\theta)y) &=\max\limits_{i}~f_i(\theta x + (1-\theta)y)\\
&\le \max\limits_{i}~\theta f_i(x) + (1-\theta)f_i(y)\\
&\le \max\limits_{i}~\theta f_i(x) +  \max\limits_{i}~(1-\theta)f_i(y)\\
&=\theta f(x) + (1-\theta)f(y)
\end{align*}
\end{proof}
\begin{example}
Define the function
\[
\begin{array}{ll}
&f:\mathbb{S}^n\to\mathbb{R}\\
\mbox{with}&f(\bm X)=\lambda_{\max}(\bm X)\triangleq\max\limits_{\|\bm v\|=1}~\bm v\trans\bm X\bm v
\end{array}
\]
This function is convex since $f$ can be written as the maximization of a collection of affines (in terms of $\bm X$):
\[
f(\bm X) = \max_{\|\bm v\|=1}f_{\bm v}(\bm X),\qquad
\text{with }f_{\bm v}(\bm X)=\bm v\trans\bm X\bm v.
\]
\end{example}
















