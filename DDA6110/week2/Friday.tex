% !TEX encoding = UTF-8 Unicode

\section{Wednesday}\index{Friday_lecture}

$K$ proper cone: convex, closed, solid, pointed.
\begin{definition}[Generalized Inequality]
Given a set $K$, 
we define $x\le_Ky$ if $y-x\in K$.
\end{definition}

Pointed cone:
\[
x\le_K y, y\le_K x\implies x=y.
\]

\begin{definition}[Dual Cone]
Given a cone $K\subseteq\mathbb{R}^n$, the dual cone is defined as
\[
K^* = \{y\in\mathbb{R}^n\mid y\trans x\ge0, \forall x\in K\}.
\]
\end{definition}
\begin{remark}
The dual cone $K^*$ is convex, closed, and pointed.
The convexity is because that it is an intersection of many half-spaces.
\end{remark}

Self-dual cones:
\[
(\mathbb{R}^n_+)^* = \mathbb{R}^n_+,\qquad
(\mathbb{S}_+^n)^* = \mathbb{S}_+^n.
\]
\begin{proof}
Let $X,Y\in\mathbb{S}_+^n$, then
\[
\inp{X}{Y} = \sum_i\lambda_iq_i\trans Yq_i\ge0.
\]
Therefore, $\mathbb{S}_+^n\subseteq (\mathbb{S}_+^n)^*$.
When $Y\notin(\mathbb{S}_+^n)$, there exists $u$ such that
\[
\bm u\trans Y\bm u<0
\implies
\inp{Y}{\bm u\bm u\trans}<0,
\]
which implies $Y\notin(\mathbb{S}_+^n)^*$.
\end{proof}

\subsection{Separation Theorems}
If $C$ and $D$ are disjoint convex sets, there exists affine function such that 
\[
\bm a\trans\bm x\le\bm b,\forall\bm x\in C,\qquad
\bm a\trans\bm x\ge\bm b,\forall \bm x\in D.
\]

\begin{theorem}
If $C$ is convex, then for any $x_0\in\partial C$, 
there exists a hyper-plane $\{\bm x\mid\bm a\trans\bm x=\bm b\}$ with $\bm a\ne\bm0$ such that
\[
a\trans\bm x_0,
\]
and $\bm a\trans\bm x\le\bm b,\forall\bm x\in C$.

The converse is true if $C$ is closed.
\end{theorem}











