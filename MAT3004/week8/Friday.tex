\chapter{Week8}
\section{Friday}\index{week7_Thursday_lecture}
\subsection{Classification in Chapter 7}
\begin{definition}
A ring $R=(R,+,\cdot)$ means that:
\begin{enumerate}
\item
$(R,+)$ is an abelian group
\item
$(R,\cdot)$ is a semi-group
\item
$R$ satisfies the \emph{distributive law}
\end{enumerate}
\begin{itemize}
\item
In addiction, if $R$ has a \emph{multiplicative identity} $1\in R$, then $R$ is a \emph{unital ring}.
\item
A ring $R$ is said to be commutative if its multiplication is commutative.
\end{itemize}
\end{definition}
\begin{proposition}
Let $(R,+)$ is a group, and $(R,\cdot)$ is a monoid, and $(R,+,\cdot)$ satisfies the distributive laws, then $+$ is \emph{commutative}.
\end{proposition}
\begin{proof}
Consider distributive laws in $(1+1)(x+y)$
\end{proof}

Since $(\mathbb{Z}_m,\cdot)$ is not necessarily a group, we assume
\begin{itemize}
\item
$(\mathbb{Z}_m,+)$ is a group
\item
$(\mathbb{Z}_m,+,\cdot)$ is a ring. (unital and commutative)
\item
$\mathbb{Z}/m\mathbb{Z}\cong\mathbb{Z}_m$
\end{itemize}
\begin{proposition}
Question on $a\equiv c(\bmod m)$ and $b\equiv d(\bmod m)$ implies
\[
a+b\equiv c+d(\bmod m),
ab\equiv cd(\bmod m)
\]
\end{proposition}
\begin{definition}[Ring of polynomials]
Let $R$ be a \emph{commutative ring}, then a polynomial over $R$ is
\[
f(x)=\sum_{i=0}^na_ix^i
\]
with $a_i\in R$. Here $f(x)\in R[x]$.
\end{definition}
\begin{remark}
The image on $R$ does not necessarily define a function $f$, e.g.,
\[
f(x)=1+x+x^2,g(x)=1\in\mathbb{Z}_2[x]
\]
\end{remark}
\begin{definition}
Let $D$ be a ring. 
\begin{itemize}
\item
A nonzero element $r\in D$ is called a \emph{zero divisor} if there exists a nonzero $s\in D$ such that $rs=0$ or $sr=0$
\item
If $D$ has no zero divisors, then $D$ is called a \emph{domain}
\item
If $D$ has no zero divisors, i.e., the product of two nonzero elements is always nonzero, and $D$ is commutative, then $D$ is called an \emph{integral domain}.
\end{itemize}
\end{definition}
\begin{remark}
\begin{itemize}
\item
$R$ is an integral domain iff $R[x]$ is an integral domain
\item
$\mathbb{Z}_6$ is not an integral domain. Note that $\mathbb{Z}_m$ is an integral domain iff $m$ is a prime.
\item
$C[-1,1]$ is not an integral domain, e.g., $f=(x)^+,g=(x)^-$.
\end{itemize}
\end{remark}
\begin{proposition}
Let $D$ be a commutative ring, TFAE
\begin{itemize}
\item
$D$ is an integral domain
\item
For any nonzero $a,b\in D$, we have $ab\ne0$
\item
$D$ satisfies the cancellation law: $ca=cb$ and $c\ne0$ implies $a=b$.
\end{itemize}
\end{proposition}
\begin{proof}
Consider the distributive laws on $c[a+(-b)]=0$; and $ab=a0$.
\end{proof}
\begin{remark}
Generalization into non-commutative rings.
\end{remark}
\begin{definition}
Let $R$ be a ring, then $a\in R$ is a unit if it has a multiplicative inverse $a^{-1}\in R$.
\end{definition}
\begin{definition}
A \emph{divison field} $R$ is a ring that all its nonzero elements are units.

If $R$ is a commutative ring in which every nonzro element is a unit, then $R$ is a field
\end{definition}
\begin{remark}
The quaternion is not commutative, and thus not a field.
\end{remark}
\begin{itemize}
\item
$\{\mbox{zero divisors in $\mathbb{Z}_m$}\} = \{k\in\mathbb{Z}_m^{*}\mid gcd(k,m)>1\}$
\item
$\{\mbox{units in $\mathbb{Z}_m$}\} = \{k\in\mathbb{Z}_m^{*}\mid gcd(k,m)=1\}$
\end{itemize}
\begin{proposition}
All finite integral domain $D$ is a field
\end{proposition}
\begin{proof}
For $D=\{a_1,\dots,a_n\}$, consider $a^n=a^m$ for $a\ne0$, which implies $1\in D$. Then consider the set
\[
\{aa_1,\dots,aa_n\}
\]
\end{proof}
\begin{definition}[Char]
Define
\[
n\circ a=\underbrace{a+\cdots+a}_{n\ge1},\qquad
0\circ a=0_R
\]
If there exists smallest positive $n$ such that
\[
n\circ a=0,\forall a\in R,
\]
then $n$ is the \emph{characteristic of the ring }$R$. Otherwise $R$ is of characteristic $0$. In particular, if $R=F$ is a field, then it is the characteristic of the field.
\end{definition}
\begin{proof}
$\mbox{char}(\mathbb{Z}_n)=n$
\end{proof}
\begin{proposition}
The characteristic of an integral domain is either $0$ or a prime.
\end{proposition}
\begin{proof}
Consider $n=km$, then $n\circ 1=(k\circ 1)(m\circ 1)$
\end{proof}
\begin{theorem}
The characteristic for a \emph{unital} ring is either the smallest $n$ s.t. $n\circ 1=0$, or 0.
\end{theorem}
\begin{proof}
$n\circ a=a(n\circ 1)=0$
\end{proof}
Given an integral domain, we want to enlarge it into a field by adding some multiplicative inverses.
\paragraph{Equivalence Relation}
For the set $R\times R_{\ne0}=\{(a,b)\mid a,b\in R,b\ne0\}$, define the operation
\[
(a,b)\sim (c,d)\mbox{ if }ad=bc
\]

\begin{definition}[Quotient Set]
Given the equivalence relation $\sim$, the quotient set $S/\sim$ is the set of all equivalence classes of $S$.
\end{definition}
Define the operation
\[
\begin{array}{l}
(a,b)+(c,d)=(ad+bc,bd)\\
(a,b)(c,d)=(ac,bd)
\end{array}
\]
we have $(a,b)\sim(a',b'), (c,d)\sim(c',d')$ implies
\begin{itemize}
\item
$(a,b)+(c,d)\sim(a',b')+(c',d')$
\item
$(a,b)(c,d)\sim(a',b')(c',d')$
\end{itemize}
\begin{definition}[Fraction Field]
Define $\mbox{Frac}(R)=(R\times R_{\ne0})/\sim$, and 
\[
\begin{aligned}
[(a,b)]+[(c,d)]&=[(ad+bc,bd)]\\
[(a,b)][(c,d)]&=[(ac,bd)]
\end{aligned}
\]
it forms a field, with additive identity $0:=[(0,1)]$, and multiplicative identity $1:=[(1,1)]$. The multiplicative inverse of a nonzero $[(a,b)]\in\mbox{Frac}(R)$ is $[(b,a)]$
\end{definition}
\begin{remark}
$\mbox{Frac}(\mathbb{Z})=\mathbb{Q}$, if identify $[(a,b)]:=a/b\in\mathbb{Q}$.
\end{remark}
\subsection{Classificiation on Chapter 8}
\begin{definition}[Ring Homomorphism]
A map $\phi:R\to R'$ is a ring homomorphism if
\begin{enumerate}
\item
$\phi(a+b)=\phi(a)+\phi(b)$
\item
$\phi(ab)=\phi(a)\phi(b)$
\end{enumerate}
\begin{itemize}
\item
\emph{Unital homomorphism}: $R,R'$ are also unital and $\phi(1_R)=1_{R'}$
\item
If $\phi$ is bijective, then $\phi$ is an \emph{isomorphism}, $R\cong R'$
\item
$R,R'$ are unital but $\phi$ does not have to be unital: 
\[
\phi: a\mapsto 0_{R'}
\]
\end{itemize}
\end{definition}
\begin{itemize}
\item
$\phi(0_R)=0_{R'}$: $\phi(0_R)=\phi(0_R+0_R)=\phi(0_R)+\phi(0_R)$
\item
$\phi(-a)=-\phi(a)$: $0_{R'}=\phi(a+(-a))=\phi(-a)+\phi(a)$
\item
If $\phi$ is unital, then $[\phi(u)]^{-1}=\phi(u^{-1})$ for each unit $u\in R$:
\[
1_{R'}=\phi(u)\phi(u^{-1})
\]
\item
$\mbox{Im}(R)=\phi(R)$ is a subring of $R'$
\end{itemize}
\begin{remark}
Let $R$ be a ring, then $\phi: \mathbb{Z}\to R$ is uniquely determined by
\[
\phi(1)=a\in R,
\]
since $\phi(n)=n\circ a$ and $\phi(-n)=-n\circ a=n\circ(-a)$
\end{remark}
\begin{proposition}
The ring $\mathbb{Q}$ and $\mathbb{Z}$ cannot be isomorphism, but the fields $\mathbb{Q}$ and $\mbox{Frac}(\mathbb{Z})$ are isomorphic.
\end{proposition}
\begin{proof}
Consider the map $\phi:\mathbb{Q}\to\mbox{Frac}(\mathbb{Z})$:
\[
\phi(a/b)=[(a,b)]
\]
First it is well-defined. Second it is homorphism. Third it is one-to-obe and onto.
\end{proof}
\begin{theorem}
Let $F$ be a field, then $\mbox{Frac}(F)\cong F$.
\end{theorem}
\begin{proof}
Consider the map $\phi:F\to\mbox{Frac}(F)$:
\[
\phi(s)=[(s,1)],\quad\forall s\in F.
\]
\end{proof}

\begin{definition}[Subring]
Let $R$ be a ring, a subset $S$ of $R$ is a \emph{subring} if it is a ring under the same operations of $R$. Or equivalently,
\begin{itemize}
\item
$a,b\in S$ implies $a-b\in S$
\item
$a,b\in S$ implies $ab\in S$
\end{itemize}
To check $S$ is unital, we need to check $S$ contains a multiplicative identity $1_S$ (not necessaruly $1_R$)
\end{definition}
\begin{proposition}
For ring $R$ and subring $S$, we have $0_S=0_R$
\end{proposition}

\begin{definition}[Kernel]
The kernel of $\phi$ is $\mbox{ker}(\phi)=\{a\in R\mid \phi(a)=0_{R'}\}$
\end{definition}
\begin{proposition}
For a ring homomorphism $\phi$,
\begin{itemize}
\item
$S$ is a subring implies $\phi(S)$ is a subring
\item
$S'$ is a subring implies $\phi^{-1}(S')$ is a subring.
\item
$\mbox{im}(\phi)$ is a subring.
\end{itemize}
\end{proposition}
\begin{corollary}
If $R,R'$ are isomorphic, then $\phi(1_R)=1_{R'}$
\end{corollary}
\begin{remark}
For unital $S'$, the $\phi^{-1}(S')$ is not necessarily unital. Example: $\phi:3\mathbb{Z}\to\mathbb{Z}_6$ defined by $\phi(x)=\bar x$.
\end{remark}
\begin{proposition}
A ring homomorphism is one-to-one iff $\mbox{ker}\phi=\{0_R\}$
\end{proposition}
\begin{definition}[Ring of polynomials]
\[
R[x,y]=\left\{
\sum_i\sum_ja_{ij}x^iy^j\middle|
a_{ij}\in R
\right\}
\]
\end{definition}
\begin{proposition}
\[
R[x,y]\cong (R[x])[y]
\]
\end{proposition}
\begin{proof}
Construct the mapping 
\[
\phi\left(\sum_i\sum_ja_{ij}x^iy^j\right)=\sum_j\left(\sum_{i}a_{ij}x^i\right)y^j
\]
\end{proof}
\begin{proof}
It is clear that is a homomorphism. The one-to-one is by showing
\[
\mbox{ker}\phi=\{0\}
\]
To show the onto, define $g=\sum_{j=0}^np_jy^j$, and let $m=\max_{j}\mbox{deg}p_j$, which implies
\[
g=\sum_{j=0}^n(\sum_{i=0}^ma_{ji}x^i)y^j
\]
\end{proof}
\begin{proposition}
A subring of a field is an integral domain.
\end{proposition}
\begin{proof}
For the subring $R\subseteq F$, suppose $a,b\in R,ab=0,a,b\ne0$, we have
\[
b=a^{-1}(ab)=0
\]
\end{proof}
\begin{remark}
The integral domain $\mathbb{Z}$ is a subring of field $\mathbb{Q}$.
\end{remark}
\begin{definition}[Ideal]
A subset $I$ in a ring $R$ is an ideal if
\begin{itemize}
\item
$(I,+)$ is a group
\item
For each $r\in R$, $rI\subseteq I$ and $Ir\subseteq I$
\end{itemize}
If $I$ is a prpoper subset, then $I$ is a proper ideal.
\end{definition}
\begin{remark}
\begin{itemize}
\item
Improper ideal: $R$, trivial ideal $\{0\}$, containing any proper non-trivial ideals: simple.
\item
The first condition is replaced by:
\[
0\in I,\quad
x-y\in I,\forall x,y\in I
\]
\item
an ideal $I$ containing 1 implies $I=R$
\item
$I=m\mathbb{Z}$ is an ideal of ring $\mathbb{Z}$ since $mn_1-mn_2=m(n_1-n_2)\in I$, and $d\cdot mn=mn\cdot d\in I$ for $\forall d\in I$.
\item
$I=\{f\in R\mid f(1/2)=0\}\subseteq C[-1,1]$ is an ideal.
\end{itemize}
\end{remark}

Now we seek a special subring $I$ such that $R/I$ forms a new ring.
\begin{theorem}
$R$ is a commutative ring, $I$ is an ideal. Let $R/I$ denote the set of equivalence classes of $R$, and each element has the form $r+I$ for $r\in R$.
\begin{align*}
(a+I)+(b+I)&=(a+b)+I\\
(a+I)(b+I)=ab+I
\end{align*}
then $R/I$ forms a ring.
\end{theorem}
\begin{proof}
Check it forms a group, associative multiplication, distributive laws.
\end{proof}
\begin{remark}
$\bar a=\bar b$ is written as $a\equiv b(\bmod I)$
\end{remark}
Every ideal is a subring, and the converse does not necessarily hold. If the converse is true, then it is a Hamiltonian ring.

\subsection{Classificatin on Chapter 10}

\begin{proposition}
The \emph{canonical homomorphism} $\pi:R\to R/I$ defined by $\pi(r)=\bar{r}$ is a surjective homomorphism with $\mbox{ker}(\pi)=I$.
\end{proposition}
\begin{theorem}[First Isomorphism Theorem]
Let $\phi:R\to R'$ be a ring homomorphism, then $\mbox{e=ker}(\phi)$ is an ideal of $R$, and
\[
R/\mbox{ker}\phi\cong\mbox{im}(\phi)
\]
\end{theorem}
\begin{proof}
Construct the mapping
\[
\bar{\phi}(\bar{a}) = \phi(a),\quad \forall a\in R
\]
Then show the well-defined, homomorphism, surjective, and one-to-one:
\[
a':=\phi(a)=\bar{\phi}(\bar a)
\]
\[
\bar{a}\in\mbox{ker}(\bar\phi)\implies\phi(a)=0,\bar a=\bar0
\]
\end{proof}
\begin{corollary}
Let $\phi$ be a surjective ring homomorphism, then 
\[
R/\mbox{ker}(\phi)\cong R'
\]
\end{corollary}
\begin{remark}
Define a homomorphism $\phi:\mathbb{Z}\to\mathbb{Z}_m$ by $\phi(n)=\bar n$. Thus $\phi$ is surjective and $\mbox{ker}(\phi)=m\mathbb{Z}$, and therefore $\mathbb{Z}/m\mathbb{Z}\cong\mathbb{Z}_m$
\end{remark}
\begin{proposition}
\[
\mathbb{Z}/10\mathbb{Z}\cong\mathbb{Z}[i]/(1+3i)
\]
\end{proposition}
\begin{proof}
Construct a mapping $\phi:\mathbb{Z}\to\mathbb{Z}[i]/(1+3i)$ by
\[
\phi(n)=\bar{n}
\]
It's clear that $\phi$ is a homomorphism.
\begin{itemize}
\item
Note that $1+3i\equiv0(\bmod <1+3i>)$ implies $i\equiv3(\bmod <1+3i>)$. Thererfore,
\[
\overline{a+bi}=\overline{a+3b}=\phi(a+3b)\implies\mbox{$\phi$ is surjective}
\]
\item
Suppose $\phi(n)=\bar0$, then
\[
n = (a+bi)(1+3i)=(a-3b)+(3a+b)i
\]
If $3a+b=0$, then $n=10a$, which implies $\mbox{ker}(\phi)\subseteq10\mathbb{Z}$
\item
For each $m\in\mathbb{Z}$,
\[
\phi(10m)=\overline{10m}=\overline{1+3i}\overline{(1-3i)m}=\bar0\implies10\mathbb{Z}\subseteq\mbox{ker}(\phi)
\]
\end{itemize}
Thus $\mbox{ker}(\phi)=10\mathbb{Z}$. Applying First Isomorphic Theorem.
\end{proof}

\begin{example}
$R[x]/(x^2+1)\cong\mathbb{C}$.

Define the map $R[x]\to\mathbb{C}$ by:
\[
\phi(\sum_{k=0}^na_kx^k)=\sum_{k=0}^na_ki^k
\]
Check homomorphism, surjective. Let $f(x)\in\mbox{ker}(\phi)$, then
\[
f(i)=0\implies f(-i)=0\implies (x^2+1)|f(x)\implies\mbox{ker}(\phi)\subseteq<x^2+1>
\]
On the other hand,
\[
f(x)=(x^2+1)g(x)\implies f(i)=0\implies <x^2+1>\subseteq\mbox{ker}(\phi)
\]
\end{example}

\begin{definition}[Maximal]
An ideal $M$ in ring $R$ is \emph{maximal} if the only ideal that properly contains $M$ is $R$ it self.
\end{definition}
\begin{proposition}
A unital commutative ring $R$ is \emph{simple} iff it is a division ring.
\end{proposition}
\begin{proof}
Consider a nonzero ring $R$.
\begin{itemize}
\item
For nonzero $a\in R$, the principle ideal $<a>=aR=\{0\}$ or $R$.
\[
a=1a\in<a>\implies <a>=aR=R
\]
Thus there exists $x\in R$ such that $ax=xa=1_R$
\item
For the converse, consider the $aa^{-1}\in R$
\end{itemize}
\end{proof}
\begin{remark}
It says that a field is simple, and a simple unital commutative ring forms a field.
\end{remark}

\begin{theorem}
A proper ideal $M$ of a unital commutative ring $R$ is \emph{maximal} iff $R/M$ is a fied.
\end{theorem}
\begin{proof}
It suffices to show $M$ is maximal iff $R/M$ is simple. 
\end{proof}
\begin{remark}
When $R$ is not unital, the theorem will not hold. For $R=2\mathbb{Z}$, $M=4\mathbb{Z}$ is a maximal ideal of $R$. $R/M$ is not a field since $\bar 2\in R/M$ and $\bar 2\bar 2=\bar 0$. The converse also holds.
\end{remark}
Question about second.
\begin{remark}
Consider $R=\mathbb{Z}_{12}$, we have proper ideals
\[
I_1=\{0,2,4,8,10\}
\quad
I_2=\{0,3,6,9\},\quad
I_3=\{0,4,8\},\quad
I_4=\{0,6\}
\]
Here $I_1,I_2$ are maximal, and $R/I_1\cong F_2$, and $R/I_2\cong F_3$.
\end{remark}

\begin{corollary}
A unital commutative ring $R$ is a field iff it has no proper non-trivial ideals.
\end{corollary}

\begin{definition}[Prime]
An ideal $P$ of a \emph{commutative} ring $R$ is \emph{prime} if
\[
ab\in P\implies a\in P\mbox{ or }b\in P,\forall amb\in R
\]
\end{definition}
\begin{theorem}
An ideal $P$ of a \emph{commutative} ring $R$ is \emph{prime} iff $R/P$ is an \emph{integral domain}
\end{theorem}
\begin{proof}
For the forward direction, assume $\bar{a},\bar{b}\in R/P$ s.t. $\bar{a}\bar{b}=\bar0$, then
\[
\bar{ab}=\bar0\implies ab\in P\implies a\in P\mbox{ or }b\in P
\]
For the converse, take $a,b\in R$ such that $ab\in P$, then
\[
\bar{a}\bar{b}=\bar{ab}=\bar0\implies \bar a=\bar0\mbox{ or }\bar b=\bar 0
\]
\end{proof}
\begin{corollary}
Every maximal ideal in a unital commutative ring is prime.
\end{corollary}

Can we drop the condition unital?
\begin{definition}[Nilpotent]
For ring $R$, an element $a\in R$ is \emph{nilpotent} if there exists $n\in\mathbb{Z}^+$ such that
\[
a^n=0
\]
If $R$ is also commutative, then the \emph{nilradical} of $R$ is the set of all its \emph{nilpotent} elements, denoted as $\mbox{Nil}(R)$
\end{definition}
\begin{remark}
Every nonzero nilpotent element is a zero divisor, but the converse is not true.
\end{remark}
\begin{proposition}
For commutative ring $R$, $\mbox{Nil}(R)$ is an ideal of $R$.
\end{proposition}
\begin{definition}[Partially Ordered Set/poset]
$\succeq$:
\begin{itemize}
\item
$a\succeq a$
\item
$a\succeq b,b\succeq c$ implies $a=b$
\item
$a\succeq b,b\succeq c$ implies $a\succeq c$
\end{itemize}
Given $A\subseteq S$, we say $u\in S$ an upper bound of $A$ if $a\succeq u$ for all $a\in A$. An element $m\in S$ is maximal if there is no any other element $s$ such that $m\succeq s$
\end{definition}
\begin{remark}
Two elements $a,b$ are \emph{comparable} if $a\succeq b$ or $b\succeq a$ happens. Not every pair of elements are comparable. If every elements in $T\subseteq S$ are comparable, then $T$ is a chain.
\end{remark}
\begin{proposition}
Suppose every chain in a nonempty poset $S$ has an upper bound in $S$, then $S$ has a maximal element.
\end{proposition}
\begin{theorem}
Let $R$ be a commutative ring. Denote by $N$ the intersection of all prime ideals of $R$, then $\mbox{Nil}(R)=N$.
\end{theorem}
\begin{proof}
\begin{itemize}
\item
First show $\mbox{Nil}(R)\subseteq N$, i.e., $a$ is a nilpotent element. Assume $a\notin P$ for some prime ideal $P$, we have $a^n=0$, which implies
\[
\bar{a}\bar{a^{n-1}}=\bar0,
\]
for $\bar{a}\in R/P$, which is a contradction.
\item
Secondly, suppose $b\in N$ but $b$ is non-nilpotenet. Define the poset
\[
S=\{\mbox{ideals of $R$ that does not contain any $b^n$ with $n\in\mathbb{Z}^+$}\}
\]
then $\{0\}\in S$, and every chain has an upper bound $\bigcup I_\alpha$. There is a maximal ideal $P$ with $b^n\notin P$ for all $n\in\mathbb{Z}^+$. 

We claim that $P$ is prime, since otherwise there exists $x,y\in R$ such that $xy\in P$ and $x,y\notin P$, and therefore
\[
P\subset(P,x),\quad P\subset(P,y)\implies
b^s\in(P,x),b^t\in(P,y)\implies
b^{s+t}\in(P,x)(P,y)\subseteq P
\]
The $b$ is not in the prime ideal $P$ implies a contradction. 
\end{itemize}
\end{proof}
\begin{remark}
The ideal $N$ is not necessarily prime, e.g., $R=\mathbb{Z}_6$, and $N=\mbox{Nil}(R)=\{0\}$ is not prime.
\end{remark}

\begin{theorem}
Let $R$ be a unital ring with $\mbox{char}(R)=n$
\begin{itemize}
\item
If $n=0$, then $R$ contains a subring isomorphic to $\mathbb{Z}$
\item
If $n>0$, then $R$ contains a subring isomorphic to $\mathbb{Z}_n$
\end{iteize}
\end{theorem}
\begin{proof}
Define a homomorphism $\phi:\mathbb{Z}\to R$ by $\phi(m)=m\circ 1_R$. Then $\mbox{ker}(\phi)=n\mathbb{Z}$.
\end{proof}
\begin{definition}
Let $F$ be a field, a set $E\subseteq F$ is a \emph{subfield} of $F$ if $E$ is a field under the same operations of $F$. A nonzero field is \emph{prime} if it has no proper nontrivial subfields.
\end{definition}
\begin{corollary}
Any \emph{nomzero} field $F$ contains a subfield isomorphic to $\mathbb{Q}$ or $\mathbb{F}_p$ for some prime $p$.
\end{corollary}
\begin{proof}
The field $F$ contains a subring $R\cong\mathbb{Z}$ or $\mathbb{Z}_p$ for some prime $p$ under $\phi$. We consider $R\cong\mathbb{Z}$

For a smallest subfield $E$ of $F$ containing $R$, define $\hat\phi:E\to\mathbb{Q}$
\[
\hat\phi=\left\{
\begin{aligned}
\phi(x),\quad\mbox{if $x\in R$}\\
\phi^{-1}(b)\phi(a)\quad\mbox{if $x=b^{-1}a$ with $a,b\in R$}
\end{aligned}
\right.
\]
\end{proof}
\begin{corollary}
Any prime field is isomorphic to $\mathbb{Q}$ or $\mathbb{F}_p$
\end{corollary}





















