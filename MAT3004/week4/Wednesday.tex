
\chapter{Week4}

\section{Reviewing}\index{week4_Friday_lecture}
\begin{definition}[partition]
Let $S$ be a non-empty set. A \emph{partition} $P$ of $S$ is a collection of subsets $\{S_i\mid i\in I\}$ of $S$ such that
\begin{enumerate}
\item
$S_i\ne\emptyset$ for all $i\in I$
\item
$S_i\bigcap S_j=\emptyset$ whenever $i\ne j$
\item
$\bigcup_{i\in I}S_i=S$
\end{enumerate}
We also say $P$ is a sub-division of $S$ into a \emph{disjoint union} of non-empty subsets, denoted as $S=\sqcup_{i\in I} S_i$
\end{definition}
\begin{definition}[Equivalence Relation]
An \emph{equivalence relation} on $S$ is a relation $\sim$ such that
\begin{enumerate}
\item
Reflexive: $a\sim a$ for all $a\in S$
\item
Symmetric: if $a\sim b$, then $b\sim a$
\item
Transitive: if $a\sim b$ and $b\sim c$, then $a\sim c$.
\end{enumerate}
\end{definition}
Partition and equivalence relation are essentially two equivalent concepts.
\begin{enumerate}
\item
Given partition $\{S_i\mid i\in I\}$ of $S$, we define $a\sim b$ whenever $a,b\in S_i$ for some $i\in I$.
\item
Suppose there is an equivalence relation $\sim$ on $S$, for each $a\in S$, we define the \emph{equivalence class }with the representative $a$, i.e., partition as:
\[
C_a:=\{b\in S\mid a\sim b\}
\]
\begin{itemize}
\item
Non-empty: since $a\sim a$, $a\in C_a$, i.e., $C_a$ is non-empty.
\item
$\bigcup_{a\in S}C_a=S$
\item
disjoint: we show that $C_a\bigcap C_b\ne\emptyset$ implies $C_a=C_b$:

Suppose $c\in C_a\bigcap C_b$, we imply $a\sim b$. Thus for any $d\in C_a$, $d\sim b$, i.e., $C_a\subseteq C_b$. Similarly, $C_a\supset C_b$.
\end{itemize}

\end{enumerate}
\begin{remark}
If $b\in C_a$, then $C_b=C_a$, i.e., \emph{any eleemebt in an equivalence class can be its representative}.
\end{remark}
\begin{proposition}
Any permutation $\sigma\in S_n$ is a product of disjoint cycles.
\end{proposition}
\begin{proof}
Let $\sigma\in S_n$ be an permutation on $X=\{1,\dots,n\}$. For any $a,b\in X$, define an equivalence relation
\[
a\sim b\mbox{ whenever }b=\sigma^k(a),\mbox{ for some }k\in\mathbb{Z}
\]
Thus $X$ is paritioned into disjoint union of equivalence classes:
\[
X=O_1\sqcup Q_2\sqcup \cdots\sqcup O_m,
\]
here $O_j$'s are called orbits of $\sigma$. For each $j\in\{`,\dots,m\}$, construct $\mu_j\in S_n$:
\[
\mu_j(a)=\left\{
\begin{aligned}
\sigma(a),&\quad\mbox{if }a\in O_j\\
a,&\quad\mbox{if }a\in O_j
\end{aligned}
\right.
\]
These $\mu_j$ are mutually disjoint cycles, and thus $\sigma=\mu_1\cdots\mu_m$.
\end{proof}
\subsection{Theorem of Lagrange}
Let $G$ be a group and $H\le G$. We are interested in the size of $H$ compared with $G$. 
\begin{proposition}
For any $a,b\in G$, define a relation $\sim_L$ on $G$:
\[
a\sim_Lb\mbox{ whenever $b=ah$ for some $h\in H$, i.e., $a^{-1}b\in H$}
\]
The operator $\sim_L$ is an equivalence relation.
\end{proposition}
Thus the group $G$ is partitioned into a disjoint union of equivalence classes w.r.t. $\sim_L$. We call these equivalence classes the \emph{left cosets} of $H$ in $G$, and each has form $aH=\{ah\mid h\in H\}$. Every $a\in G$ is a representative of the left coset $aH$.

Also, we define $\sim_R$ as:
\[
a\sim_Rb\mbox{whenever $b=ha$ for some $h\in H$}
\]
and $Hb=\{hb\mid h\in H\}.$
\begin{example}
Let $G=(\mathbb{Z},+)$ and $H=3\mathbb{Z}\le G$. The left cosets of $H$ in $G$ are:
\[
\begin{array}{ll}
3\mathbb{Z},1+3\mathbb{Z},2+3\mathbb{Z}
\end{array}
\]
The coset representatives are generally not unique.
\end{example}
In general, if $n\in\mathbb{N}^+$, the left cosets of $n\mathbb{Z}$ in $\mathbb{Z}$ are:
\[
i+n\mathbb{Z},\quad
i=0,\dots,n-1
\]
\begin{definition}
Let $G$ be a group, and $H\le G$. Denote the collection of the left cosets of $H$ in $G$ by $[G:H]$. We call the size of $[G:H]$ the \emph{index} of $H$ in $G$, denoted as $|G:H|$.
\end{definition}
\begin{remark}
The number of left cosets and right cosets are always equal to each other, if finite.
\end{remark}
\begin{example}
Let $G=\mbox{GL}(n,\mathbb{R})$, and
\[
H=\mbox{GL}^+(n,\mathbb{R}):=\{h\in G\mid \det(h)>0\}\le G.
\]
Take $a=\diag(-1,1,\dots,1)\in G$ and thus $\det(a)=-1$. Take $g\in G$, then
\begin{enumerate}
\item
If $\det(g)>0$, then $g\in H$
\item
If $\det(g)<0$, then $\det(a^{-1}g)>0$, i.e., $g\in aH$
\end{enumerate}
Thus $[G:H] = \{H,aH\}$ and $|G:H|=2$. Note that both $G$ and $H$ are infinite, but $|G:H|$ are finite.
\end{example}
\begin{example}
Let $G=\mbox{GL}(n,\mathbb{R})$ and $H=\mbox{SL}(n,\mathbb{R})$. For each $x\in\mathbb{R}^{\#}$, take $a_x=\diag(x,1,\dots,1)\in G$ and $\det(a_x) = x$. Thus for each $g\in G$,
\[
g = a_{\det(g)}\left[a_{\det(g)}^{-1}g\right]\in a_{\det(g)}H
\]
Moreover, we show that the constructed cosets are non-trivial:
\[
a_x H\bigcap a_yH=\emptyset,\forall x,y\in\mathbb{R}^{\#}
\]
and therefore $G=\sqcup_{x\in\mathbb{R}^{\#}}a_xH$, and $|G:H|$ is infinite.
\end{example}
\begin{proposition}
For the additive subgroup $\mathbb{Z}<\mathbb{R}$, we have
\[
\mathbb{R}=\sqcup_{t\in[0,1)}(t+\mathbb{Z})
\]
\end{proposition}
\begin{proof}
For the subgroup $H:=\mathbb{Z}$, construct left cosets $tH$ for $t\in[0,1)$. Thus for $\forall h_1\in t_1H$ and $h_2\in t_2H$ with distinct $t_1$ and $t_2$, we have
\[
h_1=t_1+z_1\ne t_2+z_2,\quad
h_2=t_2+z_3\ne t_1+z_4
\]
Thus $tH$'s are disjoint. Moreover, for $\forall x\in\mathbb{R}$, $t_x:=x-\lceil{x}\rceil$, and thus $x\in t_xH$.
\end{proof}
\begin{proposition}
For vector subspace $W\subseteq V$, consider the subgroup $(W,+)\le(V,+)$. The set of cosets are the translations $v+W$, $v\in V$. Let $W'\subseteq V$ be a subspace complementary to $W$ such that
\begin{enumerate}
\item
$\dim(W)+\dim(W')=\dim(V)$
\item
$W\bigcap W'=\{0\}$
\end{enumerate}
Show that $V=\sqcup_{v\in W'}(v+W).$
\end{proposition}
\begin{example}
Consider the dihedral group $D_n$, we have
\[
D_n=\{\mbox{id},r,r^2,\dots,r^{n-1},s,rs,\dots,r^{n-1}s\}
\]
Thus we have $D_n=\gen{r}\sqcup\gen{r}s$.
\end{example}
Question,
\begin{example}
Consider $S_n$ with its subgroup $A_n$, we have $S_n=A_n\sqcup\tau A_n$
\end{example}
\begin{example}
Recall that $S_3=\gen{\rho,\mu}$ with $\rho=(123)$ and $\mu=(12)$. For the cyclic subgroup $H=\gen{\mu}$, we have $[S_3:H] = \{H,\rho H,\rho^2 H\}$
\end{example}
\begin{theorem}
Let $G$ be a finite group and $H\le G$, then $|G:H|=|G|/|H|$
\end{theorem}
\begin{proof}
Suppose $G=H\sqcup a_1H\sqcup\cdots\sqcup a_mH$. It suffces to show $|a_kH| =|H|$ for $k\in\{1,\dots,m\}$.

Suppose $H=\{h_1,\dots,h_n\}$, then immediately $|a_kH|=|\bigcup_{i=1}^n\{a_kh_i\}|\le n$. For distinct $h_i,_j\in H$, $a_kh_i=a_kh_j$ implies $h_i=h_j$, contradiciton.
\end{proof}
This theorem also works for right cosets.
\begin{proposition}
Let $G$ be a group and $H\le G$ s.t.
\[
G=H\sqcup a_1H\sqcup a_2H\sqcup\cdots\sqcup a_mH,
\]
show that $G=H\sqcup Ha_1^{-1}\sqcup Ha_2^{-1}\sqcup\cdots\sqcup Ha_m^{-1}$.
\end{proposition}
\begin{proof}
It's clear that $H,Ha^{-1}_1,\dots,Ha^{-1}_m$ are disjoint. Also, for $\forall g\in G$, consider $g^{-1}$:
\[
g^{-1}=a_ih\implies
g=h^{-1}a_{i}^{-1}\in Ha_i^{-1}
\]
\end{proof}
\begin{proposition}
Let $G$ be a finite group and $g\in G$, then $|g|||G|$.
\end{proposition}
\begin{proof}
Since $|g| = |\gen{g}|$
\end{proof}
\begin{proposition}
Let $G$ be a finite group of prime order, then $G$ is cyclic.
\end{proposition}
\begin{proof}
Let $p=|G|\ge2$. There exists non-trivial element $a$ s.t. $|a|$ divides $p$, and $|a|\ne 1$, i.e., $|a|=p$.
\end{proof}
\begin{proposition}
Let $G$ be a finite group. For each $g\in G$, we have
\[
g^{|G|} = e
\]
\end{proposition}
\begin{proof}
SInce $|g|$ divides $|G|$, there exists $k$ s.t.
\[
g^{|G|}=(g^{|g|})^k=e
\]
\end{proof}
Then we give a generalization of cosets:
\begin{definition}[Product of $H$ by $K$]
For a group $G$ and $H,K\subseteq G$, the \emph{product} of $H$ by $K$ is 
\[
HK:=\{hk\mid h\in H,k\in K\}
\]
\end{definition}
Note that $(HK)L=H(KL),\forall H,K,L\subseteq G$

If $H\le G$, it's clear that $HH = H$; the converse is not true in general, e.g., $H=\{\mbox{id},(12)\}$ and $G=S_3$.

\begin{theorem}
Let $G$ be a group and $H,G\le G$ be finite, then
\[
|HK| = \frac{|H||K|}{|H\bigcap K|}
\]
\end{theorem}
\begin{proof}
It suffices to show $|H||K| = |HK||H\bigcap K|$, e.g., construct a map $\phi: H\times K\mapsto HK$, it suffices to show $\phi$ is $|H\bigcap K|$-to-1 map. 
\begin{itemize}
\item
For $h_1,h_2\in H$ and $k_1,k_2\in K$, if the output of the mapping is the same, 
\[
h_1k_1=h_2k_2\implies
h_2^{-1}h_1=k_2k_1^{-1}:=d\in H\bigcap K
\]
and therefore $h_2 = h_1d^{-1}$ and $k_2 = dk_1$.
\item
On the other hand, for any $h\in H,k\in K,d\in H\bigcup K$, if $h'=hd^{-1}$ and $k'=dk$, we find $h'k'=hk$.
\end{itemize}
\end{proof}
question.
\begin{theorem}
Let $G$ be a group and $H,K\le G$. Then $HK$ is a group iff $HK = KH$
\end{theorem}
\begin{proof}
Let $h\in H,k\in K$

Necessity. $kh=(h^{-1}k^{-1})^{-1}\in HK$, i.e., $KH\subseteq HK$. On the other hand, $(hk)^{-1}\in HK$ implies $(hk)^{-1}:=h_1k_1$, i.e., $hk=k_1^{-1}h_1^{-1}\in KH$.

Sufficiency. For any $h_1k_1,h_2k_2\in HK$, we have
\[
(h_1k_1)(h_2k_2)^{-1}=h_1k_1k_2^{-1}h_2^{-1}
\in HKKH=HKHK=HHKK=HK.
\]
\end{proof}












