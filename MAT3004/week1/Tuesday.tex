
\chapter{Week1}

\section{Monday}\index{Monday_lecture}

\subsection{Introduction to Abstract Algebra}
The basic concepts include \emph{groups, rings, fields}.

One topic is algebra, i.e., the solvability of polynomials. (From Galois Theory to analysis)

\[
\mbox{Example: }
\frac{\diff \bm y}{\diff\bm x} = g(\bm x)\implies D\bm y=g(\bm x)
\]
with operatior $D:=\frac{\diff}{\diff \bm x}$. The operator $D$ forms a ring, i.e.,
\[
\{a_n(x)D^n+a_{n-1}(x)D^{n-1}+\cdots+a_0(x)\}\mapsto\mbox{ring}
\]

Second topic is number theory

Third topic is geometry, including \textit{algebraic geometry, differential geometry, topology, finite geometry, affine geometry, algebraic graph theory,, combinatorics}, with applications to coding theory, physics, crystallography chemestry.

\subsection{Group}
\begin{definition}[Group]\label{Def:1:1}
A group $\mathcal{G}$ is a set equipped with a binary operation, i.e.,
\[
\begin{array}{ll}
*:
&
\mathcal{G}\times\mathcal{G}\mapsto
\mathcal{G}
\end{array}
\]
such that:
\begin{enumerate}
\item
Associativity: 
$(a*b)*c = a*(b*c)$ for $\forall a,b,c\in\mathcal{G}.$
\item
Existence of Identity: 
$\exists$ an identity $e\in\mathcal{G}$ s.t. $e*g = g*e = g$ for $\forall g\in\mathcal{G}$.
\item
Existence of Inverse: 
$\forall g\in\mathcal{G}$, there exists an inverse $g^{-1}$ s.t. $g^{-1}* g=g*g^{-1} = e$.
\end{enumerate}
The size(order) of $\mathcal{G}$ is denoted by $|\mathcal{G}|$.
\end{definition}
\begin{remark}
\begin{itemize}
\item
If $a*b=b*a$ for $\forall a,b$, then $\mathcal{G}$ is called an \emph{abelian group}.
\item
If $|\mathcal{G}|=1$, then $\mathcal{G}$ is said to be \emph{trivial}, otherwise $\mathcal{G}$ is \emph{nontrivial}.  
\item
Similarly, the ternary operation means:
\[
\begin{array}{ll}
*:
&
\mathcal{G}\times\mathcal{G}\times\mathcal{G}\mapsto
\mathcal{G}
\end{array}
\]
\item
The semigroup definition only requires the (1) condition; and the menoid requies the (1) and (2) conditions.
\item
Is $\emptyset$ a group? By the second condition, it is not a group.
\end{itemize}
\end{remark}
Given a set $\mathcal{S}$ with its associated operation $*$, to check $(\mathcal{S},*)$ is a group, we need to check:
\begin{enumerate}
\item
$\mathcal{S}$ is \emph{closed} under the operation $*$, i.e., $a*b\in\mathcal{S}$ for $\forall a,b\in\mathcal{S}$
\item
\emph{Associativity}.
\item
\emph{Existence of Identity}
\item
\emph{Existence of Inverse}
\end{enumerate}

\begin{proposition}
$(\mathbb{Q},+)$ is a group.
\end{proposition}
\begin{proof}
\begin{enumerate}
\item
For $\forall a,b\in\mathbb{Q}$, it is easy to show $a+b\in\mathbb{Q}$.
\item
Associativity: $(a+b)+c=a+(b+c)$ for $\forall a,b\in\mathbb{Q}$
\item
Existence of Identity: Take the identity $0\in\mathbb{Q}$, we have $0+a=a+0=a$ for $\forall a\in\mathbb{Q}$.
\item
Existence of Inverse: For $\forall a\in\mathbb{Q}$, it follows that $(-a)\in\mathbb{Q}$ s.t. $(-a)+a=a+(-a)=0$.
\end{enumerate}
\end{proof}
Note that $(\mathbb{Q},\cdot)$ is not a group since inverse does not exist.
\begin{remark}
Note that the existence of identity is unique, which will be shown in the future.
\end{remark}
\begin{proposition}
$(u_m,\cdot)$ is a group, where
\[
u_m=\{1,\xi^m,\dots,\xi_m^{m-1}\}
\] 
with $\xi^m=1$ and $\xi\ne1$.
\end{proposition}
\begin{proof}
\begin{enumerate}
\item
Note that for $\forall \xi^j,\xi^k\in u_m$, we have
\[
\xi^j\cdot\xi^k:=\xi^{j+k}=\left\{
\begin{aligned}
\xi^{j+k}, &j+k\le m-1\\
\xi^{j+k-m}, &j+k\ge m
\end{aligned}
\right.
\]
\item
The associativity is easy to show.
\item
Take the identity $e=1$.
\item
For $\forall\xi^k\in u_m$, we take the inverse $\xi^{m-k}$.
\end{enumerate}
\end{proof}
\begin{proposition}
The set $\mathcal{G} = \{\mbox{bijections of $\mathbb{R}$}\}$ associated with the \emph{conposition} operator is a group.
\end{proposition}
\begin{definition}[bijection]
The bijection contains \emph{injective}, i.e., $f(x)=f(y)$ implies $x=y$; and \emph{supjective}, i.e., $\forall y\in\mathcal{B}, \exists x\in\mathcal{A}$ s.t. $f(x)=y$.
\end{definition}
\begin{proof}
\begin{enumerate}
\item
$\forall f,g\in\mathcal{G}$, 
\begin{itemize}
\item
Injective: take $x,y\in\mathbb{R}$ s.t. $(f\odot g)(x)=(f\odot g)(y)$, it follows that 
\[
f(g(x))=f(g(y))\implies g(x)=g(y)\implies x=y.
\]
\item
Subjective: take $y\in\mathbb{R}$ s.t. $f(z)=y$. Hence, $\exists x\in\mathbb{R}$ s.t. $g(x)=z$, which implies $f(g(x))=y$.
\end{itemize}
\item
For any functions $f,g,h\in\mathcal{G}$,
\[
((f\odot g)\odot h)(x) = (f\odot g)(h(x)) = f(g(h(x))), \forall x\in\mathbb{R}
\]
Similarly,
\[
(f\odot(g\odot h))(x) = f((g\odot h)(x)) = f(g(h(x))), \forall x\in\mathbb{R}
\]
\item
Define $e:x\mapsto x$. Then $e\in\mathbb{G}$. It follows that
\[
(e*g)(x) = e(g(x)) = g(x)
\]
Similarly, $(g*e)(x) = g(x)$. Hence, $e$ is the identity.
\item
For $\forall f\in\mathcal{G}$, take $f^{-1}: f(x)\mapsto x$. Firstly verify $f^{-1}$ is a bijection. Then we have
\[
f^{-1}\odot f = f\odot f^{-1} = e.
\]
\end{enumerate}
\end{proof}

Recall a definition from Linear Algebra:
\[
\mbox{GL}(n,\mathbb{R}):=\{\bm A\in\mathcal{M}_n(\mathbb{R})\mid \det(\bm A)\ne0\}
\]
where $\mathcal{M}_n(\mathbb{R})$ denotes the set of $n\times n$ matrices over $\mathbb{R}$.
\begin{proposition}
The set $\mbox{GL}(n,\mathbb{R})$ associated with the matrix multiplication operator is the general linear group.
\end{proposition}
\begin{proof}
\begin{enumerate}
\item
$\forall \bm A,\bm B\in\mbox{GL}(n,\mathbb{R})$, we have $\bm A\bm B\in\mbox{GL}(n,\mathbb{R})$ since
\[
\det(\bm A\bm B)=\det(\bm A)\det(\bm B)\ne0
\]
\item
Associativity of matrix multiplication is easy to verify
\item
Take the identity $e:=\bm I_n$
\item
Inverse is $\bm A^{-1}$.
\end{enumerate}
\end{proof}
\begin{remark}
$\mbox{SL}(n,\mathbb{R}):=\{\bm A\in\mathcal{M}_n(\mathbb{R})\mid \det(\bm A)=1\}$ is a special linear group.
\end{remark}

\begin{proposition}
Let $n\in\mathbb{Z}^+$, for the set 
\[
\mathbb{Z}_n:=\{0,1,\dots,n-1\}
\]
associated with the operation
\[
\begin{array}{ll}
{+_n}\mbox{ such that}
&
a{+_n}b=\left\{
\begin{aligned}
a+b, &\mbox{ if $a+b\le n-1$}
\\
a+b -n , &\mbox{ if $a+b\ge n$}
\end{aligned}
\right.
\end{array}
\]
\end{proposition}
\begin{proof}
\begin{enumerate}
\item
Closed under operation
\item
Associativity: 
\[
(a+_n b+_n c) = a+b+c\in\mathbb{Z}_n\mbox{ or }a+b+c -n\in\mathbb{Z}_n\mbox{ or }a+b+c-2n\in\mathbb{Z}_n
\]
\item
Identity?
\item
Inverse?
\end{enumerate}
\end{proof}
In the future we abuse the operator $+$ to denote the $+_n$ for $\mathbb{Z}_n$.
\begin{theorem}\label{Theorem:1:1}
Given a sequence of elements $g_1,\dots,g_n$ in the group $\mathcal{G}$ associated with $\cdot$, the product is independent from adding brackets.
\end{theorem}
\begin{proof}
We show it by induction. Let $\mathcal{P}(n)$ denotes the product is the same whatever different ways of putting brackets on $g_1,\dots,g_n$
\begin{enumerate}
\item
Easy to verify $\mathcal{P}(1)$ is true.
\item
Assume $\mathcal{P}(n)$ is true for $n\le k$. Consider $n=k+1$. For $\forall m\le  n$, we have
\begin{align*}
(g_1g_2\dots g_m)(g_{m+1}\dots g_{k+1})&
\\
&=(g_1(g_2\dots g_m))(g_{m+1}\dots g_{k+1})\\
&=g_1((g_2\dots g_m)(g_{m+1}\dots g_{k+1}))\\
&=g_1(g_2\dots g_{k+1})\\
&=g_1\dots g_{k+1}
\end{align*}
\end{enumerate}
\end{proof}
\begin{remark}
Theorem (\ref{Theorem:1:1}) shows that given a sequence of elements multiplied together, we do not need to specify the order of operations that performed.

Moreover, for \emph{abelian} groups, the group is unqiue regardless of the ordering of elements.
\end{remark}

\begin{theorem}\label{Theorem:1:2}
Each group $(\mathcal{G},*)$ has the unique identity.
\end{theorem}
\begin{proof}
Let $e,e'\in\mathcal{G}$ be two identites. By definition,
\[
e' = e'* e = e.
\]
\end{proof}
\begin{theorem}\label{Theorem:1:3}
Let $\mathcal{G}$ be a group, then $g^{-1}$ is unique for any $g\in\mathcal{G}$.
\end{theorem}
\begin{proof}
Let $h_1,h_2$ be two inverses of $g\in\mathcal{G}$, by definition,
\[
h_1 = h_1\cdot e = h_1\cdot (gh_2) = (h_1g)h_2 = e\cdot h_2 = h_2.
\]
\end{proof}
\begin{remark}
Due to Theorem(\ref{Theorem:1:1}) to (\ref{Theorem:1:3}), it makes sense to define
\[
\begin{array}{lll}
\qquad g^n:=\underbrace{g\cdot g\cdots g}_{\text{$n$ times}},
\qquad
&
\qquad g^{-n}:=\underbrace{g^{-1}\cdot g^{-1}\cdots g^{-1}}_{\text{$n$ times}},
\qquad
&
\qquad g^0:=e.
\end{array}
\]
\end{remark}


\begin{proposition}
Let $(\mathcal{G},\cdot)$ be a group, then
\begin{enumerate}
\item
$(g^{-1})^{-1}=g, \forall g\in\mathcal{G}$
\item
$(ab)^{-1}=b^{-1}a^{-1}$, $\forall a,b\in\mathcal{G}$
\item
$g^m\cdot g^n = g^{m+n}$, $\forall g\in\mathcal{G}, m,n\in\mathbb{Z}$.
\end{enumerate}
\end{proposition}
In fact, we can redefine groups as the semigroups with a weaker condition.
\begin{definition}[Group]\label{Def:1:3}
\begin{itemize}
\item
A group $\mathcal{G}$ is a \emph{semigroup} with a binary operation $\cdot$ such that
\begin{itemize}
\item
There exists a left identity $e\in\mathcal{G}$ s.t.
\[
\begin{array}{ll}
e\cdot a=a,
&
\forall a\in\mathcal{G}
\end{array}
\]
\item
For each $a\in\mathcal{G}$, there exists a left inverse $a^{-1}\in\mathcal{G}$ s.t.
\[
a^{-1}a = e.
\]
\end{itemize}
\item
A set $\mathcal{G}$ equipped with a binary operation $\cdot$ is said to be a \emph{semigroup} if
\begin{enumerate}
\item
It is closed under the operation $\cdot$
\item
It satisfies the associativity.
\end{enumerate}
\end{itemize}
\end{definition}

\begin{proposition}
The definition(\ref{Def:1:1}) adn definition(\ref{Def:1:3}) are equivalent.
\end{proposition}
\begin{proof}
It suffices to show that the left identity and left inverse are essentially identity and inverse, respectively.
\begin{enumerate}
\item
Suppose $a^{-1}$ is the left inverse of $a$, we have
\begin{align*}
(a^{-1})^{-1}a^{-1}a = ((a^{-1})^{-1}a^{-1})a=ea=a
\end{align*}
It follows that
\begin{align*}
aa^{-1}&=(a^{-1})^{-1}a^{-1}aa^{-1}\\
&=(a^{-1})^{-1}(a^{-1}a)a^{-1}\\
&=(a^{-1})^{-1}ea^{-1}\\
&=e
\end{align*}
\item
Suppose $e$ is the left identity, it follows that
\[
ae = a(a^{-1}a) = (aa^{-1})a=ea=a
\]
\end{enumerate}
\end{proof}
Similarly, we can define a group with help of right identities and right inverses; but we cannot define a group by left identities and right inverses. Here is a counterexample:
\begin{example}
Let $(\mathcal{G},*)$ be a group with at least 2 elements such that
\[
\begin{array}{ll}
a*b = b,
&
\forall a,b\in\mathcal{G}
\end{array}
\]
Note that $\mathcal{G}$ is closed under $*$ and associative. The group $\mathcal{G}$ has left identities and rihgt inverses. But $\mathcal{G}$ is not a group by definition.
\end{example}

If a group is \emph{finite}, then its operation can be described by its Cayley table, or multiplication table.
\begin{example}
For a group $\mathcal{G}=\{e,a\}$ equipped with $*$, its Cayley table is given by: 
\[\begin{array}{|c||c|c|}
\hline
* & e & a  \\
\hline
e& e & a \\
\hline
a & a & e \\
\hline
\end{array}\]

For a group $(\mathbb{Z}_6,+)$, its Cayley table is given by: 
\[\begin{array}{|c||c|c|c|c|c|c|}
\hline
+&0&1&2&3&4&5
\\\hline
0&0&1&2&3&4&5
\\\hline
1&1&2&3&4&5&0
\\\hline
2&2&3&4&5&0&1
\\\hline
3&3&4&5&0&1&2
\\\hline
4&4&5&0&1&2&3
\\\hline
5&5&0&1&2&3&4
\\\hline
\end{array}\]
\end{example}
\begin{remark}
Note that a group can be described by a cayley table. But not all cayley tables define a group.
\end{remark}
\begin{remark}
It is usually messy to check whether the cayley table defines a group. There are a few necessary conditions we can examine (to exclude Cayley Tables that don't define groups):
\begin{itemize}
\item
It has a row and column that is identical to the list of the elements
\item
The elements in each row and each column must be all distinct.
\end{itemize}
\end{remark}

















