\section{Assignment One}\index{Assignment One}
\begin{enumerate}
\item
Consider the system
\begin{align*}
ax+2y+3z&=b_1\\
ax+ay+4z&=b_2\\
ax+ay+az&=b_3\\
\end{align*}
For what three values of $a$ will the \textit{elimination} fail to give the pivots? (\textit{Pivots means the first nonzero entry on rows.})
\item
It is impossible for a system of linear equations to have \textit{exactly} two solutions? Explain your answers. And you may consider the following questions as intuitions to derive your final solution.
\begin{enumerate}
\item
In $\mathbb{R}^3$ if $(x,y,z)$ and $(X,Y,Z)$ are two solutions, what is another one?
\item
In $\mathbb{R}^3$ if $25$ planes meet at two points, where else do they meet?
\item
Extend the argument to $\mathbb{R}^n.$
\end{enumerate}
\item
In the following system
\begin{align*}
x+4y-2z&=1\\
x+7y-6z&=6\\
3y+qz&=t
\end{align*}
\begin{enumerate}
\item
Which number $q$ makes this system \textit{singular}? Moreover, if this system is \textit{singular}, which right-hand side $t$ gives \textit{infinitely} many solutions?
\item
Find the solution that has $z=1.$
\end{enumerate}
\item
By \emph{trial} and \emph{error}, find examples of $2\x 2$ matrices such that:
\begin{enumerate}
\item
$\bm A^2=-\bm I$, where $\bm A$ has \textit{real} entries.
\item
$\bm B^2=\bm0$, where $\bm B\ne\bm0.$
\item
$\bm{CD}=-\bm{DC},$ where $\bm CD\ne\bm0.$
\item
$\bm{EF}=\bm0$, and no entries of $\bm E$ or $\bm F$ are zero.
\end{enumerate}
\item
For \textit{real} matrices $\bm A,\bm B,\bm C$ in \textit{finite} field, prove the \textit{associativity product rule:}
\[
(\bm{AB})\bm C=\bm A(\bm{BC}).
\]
\item
Matrices can be cut into blocks (which are smaller matrices). Here is a 4 by 6 matrix broken into blocks of size $2$ by $2$, in this example each block is just $\bm I$:
\[
\begin{aligned}
\text{\emph{4 by 6 matrix}}\\
\text{\emph{2 by 2 blocks}}
\end{aligned}\qquad
\bm A=\left[
\begin{array}{@{}cc|cc|cc@{}}
1&0&1&0&1&0\\
0&1&0&1&0&1\\
\hline
1&0&1&0&1&0\\
0&1&0&1&0&1
\end{array}
\right]=\begin{bmatrix}
\bm I&\bm I&\bm I\\\bm I&\bm I&\bm I
\end{bmatrix}.
\]
We give the definition for \textit{block multiplication}:
\begin{definition}[Block Multiplication]
If the cuts between columns of $\bm A$ match the cuts between rows of $\bm B$, then block multiplication of $\bm{AB}$ is allowed: 
\[
\begin{bmatrix}
\bm A_{11}&\bm A_{12}&\bm A_{21}&\bm A_{22}
\end{bmatrix}\begin{bmatrix}
\bm B_{11}&\cdots\\\bm B_{21}&\cdots
\end{bmatrix}=\begin{bmatrix}
\bm A_{11}\bm B_{11}+\bm A_{12}\bm B_{21}&\cdots\\
\bm A_{21}\bm B_{11}+\bm A_{22}\bm B_{21}&\cdots
\end{bmatrix}.
\]
\end{definition}
If we have $\bm A,\bm B$ such that
\[
\bm A\bm B=\left[
\begin{array}{@{}cc|c@{}}
\x&\x&\x\\
\x&\x&\x\\
\hline
\x&\x&\x
\end{array}\right]\left[
\begin{array}{@{}cc|c@{}}
\x&\x&\x\\
\x&\x&\x\\
\hline
\x&\x&\x\\
\end{array}\right],
\]
replace $\x$ by numbers to verify the block multiplication succeeds.
\item
\[
\begin{array}{@{}cc|c@{}}
4&0&4\\
6&6&8\\
\hline
-9&5&-8
\end{array}
\bm A=\begin{bmatrix}
a&a&a&a\\a&b&b&b\\a&b&c&c\\a&b&c&d
\end{bmatrix}.
\]
Seperate $\bm A$ into $\bm L$ and $\bm U$. Moreover, Find four conditions on $a,b,c,d$ to let $\bm A$ have \textit{four} pivots.
\end{enumerate}