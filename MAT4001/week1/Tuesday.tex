
\chapter{Week1}
\section{Differentiation}
\begin{definition}[Forward and Backforward Difference Formula]
\[
f'(x_0)=\frac{f(x_0+h)-f(x_0)}{h}-\frac{h}{2}f''(\xi)
\]
Consider the Taylor expansion, we have
\[
f''(x_0)=\frac{1}{h^2}[f(x_0-h)-2f(x_0)+f(x_0+h)]-\frac{h^2}{12}f^{(4)}(\xi)
\]
\end{definition}
\begin{definition}[General Derivative Approximation]
The interpolation formula gives
\[
f(x)=\sum_{k=0}^nf(x_k)L_k(x)+\frac{(x-x_0)\cdots(x-x_n)}{(n+1)!}f^{(n+1)}(\xi_x)
\]
Differentiating both sides gives ($x_j$ is one of the node points)
\[
f'(x_j)=\sum_{k=0}^nf(x_k)L_k'(x_j)+\frac{1}{(n+1)!}f^{(n+1)}(\xi_{x_j})\prod_{k=0,k\ne j}^n(x_j-x_k)
\]
\end{definition}
\begin{remark}
Three-point formula:
\begin{align*}
f'(x_j)&=f(x_0)\left[\frac{2x_j-x_1-x_2}{(x_0-x_1)(x_0-x_2)}\right]+f(x_1)\left[\frac{2x_j-x_0-x_2}{(x_1-x_0)(x_1-x_2)}\right]\\
&+f(x_2)\left[\frac{2x_j-x_0-x_1}{(x_2-x_0)(x_2-x_1)}\right]+\frac{1}{6}f^{(3)}(\xi_j)\prod_{k=0,k\ne j}^2(x_j-x_k)
\end{align*}
Substituting $x_j=x_0,x_1,x_2$, we obtain:
\begin{align*}
f'(x_0)&=\frac{1}{2h}[-3f(x_0)+4f(x_0+h)-f(x_0+2h)]+\frac{h^2}{3}f^{(3)}(\xi_0)\\
f'(x_0)&=\frac{1}{2h}[-f(x_0-h)+f(x_0+h)]-\frac{h^2}{6}f^{(3)}(\xi_1)
\end{align*}
\end{remark}
\begin{definition}[Richardson Extrapolation]
For the approximation with the form
\[
M=N_1(h)+K_1h^2+K_2h^4+\cdots
\]
we have the approximation
\[
N_j(h)=N_{j-1}(\frac{h}{2})+\frac{N_{j-1}(h/2) - N_{j-1}(h)}{4^{j-1}-1}
\]
where $N_j(h)$ has order $O(h^{2j})$
\end{definition}
round-off error in $N_1(h/2^k)$; we recommend comparing the final diagonal entries to ensure accuracy.

\begin{definition}[Quadrature formula]
\[
\int_a^bf(x)\diff x\approx\sum_{i=0}^na_if(x_i)
\]
where
\begin{align*}
a_i&=\int_a^bL_i(x)\diff x\\
E(f)&=\frac{1}{(n+1)!}\int_a^b\prod_{i=0}^n(x-x_i)f^{(n+1)}(\xi_x)\diff x
\end{align*}
\end{definition}
\begin{enumerate}
\item
Trapezoidal Rule:
\[
\int_a^bf(x)\diff x=\frac{h}{2}[f(x_0)+f(x_1)]-\frac{h^3}{12}f'(\xi)
\]
Seperating into subintervals $[x_{k-1},x_k]$, we have, with $h=(b-a)/n$,
\[
\int_a^bf(x)\diff x=\frac{h}{2}\left[f(a)+2\sum_{j=1}^{n-1}f(x_j)+f(b)\right]-\frac{b-a}{12}h^2f''(\mu)
\]
\item
Simpson's rule: for $h=(x_2-x_0)/2$,
\[
\int_{x_0}^{x_2}f(x)\diff x=\frac{h}{3}[f(x_0)+4f(x_1)+f(x_2)]-\frac{h^5}{90}f^{(4)}(\xi)
\]
Error Bound: Taylor expansion of $f(x)$, and bound the integrand
\[
\frac{1}{24}\int_{x_0}^{x_2}f^{(4)}(\xi_x)(x-x_1)^4=\left.\frac{f^{(4)}(\xi_1)}{120}(x-x_1)^5\right|_{x_0}^{x_2}=\frac{f^{(4)}(\xi_1)}{60}h^5
\]
Seperating into subintervals $[x_0,x_2],[x_2,x_4],\dots,[x_{n-2},x_n]$, we have, with $h=(b-a)/n$,
\begin{align*}
\int_a^bf(x)\diff x&=\sum_{j=1}^{n/2}\left\{\frac{h}{3}[f(x_{2j-2}+4f(x_{2j-1})+f(x_{2j}))]\right\}\\
\mbox{Error}&=-\frac{h^5}{90}\sum_{j=1}^{n/2}f^{(4)}(\xi_j)
=
-\frac{h^5}{90}\frac{n}{2}f^{(4)}\mu=-\frac{b-a}{180}h^4f^{(4)}(\mu)
\end{align*}
\item
Composite Mid-point rule: for subintervals $[x_{-1},x_1],\dots,[x_{n-1},x_{n+1}]$ with centers $x_0,x_2,\dots,x_n$, and $h=(b-a)/(n+2)$
\[
\int_a^bf(x)\diff x=2h\sum_{j=0}^{n/2}f(x_{2j})+\frac{b-a}{6}h^2f''(\mu)
\]
\end{enumerate}


\begin{definition}[Degree of Precision]
The degree of precision of a quadrature formula is $n$ if and only if the error is zero for all polynomials of degree $k = 0,1,\cdots,n$, but is not zero for some polynomial of degree $n + 1.$
\end{definition}
\begin{definition}[Romberg Method]
First column: for $k=2,\dots,n$, $h_k=(b-a)/(2^{k-1})$
\[
R_{k,1}=\frac{1}{2}\left[
R_{k-1,1}+h_{k-1}\sum_{i=1}^{2^{k-2}}f(a+(2i-1)h_k)
\right]
\]
For $k=j,j+1,\dots$,
\[
R_{k,j}=R_{k,j-1}+\frac{1}{4^{j-1}-1}(R_{k,j-1}-R_{k-1,j-1})
\]
\end{definition}
Stopping Criteria: $|R_{n-1,n-1}-R_{n,n}|<\mbox{tol}$ and $|R_{n-2,n-2}-R_{n-1,n-1}|<\mbox{tol}$. Ensure two differently generated sets of approximations agree within the specified tolerance.
\begin{definition}[Gauss-Quadrature Rule]
Generate Legendre poylnomial $P_0(x)=1,P_1(x)=x$,
\[
(n+1)P_{n+1}(x)=(2n+1)xP_n(x)-nP_{n-1}(x)
\]
Let $\{x_0,\dots,x_n\}$ be rooots of $P_{n+1}(x)$, and
\[
w_i=\int_{-1}^1l_i(x)\diff x=\int_{-1}^1\prod_{j=0,j\ne i}^n\frac{x-x_j}{x_i-x_j}\diff x,\quad i=0,1,\dots,n
\]
which implies
\[
\int_{-1}^1f(x)\diff x=\sum_{j=0}^nw_if(x_i)+\frac{f^{(2n+2)}(\xi)}{(2n+2)!}\int_{-1}^1\prod_{i=0}^n(x-x_i)^2\diff x
\]
\end{definition}
\begin{definition}[Pivoting]
Why: small $a_{kk}^{(k)}$ leads big error. 
\begin{itemize}
\item
Partial: select an element in the same column that is below the diagonal and has the largest absolute value.
\item
Scaled partial: first compute scale $s_i$ for each row; then do the same as partial
\end{itemize}
\end{definition}
\begin{proposition}
\begin{itemize}
\item
$\|\bm A\|_1=\max_{1\le j\le n}\sum_{i=1}^m|a_{ij}|$
\item
$\|\bm A\|_2=\sqrt{\lambda_{\max}(\bm A\trans\bm A)}$. If $\bm A$ symmetric, then $\|\bm A\|_2=\rho(\bm A)=\max|\lambda(\bm A)|$
\item
$\|\bm A\|_\infty=\max_{1\le i\le m}\sum_{j=1}^n|a_{ij}|$
\item
$\rho(\bm A)\le\|\bm A\|$ for any natural norm.
\item
$\bm A$ is convergent iff $\rho(\bm A)<1$.
\end{itemize}
\end{proposition}
\begin{proof}
For eigen-pair ($\lambda,\bm x$) with $\|\bm x\|=1$,
\[
|\lambda|=\|\lambda\bm x\|\le\|\bm A\|\|\bm x\|=\|\bm A\|
\]
\end{proof}
\begin{definition}[Jacobi's Method]
\[
x_i^{(k)} = \frac{1}{a_{ii}}\left[\sum_{j=1,j\ne i}^n(-a_{ij}x_j^{(k-1)})
+b_i
\right]
\]
Or for the matrix form $\bm x^{(k)}=\bm T_j\bm x^{(k-1)}+\bm c$:
\[
\bm x^{(k)}=\bm D^{-1}(\bm L+\bm U)\bm x^{(k-1)}+\bm D^{-1}\bm b,
\]
where $\bm D$ are the diagonal of $\bm A$; $-\bm L,-\bm U$ are the strictly lower and upper part of $\bm A$.
\end{definition}

\begin{definition}[Gauss-Seidel Method]
\[
x_i^{(k)}=\frac{1}{a_{ii}}
\left[
-\sum_{j=1}^{i-1}(a_{ij}x_j^{(k)})-\sum_{j=i+1}^n(a_{ij}x_j^{k-1})+b_i
\right]
\]
Or for the matrix form $\bm x^{(k)}=\bm T_g\bm x^{(k-1)}+\bm c_g$:
\[
(\bm D-\bm L)\bm x^{(k)}=\bm U\bm x^{(k-1)}+\bm b\implies
\bm x^{(k)}=(\bm D-\bm L)^{-1}\bm U\bm x^{(k-1)}+(\bm D-\bm L)^{-1}\bm b
\]
\end{definition}
\begin{proposition}
If $\rho(\bm T)<1$, then $(\bm I-\bm T)^{-1}$ exists, and
\[
(\bm I-\bm T)^{-1}=\bm I+\bm T+\bm T^2+\cdots
\]
\end{proposition}
\begin{proposition}
The iteration
$
\bm x^{(k)}=\bm T\bm x^{(k-1)}+\bm c
$
converges to the unique solution to 
\[
\bm x=\bm T\bm x+\bm c
\]
iff $\rho(\bm T)<1$. The error bound holds:
\begin{itemize}
\item
$\|\bm x-\bm x^{(k)}\|\le\|\bm T\|^k\|\bm x^{(0)}-\bm x\|$
\item
$\|\bm x-\bm x^{(k)}\|\le\frac{\|\bm T\|^k}{1-\|\bm T\|}\|\bm x^{(1)}-\bm x^{(0)}\|$
\item
$\|\bm x-\bm x^{(k)}\|\approx[\rho(\bm T)]^k\|\bm x^{(0)}-\bm x\|$
\end{itemize}
\end{proposition}
\begin{proof}
Converse: Express $\bm x^{(k)}$ as
\[
\bm x^{(k)}=\bm T^k\bm x^{(0)}+(\bm T^{k-1}+\cdots+\bm T+\bm I)\bm c
\]
Forward: Assume converge to $\bm x$, and therefore
\[
\bm x-\bm x^{(k)}=\bm T(\bm x-\bm x^{(k-1)})=\cdots=\bm T^k(\bm x-\bm x^{(0)})
\]
for arbitrary $\bm x^{(0)}.$ Taking limit implies $\lim_{k\to\infty}\bm T^k\bm z=\bm0$ for any $\bm z$.
\end{proof}

\begin{proposition}
Sufficient condition for convergence of Jacobi and Gauss-Seidel method: $\bm A$ is strictly diagonally dominant, i.e., $|a_{ii}|>\sum_{j=1,j\ne i}^n|a_{ij}|$
\end{proposition}
\begin{proposition}
For $a_{ij}\le0, i\ne j$ and $a_{ii}>0$, one and only one conditon holds:
\begin{itemize}
\item
$0\le\rho(\bm T_g)<\rho(\bm T_j)<1$
\item
$1<\rho(\bm T_j)<\rho(\bm T_g)$
\item
$\rho(\bm T_j)=\rho(\bm T_g)=0$
\item
$\rho(\bm T_j)=\rho(\bm T_g)=1$
\end{itemize}
\end{proposition}
\begin{definition}[Gauss-Seidel Method and Relaxation]
\[
 x_i^{(k)}= x_i^{(k-1)}+\frac{r_{ii}^{(k)}}{a_{ii}}
\]
Relaxation:
\[
x_i^{(k)}= x_i^{(k-1)}+\omega\frac{r_{ii}^{(k)}}{a_{ii}}
\]
Or equivalently, ($\omega<1$ is under-relaxation methods, $\omega>1$ is over-relaxation methods.)
\[
x_i^{(k)}=
(1-\omega)x_i^{(k-1)}+
\frac{\omega}{a_{ii}}
\left[
-\sum_{j=1}^{i-1}(a_{ij}x_j^{(k)})-\sum_{j=i+1}^n(a_{ij}x_j^{k-1})+b_i
\right]
\]
\[
\bm x^{(k)}=(\bm D-\omega\bm L)^{-1}[(1-\omega)\bm D+\omega\bm U]\bm x^{(k-1)}
+
\omega(\bm D-\omega\bm L)^{-1}\bm b
\]
\end{definition}
\begin{proposition}
If $a_{ii}\ne0$, then $\rho(T_\omega)\ge|\omega-1|$, o.e., the SOR method converges only when $0<\omega<2$; sufficient condition for convergence: $\bm A$ is PD and $0<\omega<2$.
\end{proposition}
\begin{proposition}
If $\bm A$ is PD and tridiagonal, then $\rho(\bm T_g)=[\rho(T_j)]^2<1$, the optimal choice is
\[
\omega=\frac{2}{1+\sqrt{1-[\rho(T_j)]^2}}
\]
under this choice, we have $\rho(\bm T_\omega)=\omega-1$
\end{proposition}
\begin{enumerate}
\item
Direct method: Gaussian Eliminiation;

Advantage: Exact method, no truncation error;

Disadvantage: Computationally expansive, large round off error

Suitable: linear systems of small dimension 
\item
Iterative method:

Advantage: efficient in terms of both computer storage and computation

Disadvantage: not so efficient for small dimension

Suitable for: large linear sparse systems
\end{enumerate}
\begin{definition}
The condition number of nonsingular matrix $\bm A$ is $K(\bm A)=\|\bm A\|\|\bm A^{-1}\|$.

If close to $1$, then $\bm A$ is well-conditioned, otherwise ill-conditioned.
\end{definition}

\begin{theorem}
For natural norm,
\[
\|\tilde{\bm x}-\bm x\|\le\|\bm A^{-1}\|\|\bm r\|=K(\bm A)\frac{\|\bm r\|}{\|\bm A\|}
\]
\[
\frac{\|\tilde{\bm x}-\bm x\|}{\|\bm x\|}\le\|\bm A\|\|\bm A^{-1}\|\frac{\|\bm r\|}{\|\bm b\|}=K(\bm A)\frac{\|\bm r\|}{\|\bm b\|},\quad
\bm x\ne0,\bm b\ne0
\]
\end{theorem}
\begin{proof}
Consider the first equality and $\|\bm b\|\le\|\bm A\|\|\bm x\|$
\end{proof}

\begin{theorem}
Suppose $\bm A$ is nonsingular and $\|\delta\bm A\|\le\frac{1}{\|\bm A\|}$, the solution $\tilde{\bm x}$ to $(\bm A+\delta\bm A)\bm x=\bm b+\delta\bm b$ has the bound
\[
\frac{\|\tilde{\bm x}-\bm x\|}{\|\bm x\|}\le\frac{K(\bm A)\|\bm A\|}{\|\bm A\|-K(\bm A)\|\delta\bm A\|}\left(
\frac{\|\delta\bm b\|}{\|\bm b\|}+\frac{\delta\bm A}{\|\bm A\|}
\right)
\]
\end{theorem}
\begin{proposition}
For $\bm A\in\mathbb{R}^{n\times n}$, $\mbox{cond}(\bm A\trans\bm A)=[\mbox{cond}(\bm A)]^2$
\end{proposition}
\begin{theorem}
For column full rank $\bm A\in\mathbb{R}^{m\times n}$, we have $\bm A=\bm Q\bm R$, for $\bm Q\in\mathbb{R}^{m\times n}$ with orthogonal columns, $\bm R\in\mathbb{R}^{n\times n}$ is an upper triangular matrix.

As a result, the least square solution becomes
\[
\bm x^*=\bm R^{-1}\bm Q\trans\bm b
\]
\end{theorem}
Gram-Schmidt Process: numerically unstable; ease of implementation
\begin{theorem}[Householder matrix]
For given vector $\bm x$ and unit vector $\bm g$, define $\b, H:=\bm I-2\bm u\bm u\trans$,
\[
\bm{Hx}=\|\bm x\|\bm g,\quad
\bm u=\frac{\bm x-\|\bm x\|g}{\|\bm x-\|\bm x\|g\|}
\]
\[
\bm H_2=\begin{pmatrix}
1&0\\0&\tilde{\bm H}_2
\end{pmatrix}
\]
Define $\bm Q=\bm H_n\cdots\bm H_1$, and $\bm Q\bm A=\bm R=[\bm R_1;\bm0]$, which implies
\[
\bm A=[\bm Q_1,\bm Q_2]\begin{pmatrix}
\bm R_1\\\bm0
\end{pmatrix}=\bm Q_1\bm R_1
\]
\end{theorem}

















