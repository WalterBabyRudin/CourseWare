%\chapter{Introduction to Linear Programming}
\chapter{Planar}

\begin{definition}[Planar]
An undirected graph is \emph{planar}
if it can be drawn in the plane without \emph{edges crossing}.
Such a way of drawing is a \emph{plane drawing} of the graph,
or an \emph{embedding} of the graph in the plane.
\end{definition}
For example, $K_4$ is a planar graph, and only two way of drawing are plane drawings.

\paragraph{Characterisation of Planar Graphs}
\begin{itemize}
\item
Relationships among vertices, edges, and faces?
\item
Smallest non-planar graph?
\item
Necessary condition and sufficient conditions for a graph to be planar?
\item
Dual of a planar graph?
\item
Measure of non-planarity?
\end{itemize}

\begin{definition}[Regular Polyhedron]
A \emph{regular polygon} is a convex figure (in 2-dimensions) where all edges are of same length and all (internal) angles at the vertices are the same.

A \emph{regular polyhedron} is a solid (convex) figure (in 3 dimensions) with \emph{all faces} being congruent regular polygons, with the same number arranged around each vertex.
\end{definition}
\begin{remark}
For any plane drawing~(embedding) of a planar graph, the plane is divided into faces. 
The face unbounded by the edges of the embedding is called the \emph{infinite} face.
A planar graph can have different embeddings, but the number of faces in any embedding is the same.
\end{remark}
\section{Euler’s Polyhedral Formula}
\begin{theorem}[Euler]
Let $G$ be a plane drawing of connected planar graph with $n$ vertices and $m$ edges.
Let $f$ be the number of faces of $G$. Then
\[
n-m+f=2
\]
\end{theorem}
\begin{proof}
The proof is by induction on the number of edges $m$.
For $m=0$, we imply $n=1,f=1$, i.e., $n-m+f=2$.
Assume that the formula holds for all planar connected graphs with at most $m-1$ edges, and let $G$ be a plane drawing of a graph with $m$ edges.
\begin{itemize}
\item
If $G$ is a tree, then $n=m+1$ and $f=1$, i.e., $n-m+f=2$
\item
If $G$ is not a tree, let $e$ be an edge in some cycle of $G$, and $G-e$ is a connected planar graph with $n$ vertices, $m-1$ edges, and $f-1$ faces. By induction hypothesis, $n-(m-1)+(f-1)=2$, which follows that $n-m+f=2$.
\end{itemize}
\end{proof}

\begin{corollary}
Let $G$ be a plane graph with $n$ vertices, $m$ edges and $f$ faces, $k$ components, then
\[
n-m+f=k+1
\]
\end{corollary}

\begin{corollary}
\begin{enumerate}
\item
Let $G$ be a simple connected planar graph with $n\ge3$ vertices and $m$ edges, then $m\le 3n-6$.
\item
If in addition, $G$ has no $3$-cycles, then $m\le 2n-4$.
\end{enumerate}
\end{corollary}
\begin{proof}
Bound the number of faces, i.e., each face is bounded by at least 3 edges, and each edge is on the boundary of at most two faces. Therefore, we have $3f\le 2m$, i.e., $3(m-n+2)\le 2m$.

If there is no $3$-cycles, then each face is bounded by at least $4$ edges, and therefore $4f\le 2m$, i.e., $4(m-n+2)\le 2m$
\end{proof}

\begin{corollary}
Each simple planar graph contaisn a vertex of degree at most $5$.
\end{corollary}
\begin{proof}
w.l.o.g., consider the connected graph with at least $3$ vertices.
If each vertex has degree at least $6$, then $6n\le 2m$, i.e., $3n\le m\le 3n-6$, which is a contradiction.
\end{proof}


\section{Non-planar Graphs}
\begin{theorem}
The graph $K_{3,3}$ is non-planar.
\end{theorem}
\begin{theorem}
The graph $K_5$ is non-planar.
\end{theorem}
\begin{proof}
We can show the non-planarity of $K_5,K_{3,3}$ by Euler’s Polyhedral Formula:
\begin{itemize}
\item
For $K_5$, $n=5,m=10$, i.e., $3n-6=9<m$, contradiction
\item
For $K_{3,3}$, $n=6,m=9$, without $3$ cycles, i.e., $2n-4=8<m$, contradiction.
\end{itemize}
\end{proof}
\begin{remark}
Every subgraph of a planar graph is planar;
Every graph with a non-planar subgraph must be non-planar;
Every graph that ‘contains’ either $K_5$ or $K_{3,3}$ must be non-planar.
\end{remark}

\begin{definition}[homeomorphism]
When we replace an edge with a $2$-edge path with a new vertex, we say that the edge has been sub-divided.
We say two graphs are \emph{homeomorphic} if they can be expanded from the same graph by edge sub-division.
\end{definition}

\begin{theorem}
A graph is planar if and only if it contains no sub-graph homeomorphic to $K_5$ or $K_{3,3}$
\end{theorem}

Consider a graph $H$ obtained from a graph $G$ by \emph{contracting} an edge.
\begin{lemma}
If the graph $G$ is planar, then so is any graph $H$ obtained from $G$ by \emph{contracting} an edge.
\end{lemma}
\begin{theorem}
If a graph $G$ is contractible to either $K_5$ or $K_{3,3}$, then $G$ cannot be planar.
\end{theorem}
\section{Geometric Dual of a Planar Graph}
\begin{definition}[Geometric Dual]
Consider a plane drawing of a planar graph $G$, its geometric dual $G^*$ is contructed as follows:
\begin{enumerate}
\item
There is a vertex $v^*$ of $G^*$ for each face of $G$
\item
If edge $e$ in $G$ separates two faces of $G$, then there is an edge $e^*$ in $G^*$ incident to the two vertices in $G^*$ corresponding to two faces
\item
If edge $e$ is incident to only one face, then there is a corresponding self-loop in $G^*$.
\end{enumerate}
\end{definition}
 
\begin{lemma}
Let $G$ be a connected planar graph, then $G^*$ is also a connected planar graph.
Let $G$ be a planar graph with $n$ vertices, $m$ edges, $f$ faces, and $G^*$ be the geometric dual with $n^*$ vertices, $m^*$ edges, and $f^*$ faces, then
\[
n^*=f,\quad
m^*=m,\quad
f^*=n
\]
\end{lemma}

\begin{theorem}
If G is a planar connected graph, then $G^{**}$ is isomorphic to G.
\end{theorem}
\begin{definition}
A planar graph may also have self-dual, i.e., isomorphic to its geometric dual.
\end{definition}
\begin{theorem}
Let $G$ be a planar graph and $G^*$ be a geometric dual of $G$.
Then a set of edges of $G$ form a cycle in $G$ if and only if the corresponding set of edges of $G^*$ form an \emph{edge cutset} of $G^*$;
\end{theorem}
\begin{proof}
Assume $G$ is connected.
If $C$ is a cycle of $G$, then $C$ encloses one or more finite faces of $G$, corresponding to a non-empty set of vertices $V^*$ in $G^*$. The edges in $G^*$ corresponding to the edges of $C$ form an edge cutset separating $V^*$ from the rest of the graph $G^*$.

If $K^*$ is a cutset of $G^*$ separting the vertex sets $V_1^*,V_2^*$ of $G^*$. 
w.l.o.g., $V_1^*$ does not contain the vertex corresponding to the infinite face. 
For any two adjacent faces in $G$ corresponding to $v_1^*\in V_1^*,v_2^*\in V_2^*$, 
the edge separating the two faces must be in the cutset $K^*$.
Therefore, the edges in $G$ corresponding to the edges in the cutset $K^*$ forms a cycle.
\end{proof}
\begin{corollary}
A set of edges in $G$ forms a cutset of $G$ iff the corresponding edges in $G^*$ form a cycle in $G^*$.
\end{corollary}
\begin{definition}
A graph $G^*$ is an abstract dual of a graph $G$ if there is a one-to-one correspondence between the edges of $G$ and $G^*$ such that:
a set of edges form a cycle in $G$ if and only if the corresponding edges form an edge cutset of $G^*$
\end{definition}
The concept of the abstract dual generalises that of a geometric dual. If $G$ is a plane graph, then its geometric dual is an abstract dual.
\begin{proposition}
$K_5,K_{3,3}$ has no abstract dual.
\end{proposition}
\begin{theorem}
A graph is planar if and only if it has an abstract dual
\end{theorem}

\section{Crossing Number}
\begin{definition}[Crossing Number]
The crossing number $\text{cr}(G)$ of a graph $G$ is the minimum number of edge crossing that can occur when the graph is drawn in the plane.
\end{definition}

\begin{lemma}
For a graph $G$ with $n\ge3$ vertices and $m$ edges, $\text{cr}(G)\ge m-3n+6$
\end{lemma}
\begin{proof}
For planar graph $m\le 3n-6$. Suppose $G$ is non-planar with $\text{cr}(G)=c$.
For a plane drawing of $G$, ``replace'' each crossing with a new vertex (and 2 new edges). The resulting graph is planar and $(m+2c)\le 3(n+c)-6$, i.e., $c\ge m-3n+6$.
\end{proof}
\begin{lemma}
For a complete bipartite graph $K_{p,q}$,
\[
\text{cr}(K_{p,q})\le\lceil p/2
\rceil
\lceil (p-1)/2
\rceil
\lceil q/2
\rceil
\lceil (q-1)/2
\rceil
\]
\end{lemma}
\begin{remark}
This bound is shown to be tight when $p\le q$ and 
\begin{itemize}
\item
either $p\le 6$
\item
or $p=7,q\le10$
\end{itemize}
\end{remark}


\begin{definition}[Thickness]
The thickness of a graph $G$ is the fewest number of planar subgraphs whose union yields $G$.
\end{definition}



















