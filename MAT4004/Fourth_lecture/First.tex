%\chapter{Introduction to Linear Programming}
\chapter{Trees}
\section{Characterisation for Trees}
\begin{definition}[Tree]
An undirected graph is \emph{acyclic} if it has no cycles.
A connected acyclic undirected graph is a \emph{tree}.
\begin{itemize}
\item
A vertex of degree 1 in a tree is called a \emph{leaf}
\item
An acyclic (undirected) graph but \emph{not connected} is called a \emph{forest	}
\end{itemize}
\end{definition}
\begin{remark}
Trees are simple graphs. They are the simplest non-trivial graphs, with some nice properties.
\end{remark}
\begin{theorem}\label{The:4:1}
Let $T$ be an undirected graph with $n$ vertices. TFAE:
\begin{enumerate}
\item
$T$ is a tree
\item
$T$ contains no cycles and has $n-1$ edges
\item
$T$ is connected and has $n-1$ edges
\item
$T$ is connected and each edge of $T$ is a bridge (i.e., a cutset of a \emph{single} edge)
\item
Any two vertices of $T$ is connected by a unique path
\item
$T$ is acylic, but the addition of any new edge creates \emph{exactly} one cycle
\end{enumerate}
\end{theorem}
\begin{proof}
(1) implies (2):
The proof is by induction on $n$.
The removal of any edge of $T$ disconnects the graph into two components, each of which is a tree.
Therefore, applying the induction hyphothesis on each components gives the desired result.
%\[
%\diff\omega_0 = \diff\frac{\omega}{k} = \frac{1}{k}\diff\omega\implies
%\tilde x(t) = \frac{1}{2\pi k}\int_{-\infty}^\infty x(j\omega)e^{j\omega t}\diff\omega
%\]
(2) implies (3):
Suppose $T$ is disconnected. Since $T$ is acyclic, then each component is a tree.
Therefore, the number of edges in each component tree is one fewer than the number of vertices, which implies the  graph $T$ has fewer than $n-1$ edges, which is a contradiction.

(3) implies (4):
Suppose the removal of edge $e$ of $T$ does not disconnect the graph.
Since $T-e$ is a tree, and the number of edges is $n-1$, we derive a contradiction.

(4) implies (5):
Since $T$ is connected, there is at least one pathbetween any pair of vertices.
If a pair of vertices is connected by two distinct paths, then they must contain a cycle.
The removal of an edge on this cycle does not disconnect the graph. Thus the edge is not a bridge.

(5) implies (6):
$T$ does not contain a cycle.
Consider an edge $e$ that is not in $V(T)$, and there is a path in $T$ connecting the two end-vertices of $e$. Therefore, the addition of $e$ creates a cycle.
If there are two cycles in $T+e$, then both cycles must contain $e$, which implies there exists a cycle in $T+e$ not containing $e$, which is a contradiction.

(6) implies (1):
Suppose that $T$ is disconnected, then it's possible to add an edge to $T$ that joins two different components but not create a cycle.
\end{proof}

\section{Spanning Trees}
Consider a simple connected graph $G$ on $n$ vertices. We are interested in the subgraphs of $G$ that are trees.
\begin{definition}[Spanning Tree]
A subgraph $T$ of $G=(V,E)$ that is a tree and contains all the vertices $V$ of $G$ is a \emph{spanning tree} of $G$.
\end{definition}


\begin{theorem}
If $T=(V,F)$ is a spanning tree of a connected graph $G=(V,E)$, then
\begin{enumerate}
\item
Each cutset of $G$ contains an edge in $T$
\item
Each cycle of $G$ contains an edge in $\bar{T}:=(V,E-F)$.
\end{enumerate}
\end{theorem}
\begin{proof}
\begin{enumerate}
\item
Suppose $K$ is a cutset of $G$ that disconnects $G$ into two components $G_1$ and $G_2$.
Since $T$ is a spanning tree, it must contain an edge joining one vertex in $G_1$ and one vertex in $G_2$
\item
If $C$ is a cycle in $G$ with no edge in common with $E-F$, then all the edges in $C$ are in $F$m which contradicts that $T$ is acyclic.
\end{enumerate}
\end{proof}

\begin{theorem}
If $T$ and $T'$ are spanning trees of a connected graph $G$ and edge $e\in E(T)-E(T')$,
then there is an edge $e'\in E(T')-E(T)$ such that $T-e+E'$ is a spanning tree of $G$.
\end{theorem}
\begin{proof}
By part (4) in Theorem~(\ref{The:4:1}), each edge of $T$ is a cutset of $T$.
Let $k$ and $k'$ be two components of $T-e$.
Since $T$ is a spanning tree of $G$, it is connected and there exists an edge $e'\in T'$ that joins a vertex in $K$ to a vertex in $K'$.
Therefore, the constructed $T-e+e'$ is connected and has $n-1$ edges, therefore a spanning tree of $G$.
\end{proof}

\section{Prufer Code}
\paragraph{Motivation}
How many different (i.e. non-isomorphic) graphs are there with a given property?
In particular, how many regular graph~(i.e., each vertex has the same degree) of degree 2?
How many regular graph of degree $n$? Unfortunately, it is almost impossible to express these results by simple closed-form formulas.

One application is in the enumeration of chemical molecules, i.e., how to count the saturated hydrocarbons with the formula $C_nH_{2n+2}$?
Note that it is a tree since it is connected with $n+(2n+2)$ vertices and $(4n+(2n+2))/2=3n+1$ edges.

Therefore, the problem turns into the enumeration of labelled trees with a given number of vertices.

In particular, we assume the graphs with label $1-2-3-4$ and $4-3-2-1$ are the same since they have the same adjacent matrix, but they are both different to $1-2-4-3$.

\begin{theorem}
There are $2^{n(n-1)/2}$ distinct labelled simple graphs with $n$ vertices;
and $n^{n-2}$ distinct labelled trees with $n$ vertices.
\end{theorem}

\subsection{Prufer's Code}
Now we aim to represent each tree with $n$ vertices as a sequence of $n-2$ numbers selected from $\{1,2,3,\dots,n\}$

\begin{algorithm}[htb] 
\caption{Prufer Coding for a tree $T$} 
\label{alg:SM} 
\begin{algorithmic}[1] %show number in each rows
\REQUIRE ~~\\ %算法的输入参数:Input
A tree $T$ with $n$ labelled vertices
\ENSURE ~~\\ %算法的输出:Output
The sequence $(p_1,\dots,p_{n-2})\subseteq\{1,2,3,\dots,n\}$
\STATE 
Initialize $T_1=T$ and $i=1$;
\STATE
\textbf{Stop} if $T_i$ has only one edge;
\STATE
Otherwise, let $v_i$ be the lowest labelled \emph{leaf} of $T_i$, and $p_i$ be its neighbour.
Let $T_{i+1}=T_i-v_i$;
\STATE
Repeat previous step;
\end{algorithmic}
\end{algorithm}

Given a sequence of $n-2$ numbers selected from $\{1,\dots,n\}$, we can decode it into the corresponding labelled tree. Consider the nontrivial case $n\ge3$:

\begin{algorithm}[htb] 
\caption{Constructing a Tree from a (Prufer) Sequence} 
\label{alg:SM} 
\begin{algorithmic}[1] %show number in each rows
\REQUIRE ~~\\ %算法的输入参数:Input
The sequence $(p_1,\dots,p_{n-2})\subseteq\{1,2,3,\dots,n\}$
A tree $T$ with $n$ labelled vertices
\ENSURE ~~\\ %算法的输出:Output
A tree $T$ with $n$ labelled vertices
\STATE 
Initialize $V_0=\{1,\dots,n\}$ and $i=1$;
\STATE
Let $P_i = (p_i,p_{i+1},\dots,p_{n-2})$, and $v_i$ be the lowest labelled vertex that do not appear in $P_i$ Join it into vertex $p_i$, and let $V_i = V_{i-1}\setminus\{v_i\}$;
\STATE
If $i=n-2$, then join the two remaining vertices in $V_i$. \textbf{STOP};
\STATE
Otherwise, increment $i$ by $1$ and repeat from step $2$;
\end{algorithmic}
\end{algorithm}

\section{Minimum Spanning Tree}
Given a network with costs labelled for each edges, we are interested in finding a spanning tree with minimum cost. The computation cost is high for finding each of the $n^{n-2}$ trees.

\paragraph{Kruskal’s Algorithm}
Let $G=(V,E)$ be a connected graph with $n$ vertices with weights on the edges. Then the following algorithm gives a spanning tree $H=(V,F)$ of $G$ with a minimum edge weight:

\begin{algorithm}[htb] 
\caption{Kruskal’s Algorithm} 
\label{alg:SM} 
\begin{algorithmic}[1] %show number in each rows
\STATE 
Initialize $E(H)=\emptyset$ and let the edges in $E$ be sorted in non-decreasing order;
\STATE
Let $e$ be an edge in $E$ of smallest weight. Add this edge to $H$ as long as it does not create a cycle in $H$;
\STATE
Repeat above step until obtaining $n-1$ edges in $H$.
\end{algorithmic}
\end{algorithm}


\begin{theorem}[Optimality for finding MST]
Let $T$ be a spanning tree of $G$.
Let $e=\{v_1,v_2\}$ be an edge of $T$.
The removal of $e$ disconnects $T$ into two components, say $T_1$ and $T_2$ with vertex sets $V_1$ and $V_2$ respectively.
Let $K(G)$ be an edge cutset of $G$ that disconnects $G$ into two components with vertex sets $V_1$ and $V_2$.

Then $T$ is a MST of $G$ if and only if
\[
\forall e\in\{v_1,v_2\}\in E(T),
w(e)\le w(f),\quad\forall f\in K(G)
\]
\end{theorem}

\paragraph{Jarnik-Prim's Algorithm}
We denote $[W,V\setminus W]$ as the edge cutset that disconnects $G$ into two components with vertex-sets $W$ and $V\setminus W$.
In other words, $[W,V\setminus W]$ is the set of edges in $G$ that joins a vertex in $W\subseteq V$ to a vertex not in $W$.
\begin{algorithm}[htb] 
\caption{Jarnik-Prim's Algorithm} 
\label{alg:SM} 
\begin{algorithmic}[1] %show number in each rows
\STATE 
Initialize $V_1=\{v_1\}$ (any vertex), $T_1=(V_1,E_1=\emptyset)$, $k=1$, $d(v_1)=0,d(v)=\infty,\forall v\ne v_1$, $n(v)=v_1,\forall v\ne v_1$;
\STATE
\textbf{Stop} if $V_k=V$;
\STATE
Otherwise for $v\ne v_k$, let $d(v)=\min\{d(v),w(\{v,v_k\})\}$, and $n(v)=v_k$ if $d(v)=w(\{v,v_k\})$
\STATE
Let $v_{k+1} =\arg\min_{v\ne V_k}d(v)$ (essentially looking for the edge of least weight in cutset $[V_k,V\setminus V_k]$), $V_{k+1}=V_k\cup\{v_{k+1}\},T_{k+1}=(V_{k+1},E_k\cup\{(v_{k+1},n(v_{k+1}))\})$.

Increment $k$ by 1 and go to step 2
\end{algorithmic}
\end{algorithm}
















