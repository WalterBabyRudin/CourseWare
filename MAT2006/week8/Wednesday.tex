
\chapter{Week7}

\section{Wednesday}\index{week7_Tuesday_lecture}
\paragraph{Announcement}
Our mid-term is on next Wednesday in Liwen Building, from 8:00am to 10:00am. We will cover everything until this Friday.
\subsection{Integrable Analysis}
\paragraph{Recap}
Given a sequence of functions $\{f_n\}$ with pointwise limit $f$, we are curious about whether the equation holds:
\[
\lim_{n\to\infty}\int_a^bf_n(x)\diff x
=
\int_a^b\left[\lim_{n\to\infty}f_n(x)\right]\diff x
\]
Let's give a counter-example to show this equaiton may not necessarily true.
\begin{example}
Let $\{f_n\}$ defined on $[0,1]$ with
\[
f_n(x)=\left\{
\begin{aligned}
n,&\quad\mbox{if $x\in(0,\frac{1}{n})$}\\
0,&\quad\mbox{otherwise}
\end{aligned}
\right.
\]
We find that $\int_0^1f_n\diff x=1$, and $f_n\to0$ as $n\to\infty$. Thus
\[
\int_0^1\left[\lim_{n\to\infty}f_n(x)\right]\diff x=0\ne
\lim_{n\to\infty}\int_0^1f_n(x)\diff x
\]
\end{example}
There is a \emph{sufficient} condition that guarantees the equation holds:
\begin{theorem}
Let $\{f_n\}$ be a sequence of Riemann integrable functions on $[a,b].$ If $f_n$ converges to $f$ uniformly as $n\to\infty$, then $f$ is also \emph{Riemann integrable}, and 
\[
\lim_{n\to\infty}\int_a^bf_n(x)\diff x
=
\int_a^bf(x)\diff x
\]
\end{theorem}
\begin{definition}
We say that $f_n$ converges to $f$ uniformly as $n\to\infty$ on $[a,b]$ if for every $\varepsilon>0$, there exists $N$ such that $|f_n(x) - f(x)|<\varepsilon$ for all $x\in[a,b]$ and for all $n\ge N$.
\end{definition}
\begin{proof}
\begin{itemize}
\item
\textbf{Step 1: }First we need to show that both $\int_a^bf_n(x)\diff x$ and $\int_a^bf(x)\diff x$ is well-defined, i.e., $f$ and $f_n$ is \emph{uniformly bounded}, i.e., there exists $M,M'>0$ such that $|f(x)|\le M$ and $|f_n(x)|\le M'$,$\forall n$. First show that $\{f_n\}$ is uniformly bounded:
\begin{subequations}
\begin{align}
|f_n(x) - f_k(x)|&=|f_n(x) - f(x)+f(x) - f_k(x)|\label{Eq:7:1:a}\\
&\le |f_n(x) - f(x)|+|f(x) - f_k(x)|
\end{align}
Due to the uniform convergence of $\{f_n\}$, we choose $\varepsilon:=1$, then there exists $N>0$ s.t. 
\begin{equation}\label{Eq:7:1:c}
|f_m(x)-f(x)|<1,\qquad \forall m\ge N.
\end{equation}
Therefore, we give a bound on (\ref{Eq:7:1:a}):
\begin{equation}
|f_n(x) - f_k(x)|<2,\quad
\forall  n,k\ge N
\end{equation}
In particular, take $k=N$, thus
\begin{equation}
|f_n(x)-f_N(x)|<2\implies |f_n(x)|<|f_N(x)|+2,\qquad
\forall n\ge N,
\end{equation}
\end{subequations}
i.e., every $f_n$ for $n\ge N$ is bounded from $|f_N(x)|$ as $2$. Therefore, we have $\{f_n\}_{n=1}^\infty$ is uniformly bounded by $M$. (just set $M=\max\{|f_1(x)|,\dots,|f_{N-1}(x)|,|f_N|+2\}$.) 

Another application of (\ref{Eq:7:1:c}) gives the uniform boundness of $f$:
\[
|f(x)|\le|f(x) - f_N(x)|+|f_N(x)|\le1+|f_N(x)|.
\]
\item
\textbf{Step 2: }Argue the Riemann integrability of $f$. Define $\varepsilon_n=\sup_{a\le x\le b}|f_n(x) - f(x)|$, and $\varepsilon_n\to0$ as $n\to\infty$. 
\begin{subequations}
Therefore, we give bounds on $f$:
\begin{equation}
-\varepsilon_n+f_n(x)\le f(x)\le \varepsilon_n+f_n(x)
\end{equation}
So that the lower and upper integrals of $f$ satisfy:
\begin{equation}
\loRiemannint{a}{b}[-\varepsilon_n+f_n(x)]\diff x
\le
\loRiemannint{a}{b}f(x)\diff x\le \upRiemannint{a}{b}f(x)\diff x\le
\upRiemannint{a}{b}[\varepsilon_n+f_n(x)]\diff x
\end{equation}
Note that $f_n$ is integrable, so we can remove the upper and lower integral symbols of $f_n\pm\varepsilon_n$:
\begin{equation}\label{Eq:7:2:c}
\int_a^bf_n(x)-\varepsilon_n\diff x
\le
\loRiemannint{a}{b}f(x)\diff x\le \upRiemannint{a}{b}f(x)\diff x\le
\int_a^bf_n(x)-\varepsilon_n\diff x
\end{equation}
Hence we give a bound on the difference of upper and lower integrals of $f$:
\begin{equation}
0\le  \upRiemannint{a}{b}f(x)\diff x-\loRiemannint{a}{b}f(x)\diff x\le2(b-a)\varepsilon_n,
\end{equation}
\end{subequations}
Since $\varepsilon_n\to0$ as $n\to\infty$, the upper and lower integrals of $f$ are equal. Thus $f\in\mathcal{R}[a,b]$.
\item
Another application of (\ref{Eq:7:2:c}) now yields
\begin{equation}
\left|\int_a^bf(x)-f_n(x)\diff x\right|\le\int_a^b\varepsilon_n\diff x=\varepsilon_n(b-a),
\end{equation}
which implies $\lim_{n\to\infty}\int_a^bf_n(x)\diff x
=
\int_a^bf(x)\diff x$.
\end{itemize}
\end{proof}
\begin{remark}
The sequence of functions also remains the question that:
\begin{quotation}
Would the equation (\ref{Eq:7:2}) holds?
\begin{equation}\label{Eq:7:2}
\lim_{n\to\infty}f_n'(x)=\left[\lim_{n\to\infty}f_n(x)\right]'
\end{equation}
\end{quotation}
Equation (\ref{Eq:7:2}) holds also depends on the uniform convergence of $\{f_n\}$.
\end{remark}

\subsection{Elementary Calculus Analysis}
\begin{theorem}[Fundamental Theorem of Calculus]
If $f:[a,b]\mapsto \mathbb{R}$ is continuous, then the function $F(x)=\int_a^xf(t)\diff t$ is \emph{differentiable} with $F'=f$.
\end{theorem}
\begin{proof}
The proof is simply by definition, keep in mind that difference quotient is useful in proofs related to differentiation.
\begin{subequations}
\begin{align}
\frac{F(x+h) - F(x)}{h}-f(x)&=
\frac{1}{h}\left[\int_a^{x+h}f(t)\diff t-\int_a^xf(t)\diff t\right]-f(x)\\
&=\frac{1}{h}\int_x^{x+h}f(t)\diff t-f(x)\\
&=\frac{1}{h}\int_x^{x+h}f(t)\diff t-\frac{1}{h}\left[\int_x^{x+h}1\diff t\right]f(x)\\
&=\frac{1}{h}\int_x^{x+h}f(t)\diff t-\frac{1}{h}\int_x^{x+h}f(x)\diff t\\
&=\frac{1}{h}\int_x^{x+h}[f(t) - f(x)]\diff t,
\end{align}
which implies that
\[
\left|\frac{F(x+h) - F(x)}{h}-f(x)\right|\le\frac{1}{h}\int_x^{x+h}|f(t) - f(x)|\diff t,
\]
\end{subequations}
Then apply continuity condition to give a bound on $|f(t)-f(x)|$:

Since $f$ is continuous at $x$, for $\varepsilon>0$, there exists $\delta>0$ such that $|f(y) - f(x)|<\varepsilon$ if $|y-x|<\delta$. Therefore,
\[
\left|\frac{F(x+h) - F(x)}{h}-f(x)\right|\le\frac{1}{h}\int_x^{x+h}|f(t) - f(x)|\diff t\le\frac{1}{h}\int_x^{x+h}\varepsilon\diff t=\varepsilon,
\]
If $h<\delta$, we imply
\[
\lim_{h\to0}\frac{F(x+h) - F(x)}{h}=f(x)
\]
\end{proof}
The integraiton by parts is an important part from Calculus, the core idea is from the product rule for differentiation.
\begin{theorem}[Integration by Parts]
Given two functions $f,g\in\mathcal{C}^1[a,b]$, (similar to $(fg)'=f'g+fg'$), we have
\[
\int_a^b(fg)'\diff x=\int_a^bf'g\diff x+\int_a^bfg'\diff x,
\]
or equivalently,
\[
(fg)(a)-(fg)(b)=\int_a^bf'g\diff x+\int_a^bfg'\diff x,
\]
i.e.,
\[
\int_a^bfg'\diff x=(fg)|_a^b-\int_a^bf'g\diff x
\]
\end{theorem}

%\begin{theorem}[Change of variables]
%
%\end{theorem}
There are two versions of change of variables in Calculus. We will discuss the difference of these.
\begin{proposition}[Change of variables,version 1]\label{Pro:7:1}
Let $\phi:[\alpha,\beta]\mapsto[a,b]$ be a continuously differentiable function such that
\[
\begin{array}{ll}
\phi(\alpha)=a,
&
\phi(\beta)=b.
\end{array}
\]
Then for every continuous function $f:[a,b]\mapsto\mathbb{R}$, we have
\[
\int_a^bf(x)\diff x=\int_\alpha^\beta f(\phi(t))\phi'(t)\diff t
\]
\end{proposition}
\begin{proof}
Define $F(x)=\int_a^xf(t)\diff t$, which implies
\[
\frac{\diff F(x)}{\diff x}=f(x),
\qquad
\int_a^bf(x)\diff x=F(b).
\]
Observe that
\[
\frac{\diff F(\phi(t))}{\diff t}=\frac{\diff F(\phi(t))}{\phi(t)}\frac{\phi(t)}{\diff t}=f(\phi(t))\phi'(t)
\]
Or equivalently,
\[
\frac{\diff}{\diff t}(F\circ\phi)(t)=f(\phi(t))\phi'(t)
\]
Therefore,
\begin{align}
\int_\alpha^\beta(F\circ\phi)'(t)\diff t
&=
\int_\alpha^\beta f(\phi(t))\phi'(t)\diff t\\
&=
(F\circ\phi)(\beta) - (F\circ\phi)(\alpha)=F(\phi(\beta))-F(\phi(\alpha))\\
&=F(b)-F(a)=F(b)\\
&=\int_a^bf(x)\diff x
\end{align}
\end{proof}
\begin{proposition}[Change of variables,version 2]\label{Pro:7:2}
Let $\phi:[\alpha,\beta]\mapsto[a,b]$ be continuously differentiable and \emph{strictly monotone}. Then for any $f\in\mathcal{R}[a,b]$, we have
\begin{enumerate}
\item
$f(\phi(t))\phi'(t)\in\mathcal{R}[\alpha,\beta]$
\item
\[
\int_\alpha^\beta f(\phi(t))\phi'(t)
=
\int_{\phi(\alpha)}^{\phi(\beta)}f(x)\diff x
\]
\end{enumerate}
\end{proposition}
\begin{remark}
\begin{itemize}
\item
Comparing proposition(\ref{Pro:7:2}) to (\ref{Pro:7:1}), note that we \emph{relax} $f$ from being continuously differentiable to being Riemann integrable; but \emph{restrict} $\phi$ to be \emph{strictly monotone}.
\item
The proof for proposition(\ref{Pro:7:2}) is messy. For most time functions we have faced is \textit{not continuous}, but we can break into finite sub-intervals and apply proposition(\ref{Pro:7:1}). Thus the benifit for proposition(\ref{Pro:7:2}) is not such huge. In practice, proposition(\ref{Pro:7:1}) is enough.
\end{itemize}
\end{remark}
Last, let's discuss a initutive fact of Riemann sum, i.e., as the mesh goes to zero, Riemann sums always converges  to their corresponding integration
\begin{theorem}
Let $f\in\mathcal{R}[a,b]$. Then a Riemann sum $S(\mathcal{P},f)$ converges to $\int_a^bf(x)\diff x$ as the mesh $\lambda(\mathcal{P})\to0$, i.e.,
\[
\sum_{i=1}^nf(t_i)\Delta x_i\to\int_a^bf(x)\diff x,\qquad
\mbox{as }\max_{1\le i\le n}\Delta x_i\to0,
\]
where $t_i\in[x_{i-1},x_i]$, $i=1,\dots,n$.
\end{theorem}
We apply this theorem to evaluate some limits:
\begin{example}
\begin{enumerate}
\item
Evaluate the limit
\[
\lim_{n\to\infty}\left[\frac{1}{n+1}+\frac{1}{n+2}+\cdots+\frac{1}{2n}\right].
\]
\begin{align*}
x_n&=\frac{1}{n+1}+\frac{1}{n+2}+\cdots+\frac{1}{2n}\\
&=\frac{1}{n}\left[\frac{n}{n+1}+\frac{n}{n+2}+\cdots+\frac{n}{2n}\right]\\
&=\frac{1}{n}\left[
\frac{1}{1+1/n}+\frac{1}{1+2/n}+\cdots+\frac{1}{1+n/n}
\right]\\
&=\Delta x_i\left[
f(\frac{1}{n})+f(\frac{2}{n})+\cdots+f(\frac{n}{n})
\right]
\end{align*}
which is essentially the Riemann sum of function $f(x)=\frac{1}{1+x}$ over interval $[0,1]$. Therefore, as $n\to\infty$,
\[
x_n\to\int_0^1\frac{1}{1+x}\diff x
\]
\item
Evaluate the limit
\[
\lim_{n\to\infty}\frac{1^\alpha+\cdots+n^\alpha}{n^{\alpha}}
\]
\begin{align*}
x_n&=\frac{1}{n}\frac{1^\alpha+\cdots+n^\alpha}{n^{\alpha}}\\
&=\frac{1}{n}\left[(\frac{1}{n})^\alpha+(\frac{2}{n})^\alpha+\cdots+(\frac{n}{n})^\alpha\right]\\
&=\Delta x_i\left[
f(\frac{1}{n})+f(\frac{2}{n})+\cdots+f(\frac{n}{n})
\right]
\end{align*}
As $n\to\infty$,
\[
x_n\to\int_0^1x^\alpha\diff x=\left.\frac{1}{\alpha+1}x^{\alpha+1}\right|_0^1=\frac{1}{\alpha+1}
\]


\end{enumerate}
\end{example}
















