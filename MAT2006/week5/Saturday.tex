
\section{Saturday: Comments on Quiz 1}\index{week5_Thursday_lecture}
The performance in Quiz 1 is not satisfying in general. The quiz one will not be counted, but the next exam will. It is designed for a rigorous mathematical course. 

\subsection{First Question}
The first is from Schroder-Bernstein Theorem: 
\begin{quotation}
$A=(-1,1)$ and $B=[-1,1]$. We are given two one-to-one mapping $f(x)=x:A\mapsto B$ and $g(x)=\frac{x}{2}:B\mapsto A$. We are required to construct a one-to-one onto mapping from $A$ to $B$.
\end{quotation}
\begin{proof}[Solution.]
\begin{align*}
B\setminus D&=\{-1,1\}\\
g(B\setminus D)&=\{-\frac{1}{2},\frac{1}{2}\}\\
gfg(B\setminus D)&=\{-\frac{1}{4},\frac{1}{4}\}\\
\cdots\cdots&\cdots\cdots
\end{align*}
Hence $S=\{\dots,-\frac{1}{4},-\frac{1}{2},\frac{1}{2},\frac{1}{4},\dots\}$, and
\[
F(x)=\left\{
\begin{aligned}
f(x)=x,& x\in A\setminus S\\
g^{-1}(x)=2x,& x\in S
\end{aligned}
\right.
\]
\end{proof}


\subsection{Second Question}
The first question appears on assginement 1, while the second question appears in diagnostic quiz. \emph{Always keep in mind that the same question or similar question may appear again in future exam.}
\begin{enumerate}
\item
Compute limit $\lim_{x\to0+}x^x$: as $x\to 0+$,
\begin{align*}
x^x&=\exp(\ln(x^x))=\exp(x\ln x)\\
&=\exp(\frac{\ln x}{1/x})=\exp(\frac{(\ln x)'}{(1/x)'})=\exp(\frac{1/x}{-1/x^2})\\
&=\exp(-x)\to 1
\end{align*}



\end{enumerate}

\subsection{Third Question}
Also appears in assignment 1. 
\begin{quotation}
Suppose $a_n\ge0,\forall n$, and the series $\sum_{n=1}^\infty a_n$ converges. The new sequence $\{A_n\}$ is given by
\[
A_n=\sqrt{\sum_{k=n}^\infty a_k} - \sqrt{\sum_{k=n+1}^\infty a_k}.
\]
Show that $\sum_{n=1}^\infty A_n$ also convetges and $a_n=o(A_n)$ as $n\to\infty$.
\end{quotation}
\begin{proof}[Solution.]
The partial sum gives
\begin{align*}
&A_1+A_2+\cdots+A_l\\
&=
(\sqrt{\sum_{k=1}^\infty a_k} - \sqrt{\sum_{k=2}^\infty a_k})
+
(\sqrt{\sum_{k=2}^\infty a_k} - \sqrt{\sum_{k=3}^\infty a_k})
+\cdots+(\sqrt{\sum_{k=l}^\infty a_k} - \sqrt{\sum_{k=l+1}^\infty a_k})\\
&=\sqrt{\sum_{k=1}^\infty a_k} - \sqrt{\sum_{k=l+1}^\infty a_k}\to\sqrt{\sum_{k=1}^\infty a_k}\qquad \mbox{as }l\to\infty
\end{align*}
Also,
\begin{align*}
\frac{a_n}{A_n}&=\frac{a_n}{\sqrt{\sum_{k=n}^\infty a_k} -\sqrt{\sum_{k=n+1}^\infty a_k}}\\
&=\frac{a_n(\sqrt{\sum_{k=n}^\infty a_k} +\sqrt{\sum_{k=n+1}^\infty a_k})}{a_n}=\sqrt{\sum_{k=n}^\infty a_k} +\sqrt{\sum_{k=n+1}^\infty a_k}\\
&=o(1)
\end{align*}
\end{proof}



\subsection{Fourth Question}
This question is about Baire-Category Theorem: 
\begin{quotation}
The set of rational numbers $\mathbb{Q}$ is not a countably intersection of open sets.
\end{quotation}
\begin{proof}
Assume it is. Suppose
\[
\mathbb{Q}=\bigcap_{n=1}^\infty O_n,
\]
where $O_n$'s are open set containing $\mathbb{Q}$, and therefore dense. 

Consider the set of irrational numbers $\mathbb{R}\setminus \mathbb{Q}=\bigcup_{n=1}^\infty(\mathbb{R}\setminus O_n)$. $\mathbb{R}\setminus O_n$ is closed $\forall n$, containing no open set, since otherwise open set $\mathbb{R}\setminus O_n$ will contain rational numbers. Therefore $\mathbb{R}\setminus O_n$ is nowhere dense, and therefore $\mathbb{R}\setminus \mathbb{Q}$ is first Category, i.e.,
\[
\mathbb{R}=(\mathbb{R}\setminus\mathbb{Q})\bigcup\mathbb{Q}
\]
is of first Category, which is a contradiction.
\end{proof}

\subsection{Fifth Question}
The growth of the uniform continuous function.It is essentially the test on mathematical maturity. You are required to show that 
\begin{quotation}
if $f:\mathbb{R}\mapsto\mathbb{R}$ is \emph{uniformly continuous}, then $f$ can grow at $\infty$ at most linearly, i.e., $\frac{f(x)}{x^2}\to0$, $x\to\infty$, i.e., $|f(x)|\le Cx$ for large $x$.
\end{quotation}
\begin{proof}
Recall the definition for uniform continuous: $\forall\varepsilon>0,\exists\delta>0$ such that $|f(x) - f(y)|\le\varepsilon$ if $|x-y|<\delta$. Take $\varepsilon=1$. Look at the postive axis: for $(n-1)\delta<x\le n\delta$, we have
\[
|f(x) - f(0)|\le n\le\frac{x}{\delta}+1
\]
and therefore
\[
|f(x)|\le |f(x)-f(0)|+|f(0)|\le \frac{x}{\delta}+1+|f(0)|\implies\frac{f(x)}{x^2}\to0.
\]

\end{proof}







\subsection{Grading policy}
We will not follow the partial grading policy in this course. 

The next exam will be similar.

Furthermore, the \emph{final} exam will be more \emph{comprehensive}.





















