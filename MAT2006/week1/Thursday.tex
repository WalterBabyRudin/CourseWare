\section{Thursday}\index{Thursday_lecture}

\subsection{Row-Echelon Form}
\subsubsection{Gaussian Elimination does't always work}

Let's discuss an example to introduce the concept for row-echelon form.
\begin{example}
We apply Gaussian Elimination to try to transfrom a  Augmented matrix:
\begin{itemize}
\item
In step one we choose the first row as pivot row (the first nonzero entry is the pivot):
\[ \left(
\begin{array}{@{}ccccc|c@{}}
\rowcolor{blue!10}
\cellcolor{black!20}1 & 1 & 1 & 1 & 1 & 1 \\
-1 & -1 & 0 & 0 & 1 & -1 \\
-2 & -2 & 0 & 0 & 3 & 1 \\
0 & 0 & 1 & 1 & 3 & -1 \\
1 & 1 & 2 & 2 & 4 & 1
\end{array}
\right)
\xLongrightarrow[\text{Add $1\times$ row 1 to row 2; Add $2\times$ row 1 to row 3}]{\text{Add $(-1)\times$ row 1 to row 5}}
\]
\[ \left[
\begin{array}{@{}ccccc|c@{}}
1 & 1 & 1 & 1 & 1 & 1 \\
\rowcolor{blue!10}
0 & 0 & \cellcolor{black!20}1 & 1 & 2 & 0 \\
0 & 0 & 2 & 2 & 5 & 3 \\
0 & 0 & 1 & 1 & 3 & -1 \\
0 & 0 & 1 & 1 & 3 & 0 \\
\end{array}
\right]
\]
\item
Then we choose second row as pivot row to continue elimination:
\[ 
\xLongrightarrow[\text{Add $(-2)\times$ row 2 to row 3; Add $(-1)\times$ row 2 to row 4}]{\text{Add $(-1)\times$ row 2 to row 5}}\left[
\begin{array}{@{}ccccc|c@{}}
1 & 1 & 1 & 1 & 1 & 1 \\
0 & 0 & 1 & 1 & 2 & 0 \\
\rowcolor{blue!10}
0 & 0 & 0 & 0 & \cellcolor{black!20}1 & 3 \\
0 & 0 & 0 & 0 & 1 & -1 \\
0 & 0 & 0 & 0 & 1 & 0 \\
\end{array}
\right]
\]
\item
Next, we choose the third row as pivot row to continue elimination:
\begin{equation} 
\xLongrightarrow[\text{Add $(-1)\times$ row 3 to row 5}]{\text{Add $(-1)\times$ row 3 to row 1; Add $(-1)\times$ row 3 to row 4}}\left[
\begin{array}{@{}ccccc|c@{}}
1 & 1 & 1 & 1 & 0 & -2 \\
\rowcolor{blue!10}
0 & 0 & \cellcolor{black!20}1 & 1 & 2 & 0 \\
\rowcolor{blue!10}
0 & 0 & 0 & 0 & \cellcolor{black!20}1 & 3 \\
0 & 0 & 0 & 0 & 0 & -4 \\
0 & 0 & 0 & 0 & 0 & -3 \\
\end{array}
\right] \label{Row-echelon_matrix}
\end{equation}

Note that the matrix (\ref{Row-echelon_matrix}) is said to be the \emph{Row Echlon form}.
\item
Finally, we set second row as pivot row then set third row as pivot row to do elimination:
\begin{equation} 
\xLongrightarrow[\text{Add $2\times$ row 3 to row 1; Add $(-2)\times$ row 3 to row 2}]{\text{Add $(-1)\times$ row 2 to row 1}}\left[
\begin{array}{@{}ccccc|c@{}}
1 & 1 & 0 & 0 & 0 & 4 \\
0 & 0 & 1 & 1 & 0 & -6 \\
0 & 0 & 0 & 0 & 1 & 3 \\
0 & 0 & 0 & 0 & 0 & -4 \\
0 & 0 & 0 & 0 & 0 & -3 \\
\end{array}
\right] \label{Reduced_Row-echelon_matrix}
\end{equation}
\end{itemize}

The matrix (\ref{Reduced_Row-echelon_matrix}) is said to be the \emph{Reduced Row Echelon form}. Or equivalently, it is said to be the \textit{singular matrix}. (Don't worry, we will introduce these concepts in future.)

You may find there exist many solutions to this system of equation, which means Gaussian Elimination \emph{doesn't} always derive \emph{unique} solution.
\end{example}
\begin{definition}[Row Echelon Form]
A matrix is said to be in \emph{row echelon form} if
\begin{itemize}
\item
 \emph{(i)} The \textcolor{blue}{first nonzero entry} in each \textcolor{blue}{nonzero row} is $1$.
 \item
\emph{(ii)} If row $k$ does not consist entirely of zeros, the number of leading zero entries in row $k + 1$ is greater than the number of leading zero entries in row $k$.
\item
\emph{(iii)} If there are rows whose entries are all zero, they are below the rows having nonzero entries.
\end{itemize}
\end{definition}
\begin{definition}[Reduced Row Echelon Form] \qquad \\
A matrix is said to be in \emph{Reduced row echelon form} if
\begin{itemize}
\item
 \emph{(i)} The matrix is in \textcolor{blue}{\textit{row echelon form}}. 
 \item
\emph{(ii)} The \textcolor{blue}{first nonzero} entry in each row is the \textcolor{blue}{only} nonzero entry in its column.
\end{itemize}
\end{definition}
For example, the matrix $\begin{pmatrix}
1 & 0 \\ 0 & 1
\end{pmatrix}$ is also of \textit{Row Echelon Form}! Moreover, it is of \textit{Reduced Row Echelon Form}.
\subsection{Matrix Multiplication}
\subsubsection{Matrix Multiplied by Vector}
Here we introduce the definition for inner product of vector:
\begin{definition}[inner product]
Given two vectors $x = (x_1,x_2,\dots,x_n)$ and $y = (y_1,y_2,\dots,y_n)$, the inner product between $x$ and $y$ is given by 
\begin{equation*}
\inp{x}{y} = x_1y_1 + x_2y_2 + \dots + x_ny_n
\end{equation*}
The notation of inner product can also be written as $x\trans y$ or $x \cdot y$.
\end{definition}
\begin{remark}
Pro. Tom Luo \textcolor{blue}{highly recommends} you to write \textit{inner procuct} as $
\inp{x}{y}$. For myself, I also try to \textcolor{red}{avoid} using notation $x \cdot y$ to avoid misunderstanding.
\end{remark}
Let's study an example for matrix multiplied by a vector:
\begin{example}
For the system of equations $
\left \{	\begin{gathered}
2x_1 + x_2 +x_3=5 	\\
4x_1 - 6x_2 = -2 \\
-2x_2+7x_2+2x_3 = 9
\end{gathered}
\right.$, 
we define 
\[
\begin{array}{lll}
\bm{x} = \begin{pmatrix}
x_1 \\ x_2 \\ x_3
\end{pmatrix},&
\bm{A} = \begin{pmatrix}
2 & 1 & 1 \\ 4 & -6 & 0 \\ -2 & 7 & 2 
\end{pmatrix}
= \begin{pmatrix}
a_1\trans \\ a_2\trans \\ a_3\trans
\end{pmatrix},&
\bm{b} = \begin{pmatrix}
5 \\ -2 \\ 9
\end{pmatrix}.
\end{array}
\]
Here $\bm{x}$ and $a_1,a_2,a_3$ are all vectors. More specifically, 
\[
\begin{array}{lll}
a_1 = \begin{pmatrix}2 \\ 1 \\ 1\end{pmatrix},&
a_2 = \begin{pmatrix}4 \\ -6 \\ 0\end{pmatrix},&
a_3 = \begin{pmatrix}-2 \\ 7 \\ 2\end{pmatrix}.
\end{array}
\]
Then we multiply matrix $\bm{A}$ with vector $\bm{x}$:
\[
\bm{A}\bm{x} = 
\begin{pmatrix}
2x_1+x_2+x_3 \\ 4x_1-6x_2 \\ -2x_1+7x_2+2x_3
\end{pmatrix}
=\begin{pmatrix}
\inp{a_1}{\bm x}\\
\inp{a_2}{\bm x}\\
\inp{a_3}{\bm x}
\end{pmatrix}
=\begin{pmatrix}
b_1 \\ b_2 \\ b_3
\end{pmatrix}
\]
Hence we finally write the system equation as:
\[
\begin{array}{ll}
\bm{A}\bm{x} = \bm{b} &\mbox{Compact Matrix Form}
\end{array}
\]
Also, if we regard $\bm{x}$ as a scalar, we can also write:
\[
\bm{b} = \bm{A}\bm{x} = \begin{pmatrix}
a_1\trans \\ a_2\trans \\ a_3\trans
\end{pmatrix}\bm{x} = \begin{pmatrix}
a_1\trans\bm x \\ a_2\trans\bm x \\ a_3\trans\bm x
\end{pmatrix}
\]
\end{example}
\subsubsection{Matrix Multiply Matrix}
\begin{remark}
Note that an $m\times n$ matrix $\bm A$ can be written as $\begin{bmatrix}
a_{ij}
\end{bmatrix}$, where $a_{ij}$ denotes the entry of $i$th row, $j$th column of $\bm A $.\end{remark}

Notice that matrix $\bm A$ and $\bm B$ can do multiplication operator if and only if \emph{the \# for column of $\bm A$ equal to the \# for row of $\bm B$.} Moreover, for $m \times n$ matrix $\bm A$ and $n\times k$ matrix $\bm B$, we can do multiplication as follows:
\[
\bm A\bm B = \bm A \begin{pmatrix}
b_1 &b_2 & \dots & b_k
\end{pmatrix} = \begin{pmatrix}
\bm Ab_1 &\bm Ab_2 & \dots & \bm Ab_k
\end{pmatrix}
\]

The result is a $m\times k$ matrix. Thus for matrix multiplication, it suffices to calculate matrix multiplied by vectors.
\begin{example}
We want to calculate the result for $m \times n$ matrix $\bm A$ multiply $n\times k$ matrix $\bm B$, which is written as 
\[\bm A\bm B = \bm C = \begin{pmatrix}
\bm Ab_1 &\bm Ab_2 & \dots & \bm Ab_k
\end{pmatrix}\]
Hence the $i$th row, $j$th column of $\bm C $ is given by
\[
c_{ij} = \sum_{l=1}^{n}a_{il}b_{lj} = \inp{a_i}{b_j}
\]
You should understand this result, this means the $i$th row, $j$th column entry of $\bm C $ is given by the $i$th row of $\bm A $ multiplying the $j$th column of $\bm B $.
\end{example}
\begin{remark}
\textbf{Time Complexity Analysis}
\begin{itemize}
\item
To Calculate the single entry of $\bm C$, you need to do $n$ times multiplication.
\item There exists $n^2$ entries in $\bm C$
\item
Hence it takes $n \times n^2 \sim O(n^3)$ operations to compute $\bm C$. (Moreover, using more advanced algorithm, the time complexity could be reduced.
\end{itemize}
\end{remark}

\subsection{Special Matrices}
Here we introduce several special matrices:
\begin{definition}[Identity Matrix]
The $n \x n$ identity matrix is the matrix $\bm I = [m_{ij}]$, where 
\[
m_{ij} = \begin{dcases}
1, & \text{if } i =j; \\
0, & \text{if } i\ne j.
\end{dcases}
\]
\end{definition}
\begin{proposition}
Identity Matrix has the following properties:
\[
\begin{array}{ll}
\bm I \bm B = \bm B,&
\bm A \bm I = \bm A,
\end{array}
\]
where $\bm A$ and $\bm B$ coud be any size-suitable matrix.
\end{proposition}

\begin{definition}[Elementary Matrix of type III]
An elementary matrix $\bm{E}_{ij}$ of type III is a matrix such that 
\begin{itemize}
\item
its \textcolor{blue}{diagonal entries} are all $1$
\item
the $i$th row $j$ th column is a scalar
\item
the remaining entries are all \textcolor{blue}{zero}.
\end{itemize}
\end{definition}
For example, the matrix $\bm A = \begin{pmatrix}
2 & 1 & 1 \\ 4 & -6 & 0 \\ -2 & 7 & 2
\end{pmatrix}$ is elementary matrix of type III.t

\begin{remark}
If $\bm A$ is a matrix, then \textcolor{blue}{postmultiplying} with $\bm E_{ij}$ has the \textcolor{blue}{same} effect of performing row operation on $\bm A$. 

For example, given an elementary matrix of type III and a matrix $\bm A$:
\[
\begin{array}{ll}
\bm E_{21} = \begin{pmatrix}
1 & 0 & 0 \\ -2 & 1 & 0 \\ 0 & 0 & 1
\end{pmatrix}, &A = \begin{pmatrix}
2 & 1 & 1 \\ 4 & -6 & 0 \\ -2 & 7 & 2
\end{pmatrix}
\end{array}\]
Then the effect of $\bm E \bm A$ has the same effect of adding $(-2)\times$ row 1 to row 2:
\[\qquad \bm E_{21}A = \begin{pmatrix}
2 & 1 & 1 \\ 0 & -8 & -2 \\ -2 & 7 & 2
\end{pmatrix}\]
Moreover, if we define $\bm E = \begin{pmatrix}
1 & 0 & 0 \\ 0 & 1 & 0 \\ 1 & 0 & 1
\end{pmatrix}$, then continuing postmultiplying $\bm E_{31}$ is just like doing Gaussian Elimination:
\[
\bm E_{31}\bm E_{21} A = 
\begin{pmatrix}
2 & 1 & 1 \\ 0 & -8 & -2 \\ 0 & 8 & 3
\end{pmatrix}
\]
\end{remark}