
\section{Friday}\index{week8_Thursday_lecture}
\subsection{Analysis on Constraint Optimization}
\begin{theorem}[Lagrange Multiplier Theorem]
Let $f,g\in\mathcal{C}^1(U,\mathbb{R})$, where $U\subseteq\mathbb{R}^m$. Suppose $\bm x_0\in U$, and $g(\bm x_0)=\bm c_0$, $\mathcal{S}=g^{-1}(\bm c_0)$, and $\nabla g(\bm x_0)\ne\bm0$. If $f|_{\mathcal{S}}$ (the range of $f$ over the support $\mathcal{S}$) has a (local) maximum or (local) minimum at $\bm x_0$, then there exists $\lambda\in\mathbb{R}$ such that $\nabla f(\bm x_0)=\lambda\nabla g(\bm x_0)$.
\end{theorem}
\begin{proof}\qquad\\
\begin{enumerate}
\item[Step 1:]
Note that the tangent plane of the constraint set $\mathcal{S}$ at $\bm x_0$ is given by:
\begin{equation}\label{Eq:13:10}
T_{\bm x_0}(\mathcal{S}):=\{\bm v\mid \text{there exists a $\mathcal{C}^1$ path $h(t)$ on $\mathcal{S}$ such that $h(0)=\bm x_0$ and $h'(0)=\bm v$}\},
\end{equation}
i.e., the tangent plane of $\bm x_0$ is a set of vectors which are the \emph{velocities} for some smooth paths passing through $\bm x_0$. We claim that the definition (\ref{Eq:13:10}) is equivalent to say
\begin{equation}\label{Eq:13:11}
T_{\bm x_0}(\mathcal{S}):=\{\bm v\in\mathbb{R}^m\mid \inp{\nabla g(\bm x_0)}{\bm v}=0\}
\end{equation}
\begin{itemize}
\item
First show the forward diection. Consider any $\bm v$ satisfying (\ref{Eq:13:10}), i.e., there exists a $\mathcal{C}^1$ path $h(t)$ on $\mathcal{S}$ such that $h(0)=\bm x_0$ and $h'(0)=\bm v$. Define $\psi(t)=g(h(t))$, and then $\psi(t)=\bm c_0$ as $h(t)\in\mathcal{S}$. The derivative of $\psi$ is given by:
\[
\psi'(t)=0=\inp{\nabla g(h(t))}{h'(t)}\implies
\psi'(0)=0=\inp{\nabla g(h(0))}{h'(0)}=\inp{\nabla g(\bm x_0)}{\bm v}
\]
\item
For the reverse direction, we pick any $\bm v\in\mathbb{R}^m$ such that $\bm v\perp\nabla g(\bm x_0)$. We need to construct a $\mathcal{C}^1$ path $h(t)$ with $h(0)=\bm x_0$ and $h'(0)=\bm v$.  Since $\nabla g(\bm x_0)\ne0$, we assume w.l.o.g. $\frac{\partial g}{\partial x_m}(\bm x_0)\ne0$.

By applying IFT, there exists a neighborhood of $\bm x_0$, say $N(\bm x_0)$, and a $\mathcal{C}^1$ function $p(x_1,\dots,x_{m-1})$ such that in $N(\bm x_0)$ we can solve $g(\bm x_0)-\bm c_0=\bm0$ for $x_m$:
\begin{subequations}
\begin{align}
x_m&=p(x_1,\dots,x_{m-1})\\
x_{m,0}&=p(x_{1,0},\dots,x_{m-1,0})\\
\nabla p(x_1,\dots,x_{m-1})&=-\frac{1}{\frac{\partial g}{\partial x_m}}
\begin{pmatrix}
\frac{\partial g}{\partial x_1}
&
\cdots
&
\frac{\partial g}{\partial x_{m-1}}
\end{pmatrix}\trans\label{Eq:13:12:c}
\end{align}
\end{subequations}
Thus we construct function $h(t):\mathbb{R}\to\mathbb{R}^m$:
\[
h(t)=\begin{pmatrix}
\underbrace{x_{0,1}+v_1t}_{h_1(t)}
&
\cdots
&
\underbrace{x_{0,m-1}+v_{m-1}t}_{h_{m-1}(t)}
&
\underbrace{
p(h_1(t),\dots,h_{m-1}(t))
}_{h_{m}(t)}
\end{pmatrix}
\]
Therefore, it is clear that $h(0)=\bm x_0$, and
\begin{subequations}
\begin{align}
h'(0)&=\begin{pmatrix}
v_1&\cdots&v_{m-1}&\inp{\nabla p(h_1(t),\dots,h_{m-1}(t)}{(h_1',\dots,h_{m-1}')(t)}|_{t=0}\label{Eq:13:13:a}
\end{pmatrix}\\
&=
\begin{pmatrix}
v_1&\cdots&v_{m-1}&\inp{\nabla p(h_1(0),\dots,h_{m-1}(0))}{(h_1',\dots,h_{m-1}')(0)}
\end{pmatrix}\label{Eq:13:13:b}\\
&=\begin{pmatrix}
v_1&\cdots&v_{m-1}&\inp{\nabla p(x_{1,0},\dots,x_{m-1,0})}{(v_1,\dots,v_{m-1})}
\end{pmatrix}\label{Eq:13:13:c}
\end{align}
and
\begin{align}
\inp{\nabla p(x_{1,0},\dots,x_{m-1,0})}{(v_1,\dots,v_{m-1})}&=
-\frac{1}{\frac{\partial g}{\partial x_m}}
\begin{pmatrix}
\frac{\partial g}{\partial x_1}
&
\cdots
&
\frac{\partial g}{\partial x_{m-1}}
\end{pmatrix}(\bm x_0)\cdot(v_1,\dots,v_{m-1})\label{Eq:13:13:d}\\
&=
-\frac{1}{\frac{\partial g}{\partial x_m}}
\left[
v_1\frac{\partial g}{\partial x_1}+\cdots+v_{m-1}\frac{\partial g}{\partial x_{m-1}}
\right](\bm x_0)\label{Eq:13:13:e}\\
&=v_m\label{Eq:13:13:f}
\end{align}
\end{subequations}
Therefore we obtain $h'(0)=(v_1,\dots,v_{m-1},v_m)$.

Note that (\ref{Eq:13:13:a}) is by applying the chain rule; (\ref{Eq:13:13:c}) is by applying $h_(0)=\bm x_0$ and $h'_1(0)=v_1,\dots,h'_{m-1}(0)=v_{m-1}$; (\ref{Eq:13:13:d}) is by applying (\ref{Eq:13:12:c}); (\ref{Eq:13:13:f}) is by applying the condition $\bm v\perp\nabla g(\bm x_0)$ and then arranging terms.
\end{itemize}



\item[Step 2:]
We claim that $\nabla f(\bm x_0)\perp\bm v$ for $\forall\bm v\in T_{\bm x_0}(\mathcal{S})$

This is because for fixed $\bm v$, there exists a path $h(t)$ in $\mathcal{S}$, with $h(0)=\bm x_0$, and $h'(0)=\bm v$. Define $\phi(t)=f(h(t))$, which implies that
\[
\begin{aligned}
\phi'(t)&=\inp{\nabla f(h(t))}{h'(t)}
\end{aligned}
\]
If $\bm x_0$ is an extreme point of the function $f|_{\mathcal{S}}$, then the smooth function $\phi(t)=f(h(t))$ must have an extreme at $t=0$, which follows that 
\[
0=\phi'(0)=\inp{\nabla f(h(0))}{h'(0)}=\inp{\nabla f(\bm x_0)}{\bm v}
\]
\end{enumerate}
Therefore, it is clear that $\nabla f(\bm x_0)$ is co-linear with $\nabla g(\bm x_0)$, since otherwise we have $\nabla f(\bm x_0)=\mu_1\nabla g(\bm x_0)+\mu_2\nabla g(\bm x_0)^\perp$ and $\inp{\nabla f(\bm x_0)}{\bm v}=\mu_2\inp{\nabla g(\bm x_0)^\perp}{\bm v}\ne0$ 
\end{proof}
\begin{remark}
\begin{enumerate}
\item
The proof above is so called the \emph{elimination approach}. Here we view the constraints as a system of 1 equation with $m$ unknowns, and we express one of the variables in terms of the remaining $m-1$, and thereby essentially reducing this problem into an unconstrainted problem. This approach requires the use of implicit theorem.
\item
Another way of proof is to disgard the constraint and consider the unconstraint minimization over
\[
f(\bm x)+k\|g(\bm x)-\bm c_0\|^2+\frac{\alpha}{2}\|\bm x-\bm x_0\|
\] 
where $\bm x_0$ is supposed to be a local minimum satisfying $h(\bm x_0)=\bm c_0$, and $\alpha>0$. By \textit{writing the necessary condition for such unconstraint minimization problem} and \textit{taking the limit $k\to\infty$}, we obtain the desired result.
\end{enumerate}
We encourage the reader to read the book \textit{Nonlinear programming} from page $349$ to $355$, or Prof. Zhiquan Luo's note for Lecture 7 for details about these two approaches above.
\end{remark}

\subsection{Analysis on compactness}
Then we discuss the description on compactness over the continuous function space.
\paragraph{Notations}Let $\mathcal{C}(A,\mathbb{R})$ denote the set of all \emph{continuous} functions from a \emph{compact} set $A$ to $\mathbb{R}$. Define the associated metric $\bm d:\mathcal{C}(A,\mathbb{R})\times \mathcal{C}(A,\mathbb{R})\to\mathbb{R}$:
\[
\bm d(u,v)=\max_{x\in A}|u(x)-v(x)|,\quad
\forall u,v\in \mathcal{C}(A,\mathbb{R})
\]
It is easy to show that $(\mathcal{C}(A,\mathbb{R}),\bm d)$ defines a metric space and is complete. We discuss the \emph{equicontinuity} under this setting.

\begin{definition}[Equi-continuous]
Let $\mathcal{B}$ be a subset of $\mathcal{C}(A,\mathbb{R})$.
\begin{enumerate}
\item
We say $\mathcal{B}$ is \emph{uniformly bounded} if there exists $M_1>0$ such that
\[
|\phi(x)|\le M_1,\quad \forall x\in A,\forall \phi\in\mathcal{B}
\]
\item
We say $\mathcal{B}$ is a \emph{equi-continuous} family of functions, if for any $\varepsilon>0$, there exists $\delta:=\delta(\varepsilon)>0$, such that
\[
d(f(x),f(y))<\varepsilon,\quad
\mbox{provided that }x,y\in A,d(x,y)<\delta,\forall f\in\mathcal{B}
\]
\end{enumerate}
\end{definition}
\begin{theorem}[Arela-Ascoli Theorem]
Let $A\subseteq\mathbb{R}^m$ be a compact set, and $\mathcal{B}\subseteq\mathcal{C}(A,\mathbb{R})$. Suppose $\mathcal{B}$ is \emph{uniformly bounded} and \emph{equi-continuous}, then any sequence in $\mathcal{B}$ has a \emph{uniformly convergent} subsequence ($\mathcal{B}$ is compact as a result). 
\end{theorem}
%\begin{example}
%Suppose $\Omega\subseteq\mathbb{R}^n$ is a bounded open convex set. For fixed $M_1,M_2>0$, we derive that the set
%\[
%\mathcal{B}:=\{f\in\mathcal{C}(\bar{\Omega},\mathbb{R})\mid |f(x)|\le M_1,|\nabla f(x)|\le M_2,\forall x\in\Omega\}
%\]
%is a compact set.
%\end{example}

















