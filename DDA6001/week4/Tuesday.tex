
\chapter{Week 4}

\section{Monday}\index{Monday_lecture}


\subsection{Basic Convergence Results}

In order to show CLT, first study basic convergence results for $\mathbb{E}[X_n]\to\mathbb{E}[X]$.
The reason is that convergence of expectations has wide applications.
We may apply those convergence results for the sequence of $f(X_n)$.
For instance, if using the test function $f(x) = e^{i\theta x}$, then $\mathbb{E}[f(X_n)]$ is the characteristic function of the r.v. $X_n$.
Thus the proof of CLT is reduced to show that $\mathbb{E}[f(X_n)]\to\mathbb{E}[f(X)]$.


The expectation for a random variable $X:\Omega\to\mathbb{R}$ is defined as follows:
\begin{enumerate}
\item
When $X$ is simple, i.e., $X:\Omega\to\{x_i\}_{i=1}^{\ell}$, define
\[
\mathbb{E}[X] = \sum_{i=1}^{\ell}x_i\mathbb{P}(X=x_i)
\]
\item
When $X\ge0$, define an approximation of $X$:
\[
X_n = \left\{
\begin{aligned}
(k-1)2^{-n},&\quad\text{if $(k-1)2^{-n}\le X< k2^{-n}, k=1,2,\ldots,n2^n+1$}\\
n,&\quad\text{if $X\ge n$}
\end{aligned}
\right.
\]
Then $X_n$ is simple and $X_n\uparrow X$. Then define $\mathbb{E}[X]\triangleq \lim_{n\to\infty}\mathbb{E}[X_n]$.
\item
When $X$ is general, define $X = X^+-X^-$ and take $\mathbb{E}[X] = \mathbb{E}[X^+] - \mathbb{E}[X^-]$.
\end{enumerate}

\begin{proposition}\label{pro:4:1}
Suppose that $X\ge0$ and consider a sequence of simple non-negative r.v.'s $\{Y_n\}_{n\ge1}$.
If $Y_n\uparrow X$ as $n\to\infty$, then $\lim_{n\to\infty}\mathbb{E}[Y_n] = \mathbb{E}[X]$.
\end{proposition}
\begin{proof}
Define $X_k = \eta_k(X)$ and $Y_{n,k} = \eta_k(Y_n)$. Therefore,
\[
\lim_{n\to\infty}\mathbb{E}[Y_n]=\lim_{n\to\infty}\lim_{k\to\infty}\mathbb{E}[Y_{n,k}]
\le\lim_{k\to\infty}\mathbb{E}[X_k]=\mathbb{E}[X],
\]
where the inequality is because that $Y_n\le X$ implies $\eta_k(Y_n)\le \eta_k(X)$, i.e., $Y_{n,k}\le X_k$.

For the reverse direction, let $Z$ be a simple r.v. such that $0\le Z\le X$. Define $b = \max_{\omega\in\Omega}Z(\omega)$. Note that
\[
\mathbb{E}[Z] = \mathbb{E}[Z1(Y_n+\varepsilon>Z)] + \mathbb{E}[Z1(Y_n+\varepsilon\le Z)]
\]
Therefore, for any $\varepsilon>0$,
\[
\mathbb{E}[Z] \le \mathbb{E}[Y_n+\varepsilon] + \mathbb{E}[Z1(Y_n+\varepsilon\le X)]\le
\mathbb{E}[Y_n] + \varepsilon + b\mathbb{P}(\varepsilon\le X-Y_n).
\]
Taking $n\to\infty$, $\mathbb{E}[Z] \le \lim_{n\to\infty}\mathbb{E}[Y_n] +\varepsilon$. Thus $\mathbb{E}[Z] \le \lim_{n\to\infty}\mathbb{E}[Y_n]$. Take $Z= X_k$, we have
\[
\mathbb{E}[X] = \lim_{k\to\infty}\mathbb{E}[X_k]\le  \lim_{n\to\infty}\mathbb{E}[Y_n].
\]

\end{proof}
The linearity of expectation can be shown by applying this proposition.

\begin{proposition}[Expectations of almost surely identical r.v.'s]\label{pro:4:2}
If $X=Y$ a.s., and their expectations are finite, then $\mathbb{E}[X]=\mathbb{E}[Y]$.
\end{proposition}
\begin{proof}
It suffices to show that $\mathbb{P}(X=0)=1$ implies $\mathbb{E}[X]=0$.
Let $X=X^+-X^-$, and it suffices to show $\mathbb{E}[X^+]=0$ and $\mathbb{E}[X^-]=0$.
Construct $X_n^+ = \eta_n(X^+)$, which is a simple r.v. and $0\le X_n^+\uparrow X^+$.
As a result,
\[
\mathbb{E}[X^+] = \lim_{n\to\infty}\mathbb{E}[X_n^+] 
\]
Note that $\mathbb{P}(X_n^+>0) = \mathbb{P}(X^+\ge 2^{-n}) = \mathbb{P}(X\ge 2^{-n}) \le 1-\mathbb{P}(X=0)=0$, we imply $\mathbb{P}(X_n^+=0)=1,\forall n$, i.e., $\mathbb{E}[X^+]=0$.
\end{proof}


\begin{proposition}
Suppose that $X\ge0$ a.s. and consider a sequence of a.s. simple non-negative r.v.'s $\{Y_n\}_{n\ge1}$.
If $Y_n\uparrow X$ a.s. as $n\to\infty$, then $\lim_{n\to\infty}\mathbb{E}[Y_n] = \mathbb{E}[X]$.
\end{proposition}
\begin{proof}
Define
\[
\Omega_0 = \{X\ge0, \text{$Y_n$ is simple for all $n$, $Y_n\uparrow X$}\}.
\]
Since $\mathbb{P}(\cap_{\ell=1}^\infty A_\ell)=1$ when $\mathbb{P}(A_\ell)=1$, we imply $\mathbb{P}(\Omega_0)=1$.
Repeat the same proof as in Proposition~\ref{pro:4:1} and then apply Proposition~\ref{pro:4:2}.

\end{proof}

\begin{theorem}[Bounded Convergence Theorem]
Suppose that $X_n\xrightarrow{p}X$ as $n\to\infty$, and $|X_n|\le b$ a.s., then
\[
\lim_{n\to\infty}\mathbb{E}[X_n] = \mathbb{E}[X].
\]
\end{theorem}
\begin{proof}
Firstly argue that there exists a sub-sequence of $X_n$ that converges to $X$ a.s.
Construct a subsequence $X_{n_k}$ of $X_n$ such that $\mathbb{P}(|X_{n_k} - X|>\varepsilon)<2^{-k}$, which implies 
\[
\sum_{k\ge1}\mathbb{P}(|X_{n_k} - X|>\varepsilon)<1\implies
X_{n_k}\xrightarrow{a.s.}X.
\]
Therefore, $|X|\le b$ a.s. Then
\begin{align*}
|\mathbb{E}[X_n] - \mathbb{E}[X]|&\le \mathbb{E}[|X_n - X|]\\
&=\mathbb{E}[|X_n - X|1(|X_n - X|>\varepsilon)] + \mathbb{E}[|X_n - X|1(|X_n - X|\le\varepsilon)] 
\\&\le 2b\mathbb{P}(|X_{n_k} - X|>\varepsilon)+\varepsilon
\end{align*}
Take $n\to\infty$ both sides, and take $\varepsilon\to0$, the desired result holds.
\end{proof}

\begin{theorem}[Monotone Convergence Theorem]
Suppose that $X_n\ge0$ a.s., and $X_n\uparrow X$ a.s., then
\[
\lim_{n\to\infty}\mathbb{E}[X_n] = \mathbb{E}[X].
\]
\end{theorem}
\begin{proof}
The condition $X_n\le X$ implies $\mathbb{E}[X_n] \le \mathbb{E}[X]$, i.e.,
$\limsup_{n\to\infty}\mathbb{E}[X_n] \le \mathbb{E}[X]$.
It suffices to show that 
\[
\mathbb{E}[X]\le \liminf_{n\to\infty}\mathbb{E}[X_n]
\]
It suffices to construct simple r.v.'s $Y_k\ge0$ such that 
\[
Y_k\le X_k,\forall k\ge1, \quad\text{and $Y_k\uparrow X$ as $k\to\infty$.}
\] 
Therefore, $\mathbb{E}[Y_k]\to\mathbb{E}[X]\le \liminf_{k\to\infty}\mathbb{E}[X_k]$.

Construct $X_{n,k} = \eta_k(X_n)$ and $X_{n,k}\uparrow X_n$. Define $Y_k = \max(X_{1,k},X_{2,k},\ldots,X_{k,k})$, which implies $X_{n,k}\le Y_k\le X_k$.


\end{proof}

\begin{theorem}[Fatou's Lemma]
Suppose that $X_n\ge0$ a.s., then
\[
\mathbb{E}\left(
\liminf_{n\to\infty}X_n
\right)\le \liminf_{n\to\infty}\mathbb{E}[X_n]
\]
\end{theorem}
\begin{proof}
Take $Y_n=\inf_{\ell\ge n}X_{\ell}$, then $Y_n\uparrow Y\triangleq \liminf_{n\to\infty}X_n$.
Therefore,
\[
\lim_{n\to\infty}\mathbb{E}[Y_n] = \mathbb{E}[Y].
\]
Considering that $Y_n\le X_n$, we imply $\mathbb{E}[Y_n] \le \mathbb{E}[X_n]$.
Taking liminf both sides gives the desired result.
\end{proof}

\begin{theorem}[Dominated Convergence Theorem]
Suppose that $X_n\xrightarrow{a.s.}X$ as $n\to\infty$, and $|X_n|\le Y$ a.s. for some $Y$ with $\mathbb{E}[Y]<\infty$, then $\lim_{n\to\infty}\mathbb{E}[X_n] = \mathbb{E}[X].$
\end{theorem}
\begin{proof}
Apply Fatou's lemma on $Y+X_n$ and $Y-X_n$ gives the desired result.
\end{proof}
\clearpage
\section{Thursday}

\subsection{Characteristic Function}
Define the complex valued r.v. $Z$ as
\[
Z(\omega) = X(\omega) +iY(\omega),\quad\omega\in\Omega.
\]
When $X,Y$ have finite expectations, define the expectation of $Z$ as
\[
\mathbb{E}(Z)=\mathbb{E}(X)+i\mathbb{E}(Y)
\] 

\begin{definition}[Characteristic Function]
Given a distribution function $F$, define the characteristic function $\phi$ as
\begin{align*}
\phi(\theta)&=\int_{-\infty}^{+\infty}e^{i\theta x}F(\diff x)\\
&=\int_{-\infty}^{+\infty}\cos(\theta x)F(\diff x) + i\int_{-\infty}^{+\infty}\sin(\theta x)F(\diff x) 
\end{align*}
\end{definition}

For random variable $X$ subject to $F$, we have 
\[
\phi(\theta) = \mathbb{E}(\cos(\theta X))+i\mathbb{E}(\sin(\theta X)).
\]

\begin{proposition}
If complex valued random variable $Z$ has a finite expectation, then 
\[
|\mathbb{E}(Z)|\le \mathbb{E}(|Z|).
\]
\end{proposition}

\begin{proof}
The definition for real valued random variable has three steps.
Now we only check the first step for complex valued random variable.
Suppose that $Z$ takes finite many values $z_1,z_2,\ldots,z_m$, and $z_j = x_j + iy_j$, then
\begin{align*}
\mathbb{E}[Z]&=\sum_{j=1}^m\sum_{k=1}^m(x_j + iy_k)\mathbb{P}(X=x_j,Y=y_k).
\end{align*}
Apply the triangle inequality,
\begin{align*}
|\mathbb{E}(Z)|&\le \sum_{j=1}^m\sum_{k=1}^m|x_j + iy_k|\mathbb{P}(X=x_j,Y=y_k)\\
&=\sum_{\ell=1}^m\sum_{j=1}^m\sum_{k=1}^m|x_j + iy_k|\mathbb{P}(Z=z_\ell,X=x_j,Y=y_k)\\
&=\sum_{\ell=1}^m|z_\ell|\mathbb{P}(Z=z_\ell)=\mathbb{E}(|Z|).
\end{align*}

\end{proof}

\begin{corollary}
For a characteristic function $\phi$,
\[
|\phi(\theta)|\le 1,\quad\forall\theta\in\mathbb{R}.
\]
Therefore, $\phi(\theta)$ is well-defined and finite for any $\theta$.
\end{corollary}

\begin{proposition}
Suppose that $X,Y$ are independent, then 
\[
\phi_{X+Y}(\theta) = \phi_X(\theta)\phi_Y(\theta).
\]
\end{proposition}
\begin{proof}
It suffices to verify that $\mathbb{E}(e^{i\theta(X+Y)})=\mathbb{E}(e^{i\theta X})\mathbb{E}(e^{i\theta Y})$:
\begin{align*}
\mathbb{E}(e^{i\theta X})\mathbb{E}(e^{i\theta Y})&=(\mathbb{E}[\cos\theta X]+i\mathbb{E}[\sin\theta X])\mathbb{E}(e^{i\theta Y})
\end{align*}
Verify that 
\[
\mathbb{E}\left(
\cos\theta X\cdot(\cos\theta Y + i\sin\theta Y)
\right)
=
\mathbb{E}[\cos\theta X]\mathbb{E}(e^{i\theta Y})
\]
and
\[
\mathbb{E}\left(
\sin\theta X\cdot(\cos\theta Y + i\sin\theta Y)
\right)
=
\mathbb{E}[\sin\theta X]\mathbb{E}(e^{i\theta Y})
\]
\end{proof}

We may compute the characteristic function $\phi(\theta)$ for normal distribution.
Suppose that $f$ is the density of $\mathcal{N}(0,1)$, then
\[
\phi(\theta) = \int\cos(\theta x)f(x)\diff x + i \int\sin(\theta x)f(x)\diff x =\int\cos(\theta x)f(x)\diff x,
\]
where the last equality is because that $f$ is an even function.
We study the derivative of $\phi'(\theta)$:
\begin{align*}
\phi'(\theta)&=\int(-x\sin(\theta x))f(x)\diff x=\int\sin(\theta x)f'(x)\diff x\\
&=[\sin(\theta x)f(x)]_{-\infty}^\infty - \int (\sin(\theta x))'f(x)\diff x = -\theta\phi(\theta).
\end{align*}
Solving this differential equation with $\phi(0)=1$, we imply 
\[
\phi(\theta) = \exp\left(-\frac{1}{2}\theta^2\right).
\]

\subsection{Inversion Formula}

Based on the distribution function $F$, we can define the characterisic fucntion.
The inverse is also possible. 
Let $C_F$ be the set of all continuous points of $F$.

\begin{theorem}[Inversion Formula]
For $x,y\in C_F$ satisfying $x<y$,
\[
\lim_{a\downarrow 0}\frac{1}{2\pi}\int_{-\infty}^\infty\frac{1}{i\theta}(e^{-i\theta x} - e^{-i\theta y})
\phi(\theta)\exp\left(-\frac{a^2\theta^2}{2}\right)\diff\theta = F(y) - F(x).
\]
Hence, $F$ is determined by $\phi$.
\end{theorem}
The formula above is slightly different from the typical inversion formula:
\[
\lim_{n\to\infty}\frac{1}{2\pi}\int_{-n}^n\frac{1}{i\theta}(e^{-i\theta x} - e^{-i\theta y})\phi(\theta)\diff\theta = F(y) - F(x).
\]


\begin{theorem}[Parseval's Lemma]
For $a>0$ and $u,\theta\in\mathbb{R}$,
\begin{equation}\label{Eq:parseval}
\frac{1}{2\pi}\int_{-\infty}^\infty e^{-i\theta u}\phi(\theta)\exp\left(-\frac{a^2\theta^2}{2}\right)\diff\theta
=
\int_{-\infty}^\infty\frac{1}{\sqrt{2\pi}a}\exp\left(-\frac{(u-s)^2}{2a^2}\right)F(\diff s)
\end{equation}
\end{theorem}
\begin{proof}
Direct calculation on the LHS gives
\begin{align*}
\frac{1}{2\pi}\int_{-\infty}^\infty e^{-i\theta u}\phi(\theta)\exp\left(-\frac{a^2\theta^2}{2}\right)\diff\theta
&=
\frac{1}{2\pi}\int_{-\infty}^\infty e^{-i\theta u}\int_{-\infty}^\infty e^{i\theta s}F(\diff s)\exp\left(-\frac{a^2\theta^2}{2}\right)\diff\theta\\
&=\frac{1}{2\pi}\int_{-\infty}^\infty\int_{-\infty}^\infty e^{i(s-u)\theta}F(\diff s)\exp\left(-\frac{a^2\theta^2}{2}\right)\diff\theta\\
&=\frac{1}{\sqrt{2\pi}a}\int_{-\infty}^\infty\int_{-\infty}^\infty e^{i(s-u)\theta}\frac{a}{\sqrt{2\pi}}
\exp\left(-\frac{a^2\theta^2}{2}\right)\diff\theta F(\diff s)
\end{align*}
Since $\frac{a}{\sqrt{2\pi}}
\exp\left(-\frac{a^2\theta^2}{2}\right)$ is the pdf of $\mathcal{N}(0,1/a^2)$,
\[
\int_{-\infty}^\infty e^{i(s-u)\theta}\frac{a}{\sqrt{2\pi}}
\exp\left(-\frac{a^2\theta^2}{2}\right)\diff\theta = \exp\left(-\frac{1}{2a^2}(s-u)^2\right).
\]
As a result,
\[
\frac{1}{2\pi}\int_{-\infty}^\infty e^{-i\theta u}\phi(\theta)\exp\left(-\frac{a^2\theta^2}{2}\right)\diff\theta
=
\frac{1}{\sqrt{2\pi}a}\int_{-\infty}^\infty\exp\left(-\frac{1}{2a^2}(s-u)^2\right)F(\diff s).
\]

\end{proof}

\begin{remark}
Setting $g$ as the pdf of $\mathcal{N}(u,a^2)$, then the characteristic function of $g$ us $e^{iu\theta - \frac{1}{a^2}\theta^2}$. Therefore, the Parseval's formula can be written as
\[
\frac{1}{2\pi}\int_{-\infty}^\infty \phi(\theta)\overline{\psi(\theta)}\diff\theta
=
\int_{-\infty}^\infty g(s)F(\diff s).
\]
This formula holds for any continous pdf $g$ under an integrability condition on $|\phi(\theta)\psi(\theta)|$.
If $F$ has a density, this fact can be interpreted that the inner products are proportional for the Hilbert spaces of characteristic functions and that of probability density functions, i.e., two Hilbert spaces are isometric.
\end{remark}

Based on the Parseval's formula, we may begin to show the inversion formula:
\begin{proof}
Integrate both sides of (\ref{Eq:parseval}) within the interval $[x,y]$ by $\diff u$ .
The LHS becomes:
\[
\frac{1}{2\pi}\int_{-\infty}^\infty \frac{1}{i\theta}[e^{-i\theta x} - e^{-i\theta y}]\phi(\theta)\exp\left(-\frac{a^2\theta^2}{2}\right)\diff\theta.
\]
he RHS of (\ref{Eq:parseval}) becomes:
\begin{align*}
\int_{-\infty}^\infty\int_x^y\frac{1}{\sqrt{2\pi}a}\exp\left(-\frac{(u-s)^2}{2a^2}\right)&\diff uF(\diff s)
=
\int_{-\infty}^\infty\int_{(x-s)/a}^{(y-s)/a}\frac{1}{\sqrt{2\pi}}\exp\left(-\frac{v^2}{2}\right)\diff vF(\diff s)\\
&=
\int_{-\infty}^\infty\int_{-\infty}^\infty 1(x-av<s\le y-av)F(\diff s)\frac{1}{\sqrt{2\pi}}e^{-\frac{v^2}{2}}\diff v\\
&=
\int_{-\infty}^\infty[F(y-av) - F(x-av)]\frac{1}{\sqrt{2\pi}}e^{-\frac{v^2}{2}}\diff v
\end{align*}
Taking $a\to0$ both sides,
\[
\lim_{a\downarrow0}\frac{1}{2\pi}\int_{-\infty}^\infty\frac{1}{i\theta}(e^{-i\theta x} - e^{-i\theta y})
\phi(\theta)\exp\left(-\frac{a^2\theta^2}{2}\right)\diff\theta
=F(y)-F(x)
\]

\end{proof}






















