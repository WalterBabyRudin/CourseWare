% !TEX encoding = UTF-8 Unicode

\section{Thursday}\index{Thursday_lecture}

We have defined what is a stochastic process.
Now we study several special stochastic processes.
The first example is a random walk, a discrete time stochastic process:

\subsection{Random Walk}
Firstly, we need to define the independence among random variables:
\begin{definition}[Independence]
\begin{itemize}
\item
Events $A,B\in\mathcal{F}$ are independent if $\mathbb{P}(A\cap B) = \mathbb{P}(A)\mathbb{P}(B)$;
\item
Random variables $X,Y: \Omega\to S$ are independent if events $\{X\in B_1\}$ and $\{X\in B_2\}$ are independent for any $B_1,B_2\in\mathcal{B}(S)$;
\item
A sequence of random variables $\{X_n\}_{n\ge0}$ is independent if for 
each finite $n$, $\{X_i\in B_i\}$ for $i=1,2,\dots,$ are independent for any $B_i\in\mathcal{B}(S)$.
% a typo here.
\end{itemize}
\end{definition}
\begin{remark}
If random variables $X_1,X_2$ are independent and $\mathbb{E}[f_k(X_k)]$ are finite for $k=1,2$, then
\[
\mathbb{E}[f_1(X_1)f_2(X_2)] = \mathbb{E}[f_1(X_1)]\mathbb{E}[f_2(X_2)].
\]
\end{remark}
Now we start defining a random walk based on coin tossing.
Let $\{U_n\}_{n\ge1}$ be the unlimited coin tossing, i.e., 
$U_n$ is i.i.d. with $\mathbb{P}(U_n=1)=\mathbb{P}(U_n=-1)=1/2$.
Then define 
\[
X_0=0,~
X_n = \sum_{i=1}^nU_i.
\]
Here the stochastic process $\{X_n:~n\ge0\}$ is called a simple symmetric random walk with the initial state $0$.
Define the random variable 
\[
T(\omega) = \inf\{n\ge1:~~X_n=0\},\quad\omega\in\Omega.
\]
The mapping $T:\Omega\to\mathbb{Z}_{+}$ is called the \emph{first return time} at the origin.
Next, we study the distribution of $T$ and compute $\mathbb{E}[T]$.
The calculation of the distribution is by means of the generating function
$g(z)\equiv \sum_{n=0}^\infty z^na_n$ for a non-negative bounded sequence $\{a_n\}_{n\ge0}$ and $|z|<1$.

\paragraph{Determine the Distribution of $T$}
Define the generating functions:
\begin{align*}
g(z)&=\sum_{n=0}^\infty\mathbb{P}(X_n=0)z^n\\
h(z)&=\sum_{n=0}^\infty\mathbb{P}(T=n)z^n
\end{align*}
Here 
\[
\mathbb{P}(X_n=0)=\left\{
\begin{aligned}
1,&\quad\text{if $n=0$}\\
\sum_{\ell=1}^n\mathbb{P}(T=\ell)\mathbb{P}(X_{n-\ell}=0),&\quad\text{if $n\ge1$}
\end{aligned}
\right.
\]
Substituting $\mathbb{P}(X_n=0)$ into $g(z)$, we imply
$g(z) = 1+h(z)g(z)$.

Moreover, the direct calculation of $\mathbb{P}(X_n=0)$ gives
\begin{align*}
\mathbb{P}(X_{2n}=0)&=\binom{2n}{n}2^{-2n}\\
\mathbb{P}(X_{2n+1}=0)&=0
\end{align*}
Substituting $\mathbb{P}(X_n=0)$ into $g(z)$, we imply
\[
g(z) = \sum_{n=0}^\infty\binom{2n}{n}2^{-2n}z^{2n}
=
\sum_{n=0}^\infty\frac{1}{n!}\frac{1}{2^{2n}}\frac{(2n)!}{n!}z^{2n} = (1-z^2)^{-1/2}.
\]
As a result, 
\[
h(z) = 1 - \frac{1}{g(z)} = 1-(1-z^2)^{1/2}.
\]
\begin{itemize}
\item
The distribution of $T$ can be obtained by taking derivative of $h$:
\[
\left.h^{(k)}(z)\right|_{z=0} = k!\mathbb{P}(T=k)
\]
\item
An alternative approach for determining the distribution of $T$ is the following.
Consider the generating function induced by $\mathbb{P}(T>2n)$:
\begin{align*}
\sum_{n=0}^\infty\mathbb{P}(T>2n)z^{2n}
&=
\sum_{n=0}^\infty\sum_{\ell=n+1}^\infty\mathbb{P}(T=2\ell)z^{2n}\\
&=
\sum_{\ell=1}^\infty\sum_{n=0}^{\ell-1}\mathbb{P}(T=2\ell)z^{2n}\\
&=\sum_{\ell=1}^\infty\mathbb{P}(T=2\ell)\frac{1-z^{2\ell}}{1-z^2}\\
%&=(1-z^2)^{-1/2}=g(z).
\end{align*}
Also observe that 
\[
h(1)-h(z)=\sum_{n=0}^\infty\mathbb{P}(T=n)\cdot(1-z^n)
=
\sum_{n=1}^\infty\mathbb{P}(T=2n)\cdot(1-z^{2n})
\]
Therefore,
\[
\sum_{n=0}^\infty\mathbb{P}(T>2n)z^{2n}
=
\sum_{\ell=1}^\infty\mathbb{P}(T=2\ell)\frac{1-z^{2\ell}}{1-z^2}
=
\frac{1-h(z)}{1-z^2}=(1-z^2)^{-1/2}=g(z).
\]
As a result,
\[
\mathbb{P}(T>2n)
=
\mathbb{P}(X_{2n}=0)
=\frac{1}{2^{2n}}\binom{2n}{n}.
\]
Apply the Stirling's formula that 
\[
n!\sim \sqrt{2\pi}n^{n+1/2}e^{-n}
\implies
\mathbb{P}(T>2n)\sim \frac{1}{\sqrt{n\pi}}.
\]
\end{itemize}
The expectation $\mathbb{E}[T]$ can be obtained as the following:
\[
\mathbb{E}[T]
=\sum_{n=0}^\infty n\mathbb{P}(T=n)
=\sum_{n=1}^\infty n[\mathbb{P}(T>n) - \mathbb{P}(T>n-1)]
=
-\sum_{n=0}^\infty \mathbb{P}(T>n)=\infty.
\]
%
%
%
%
%In order to find the distribution of $T$, use generating functions:
%\begin{align*}
%g(z)&=\sum_{n=0}^\infty\mathbb{P}(X_n=0)z^n\\
%h(z)&=\sum_{n=0}^\infty\mathbb{P}(T=n)z^n
%\end{align*}
%It follows that 
%\[
%\left.h^{(k)}(z)\right|_{z=0} = k!\mathbb{P}(T=k)
%\]
%It's only possible when $X_{2n}=0$, with probability
%\[
%\mathbb{P}(X_{2n}=0) = \binom{2n}{n}
%\frac{1}{2^n}
%\]
%which implies
%\[
%g(z) = \sum_{n=0}^{\infty}\frac{1}{n!}\frac{(2n)!}{n!}\frac{(z^2)^n}{2^{2n}}
%\]
%Observe that 
%\[
%[(1-z)^{-1/2}]^{(n)} = \frac{1}{2}\frac{3}{2}\cdots\frac{2n-1}{2}(1-z)^{-(2n+1)/2}
%=\frac{1}{2^nn!}\frac{(2n)!}{2^n}(1-z)^{-(2n+1)/2}
%\]
%which follows that
%\[
%g(z) = (1-z^2)^{-1/2}
%\]
%We consider $\mathbb{P}(T>2n)$:
%\begin{align*}
%\sum_{n=0}^\infty\mathbb{P}(T>2n)z^{2n}
%&=
%\sum_{n=0}^\infty\sum_{\ell=n+1}^\infty\mathbb{P}(T=2\ell)z^{2n}\\
%&=
%\sum_{\ell=1}^\infty\sum_{n=0}^{\ell-1}\mathbb{P}(T=2\ell)z^{2n}\\
%&=\sum_{\ell=1}^\infty\mathbb{P}(T=2\ell)\frac{1-z^{2\ell}}{1-z^2}\\
%&=(1-z^2)^{-1/2}=g(z).
%\end{align*}
%As a result, 
%\[
%\mathbb{P}(T>2n) = \mathbb{P}(X_{2n}=0)
%=\frac{1}{2^{2n}}\binom{2n}{n}
%\]
The key observation of the deviation above is 
\[
\mathbb{P}(T>2n) = \mathbb{P}(X_{2n}=0),
\]
which can also be explained by the reflection principle.


\subsection{Basic Tips on Probability Theory}
We are going to study two important limit theorems:
\begin{itemize}
\item
Law of large numbers (LLN);
\item
Central Limit Theorem (CLT);
\end{itemize}
Firstly we need to review fundamentals of probability theory:
\begin{itemize}
\item
Limiting operations on events;
\item
Continuity of probabilities;
\item
Relation between distribution and random variable.
\end{itemize}
\paragraph{Limiting operations on events}
The difficulty for defining limit on events is as follows.
Given two sets $A,B$, they may be ordered by inclusion, but may not be always.
A rescue is by the union and intersection.
Define
\[
\underline{A}_n = \bigcap_{\ell=n}^{\infty}A_{\ell},\qquad
\overline{A}_n = \bigcup_{\ell=n}^{\infty}A_{\ell}.
\]
and
\[
\liminf_{m\to\infty}A_m:=\bigcup_{n=1}^\infty\bigcap_{\ell=n}^{\infty}A_{\ell},\qquad
\limsup_{n\to\infty}A_n:=\bigcap_{n=1}^\infty\bigcup_{\ell=n}^{\infty}A_{\ell}
\]
By definition, $\underline{A}_m\subseteq
A_m\subseteq
\overline{A}_m$.
If $\liminf A_m=\limsup A_m = A$, then $\{A_n;~n\ge1\}$ is said to converge to $A$, denoted as
\[
\lim_{m\to\infty}A_m=A.
\]

\begin{theorem}[Continuity of Probability]
If $A_n$ converges to $A$ as $n\to\infty$, then
\[
\lim_{n\to\infty}\mathbb{P}(A_n) = \mathbb{P}(A)
\]
In other words,
\[
\lim_{n\to\infty}
\int
\bm 1\{A_n\}
\diff m 
=
\int\lim_{n\to\infty}\bm 1\{A_n\}\diff m 
\]
\end{theorem}


\begin{proof}
\begin{enumerate}
\item
Firstly, assume that $A_n\uparrow A$ and show $\lim_{n\to\infty}\mathbb{P}(A_n) = \mathbb{P}(A)$.
\item
Secondly, when $A_n\downarrow A$, $A_n^c\uparrow A^c$. Thus the same result can be shown.
\item
The general $A_n$ can also be shown by considering
\[
\mathbb{P}\left(
\cap_{\ell=n}^\infty A_{\ell}
\right)
\le
\mathbb{P}(A_n)
\le
\mathbb{P}\left(
\cup_{\ell=n}^\infty A_{\ell}
\right)
\]
\end{enumerate}
\end{proof}


\begin{definition}[Distribution Function]
A \emph{distribution function} $F:\mathbb{R}\to[0,1]$ is defined as follows:
\begin{enumerate}
\item
Nondecreasing: $F(x)\le F(y)$ for all $x<y$;
\item
Right-continuous: $F(x)=F(x+)\equiv\lim_{\varepsilon\to 0}F(x+\varepsilon)$;
\item
$\lim_{x\to-\infty}F(x)=0$ and $\lim_{x\to+\infty}F(x)=1$.
\end{enumerate}
\end{definition}

\begin{proposition}
For distribution function $F$, there uniquely exists a probability measure $\mathbb{P}$
on $(\mathbb{R},\mathcal{B}(\mathbb{R}))$ such that 
\[
\mathbb{P}((a,b]) = F(b) - F(a),\qquad\forall a,b\text{ with }a\le b.
\]
\end{proposition}

The random variable $X$ associated with distribution function $F$ does not necessarily exist
on a given probability space $(\Omega,\mathcal{F},\mathbb{P})$.
The answer is yes when a random variable $U$ with uniform distribution on $[0,1]$
exists on it.
The explicit construction of $X$ is the following.
Define the inverse function $h:[0,1]\to\mathbb{R}$ such that
\[
h(y) = \inf\{x\in\mathbb{R}:~y\le F(x)\},\qquad y\in[0,1].
\]
Then define $X=h(U)$, which follows that
\[
\mathbb{P}(X\le x) = \mathbb{P}(h(U)\le x) = \mathbb{P}(U\le F(x)) = F(x).
\]









