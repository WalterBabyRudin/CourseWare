% !TEX encoding = UTF-8 Unicode

\section{Thursday}\index{Thursday_lecture}

\subsection{Central Limit Theorem}
The strong LLN indicates that $\overline{X}_n\to m$ a.s.
It can be applied to real data $\{X_n(\omega), n\ge1\}$ to estimate $m$.
We need to see how big $n$ is sufficient. 
This may be answered by computing for a given error level $\varepsilon>0$,
$\mathbb{P}(|\overline{X}_n- m|>\varepsilon)$.
Therefore, we need to know the general distribution of $\overline{X}_n$, which is independent of the distribution of $X_n$.

\begin{theorem}[Central Limit Theorem]
Let $\{X_n\}_{n\ge1}$ be a sequence of i.i.d. random variables with finite $m=\mathbb{E}[X_1]$
and finite $\sigma = \sqrt{V(X_1)}$, then the sample average $\overline{X}_n$ satisfies
\[
\mathbb{E}[\overline{X}_n]=m,\qquad
V(\overline{X}_n) = \frac{\sigma^2}{n}
\]
Define the normalized random variable of $\overline{X}_n$ as
\[
Z_n = \frac{\overline{X}_n - m}{\sqrt{V(\overline{X}_n)}} = \frac{S_n - mn}{\sqrt{n}\sigma}.
\]
Then
\[
\lim_{n\to\infty}\mathbb{P}(Z_n\le x) = \Phi(x),\qquad x\in\mathbb{R},
\]
where $\Phi$ is the standard normal distribution function:
\[
 \Phi(x) = \int_{-\infty}^x\frac{1}{\sqrt{2\pi}}\exp\left(-\frac{1}{2}u^2\right)\diff u.
\]
\end{theorem}
As a result, we have the following approximation:
\[
\mathbb{P}(|\overline{X}_n- m|>\varepsilon) 
=\mathbb{P}\left(
|Z_n|>\frac{\sqrt{n}}{\sigma}\varepsilon
\right) = 2\left(
1 - \Phi(\frac{\sqrt{n}}{\sigma}\varepsilon)
\right).
\]

We have the following questions about central limit theorem:
\begin{itemize}
\item
Is the $\Phi$ a distribution function?
\item
How to define convergence for distribution functions?
\item
How to show the central limit theorem?
\end{itemize}

\begin{proposition}
The function $\Phi$ is a distribution function, where 
\[
 \Phi(x) \triangleq \int_{-\infty}^x\frac{1}{\sqrt{2\pi}}\exp\left(-\frac{1}{2}u^2\right)\diff u.
\]
\end{proposition}
\begin{proof}
It suffices to show that $\lim_{x\to\infty}\Phi(x)=1$. In other words, it suffices to show 
\[
\int_{-\infty}^\infty\int_{-\infty}^\infty\exp\left(-\frac{1}{2}u^2-\frac{1}{2}v^2\right)\diff u\diff v=2\pi.
\]
By the change of variable $\diff u\diff v = r\diff r\diff \theta$,
\[
\int_{-\infty}^\infty\int_{-\infty}^\infty\exp\left(-\frac{1}{2}u^2-\frac{1}{2}v^2\right)\diff u\diff v
=
\int_{0}^{2\pi}\int_0^\infty \exp\left(-\frac{1}{2}r^2\right)r\diff r\diff\theta=2\pi.
\]
\end{proof}

\begin{definition}[Convergence for distribution functions]
For a distribution function $F$, define $C_F$ as the set of all continuous points of $F$:
\[
C_F = \{x\in\mathbb{R}:~F(x-)=F(x)\}.
\]
\begin{enumerate}
\item
For distributions $F_n, n\ge1$ and $F$, we say $F_n\xrightarrow{w}F$ if
\[
\lim_{n\to\infty}F_n(x) = F(x).
\]
\item
For random variables $X_n, n\ge1$ and $X$, we say $X_n\xrightarrow{d}X$ if the distribution of $X_n$ weakly converges to that of $X$.
\end{enumerate}
\end{definition}
The discontinuous points are excluded for defining weak convergence, since otherwise the convergence mode will be too limited.

\begin{example}[Examples of weak convergence]
For a constant $a$, define distribution functions $F_n,F$ as
\[
F_n(x) = \left\{
\begin{aligned}
0,&\quad\text{if $x<a-1/n$}\\
\frac{n}{2}(x - a + \frac{1}{n}),&\quad\text{if $a-1/n\le x < a+1/n$}\\
1,&\quad\text{if $a+1/n\le x$}
\end{aligned}
\right.\qquad
F(x)=\left\{
\begin{aligned}
0,&\quad\text{if $x<a$}\\
1,&\quad\text{if $a\le x$}
\end{aligned}
\right.
\]
As a result,
\[
\lim_{n\to\infty}F_n(x) = \left\{
\begin{aligned}
0,&\quad\text{if $x<a$}\\
\frac{1}{2},&\quad\text{if $x=a$}\\
1,&\quad\text{if $x>a$}
\end{aligned}
\right.
\]
Therefore, $F_n(x)$ converges to $F(x)$ except for $x=a$.
\end{example}

\paragraph{General Guidance for Proving Central Limit Theorem}
\begin{enumerate}
\item
Firstly define the characteristic function $\phi:\mathbb{R}\to\mathbb{C}$:
\[
\phi(\theta)\triangleq \int_0^\infty e^{-i\theta x}F(\diff x)
\]
Then showing the following facts.
\item
$\phi$ uniquely determines distribution function $F$.
\item
$F_n\xrightarrow{w}F$ if and only if $\phi_n(\theta)\to\phi(\theta),\forall \theta\in\mathbb{R}$.
\item
$\phi_n(\theta)$ for the distribution of $\frac{1}{\sigma\sqrt{n}}(S_n - mn)$ converges to $\exp(-\frac{1}{2}\theta^2)$ for all $\theta\in\mathbb{R}$.
\end{enumerate}

\subsection{Complex Numbers and Functions}

\begin{definition}[Exponential Function]
Define the \emph{exponential funciton} with variable $z\in\mathbb{C}$ as
\[
e^z = \sum_{n=0}^\infty\frac{1}{n!}z^n.
\]
Exponential function admits the Euler's formula:
\[
e^{iz} = \sum_{n=0}^\infty\frac{1}{n!}(iz)^n = \cos z + i\sin z.
\]
\end{definition}







%
%Normalization of r.v. $X$:
%when $m\equiv \mathbb{E}[X]$ and $\sigma^2\equiv V(X)$, then the normalization of $X$ is
%\[
%Z \equiv \frac{1}{\sigma}(X-m)
%\]
%Consider the sample average $\overline{X}_n$,
%\[
%\mathbb{E}[\overline{X}_n]=m,\qquad
%V(\overline{X}_n) = \frac{\sigma^2}{n}
%\]
%Thus the normalized r.v. of $\overline{X}_n$ is
%\[
%Z_n = \frac{\overline{X}_n - m}{\sqrt{V(\overline{X}_n)}} = \frac{S_n - mn}{\sqrt{n}\sigma}
%\]