
\chapter{Week3}

\section{Tuesday}

\subsection{Reviewing}
\begin{definition}
For $p\ge1$, we say a random variable $X\in\mathcal{L}^p$ if
\[
\|X\|_p^p\triangleq \mathbb{E}[|X|^p]<\infty.
\]
Particularly, when $X\in\mathcal{L}^1$, the random variable $X$ is said to be \emph{integrable}.
\end{definition}
A useful propoerty of integrability is the following:
\begin{proposition}\label{Pro:3:1}
Suppose that a random variable $X$ is integrable, then 
for any $\varepsilon>0$, there exists $\delta>0$ such that for any $F\in\mathcal{F}$ with $\mathbb{P}(F)<\delta$, we have
\[
\mathbb{E}[|X|;F]\triangleq\mathbb{E}[|X|1_F]=\int_F|X|\diff\mathbb{P}<\varepsilon
\]
\end{proposition}
Since $\{|X|>K\}$ happens with small probability, we have the following corollary:
\begin{corollary}\label{Cor:3:1}
Suppose that $X\in L^1(\Omega,\mathcal{F},\mathbb{P})$,
then for any $\varepsilon>0$, there exists $K>0$,
such that 
\[
\mathbb{E}[|X|;|X|>K]:=\int_{\{|X|>K\}}|X|\diff\mathbb{P}<\varepsilon.
\]
\end{corollary}
\begin{definition}
Consider a collcetion of random variables instead, denoted as $\mathcal{C}$:
\begin{itemize}
\item
$\mathcal{C}$ is said to be \emph{$L^p$-bounded} if there exists \emph{a} finite $M$ such that
\[
\mathbb{E}[|X|^p]<M,\quad\forall X\in\mathcal{C}.
\]
\item
$\mathcal{C}$ is said to be \emph{uniformly integrable} if any given $\varepsilon>0$, there exists \emph{a} $K\ge0$ such that
\[
\mathbb{E}[|X|1_{\{|X|>K\}}]<\varepsilon,\qquad\forall X\in\mathcal{C}.
\]
\end{itemize}
\end{definition}
\begin{remark}
UI implies $L^1$-boundedness: Try to upper bound $\mathbb{E}[|X|]$.
However, the converse is not true:
One counter-example is $\mathcal{C}=\{X_n\}_n$ with $X_n =n\cdot 1_{(0,1/n)}$.
\end{remark}

\begin{proposition}
\begin{itemize}
\item
$L^p$-boundedness for $p>1$ will imply UI;
\item
The class of random variables dominated by an integrable random variable is UI.
\end{itemize}
\end{proposition}
Recall the proof stated in Theorem~\ref{The:2:6} and Theorem~\ref{The:2:7} in detail.
\begin{proof}[Proof Outline]
\begin{enumerate}
\item
The first statement is by applying the $L^p$-boundedness on the following formula:
\[
\mathbb{E}[|X|1_{\{|X|>K\}}]
=
\int_{\{|X|>K\}}|X|\diff\mathbb{P}
\le \frac{1}{K^{p-1}}\int_{\{|X|>K\}}|X|^p\diff\mathbb{P}.
\]
\item
Firstly show that
\[
\mathbb{E}[|X|1_{\{|X|>K\}}]
=
\int_{\{|X|>K\}}|X|\diff\mathbb{P}\le \int_{\{|Y|>K\}}|Y|\diff\mathbb{P}.
\]
Apply Corollary~\ref{Cor:3:1} concludes the proof.
\end{enumerate}
\end{proof}



\subsection{Necessary and Sufficient Conditions for UI}

Our first result is about sufficient conditions for the UI on a collection of conditional expectations:

\begin{theorem}\label{The:UI:condition}
Suppose that $X\in\mathcal{L}^1(\Omega,\mathcal{F},\mathbb{P})$ and $\{\mathcal{G}_{\alpha}\}_{\alpha\in\mathcal{A}}$ is a collection of $\sigma$-algebras such that $\mathcal{G}_{\alpha}\subseteq\mathcal{F}$.
Then the collection of random variables 
\[
\mathcal{C} = \bigg\{
\mathbb{E}[X\mid\mathcal{G}_{\alpha}]:~~\alpha\in\mathcal{A}
\bigg\}
\]
is uniformly integrable.
\end{theorem}

\begin{proof}
\begin{itemize}
\item
Apply proposition~\ref{Pro:3:1} on $X$:
For given $\varepsilon>0$, there exists $\delta>0$ such that when $\mathbb{P}(F)<\delta$ with $F\in\mathcal{F}$, $\mathbb{E}[|X|\cdot 1_F]<\varepsilon$.
%Assume on the contrary not, then
%there exists a $\varepsilon>0$ such that for any $\delta>0$, whenever $\mathbb{P}(F)\le\delta$ with $F\in\mathcal{F}$, $\mathbb{E}[|X|\cdot 1_F]>\varepsilon$.
%Pick a sequence of sets $\{F_n: F_n\in\mathcal{F}\}$ with $\mathbb{P}(F_n)\to 0$, then
%\[
%\lim_{n\to\infty}\mathbb{E}[|X|\cdot 1_{F_n}]
%=
%\lim_{n\to\infty}\int_{\Omega}|X|1_{F_n}\diff\mathbb{P}
%\le
%\int_{\Omega}\lim_{n\to\infty}|X|1_{F_n}\diff\mathbb{P}=0.
%\]
%Therefore, $\mathbb{E}[|X|\cdot 1_{F_n}]\to0$, which contradicts to the previous statement.
\item
Define $Y_{\alpha}=\mathbb{E}[X\mid\mathcal{G}_{\alpha}]$. By Jensen's inequality,
$|Y_{\alpha}|\le \mathbb{E}[|X|\mid\mathcal{G}_{\alpha}]$, which motivates us to upper bound the following integral:
\begin{align*}
\mathbb{E}\bigg[
|\mathbb{E}[X\mid\mathcal{G}_{\alpha}]|;~
|\mathbb{E}[X\mid\mathcal{G}_{\alpha}]|>K
\bigg]
&=
\int_{\{|Y_{\alpha}|>K\}}|Y_{\alpha}|\diff\mathbb{P}\\
&\le
\int_{\{|Y_{\alpha}|>K\}}\mathbb{E}[|X|\mid\mathcal{G}_{\alpha}]\diff\mathbb{P}\\
&\le
\int_{\{\mathbb{E}[|X|\mid\mathcal{G}_{\alpha}]>K\}}\mathbb{E}[|X|\mid\mathcal{G}_{\alpha}]\diff\mathbb{P}\\
&=
\int_{\{\mathbb{E}[|X|\mid\mathcal{G}_{\alpha}]>K\}}|X|\diff\mathbb{P}
\end{align*}
where the last equality is because of the definition for conditional expectation and that $\{\mathbb{E}[|X|\mid\mathcal{G}_{\alpha}]>K\}\in\mathcal{G}_{\alpha}$.
\item
Then consider upper bounding $\mathbb{P}\{\mathbb{E}[|X|\mid\mathcal{G}_{\alpha}]>K\}$ using Markov inequality:
\[
\mathbb{P}\{\mathbb{E}[|X|\mid\mathcal{G}_{\alpha}]>K\}
\le
\frac{\mathbb{E}[\mathbb{E}[|X|\mid\mathcal{G}_{\alpha}]]}{K}
=
\frac{\mathbb{E}[|X|]}{K},
\]
where the equality is by the tower property of conditional expectation.
Here we choose $K$ such that $\frac{\mathbb{E}[|X|]}{K}<\delta$, which implies 
$\mathbb{P}\{\mathbb{E}[|X|\mid\mathcal{G}_{\alpha}]>K\}<\delta$.
By applying the result on the first part, we have
\[
\mathbb{E}\bigg[
|\mathbb{E}[X\mid\mathcal{G}_{\alpha}]|;~
|\mathbb{E}[X\mid\mathcal{G}_{\alpha}]|>K
\bigg]
\le 
\int_{\{\mathbb{E}[|X|\mid\mathcal{G}_{\alpha}]>K\}}|X|\diff\mathbb{P}
\le\varepsilon.
\]

\end{itemize}
\end{proof}
% Prof. Hu's note unclear in the proof
\begin{remark}
A class $\mathcal{C}$ of random variables is uniformly integrable if and only if 
\[
\lim_{k\to\infty}\sup_{X\in\mathcal{C}}\int_{\{|X|>K\}}|X|\diff\mathbb{P}=0.
\]
\end{remark}
\subsection{Convergence of random variables}

In the following part we study several convergence versions shown in probability theory.
\begin{definition}[Convergence in probability]
Let $\{X_n\}$ be a sequence of random variables. 
\begin{itemize}
\item
We call $\{X_n\}$ converges to a random variable $X$ in probability, denoted as $X_n\to X$ in prob., if for any $\varepsilon>0$,
\[
\lim_{n\to\infty}\mathbb{P}\bigg(
|X_n-X|>\varepsilon
\bigg)=0.
\]
\item
We call $\{X_n\}$ converges to a random variable $X$ a.s., denoted as $X_n\to X$ a.s.., if 
\[
\mathbb{P}\bigg(
\left\{\omega:~
\lim_{n\to\infty}X_n(\omega)=X(\omega)
\right\}
\bigg)=1.
\]
\item
We call $\{X_n\}$ converges to a random variable $X$ in $L^1$, denoted as $X_n\to X$ in $L^1$, if
\[
\lim_{n\to\infty}\|X_n-X\|_1=0.
\]
\end{itemize}
\end{definition}
\begin{remark}
\begin{itemize}
\item
$X_n\to X$ a.s. implies $X_n\to X$ in prob.;
\item
$X_n\to X$ in $L^1$ implies $X_n\to X$ in prob.;
\item
A natural question is what is the connection between convergence a.s. and convergence in $L^1$.
The dominated convergence theorem provides the following characterization:
\[
{\displaystyle \left.{\begin{matrix}X_{n}{\xrightarrow {\overset {}{\text{a.s.}}}}X\\|X_{n}|<Y\\\mathrm {E} (Y)<\infty \end{matrix}}\right\}\quad \Rightarrow \quad X_{n}{\xrightarrow {L^{1}}}X}
\]
\end{itemize}
\end{remark}
Then we provide sufficient conditions for convergence in probability to imply convergence in $L^1$:
\begin{theorem}[Bounded Convergence Theorem]\label{The:BCT}
Let $\{X_n\}$ be a sequence of random variables converging to $X$ in probability.
Suppose that $\{X_n\}$ is bounded by $M$, i.e., $|X_n(\omega)|\le M,\forall \omega\in\Omega, n\ge1$. Then $\{X_n\}$ converges to $X$ in $L^1$:
\[
\lim_{n\to\infty}\mathbb{E}[|X_n-X|]=0.
\]
\end{theorem}
% how to proof?
\begin{remark}
Note that it is a stronger version of bounded convergence theorem compared with the one studied in MAT3006. 
In the theorem above, we only require convergence in probability rather than convergence a.s.
% is it possible to relax M into a general integrable r.v.?
\end{remark}

The relevance between uniform integrability and convergence of random variables is explained by the following theorem:

\begin{theorem}\label{The:UI:Converge}
Let $\{X_n\}$ be a sequence of random variables with $X_n\in\mathcal{L}^1$, and let $X\in\mathcal{L}^1$.
The sequence $\{X_n\}$ converges to $X$ in ${L}^1$ if and only if
\begin{enumerate}
\item
$X_n\to X$ in probability, and
\item
$\{X_n\}$ is uniformly integrable.
\end{enumerate}
\end{theorem}
% do not need X is L1
\begin{proof}[Proof for the Reverse Direction]
For $K>0$, construct a function $\phi_K:\mathbb{R}\to[-K,K]$:
\[
\phi_K(x)=\left\{
\begin{aligned}
K,&\quad\text{if $x>K$}\\
x,&\quad\text{if $|x|\le K$}\\
-K,&\quad\text{if $x<-K$}
\end{aligned}
\right.
\]
By the triangle inequality, 
\[
|X_n-X|\le |X_n - \phi_K(X_n)| + |\phi_K(X_n) - \phi_K(X)| + |\phi_K(X) - X|.
\]
It suffices to upper bound three terms on the RHS for the following formula:
\begin{align}
\mathbb{E}[|X_n-X|]&\le \mathbb{E}[|X_n - \phi_K(X_n)|]+\mathbb{E}[|\phi_K(X_n) - \phi_K(X)|]+\mathbb{E}[|\phi_K(X) - X|]\nonumber\\
&=\int_{\{|X|>K\}}[|X|-K]\diff\mathbb{P}+\mathbb{E}[|\phi_K(X_n) - \phi_K(X)|]+\int_{\{|X_n|>K\}}[|X_n|-K]\diff\mathbb{P}\nonumber\\
&\le \int_{\{|X|>K\}}[|X|]\diff\mathbb{P}+\mathbb{E}[|\phi_K(X_n) - \phi_K(X)|]+\int_{\{|X_n|>K\}}[|X_n|]\diff\mathbb{P}
\end{align}
\begin{itemize}
\item
For the first term, by choosing sufficiently large $K$, by Corollary~\ref{Cor:3:1}, it can be upper bounded by $\varepsilon/3$;
\item
For the third term, when $K$ is large enough, by the uniform integrability of $\{X_n\}$, it can be upper bounded by $\varepsilon/3$;
\item
Observe the following inequality holds: 
\[
|\phi_K(x) - \phi_K(y)|\le|x-y|,\forall x,y\implies
\{|\phi_K(X_n) - \phi_K(X)|>\varepsilon\}\subseteq\{|X_n-X|>\varepsilon\},
\]
which means that $\mathbb{P}(\{|\phi_K(X_n) - \phi_K(X)|>\varepsilon\})\le \mathbb{P}(\{|X_n-X|>\varepsilon\})$.
As a result, $X_n\to X$ in prob. implies $\phi_K(X_n)\to\phi_K(X)$ in prob.\footnote{Following the similar method, we can show that as long as $f$ is continuous and $X_n\to X$ in prob., we have $f(X_n)\to f(X)$ in prob.}

By the Bounded Convergence Theorem~\ref{The:BCT}, $\lim_{n\to\infty}\mathbb{E}[|\phi_K(X_n) - \phi_K(X)|]=0$.
Thus for sufficiently large $n$,
\[
\mathbb{E}[|\phi_K(X_n) - \phi_K(X)|]<\frac{\varepsilon}{3}
\]
\end{itemize}
Combining these three bounds above, for fixed $\varepsilon>0$, we can pick $K>0$ and there exists sufficiently large $n$ such that 
\[
\mathbb{E}[|X_n-X|]\le\varepsilon.
\]
\end{proof}

\begin{proof}[Proof for the Forward Direction]
\begin{itemize}
\item
Firstly we show that $\{X_n\}$ is $L^1$-bounded, which suffices to show that 
$\mathbb{E}[|X_n|]\to\mathbb{E}[|X|]$, which is because of the following observation:
\[
\bigg|
\mathbb{E}[|X_n|]-\mathbb{E}[|X|]
\bigg|
\le\mathbb{E}||X_n| - |X||\le \mathbb{E}|X_n-X|\to0.
\]
\item
Then we show the uniform integrability result. By the $L^1$-convergence, for fixed $\varepsilon>0$, there exists $N_0>0$ such that 
\[
\mathbb{E}\bigg[
|X_n-X|
\bigg]<\frac{\varepsilon}{2},\quad\forall n>N_0.
\]
Similar as the previous proof for the uniform integrability results, we should apply proposition~\ref{Pro:3:1} on \textit{finitely many} random variables:
for fixed $\varepsilon>0$, there exists \emph{a} $\delta>0$ such that whenever $\mathbb{P}(F)<\delta, F\in\mathcal{F}$,
\begin{subequations}
\begin{align}
\mathbb{E}[
|X|1_F
]&<\frac{\varepsilon}{2}\label{Eq:3:2:a}\\
\mathbb{E}[
|X_n|1_F
]&<\frac{\varepsilon}{2},\quad\forall n\le N_0\label{Eq:3:2:b}
\end{align}
\end{subequations}
\item
Construct \emph{a} $K$ such that $\mathbb{P}(|X_n|>K)$ is small for any $n$:
\[
\mathbb{P}(|X_n|>K)\le\frac{\mathbb{E}|X_n|}{K}\le \frac{\sup_{n}\mathbb{E}|X_n|}{K}
\]
Therefore, we choose $K$ such that $\frac{\sup_{n}\mathbb{E}|X_n|}{K}<\delta$, and then 
$\mathbb{P}(|X_n|>K)<\delta$.
\item
Now we can conclude the uniform integrability result:
For $n\le N_0$, by the construction of $K$ and (\eqref{Eq:3:2:b}),
\[
\mathbb{E}[
|X_n|1_{\{|X_n|>K\}}
]<\varepsilon.
\]
For $n>N_0$,
\begin{align*}
\mathbb{E}\bigg[
|X_n|1_{\{|X_n|>K\}}
\bigg]
&\le
\mathbb{E}\bigg[
|X-X_n|1_{\{|X_n|>K\}}
\bigg]
+
\mathbb{E}\bigg[
|X|1_{\{|X_n|>K\}}
\bigg]\\
&\le
\mathbb{E}\bigg[
|X-X_n|
\bigg]
+
\mathbb{E}\bigg[
|X|1_{\{|X_n|>K\}}
\bigg]\\
&<\frac{\varepsilon}{2}+\frac{\varepsilon}{2}=\varepsilon.
\end{align*}
where the last inequality is because of the $L^1$-convergence and (\eqref{Eq:3:2:a}).
\item
Finally, the convergence in probability can be shown by the Markov inequality:
\[
\mathbb{P}\bigg(
|X_n-X|>\varepsilon
\bigg)\le
\frac{\mathbb{E}[|X_n-X|]}{\varepsilon}\to0,\text{ as $n\to\infty$}.
\]
\end{itemize}
\end{proof}


\subsection{Martingales in Discrete Time}

\begin{definition}[Stochastic Process]
Let $(\Omega,\mathcal{F},\mathbb{P})$ be a probability space.
We describe random phenomena in discrete time 
by a collection of random variables $\{X_n:~n\ge1\}$ and increasing sequence of
sub $\sigma$-fields $\{\mathcal{F}_n:~\mathcal{F}_n\subseteq\mathcal{F}\}$.
\begin{itemize}
\item
$X(\cdot)\triangleq \{X_n:~n\ge1\}$ is called a \emph{stochastic process};
\item
$\mathbb{F}\triangleq \{\mathcal{F}_n:~\mathcal{F}_n\subseteq\mathcal{F}\}$ is called a \emph{filtration}.
\end{itemize}
A probability space $(\Omega,\mathcal{F},\mathbb{P})$ associated with a filtration $\mathbb{F}$
is called a \emph{filtered} probability space, written as $(\Omega,\mathcal{F},\mathbb{F},\mathbb{P})$
\end{definition}
\begin{remark}
A typical example of $\mathcal{F}$ is defined by generated $\sigma$-algebra:
\[
\mathcal{F}_n^{X}\triangleq\sigma(X_t:~t\le n),\qquad\forall n\ge0.
\]
This natural filteration is the sequence of smallest $\sigma$-algebras such that $X_n$ is $\mathcal{F}_n$-measurable for all $n$.
\end{remark}

\begin{definition}[Predictable process]
\begin{itemize}
\item
A stochastic process $X(\cdot)\triangleq \{X_n:~n\ge1\}$ is said to be \emph{adapted} to the 
filtration $\mathbb{F}\triangleq \{\mathcal{F}_n:~n\ge1\}$ if $X_n$ is $\mathcal{F}_n$-measurable for each $n$.
We call $X$ an \emph{adapted process} with respect to $\mathbb{F}$.
\item
If $X_n$ is $\mathcal{F}_{n-1}$-measurable for each $n$ and $X_0$ is $\mathcal{F}_0$-measurable, $X$ is said to be a \emph{predictable process}.
\end{itemize}
\end{definition}



\section{Thursday}

\subsection{Stopping Time}


\begin{definition}[Stopping Time]
A mapping $T:\Omega\to\{0,1,2,\ldots,\infty\}$ is called a stopping time with respect to the filteration 
$\{\mathcal{F}_n\}_{n\ge0}$ if
\[
\{T\le n\}\triangleq
\{\omega\in\Omega:~T(\omega)\le n\}\in\mathcal{F}_n,\qquad\forall n.
\]
\end{definition}
\begin{remark}
\begin{enumerate}
\item
$T$ can take the infinite value 
\item
An equivalent definition for a stopping time $T$ is $\{T= n\}\in\mathcal{F}_n,\forall n$.
\begin{proof}
\begin{enumerate}
\item
Suppose that $\{T\le n\}\in\mathcal{F}_n,\forall n$, then
\[
\{T\le n-1\}\in\mathcal{F}_{n-1}\subseteq \mathcal{F}_n
\implies
\{T= n\}=\{T\le n\}\setminus \{T\le n-1\}\in\mathcal{F}_n.
\]
\item
Suppose that $\{T= n\}\in\mathcal{F}_n,\forall n$, then
\[
\{T= k\}\in\mathcal{F}_k\subseteq \mathcal{F}_n,\forall k\le n
\implies
\{T= n\}=\bigcup_{k\le n}\{T= k\}\in\mathcal{F}_n.
\]
\end{enumerate}
\end{proof}
\item
A constant mapping $T\equiv N$ with $N\in\mathbb{Z}_+$ is always a stopping time.
\end{enumerate}
\end{remark}

\begin{example}
Let $\{X_n\}_{n\ge0}$ be an adapted process on a filtered probability space 
$(\Omega,\mathcal{F},\{\mathcal{F}_n\}_{n\ge0},\mathbb{P})$.
Let $B\in\mathcal{B}(\mathbb{R})$ be a Borel set.
Define
\[
T(\omega)\triangleq \inf~\{n\ge0:~X_n(\omega)\in B\}.
\]
Here $T$ denotes the first time that $\{X_n\}_{n\ge0}$ enters into set $B$.
Define $\inf(\emptyset)=\infty$ by default, i.e., $T=\infty$ when $\{X_n\}_{n\ge0}$ never enters into $V$.
To check that $T$ is a stopping time, observe that 
\begin{align*}
\{T=n\}&=\{X_0\in B^c,X_1\in B^c,X_2\in B^c,\ldots,X_{n-1}\in B^c, X_n\in B\}
\\&=
\{X_n\in B\}\cup\left(
\cup_{0\le k\le n-1}\{X_k\in B^c\}\right)
\end{align*}
Since $\{X_n\}$ is adapted, $\{X_k\in B^c\}\in\mathcal{F}_k\subseteq\mathcal{F}_n$ for $0\le k\le n-1$. Moreover, $\{X_n\in B\}\in\mathcal{F}_n$.
Therefore, $\{T=n\}\in\mathcal{F}_n$ for each $n$.
% proof in this example is wrong


\end{example}

\begin{definition}[Stopping Time $\sigma$-algebra]
Define the stopping time $\sigma$-algebra for a given stopping time $T$ as the following:
\[
\mathcal{F}_T = \{A\in\mathcal{F}:~
A\cap\{T\le n\}\in\mathcal{F}_n,\forall n
\}.
\]
Here $\mathcal{F}_T$ represents the information available up to a random time $T$.
\end{definition}
\begin{proposition}
\begin{enumerate}
\item
$\mathcal{F}_T$ is a $\sigma$-algebra;
\item
$T$ is $\mathcal{F}_T$-measurable;
\item
When $T_1,T_2$ are two stopping times with $T_1\le T_2$ a.s., 
$\mathcal{F}_{T_1}\subseteq\mathcal{F}_{T_2}$.
\end{enumerate}
\end{proposition}
\begin{proof}
\begin{enumerate}
\item
It is trivial that $\emptyset\in\mathcal{F}_T$.
Suppose that $A\in\mathcal{F}_T$, then $(A\cap\{T\le n\})^c\in\mathcal{F}_n$, which implies that
\[
A^c\cap\{T\le n\} = \bigg(
A\cap\{T\le n\}
\bigg)^c\cap\{T\le n\}\in\mathcal{F}_n.
\]
Suppose that $A_k\in\mathcal{F}_T, k\ge1$, then
\[
\bigg(
\bigcup_{k\ge1}A_k
\bigg)\cap\{T\le n\} = \bigcup_{k\ge1}(A_k\cap \{T\le n\})\in\mathcal{F}_n.
\]
\item
It suffices to show that $\{T\le m\}\in\mathcal{F}_T$ for any $m$.
This is true because for any $n$, 
\[
\{T\le m\}\cap\{T\le n\} = \{T\le m\land n\}\in\mathcal{F}_{m\land n}\subseteq\mathcal{F}_n.
\]
\item
Consider any $A\in\mathcal{F}_{T_1}$, then $A\cap\{T_1\le n\}\in\mathcal{F}_n$ for any $n$.
Moreover, 
\[
\{T_2\le n\}\subseteq\{T_1\le n\}\implies
A\cap\{T_2\le n\}\subseteq A\cap\{T_1\le n\}\in\mathcal{F}_n,
\]
which implies the desired result.
\end{enumerate}
\end{proof}
\begin{theorem}
Let $\{X_n\}_{n\ge0}$ be an adapted process on $(\Omega,\mathcal{F},\{\mathcal{F}_n\}_{n\ge0},\mathbb{P})$.
Let $T$ be a stopping time w.r.t. $\{\mathcal{F}_n\}_{n\ge0}$.
Define a random variable $X_T$:
\[
X_T(\omega)\triangleq X_{T(\omega)}(\omega),\qquad\forall\omega\in\Omega.
\]
Then $X_T$ is $\mathcal{F}_T$-measurable.
\end{theorem}
\begin{proof}
It suffices to check $\{X_T\le a\}\in\mathcal{F}_T, \forall a\in\mathbb{R}$. By definition of the stopping time $\sigma$-algebra, it suffices to check
\[
\{X_T\le a\}\cap \{T\le n\}\in\mathcal{F}_n,\forall n\quad
\Longleftarrow
\quad
\bigcup_{0\le k\le n}\{X_k\le a\}\cap \{T=k\}\in\mathcal{F}_n,\forall n.
\]
Since $\{X_n\}$ is adapted, $\{X_k\le a\}\in\mathcal{F}_k,\forall k$.
By definition of the stopping time, $\{T=k\}\in\mathcal{F}_k,\forall k$.
Therefore,
\[
\{X_k\le a\}\cap \{T=k\}\in\mathcal{F}_k\subseteq \mathcal{F}_n.
\]
The proof is complete.
\end{proof}

\begin{definition}[Martingale]
Let $\{X_n\}_{n\ge0}$ be an adapted process on $(\Omega, \mathcal{F},\{\mathcal{F}_n\}_{n\ge0},\mathbb{P})$.
A stochastic process $\{X_n\}_{n\ge0}$ is called a \emph{martingale} if
\begin{enumerate}
\item
$X_n\in\mathcal{L}^1,\forall n$;
\item
$\mathbb{E}[X_{n+1}\mid\mathcal{F}_n] = X_n$ a.s., for all $n$.
\end{enumerate}
If in the last definition, ``$=$'' is replaced by ``$\le$'' or ``$\ge$'', then $\{X_n\}_{n\ge0}$
is said to be a \emph{supermartingale} or \emph{submartingale}, respectively.
\end{definition}

\begin{remark}
\begin{itemize}
\item
A supermartingale goes downward on average, and a submartigale goes upward on average.
\item
$\{X_n\}_{n\ge0}$ is a supermartingale if and only if $\{-X_n\}_{n\ge0}$ is a submartingale.
\item
$\{X_n\}_{n\ge0}$ is a martingale if and only if it is both a supermartingale and a submartingale.
\end{itemize}
\end{remark}


\begin{example}
Let $\{Y_n\}_{n\ge1}$ be a sequence of independent random variables with 
$\mathbb{E}[|Y_k|]<\infty$ and $\mathbb{E}[Y_k]=0, \forall k$.
Define $\mathcal{F}_n = \sigma(Y_1,Y_2,\ldots,Y_n)$ for $n\ge1$ and $\mathcal{F}_0=\{\emptyset,\Omega\}$.
Define $X_n = Y_1+Y_2+\cdots+Y_n,\forall n\ge1$ and $X_0=0$.
Then $\{X_n\}_{n\ge0}$ is a martingale:
\begin{enumerate}
\item
$\mathbb{E}[|X_n|]\le\sum_{i=1}^n\mathbb{E}[|X_i|]<\infty$, which means that $X_n$ is integrable;
\item
Check that
\begin{align*}
\mathbb{E}[X_{n+1}\mid\mathcal{F}_n]&=\mathbb{E}[X_n + Y_{n+1}\mid\mathcal{F}_n]\\
&=\mathbb{E}[X_n\mid\mathcal{F}_n] + \mathbb{E}[Y_{n+1}\mid\mathcal{F}_n]\\
&=X_n + \mathbb{E}[Y_{n+1}]=X_n,
\end{align*}
where the third equality is because that $X_n$ is $\mathcal{F}_n$-measurable
and $Y_{n+1}$ is independent of $\mathcal{F}_n$.
\end{enumerate}
\end{example}

\begin{example}
Let $\{Y_n\}_{n\ge1}$ be a sequence of independent random variables with 
$Y_k\ge0$ a.s. and $\mathbb{E}[Y_k]=1, \forall k$.
Define $\mathcal{F}_n = \sigma(Y_1,Y_2,\ldots,Y_n)$ for $n\ge1$ and $\mathcal{F}_0=\{\emptyset,\Omega\}$.
Define $X_n = Y_1\cdot Y_2\cdots Y_n,\forall n\ge1$ and $X_0=1$.
Then $\{X_n\}_{n\ge0}$ is a martingale:
\begin{enumerate}
\item
$\mathbb{E}[|X_n|]=\mathbb{E}[X_n]=\prod_{k=1}^n\mathbb{E}[Y_k]=1<\infty$, which means that $X_n$ is integrable;
\item
Check that
\begin{align*}
\mathbb{E}[X_{n+1}\mid\mathcal{F}_n]&=\mathbb{E}[X_n\cdot Y_{n+1}\mid\mathcal{F}_n]\\
&=X_n\cdot \mathbb{E}[Y_{n+1}\mid\mathcal{F}_n]\\
&=X_n\cdot \mathbb{E}[Y_{n+1}]=X_n,
\end{align*}
where the second equality is because that $X_n$ is $\mathcal{F}_n$-measurable;
the third equality is because that $Y_{n+1}$ is independent of $\mathcal{F}_n$.
\end{enumerate}
\end{example}

















