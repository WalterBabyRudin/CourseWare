
\chapter{Week2}

\section{Tuesday}\index{Tuesday_lecture}
\subsection{More on Stochastic Process}
For simplicity of notation, we write
\[
\{X\in F\}\triangleq\{\omega: X(\omega)\in F\}=X^{-1}(F).
\]
\begin{definition}[Joint Distribution of a Stochastic Process]
Let $\{X_t\}$ be a stochastic process.
Let $0=t_0\le t_1\le\cdots\le t_k$.
The joint distribution of random variables 
$X_{t_1},\ldots,X_{t_k}$ is defined as
\[
\mu_{t_1,t_2,\dots,t_k}(F_1\times\cdots\times F_k)
=
\mathbb{P}(X_{t_1}\in F_1,\cdots,X_{t_k}\in F_k),
\]
where $F_1,\dots,F_k$ are all Borel sets in $\mathbb{R}^n$.
\end{definition}

\begin{remark}
The measure $\mu_{t_1,t_2,\dots,t_k}$ is the \emph{finite-dimensional distribution}. In particular, $\mu_{t_1,t_2,\dots,t_k}$ is a probability measure on the product space $\mathbb{R}^n\times\cdots\times\mathbb{R}^n$.
\end{remark}

\begin{example}[Brownian Motion]
Consider a probability space $(\Omega,\mathcal{F},\mathbb{P})$. 
Define the function 
\[
\mathsf{P}(t,x,y)=\frac{1}{\sqrt{2\pi t}}\exp\left(
-\frac{(y-x)^2}{2t}\right),\qquad x,y\in\mathbb{R},t>0
\]
The Brownian motion
\footnote{now consider only the Brownian motion with independent, normally distributed increment.}
 is denoted by $\{B_t\}_{t\ge0}$.
Then the joint distribution of $\{B_t\}$ at time $t_1,t_2,\ldots,t_k$ is given by:
\[
\begin{aligned}
\mathbb{P}&(B_{t_1}\in F_1,\dots,B_{t_k}\in F_k)
=\\
&\int_{F_1\times\cdots\times F_k}
\mathsf{P}(t_1,0,x_1)
\mathsf{P}(t_2-t_1,x_1,x_2)
\cdots
\mathsf{P}(t_l-t_{k-1},x_{k-1},x_k)
\diff x_1\diff x_2\cdots \diff x_k
\end{aligned}
\]
\end{example}

\begin{definition}[Measurable Set]
Let $(S,\mathcal{F})$ be a pair, with $S$ being a set and $\mathcal{F}$ is a $\sigma$-algebra on $S$. Then the set $\mathcal{F}$ is called a \emph{measurable space}, and an element of $\mathcal{F}$ is called a $\mathcal{F}$-\emph{measurable} subset of $S$.
\end{definition}

\begin{remark}
Cosnider a stochastic process $\{X_t\}$ in continuous time, e.g., a Brownian motion.
Consider the space $(\mathbb{R}^{[0,\infty)},\mathcal{B}(\mathbb{R}^{[0,\infty)}))$, and define the collection of outcomes
\[
F=\{\omega\in\Omega\mid X_t(\omega)\in[0,1],\forall t\le1\}
\]
The issue is that this event $F$ is not necessarily $\mathcal{B}(\mathbb{R}^{[0,\infty)})$-measurable.
Sometimes we need some extra conditions on the stochastic process to make $F$ measurable.
The significance of $F$ will also discussed in the future.
%
%
%
%Such set $F$ may be not measurable, i.e., $F$ may not be an event. Then $\mathbb{P}(F)$ does not make sense. Therefore, we need the additional conditions.
\end{remark}

\begin{proposition}
Suppose that $\{X_t\}$ is a continuous-time stochastic process.
Let $\mathcal{T}$ be a countable subset of $[0,\infty)$, then given $B\in\mathcal{B}(\mathbb{R}^n)$,
\begin{itemize}
\item
The set $\{\omega:~
X_t(\omega)\in B
\text{ for any } t\in\mathcal{T}\}$ is measurable;
\item
The function $h=\sup_{t\in\mathcal{T}}|X_t|$ is $\mathcal{F}$-measurable.
\end{itemize}
\end{proposition}
\begin{proof}
For fixed $t\in\mathcal{T}$, because of the $\mathcal{F}$-measurability of $X_t$, the set
\[
\{X_t\in B\}:=\{\omega:~
X_t(\omega)\in B
\}\text{ is measurable}.
\]
It is easy to see that the countably intersection $\cap_{t\in\mathcal{T}}\{X_t\in B\}$ is measurable as well.
For the second assertion, it suffices to check that $h^{-1}([-\infty,a))=\cap_{t\in\mathcal{T}}\{X_t<a\}$ is measurable.
\end{proof}
However, when $\mathcal{T}$ is uncountable, it is problematic to show the measurability.
It is even difficult to show that for almost all $\omega$, $t\mapsto X_t(\omega)$ is continuous.
In order to obtain a ``continuous'' process, we need the following important concept:
%Let $\{x_t\}_{t\ge0}$ be a stochastic process on $(\Omega,\mathcal{F},\mathbb{P})$ and take values in $\mathbb{R}^n$. Let $\bm B$ be a Borel subset. If $T$ is a countable set (or can be finite), then
%\[
%\{\omega\mid x_t(\omega)\in\bm B,\forall t\in T\}\text{ is measurable} = \bigcap_{t\in T}\{\omega:X_t(\omega)\in\bm B\}.
%\]
%We can also show that $\sup_{t\in T}|x_t(\omega)|$ is measurable.

\begin{definition}[Equivalent random variables]
Let $\{X_t\}_{t\ge0}$ and $\{Y_t\}_{t\ge0}$ be two stochastic process on $(\Omega,\mathcal{F},\mathbb{P})$.
Then $\{Y_t\}$ is called an \emph{equivalent} (a \emph{version}) of 
$\{X_t\}$ if 
\[
\mathbb{P}(\{w\mid X_t(\omega)=Y_t(\omega)\})=1,\qquad
\text{for any time $t$}.
\]
\end{definition}
\begin{remark}
It is easy to see that when $\{X_t\}_{t\ge0}$ is a version of $\{Y_t\}_{t\ge0}$, they have the same finite-dimensional distributions, but their path properties may be different, e.g., for almost all $\omega$, 
$t\mapsto X_t(\omega)$ may be continuous while 
$t\mapsto Y_t(\omega)$ may not.
\end{remark}

\subsection{Conditional Expectation}
\begin{definition}[Conditional Expectation]\label{Def:2:4}
Suppose that $(\Omega,\mathcal{F},\mathbb{P})$ is a probability space.
$\mathcal{G}$ is a sub $\sigma$-algebra of $\mathcal{F}$, i.e., $\mathcal{G}\subseteq\mathcal{F}$.
Let $X:\Omega\to\mathbb{R}^n$ be an \textit{integrable} random variable, and the \emph{conditional expectation} $X$ given $\mathcal{G}$, denoted as 
$\mathbb{E}[X\mid\mathcal{G}]$, is a random variable 
satisfying the following conditions:
\begin{enumerate}
\item
$\mathbb{E}[X\mid\mathcal{G}]$ is $\mathcal{G}$-measurable;
\item
For any event $A\in\mathcal{G}$,
\[
\int_A\mathbb{E}[X\mid\mathcal{G}]\diff\mathbb{P}
=
\int_AX\diff\mathbb{P}
\]
In other words,
\[
\mathbb{E}[\mathbb{E}[X\mid\mathcal{G}]1_A]=\mathbb{E}[X1_A].
\]
\end{enumerate}
\end{definition}


\begin{remark}
Let $X$ be an integrable random variable.
Then for each sub $\sigma$-algebra $\mathcal{G}\subseteq\mathcal{F}$, 
the conditional expectation $\mathbb{E}[X\mid\mathcal{G}]$ exists and 
is unique up to $\mathcal{V}$-measurable sets of probability zero.
The proof is based on the Radon-Nikodym theorem.

In other words,
suppose that $Y$ is another random variable satisfying the condition mentioned in Definition~\ref{Def:2:4}, i.e.,
\begin{itemize}
\item
$Y$ is $\mathcal{G}$-measurable;
\item
$\mathbb{E}[Y1_A] = \mathbb{E}[X1_A]$ for any $A\in\mathcal{G}$;
\end{itemize}
then we can assert that $Y = \mathbb{E}[X\mid\mathcal{G}]$ a.s., and $Y$ is called a \emph{version} of $\mathbb{E}[X\mid\mathcal{G}]$.
\end{remark}
Conditional expectation has many of the same properties that ordinary expectation does:
\begin{theorem}[Properties of Conditional Expectation]
Let $X$ be a random variable defined on $(\Omega,\mathcal{F},\mathbb{P})$, and $\mathcal{G}$ is a sub $\sigma$-algebra of $\mathcal{F}$, then the following holds:
\begin{enumerate}
\item
$\mathbb{E}[\mathbb{E}[X\mid\mathcal{G}]] = \mathbb{E}[X]$
\item
If $X$ is $\mathcal{G}$-measurable, then 
$\mathbb{E}[X\mid\mathcal{G}] = X$ a.s..
\item
(Linearity)
For any $a_1,a_2\in\mathbb{R}$, 
\[
\mathbb{E}[a_1X_1 + a_2X_2\mid \mathcal{G}]
=
a_1\mathbb{E}[X_1\mid \mathcal{G}]
+
a_2\mathbb{E}[X_2\mid \mathcal{G}],\quad\mbox{a.s.}
\]
\item
(Positivity)
If $X\ge0$, then $\mathbb{E}[X\mid\mathcal{G}]\ge0$.
\item
(Jensen Inequality)
If $\phi:\mathbb{R}\to\mathbb{R}$ is a convex function, 
then 
\[
\mathbb{E}[\phi(X)\mid\mathcal{G}]\ge\phi(\mathbb{E}[X\mid\mathcal{G}]).
\]
\item
(Tower Property)
Let $\mathcal{H}$ be a sub $\sigma$-algebra of $\mathcal{G}$.
Then 
\[
\mathbb{E}\bigg[\mathbb{E}[X\mid\mathcal{G}]\mid\mathcal{H}\bigg]
=
\mathbb{E}[X\mid\mathcal{H}],\quad\mbox{a.s.}
\]
\item
(Conditional Independence)
Suppose that $\mathcal{H}$ is a $\sigma$-algebra independent of $\sigma(\sigma(X),\mathcal{G})$,
then 
\[
\mathbb{E}\bigg[
X\bigg|
\sigma(\mathcal{G},\mathcal{H})
\bigg]
=
\mathbb{E}[X\mid\mathcal{G}].
\]
In particular, $\mathbb{E}[X\mid\mathcal{H}] = \mathbb{E}[X]$ if $\mathcal{H}$ is independent of $X$.
\end{enumerate}
\end{theorem}
\begin{proof}
\begin{enumerate}
\item
Recall the definition of $\mathbb{E}[X\mid\mathcal{G}]$ and take $A=\Omega$,
\[
\mathbb{E}[\mathbb{E}[X\mid\mathcal{G}]]=
\mathbb{E}[\mathbb{E}[X\mid\mathcal{G}]1_{\Omega}] = \mathbb{E}[X].
\]
\item
It suffices to verify that $X$ satisfies 1) and 2) in Definition~\ref{Def:2:4}, and the result holds by the uniqueness of conditional expectation.
\item
Again, verify the RHS satisfies 1) and 2) in Definition~\ref{Def:2:4}, and the result holds by the uniqueness of conditional expectation.
\item
For fixed $\omega\in\Omega$,
\[
\mathbb{E}[X\mid\mathcal{G}](\omega)
=
\mathbb{E}[\mathbb{E}[X\mid\mathcal{G}]1_{\{\omega\}}]=
\mathbb{E}[X1_{\{\omega\}}]=X(\omega)\ge0.
\]
\item
\begin{itemize}
\item
Assume that we can construct a collection of affine functions $\mathcal{L}=\{L(x):~L(x)=ax+b\}$, such that $\phi(x)=\sup_{L\in\mathcal{L}}L(x)$.
As a result, for any $L\in\mathcal{L}$,
\begin{align*}
\mathbb{E}[\phi(X)\mid\mathcal{G}]
&\ge \mathbb{E}[L(X)\mid\mathcal{G}]=L(\mathbb{E}[X\mid\mathcal{G}])
\end{align*}
Taking the supermum over all $L\in\mathcal{L}$, the desired result holds.
\item
Here we give an explicit construction of $\mathcal{L}$:
\[
\mathcal{L} = \{x\mapsto \phi(x_0)+g\trans(x-x_0)\mid x_0\in\text{dom}(\phi), g\in\partial\phi(x_0)\}
\]
Note that $L(x)\le\phi(x)$ for any $L\in\mathcal{L}$ since the subgradient inequality holds for convex functions. Reversely, $[\phi(x_0)+g\trans(x-x_0)]\mid_{x=x_0}=\phi(x_0)$.
Therefore, $\phi(x)=\sup_{L\in\mathcal{L}}L(x)$.
\end{itemize}
\item
It suffices to show that $\mathbb{E}[X\mid\mathcal{H}]$ is a version of $\mathbb{E}\bigg[
\mathbb{E}[X\mid\mathcal{G}]\mid\mathcal{H}
\bigg]$.
The key is to show that for all $A\in\mathcal{H}$, 
\[
\mathbb{E}[\mathbb{E}[X\mid\mathcal{H}]1_A] 
=\mathbb{E}[\mathbb{E}[X\mid\mathcal{G}]1_A].
\]
Verify that both sides equal to $\mathbb{E}[X 1_A]$.
%
%Notice that the LHS is $\mathcal{H}$-measurable, and for any $A\in\mathcal{H}$,
%similar as in property (a), argue that 
%\[
%\mathbb{E}
%\bigg[
%\mathbb{E}[X\mid\mathcal{G}]1_{A}
%\bigg]=\mathbb{E}
%[X1_{A}].
%\]
%Moreover, $\mathbb{E}[\mathbb{E}[X\mid\mathcal{H}]1_A]=\mathbb{E}
%[X1_{A}]$, which implies
%\[
%\mathbb{E}
%\bigg[
%\mathbb{E}[X\mid\mathcal{G}]1_{A}
%\bigg]=\mathbb{E}[\mathbb{E}[X\mid\mathcal{H}]1_A]
%\]
%By the uniqueness of conditional expectation, the desired result holds.
\item
It suffices to show that $\mathbb{E}[X\mid\mathcal{G}]$ is a version of $\mathbb{E}\bigg[
X\bigg|
\sigma(\mathcal{G},\mathcal{H})
\bigg]$, i.e., for any $A\in\sigma(\mathcal{G},\mathcal{H})$,
\[
\mathbb{E}[X1_A]
=
\mathbb{E}
\bigg[
\mathbb{E}[X\mid\mathcal{G}]1_A
\bigg].
\]
\end{enumerate}
\end{proof}

\subsection{Tips about Probability Theory}
Suppose that $\{E_n\}$ is a seuqence of events.
We aim to define the limit of this sequence. 
A key issue is that two sets may loss orders. For instance, it is possible that neither $A\subseteq B$ nor $B\subseteq A$.
Therefore, based on a seuqence of events, we first define monotone increasing/decreasing sequence of events as follows:
\[
\overline{E}_m = \bigcup_{n\ge m}E_n,\qquad
\underline{E}_m = \bigcap_{n\ge m}E_n
\]
Then $\{\overline{E}_m\}$ and $\{\underline{E}_m\}$ are montone decreasing/increasing, and it is easy to define their limits:
\[
\limsup_{n\to\infty}E_n=\cap_{m}\overline{E}_m,\qquad
\liminf_{n\to\infty}E_n=\cup_m\underline{E}_m.
\]
According to this definition, we have:
\begin{align*}
\limsup_{n\to\infty}E_n&\triangleq 
\{\omega:~~\omega\in E_n\text{ for infinitely many $n$}\}\\
\liminf_{n\to\infty}E_n&\triangleq
\{\omega:~~\omega\in E_n\text{ for all large enough $n$}\}
\end{align*}


\begin{theorem}[Borel-Cantelli Lemma]\label{The:BC}
If $\{E_n\}$ is a sequence of events satisfying 
$\sum_{n=1}^\infty\mathbb{P}(E_n)<\infty$, 
then
\[
\mathbb{P}\left(\limsup_{n\to\infty} E_n\right) = 0.
\]
\end{theorem}

\begin{proof}
Define $\overline{E}_m$ as above, and thus $\limsup_{n\to\infty}E_n=\cap_m\overline{E}_m$.
As a result, for any $m$,
\[
\mathbb{P}\left(\limsup_{n\to\infty} E_n\right)
=
\mathbb{P}(\cap_m\overline{E}_m)\le \mathbb{P}(\overline{E}_m)\le\sum_{n=m}^\infty\mathbb{P}(E_n).
\]
%
%
%
%
%Define the set 
%\[
%G_m = \bigcup_{n\ge m}E_n.
%\]
%Then $\{G_m\}$ is a decreasing sequence of events, i.e., $G_m\supseteq G_{m+1}\supseteq\cdots$.
%Also, define
%\[
%G = \bigcap_m E_m:=\limsup E_n.
%\]
%Then $\mathbb{P}(G_m)$ is decreasing and $\lim_{m\to\infty}\mathbb{P}(G_m) = \mathbb{P}(G)$.
%It implies that for any $m$,
%\[
%\mathbb{P}(G)\le \mathbb{P}(G_m)
%\le\sum_{n=m}^\infty\mathbb{P}(E_n)
%\]
Because of the condition $\sum_{n=1}^\infty\mathbb{P}(E_n)<\infty$, as $m\to\infty$,
\[
\sum_{n=m}^\infty\mathbb{P}(E_n)\to0\implies
\mathbb{P}\left(\limsup_{n\to\infty} E_n\right)=0.
\]
\end{proof}

\subsection{Reviewing on Real Analysis}

\begin{theorem}[Monotone Convergence Theorem]
Let $\{f_n\}$ be a sequence of non-negative measurable functions on $(S,\Sigma,\mu)$ satisfying
\begin{itemize}
\item
$f_1(x)\le f_2(x)\le\cdots$ for almost all $x\in S$;
\item
$f_n(x)\to f(x)$ for almost all $x\in S$, for some measurable function $f$.
\end{itemize}
Then 
\[
\int_Sf\diff\mu = \lim_{n\to\infty}\int_Sf_n\diff\mu.
\]
\end{theorem}
The proof for the monotone convergence theorem~(MCT) can be found in the website
\begin{quotation}
Daniel Wong, Jie Wang. (2019) Lecture Notes
for MAT3006: Real Analysis, Lecture 21. Available at
the link 

https://walterbabyrudin.github.io/information/Updates/Updates.html
\end{quotation}
We can apply MCT to show the Fatou's lemma, in which the required condition is weaker:
\begin{theorem}[Fatou's Lemma]
Suppose that $\{f_n\}$ is a sequence of measurable, non-negative functions. Then
\[
\int_S \liminf_{n\to\infty}f_n\diff\mu
\le
\liminf_{n\to\infty}\int_S f_n\diff\mu.
\]
\end{theorem}

\begin{proof}
Define the function $g_n = \inf_{k\ge n}f_k$.
Then $\{g_n\}$ is a non-decreasing sequence of non-negative functions.
Then 
\begin{align*}
\int \liminf_{n\to\infty}f_n\diff\mu
&=
\int\lim_{n\to\infty}g_n\diff\mu=
\lim_{n\to\infty}\int g_n\diff\mu\\
&=\liminf_{n\to\infty}\int g_n\diff\mu\\
&\le\liminf_{n\to\infty} \int f_n\diff\mu
\end{align*}
where the second equality is by MCT, and the last equality is because that $g_n\le f_n,\forall n$.
\end{proof}

\begin{example}
In general the integral of the limit-inf on a sequence of functions is smaller.
For instance, consider a sequence of functions on $\mathbb{R}$:
\[
f_n(x)=\left\{
\begin{aligned}
\bm 1_{[0,1/2]}(x),&\qquad\text{when $n$ is odd}\\
\bm1_{[1/2,1]}(x),&\qquad\text{when $n$ is even}
\end{aligned}
\right.
\]
Then 
\[
\liminf_{n\to\infty}f_n=1_{\{1/2\}}\implies
\int\liminf_{n\to\infty}f_n\diff m=0,
\]
while $\int_{[0,1]}f_n\diff m=1/2$ for each $n$.
\end{example}

\begin{remark}
We also have the reversed fatou's lemma, saying that in general the integral of the limit-sup on a sequence of functions is bigger:
\[
\int_S\limsup_{n\to\infty}f_n\diff\mu \ge \limsup_{n\to\infty}\int_Sf_n\diff\mu.
\]
\end{remark}

\begin{theorem}[Dominated Convergence Theorem]
Let $\{f_n\}$ be a sequence of measurable functions on $(S,\Sigma,\mu)$ satisfying
\begin{enumerate}
\item
$f_n$ is dominated by an integrable function $g$, i.e., 
\[
|f_n(x)|\le g(x)
\]
for almost all $x\in S$, with $\int_S|g|\diff\mu<\infty$.
\item
$f_n$ converges to $f$ almost everywhere for some measurable function $f$.
\end{enumerate}
Then $f$ is integrable and $f_n\to f$ in $L^1$, i.e., 
$
\lim_{n\to\infty}\int_S|f_n-f|\diff\mu=0,
$
which implies that 
\[
\int_Sf\diff\mu = \lim_{n\to\infty}f_n\diff\mu.
\]
\end{theorem}

\begin{proof}
\begin{itemize}
\item
The integrability of $f$ is because that $|f|\le g$ a.e.;
\item
The $L^1$-convergence for $f_n$ is by the reversed fatou's lemma:
\[
\limsup\int|f_n-f|\diff m\le \int\limsup|f_n-f|\diff m=0.
\]
\item
The remaining part is by applying Fatou's lemma on a sequence of functions $\{g+f_n\}$ and $\{g-f_n\}$. The details are in the reference
\begin{quotation}
Daniel Wong, Jie Wang. (2019) Lecture Notes
for MAT3006: Real Analysis, Lecture 23. Available at
the link 

https://walterbabyrudin.github.io/information/Updates/Updates.html
\end{quotation}
\end{itemize}
\end{proof}















