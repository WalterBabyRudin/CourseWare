
\chapter{Week12}

\section{Tuesday}\index{week7_Tuesday_lecture}
\subsection{Ito's Formula}
In this lecture, we will study the Ito's formula, which is very useful for evaluating Ito's integrals.
The following elementary identity will be used frequently:
\begin{equation}\label{Eq:12:1}
X[t_{j+1}]^2 - X[t_{j}]^2 = (X[t_{j+1}] - X[t_{j}])^2 + 2X[t_{j}](X[t_{j+1}] - X[t_{j}])
\end{equation}
Now we show how to compute the Ito integral for $X_t^2$ based on this identity:
\begin{example}
Suppose that $\{X_t\}_{t\ge0}\in\mathcal{U}_c^2$, and $\Pi=\{t_0<t_1<\cdots<t\}$ is a partition on the interval $[0,t]$. Then take the summation on \eqref{Eq:12:1} both sides yields
\[
X_t^2 - X_0^2 = 2\sum_jX[t_{j}](X[t_{j+1}] - X[t_{j}]) + \sum_j(X[t_{j+1}] - X[t_{j}])^2.
\]
Taking the limit both sides as $\|\Pi\|\to0$, we have
\[
X_t^2-X_0^2 = 2\int_0^tX_s\diff X_s + \langle X\rangle_t.
\]
It is the Ito's formula applied to $X_t^2$.
\end{example}

\begin{theorem}\label{The:12:1}
Let $\{B_t\}_{t\ge0}$ be a Brownian motion and $f\in\mathcal{C}^2(\mathbb{R})$.
Then
\[
f(B_t) = f(0) + \int_0^tf'(B_u)\diff B_u + \frac{1}{2}\int_0^tf''(B_u)\diff u,
\]
almost surely for any $t\ge0$.
\end{theorem}
\begin{proof}
Let $\Pi=\{t_0<t_1<\cdots<t_n=t\}$ be a partition on $[0,t]$, then $f(B_t)$ admits the expansion
\begin{equation}\label{Eq:12:2}
f(B_t) = f(0) + \sum_{j=0}^{n-1}[f(B[t_{j+1}])- f(B[t_j])].
\end{equation}
By Taylor expansion on the RHS, there exists $\theta_{t_j}(\omega)\in[B[t_j](\omega), B[t_{j+1}](\omega)]$ such that
\begin{equation}\label{Eq:12:3}
f(B[t_{j+1}])- f(B[t_j]) = f'(B[t_j])(B[t_{j+1}] - B[t_{j}]) + \frac{1}{2}f''(\theta_{t_j})\cdot(B[t_{j+1}] - B[t_{j}])^2.
\end{equation}
Substituting (\eqref{Eq:12:3}) into (\ref{Eq:12:2}) yields
\[
f(B_t) - f(0) = \sum_jf'(B[t_j])(B[t_{j+1}] - B[t_{j}]) + \frac{1}{2}\sum_jf''(\theta_{t_j})\cdot(B[t_{j+1}] - B[t_{j}])^2.
\]
As $\|\Pi\|\to0$, the term $\sum_jf'(B[t_j])(B[t_{j+1}] - B[t_{j}])\to \int_0^tf'(B_u)\diff B_u$ in probability.
Then we begin to compute the limit for the second term on RHS.
\begin{itemize}
\item
We first consider the case where $\theta_{t_j}(\omega)\equiv B_{t_j}$.
\begin{align*}
&\mathbb{E}\left[
\sum_jf''(B[t_{j}])\cdot(B[t_{j+1}] - B[t_{j}])^2
-
\sum_jf''(B[t_{j}])\cdot(t_{j+1}-t_{j})
\right]^2\\
=&
\mathbb{E}\left[
\sum_jf''(B[t_{j}])\cdot\left((B[t_{j+1}] - B[t_{j}])^2
-(t_{j+1}-t_{j})\right)
\right]^2\\
=&\sum_j\mathbb{E}\left[
f''(B[t_{j}])\cdot\left((B[t_{j+1}] - B[t_{j}])^2
-(t_{j+1}-t_{j})\right)
\right]^2\\
&\quad + 2\sum_{j<k}\mathbb{E}\bigg[
f''(B[t_{j}])\cdot\left((B[t_{j+1}] - B[t_{j}])^2
-(t_{j+1}-t_{j})\right)
\bigg]\\
&\qquad\qquad \times\bigg[
f''(B[t_{k}])\cdot\left((B[t_{k+1}] - B[t_{k}])^2
-(t_{k+1}-t_{k})\right)
\bigg]
\end{align*}
The second term vanishes because for any $j<k$, 
\begin{align*}
&\mathbb{E}\bigg[
f''(B[t_{j}])\cdot\left((B[t_{j+1}] - B[t_{j}])^2
-(t_{j+1}-t_{j})\right)
\bigg]\\&\qquad\qquad\times\bigg[
f''(B[t_{k}])\cdot\left((B[t_{k+1}] - B[t_{k}])^2
-(t_{k+1}-t_{k})\right)
\bigg]\\
=&\mathbb{E}\mathbb{E}\bigg[
f''(B[t_{j}])\cdot\left((B[t_{j+1}] - B[t_{j}])^2
-(t_{j+1}-t_{j})\right)\bigg|\mathcal{F}_{t_j}
\bigg]\\&\qquad\qquad\times\bigg[
f''(B[t_{k}])\cdot\left((B[t_{k+1}] - B[t_{k}])^2
-(t_{k+1}-t_{k})\right)
\bigg]\\
=&\mathbb{E}f''(B[t_{j}])\mathbb{E}\bigg[
\left((B[t_{j+1}] - B[t_{j}])^2
-(t_{j+1}-t_{j})\right)\bigg|\mathcal{F}_{t_j}
\bigg]\\&\qquad\qquad\times\bigg[
f''(B[t_{k}])\cdot\left((B[t_{k+1}] - B[t_{k}])^2
-(t_{k+1}-t_{k})\right)
\bigg]=0,
\end{align*}
where the first equality is by tower property and the last equality is because $\{B_t^2-t\}_{t\ge0}$ is a martingale.
To simplify the first term, observe that 
\begin{align*}
&\mathbb{E}\left[
f''(B[t_{j}])\cdot\left((B[t_{j+1}] - B[t_{j}])^2
-(t_{j+1}-t_{j})\right)
\right]^2\\
=&\mathbb{E}\left[
f''(B[t_{j}])\cdot\mathbb{E}\left((B[t_{j+1}] - B[t_{j}])^2
-(t_{j+1}-t_{j})\middle|\mathcal{F}_{t_j}\right)
\right]^2\\
=&\mathbb{E}\left[
f''(B[t_{j}])\cdot2(t_{j+1} - t_j)^2
\right]^2=2(t_{j+1} - t_j)^2\mathbb{E}\left[
f''(B[t_{j}])\right]^2
\end{align*}
It follows that
\begin{align*}
&\sum_j\mathbb{E}\left[
f''(B[t_{j}])\cdot\left((B[t_{j+1}] - B[t_{j}])^2
-(t_{j+1}-t_{j})\right)
\right]^2\\
=&2\sum_j(t_{j+1} - t_j)^2\mathbb{E}\left[
f''(B[t_{j}])\right]^2\\
\le&2\|\Pi\|\sum_j(t_{j+1} - t_j)\mathbb{E}\left[
f''(B[t_{j}])\right]^2=2t\|\Pi\|\mathbb{E}\left[
f''(B[t_{j}])\right]^2.
\end{align*}
\begin{itemize}
\item[(a)]
Suppose that $f''$ is bounded, i.e., $|f''(x)|\le K$ for any $x\in\mathbb{R}$, then as $\|\Pi\|\to0$,
\[
\sum_j\mathbb{E}\left[
f''(B[t_{j}])\cdot\left((B[t_{j+1}] - B[t_{j}])^2
-(t_{j+1}-t_{j})\right)
\right]^2\le 2t\|\Pi\|K^2\to0,
\]
which means that $\sum_jf''(B[t_{j}])\cdot(B[t_{j+1}] - B[t_{j}])^2
\xrightarrow{L^2}
\sum_jf''(B[t_{j}])\cdot(t_{j+1}-t_{j})$ and therefore in probability.
Also, by the Lebesgue integration knowledge, since $f''$ is continuous and $B_t$ is almost surely continuous, the term $\sum_jf''(B[t_{j}])\cdot(t_{j+1}-t_{j})$ converges to $\int_0^tf''(B_u)\diff u$ almost surely.
As a result, \[\sum_jf''(B[t_{j}])\cdot(B[t_{j+1}] - B[t_{j}])^2
\xrightarrow{P}\int_0^tf''(B_u)\diff u.\]
\item[(b)]
Now consider the case where $f''$ is unbounded, then for any $K$, we wish to show that 
\[
\sum_jf''(B[t_{j}])\cdot\left[(B[t_{j+1}] - B[t_{j}])^2
-(t_{j+1}-t_{j})
\right]
\xrightarrow{P}
0.
\]
We apply the truncation technique so that
\begin{align*}
&f''(B[t_{j}])\cdot\left[(B[t_{j+1}] - B[t_{j}])^2
-(t_{j+1}-t_{j})
\right]\\=&
f''(B[t_{j}])1\{|f''(B[t_{j}])|\le K\}\cdot\left[(B[t_{j+1}] - B[t_{j}])^2
-(t_{j+1}-t_{j})
\right]\\
&\quad  + f''(B[t_{j}])1\{|f''(B[t_{j}])|> K\}\cdot\left[(B[t_{j+1}] - B[t_{j}])^2
-(t_{j+1}-t_{j})
\right]
\end{align*}
By applying the result in the part (a), as $\|\Pi\|\to0$, 
\[
\sum_jf''(B[t_{j}])1\{|f''(B[t_{j}])|\le K\}\cdot\left[(B[t_{j+1}] - B[t_{j}])^2
-(t_{j+1}-t_{j})
\right]\to0.
\]
The remainder term can be upper bounded as the following:
\begin{align*}
&\bigg|
\sum_jf''(B[t_{j}])1\{|f''(B[t_{j}])|> K\}\cdot\left[(B[t_{j+1}] - B[t_{j}])^2
-(t_{j+1}-t_{j})
\right]
\bigg|\\
\le&\max_j\bigg|
f''(B[t_{j}])1\{|f''(B[t_{j}])|> K\}
\bigg|\cdot\sum_j\bigg|
(B[t_{j+1}] - B[t_{j}])^2
-(t_{j+1}-t_{j})
\bigg|\\
\le&\max_{0\le u\le t}\bigg|
f''(B[t_{j}])1\{|f''(B[t_{j}])|> K\}
\bigg|\cdot\left[
\sum_j(B[t_{j+1}] - B[t_{j}])^2+t
\right]
\end{align*}
Applying Theorem~\ref{The:7:8} gives $\sum_j(B[t_{j+1}] - B[t_{j}])^2\to t$ in probability.
By the continuity of $f''$ and $B_t$ on the interval $[0,t]$, 
\[
\max_{0\le u\le t}\bigg|
f''(B[t_{j}])1\{|f''(B[t_{j}])|> K\}
\bigg|\to0,\qquad\text{as }K\to\infty.
\]
Hence,
\[
\sum_jf''(B[t_{j}])1\{|f''(B[t_{j}])|\le K\}\left[(B[t_{j+1}] - B[t_{j}])^2
-(t_{j+1}-t_{j})
\right]\xrightarrow{P}0,\quad\text{as }\|\Pi\|\to0.
\]
As a result, 
\[\sum_jf''(B[t_{j}])\cdot(B[t_{j+1}] - B[t_{j}])^2\xrightarrow{P}
\sum_jf''(B[t_{j}])\cdot(t_{j+1}-t_{j})\xrightarrow{P}\int_0^tf''(B_u)\diff u.
\]
\end{itemize}
\item
Now consider the case where $\theta_{t_j}\ne B_{t_j}$. 
It remains to show that $\sum_jf''(\theta[t_j])(B[t_{j+1}] - B[t_j])^2$ converges to $\int_0^tf''(B_u)\diff u$ in probability:
\begin{align*}
&\bigg|
\sum_jf''(\theta[t_j])(B[t_{j+1}] - B[t_j])^2-\sum_jf''(B[t_j])(B[t_{j+1}] - B[t_j])^2
\bigg|\\
=&
\bigg|
\sum_j[f''(\theta[t_j]) - f''(B[t_j])](B[t_{j+1}] - B[t_j])^2
\bigg|\\
\le&\max_{j}\left|
f''(\theta[t_j]) - f''(B[t_j])
\right|\cdot \sum_j(B[t_{j+1}] - B[t_j])^2
\end{align*}
By Theorem~\ref{The:7:8} we can see that $\sum_j(B[t_{j+1}] - B[t_j])^2\xrightarrow{P}t$.
For almost all $\omega\in\Omega$, by the continuity and thus uniform continuity of $f''(B_t)$, 
\[
\max_{j}\left|
f''(\theta[t_j]) - f''(B[t_j])
\right|\to0.
\]
Hence, 
\[
\sum_jf''(\theta[t_j])(B[t_{j+1}] - B[t_j])^2\xrightarrow{P}\sum_jf''(B[t_j])(B[t_{j+1}] - B[t_j])^2
\xrightarrow{P}\int_0^tf''(B_u)\diff u.
\]

\end{itemize}
\end{proof}


\begin{definition}[Ito Processes]
An Ito process is a stochastic process $X_{\bullet}$ on $(\Omega,\mathcal{F},\mathbb{F},\mathbb{P})$ of the form 
\begin{equation}\label{Eq:12:4}
X_t = X_0 + \int_0^t\mu[u]\diff u + \int_0^t\sigma[u]\diff B[u],\quad 0\le t\le T,
\end{equation}
where $X_0$ is a $\mathcal{F}_0$-measurable random variable; $\{\mu_t\}_{t\ge0}$ is $\{\mathcal{F}_t\}$-adapted so that 
\[
\mathbb{P}\bigg[
\int_0^t|\mu_u|\diff u<\infty,\quad\text{for all }t\ge0
\bigg]=1;
\]
and $\{\sigma_t\}_{t\ge0}$ is $\{\mathcal{F}_t\}$-adapted so that 
\[
\mathbb{P}\bigg[
\int_0^t\sigma_u^2\diff u<\infty,\quad\text{for all }t\ge0
\bigg]=1.
\]
Sometimes we also write (\eqref{Eq:12:4}) as the stochastic differential form
\[
\diff X_t = \mu_t\diff t + \sigma_t\diff B_t.
\]
\end{definition}

Now we are ready to show the main result of this lecture:
\begin{theorem}[Ito Formula for $1$-dimension]
Let $\{X_t\}_{t\ge0}$ be an Ito process with stochastic differential 
\[
\diff X_t= \mu_t\diff t + \sigma_t\diff B_t.
\]
Let $f\in\mathcal{C}^2(\mathbb{R})$, then $f(X_t)$ is again an Ito process, with
the following equality holds almost surely:
\[
f(X_t) = f(X_0) + \int_0^tf'(X_u)\diff X_u + \frac{1}{2}\int_0^tf''(X_u)(\diff X_u)^2.
\]
In particular, $(\diff X_t)^2$ can be computed according to the rules
\[
(\diff t)^2=(\diff t)(\diff B_t)=(\diff B_t)(\diff t)=0,\quad (\diff B_t)^2=\diff t.
\]
So we further have the representation
\[
f(X_t) = f(X_0) + \int_0^tf'(X_u)\mu_u\diff u + \frac{1}{2}\int_0^tf''(X_u)\sigma_u^2\diff u 
+\int_0^tf'(X_u)\sigma_u\diff B_u,\quad\forall t\ge0
\]
\end{theorem}

\begin{remark}
The Ito process $\{f(X_t)\}_{t\ge0}$ has the stochastic differential form
\[
\diff f(X_t)=\bigg[
f'(X_t)\mu_t+\frac{1}{2}f''(X_t)\sigma_t^2
\bigg]\diff t + f'(X_t)\sigma_t\diff B_t.
\]
\end{remark}

The proof follows the similar idea as stated in Theorem~\ref{The:12:1}. Here we only provide a sketrch of the proof:
\begin{proof}[Outline of Proof]
Let $\Pi=\{t_0<t_1<\cdots<t_n=t\}$ be a partition on $[0,t]$, then $f(X_t)$ admits the expansion
\begin{equation}\label{Eq:12:5}
\begin{aligned}
f(X_t)&=f(X_0) + \sum_j\bigg[
f(X[t_{j+1}]) - f(X[t_j])
\bigg]\\
&=f(X_0) + \sum_j\bigg[
f'(X[t_j])(X[t_{j+1}] - X[t_j]) + \frac{1}{2}f''(\theta[t_j])(X[t_{j+1}] - X[t_j])^2
\bigg]
\end{aligned}
\end{equation}
\begin{enumerate}
\item[(a)]
First consider the case where $\{\mu_t\}$ and $\{\sigma_t\}$ are simple processes, then
\[
X[t_{j+1}] - X[t_j] = \mu[t_j](t_{j+1} - t_{j}) + \sigma[t_j](B[t_{j+1}] - B[t_{j}]) 
\]
Substituting this into (\eqref{Eq:12:5}) gives
\begin{align*}
f(X_t)&=f(X_0) + \sum_jf'(X[t_j])\mu[t_j](t_{j+1} - t_{j}) + \sum_jf'(X[t_j])\sigma[t_j](B[t_{j+1}] - B[t_{j}])\\
&\quad + \frac{1}{2}\sum_jf''(\theta[t_j])(X[t_{j+1}] - X[t_j])^2.
\end{align*}
Assume that $\theta[t_j]=X[t_j]$, then we first show $\sum_jf''(X[t_j])(X[t_{j+1}] - X[t_j])^2\xrightarrow{P}\sum_jf''(X[t_j])\sigma[t_j]^2(t_{j+1}-t_j)$, and then show that 
\[
\sum_jf''(X[t_j])\sigma[t_j]^2(t_{j+1}-t_j)\xrightarrow{a.s.}\frac{1}{2}\int_0^tf''(X_u)\sigma_u^2\diff u.
\]
For the case where $\theta[t_j]\ne X[t_j]$, we will show that
\[
\bigg|
\sum_jf''(\theta[t_j])(X[t_{j+1}] - X[t_j])^2
-
\sum_jf''(X[t_j])(X[t_{j+1}] - X[t_j])^2
\bigg|\xrightarrow{a.s.}0.
\]
\item[(b)]
For general processes $\{\mu_t\}_{t\ge0}$ and $\{\sigma_t\}_{t\ge0}$, we will use the approximation of simple processes.
\end{enumerate}
\end{proof}

\begin{example}
Let $\{X_t\}_{t\ge0}$ be the drifted Brownian motion:
\[
X_t=\mu t + B_t,\quad \mu\in\mathbb{R}.
\]
Then we apply the Ito's formula to compute the stochastic differential form of $X^2$:
\begin{itemize}
\item
Take $f(x)=x^2$ and $f'(x)=2x, f''(x)=2$. Hence,
\begin{align*}
\diff f(X_t)&=\diff X_t^2\\
&=(2\mu X_t + 1)\diff t + 2X_t\diff B_t.
\end{align*}
\end{itemize}
\end{example}

The Ito's formula can also be generalized into multiple processes:
\begin{theorem}[Ito's Formula]\label{The:12:3}
Let $\{X_t^{(1)}\}_{t\ge0},\ldots, \{X_t^{(d)}\}_{t\ge0}$ be continuous semi-martingales, and $f\in\mathcal{C}^2(\mathbb{R})$, then
\begin{align*}
f(X_t^{(1)},\ldots, X_t^{(d)})&=f(X_0^{(1)},\ldots, X_0^{(d)})\\
&\quad + \sum_{j=1}^d\int_0^t\frac{\partial f}{\partial X_j}(X_u^{(1)},\ldots, X_u^{(d)})\diff X_u^{(j)}\\
&\quad + \frac{1}{2}\sum_{j,k=1}^d\int_0^t\frac{\partial^2 f}{\partial X_j\partial X_k}(X_u^{(1)},\ldots, X_u^{(d)})\diff\inp{X^{(j)}}{X^{(k)}}_u
\end{align*}
In particular, $\diff\inp{B_i}{B_j}=1(i=j)\diff t$ and $\diff\inp{B_i}{t}=\diff\inp{t}{B_i}=0$.
\end{theorem}
Recall that in discrete case, any real-valued process is a semi-martingale, while it is not true for continuous case. We define the semi-martingale for continuous case as the following:
\begin{definition}[Semi-martingale]
We say $X_{\bullet}$ is a semi-martingale if it admits the decomposition
\[
X_t = A_t + M_t,
\]
where $M_{\bullet}$ is a continuous local martingale, and $A_{\bullet}$ is an adapted process of finite variations:
\[
|A|(t)\equiv\sup_{\delta>0, t_0=0}\sup_{t_n - t_{n-1}\ge\delta}
\sum_{n=1}^\infty1(t_n\le t)|A[t_n] - A[t_{n-1}]|<\infty,\quad\forall t\ge0.
\]
\end{definition}

\begin{remark}
Suppose that $X^{(j)}$ admits the decomposition $X^{(j)}_t=A_t^{(j)} + M_t^{(j)}$, then
\begin{align*}
\sum_{j=1}^d\int_0^t\frac{\partial f}{\partial X_j}(X_u^{(1)},\ldots, X_u^{(d)})\diff X_u^{(j)}
&=\sum_{j=1}^d\int_0^t\frac{\partial f}{\partial X_j}(X_u^{(1)},\ldots, X_u^{(d)})\diff M_u^{(j)}\\
&\quad + \sum_{j=1}^d\int_0^t\frac{\partial f}{\partial X_j}(X_u^{(1)},\ldots, X_u^{(d)})\diff A_u^{(j)}
\end{align*}
Moreover, we can show that $\inp{X^{(j)}}{X^{(k)}} = \inp{M^{(j)}}{M^{(k)}}$.
Then the process $\{f(X_t^{(1)},\ldots, X_t^{(d)})\}_{t\ge0}$ also admits the semi-martingale decomposition:
\[
f(X_t^{(1)},\ldots, X_t^{(d)}) = A_t^f + M_t^f,
\]
where
\begin{align*}
M_t^f &= f(X_0^{(1)},\ldots, X_0^{(d)}) + \sum_{j=1}^d\int_0^t\frac{\partial f}{\partial X_j}(X_u^{(1)},\ldots, X_u^{(d)})\diff M_u^{(j)},\\
A_t^f&=\sum_{j=1}^d\int_0^t\frac{\partial f}{\partial X_j}(X_u^{(1)},\ldots, X_u^{(d)})\diff A_u^{(j)}\\
&\quad + \frac{1}{2}\sum_{j,k=1}^d\int_0^t\frac{\partial^2 f}{\partial X_j\partial X_k}(X_u^{(1)},\ldots, X_u^{(d)})\diff\inp{M^{(j)}}{M^{(k)}}_u
\end{align*}
\end{remark}

\begin{example}
We can also apply Theorem~\ref{The:12:3} to obtain an ``integration by parts'' formula.
Suppose that $X_{\bullet}$ and $Y_{\bullet}$ are semi-martingales, then by direct computation with $f(X_t,Y_t)= X_tY_t$,
\[
X_tY_t - X_0Y_0 = \int_0^tX_s\diff Y_s + \int_0^tY_s\diff X_s + \inp{X}{Y}_t
\]
Suppose that $X_t$ and $Y_t$ admits the semi-martingale decomposition
\[
X_t = M_t + A_t,\quad Y_t = N_t + W_t,
\]
then $XY$ also admits the semi-martingale decomposition
\[
X_tY_t = \left[X_0Y_0 + \int_0^tX_s\diff N_s + \int_0^tY_s\diff M_s\right]
+\left[
\int_0^tX_s\diff W_s + \int_0^tY_s\diff A_s + \inp{M}{N}_t
\right]
\]
\end{example}

\subsection{Applications of Ito's Formula}

Here we presents some examples for how to use Ito's formula.

\begin{example}
We aim to sovle the following stochastic differential equation with $X_t=M_t+A_t$ being a continuous semi-martingale;
\begin{equation}
\diff Z_t = Z_t\diff X_t,\quad  Z_0=1.
\end{equation}
This equation is called the stochastic exponential of $X$,
which can be re-written as the integral form:
\begin{equation}\label{Eq:12:7}
Z_t = 1 + \int_0^tZ_u\diff X_u,
\end{equation}
where the integration refers to the Ito's integral.
We guess the solution should be $Z_t=\exp(X_t + V_t)$, with $V_t$ to be determined.
Applying Ito's formula on $Z_t$ with $f(\zeta)=\exp(\zeta)$ $(\zeta=X+V)$ gives
\begin{align*}
Z_t &= 1 + \int_0^tZ_u\diff(X_u+V_u) +\frac{1}{2}\int_0^tZ_u\diff\langle X+V\rangle_u\\
&=1 + \int_0^tZ_u\diff X_u +  \int_0^tZ_u\diff V_u +\frac{1}{2}\int_0^tZ_u\diff\langle M+V\rangle_u
\end{align*}
In order to satisfy (\eqref{Eq:12:7}), we take $V_t=-\frac{1}{2}\langle M\rangle_t$. As a result, $\exp\left(
X_t - \frac{1}{2}\langle M\rangle_t
\right)$ is the solution to (\eqref{Eq:12:7}), which is called the stochastic exponential.
\end{example}


\begin{example}[Levy's characterization of Brownian Motion]
Levy states that the quadratic variational process of a continuous local martingale will characterize the Brownian motion uniquely:
\begin{theorem}[Levy's Theorem]
Consider the filtered probability space $(\Omega, \mathcal{F},\mathbb{F},\mathbb{P})$ satisfying the usual condition. Let $\{M_t\}_{t\ge0}$ be a stochastic process on this filtered probability space, with $M_0=0$ almost surely.
Then the process $\{M_t\}_{t\ge0}$ is a Brownian motion if and only if:
\begin{enumerate}
\item
$\{M_t\}_{t\ge0}$ is a continuous local martingale with respect to $\mathbb{F}$;
\item
The quadratic variation $\langle M\rangle_t=t$ almost surely.
\end{enumerate}
\end{theorem}
\begin{proof}
In order to show the sufficiency, we construct $Z_t^{(\zeta)}=\exp\left(
\zeta M_t - \frac{\zeta^2}{2}t
\right)$.
By Ito's formula on $Z_t^{(\zeta)}$ and $f(x)=e^x$, for $0\le s<t$, we have
\[
\diff Z_t^{(\zeta)}=\zeta Z_t^{(\zeta)}\diff M_t.
\]
It indicates that $Z_t^{(\zeta)}$ is a martingale, by the uniqueness of semi-martingale decomposition. Re-arranging the term $\mathbb{E}[Z_t^{(\zeta)}\mid\mathcal{F}_s]=Z_s^{(\zeta)}$ yields
\[
\mathbb{E}[\exp(\zeta(M_t - M_s))\mid\mathcal{F}_s] = \exp\left(
\frac{\zeta^2}{2}(t-s)
\right),
\]
which, together with the uniqueness of characterization function, explains that $M_t$ is a Brownian motion.
\end{proof}

\end{example}

A third application indicates that when a continuous local martingale is applied time-change, it is a Brownian motion:
\begin{theorem}
Consider a stochastic process $\{M_t\}_{t\ge0}$ on $(\Omega, \mathcal{F},\mathbb{F},\mathbb{P})$,
where $\{M_t\}_{t\ge0}$ is a continuous local martingale with initial value equal to $0$ and $\langle M\rangle_\infty=\infty$.
Let $\tau_t=\inf\{u:~\langle M\rangle_u>t\}$.
Then for $\forall t\ge0$, $\tau_t$ is a stopping time, and $B_t\equiv M_{\tau_t}$ is a $\mathcal{F}_{\tau_t}$-Brownian motion with $M_t=B_{\langle M\rangle_t}$.
\end{theorem}
\begin{proof}[Proof Outline]
We call the increasing sequence of stopping time $\{\tau_t\}_{t\ge0}$ the time-change.
Since $\langle M\rangle_\infty=\infty$ almost surely, we can assert that each $\tau_t$ is finite almost surely.
By the continuity of $\{M_t\}_{t\ge0}$, we have $\langle M\rangle_{\tau_t}=t$.
Applying the optional samping theorem on $\{M_{u\land\tau_t}\}_{u\ge0}$ gives
\[
\mathbb{E}[M_{\tau_t}\mid\mathcal{F}_{\tau_u}]=M_{\tau_u},
\]
which implies that $B_t$ is a $\mathcal{F}_{\tau_t}$-local martingale. 
%%%
Applying the optional samping theorem on $\{M_{u\land\tau_t}^2 - \langle M\rangle_{u\land\tau_t}\}_{u\ge0}$ gives
\[
\mathbb{E}[
M_{\tau_t}^2 -  \langle M\rangle_{\tau_t}\mid\mathcal{F}_{\tau_u}
]
=
M_{\tau_u}^2 -  \langle M\rangle_{\tau_u}.
\]
This implies that $\{B_t^2-t\}$ is a $\mathcal{F}_{\tau_t}$-local martingale. 
%% why it is necessary?
By Levy's characterization, together with the continuity of $B_t$, we conclude that $\{B_t\}_{t\ge0}$ is a $\mathcal{F}_{\tau_t}$-Brownian motion.
\end{proof}

\begin{example}
Consider a probability space $(\Omega, \mathcal{F},\mathbb{F},\mathbb{P})$. Let $T>0$ and $Q$ be a probability measure on $(\Omega,\mathcal{F}_{T})$ that is absolutely continuous with respect to $\mathbb{P}$. Denote $\zeta = \left.\frac{\diff Q}{\diff\mathbb{P}}\right|_{\mathcal{F}_{T}}$.
Then for any $\mathcal{F}_T$-measurable bounded random variable $X$, we have
\[
\int_{\Omega}X(\omega)\diff Q(\omega) = \int_{\Omega} X(\omega)\zeta(\omega)\diff\mathbb{P}(\omega).
\]
Or equivalently, $\mathbb{E}_{X\sim Q}[X] = \mathbb{E}_{X\sim \mathbb{P}}[\zeta X]$.

Conversely, let $T>0$ and $\{Z_t\}_{t\ge0}$ be a continuous martingale on $(\Omega, \mathcal{F},\mathbb{F},\mathbb{P})$ with $Z_0\equiv1$ and $Z_t(\omega)>0,\forall (t,\omega)\in[0,T]\times\Omega$.
Define a probability measure $Q$ on $(\Omega,\mathcal{F}_T,\mathbb{P})$ to be $Q(A)=\mathbb{E}[Z_T1_A], A\in\mathcal{F}_T$. Then we have $\left.\frac{\diff Q}{\diff\mathbb{P}}\right|_{\mathcal{F}_{T}}=Z_T$. Because $\{Z_t\}_{t\ge0}$ is a martingale, then for any $t\le T$, we have 
\[
\left.\frac{\diff Q}{\diff\mathbb{P}}\right|_{\mathcal{F}_{t}}=Z_t.
\]
\end{example}

\begin{theorem}[Girsanov]
Let $M_{\bullet}$ be a continuous local martingale on $(\Omega, \mathcal{F},\mathbb{F},\mathbb{P})$ with $0\le t\le T$, and $Z_{\bullet}$ be a continuous martingale, strictly positive, and initial value equal to $1$.
Then the process $X_t\equiv M_t - \int_0^tZ_u\diff\inp{M}{Z}_u$ is a continuous local martingale on $(\Omega, \mathcal{F},\mathbb{F},\mathbb{Q})$.
\end{theorem}
\begin{proof}[Proof Outline]
By the technique of localization, we assume that $M,Z,\frac{1}{Z}$ are all bounded.
It suffices to show that $X$ is a martingale w.r.t. the probability measure $Q$, i.e., for any $0\le u<t\le T$, $\mathbb{E}^{\mathbb{Q}}[1_A(X_t - X_u)]=0,\forall A\in\mathcal{F}_u$.
By definition,
\[
\mathbb{E}^{\mathbb{Q}}[1_A(X_t - X_u)]=\mathbb{E}^{\mathbb{P}}[1_A(Z_tX_t - Z_uX_u)].
\]
It remains to show that $\{Z_tX_t\}$ is a martingale w.r.t. the probability measure $\mathbb{P}$.
Applying the integration by parts gives
\begin{align*}
Z_tX_t&=Z_0X_0+\int_0^tZ_s\diff X_s + \int_0^tX_s\diff Z_s + \inp{X}{Z}_t\\
&=Z_0X_0+\int_0^tZ_s\left(\diff M_s - \frac{1}{Z_s}\diff \inp{M}{Z}_s\right) + \int_0^tX_s\diff Z_s + \inp{X}{Z}_t\\
&=Z_0X_0 + \int_0^tZ_s\diff M_s + \int_0^tX_s\diff Z_s.
\end{align*}
By the uniqueness of semi-martingale decomposition, we can see that $\{Z_tX_t\}$ is a martingale.
\end{proof}

\paragraph{Motivation for Martingale Representation Theorem}
The last application we will discuss is the martingale representation theorem. 
Previously we noticed that if the stochastic process $\{f_t\}_{t\ge0}\in\mathcal{L}^2$, i.e., is square-integrable, then the process $X_t=X_0+\int_0^tf_s\diff B_s$ is always a martingale w.r.t. $\mathcal{F}_t$.\footnote{This can be shown either by directly applying definition or by the uniqueness of semi-martingale decomposition.}
The martingale representation theorem states that the converse is also true: any $\mathcal{F}_t$-martingale can be represented as an Ito's integral.
\begin{theorem}[Martingale Representation Theorem]\label{The:12:7}
Let $B_{\bullet}$ be a standard Brownian motion on a complete probability space $(\Omega, \mathcal{F},\mathbb{P})$, and $\mathbb{F}^0\equiv\{\mathcal{F}_t^0\}_{t\ge0}$ be the natural filtration generated by $B_{\bullet}$, with $\mathcal{F}_\infty^0=\sigma\left(
\cup_{t\ge0}\mathcal{F}_t^0
\right)$. Let $\mathcal{F}_\infty,\mathcal{F}$ be the completion of $\mathcal{F}_\infty^0,\mathcal{F}^0$, respectively, and denote $\mathbb{F}\equiv \{\mathcal{F}_t\}_{t\ge0}$.
Then consider a square-integrable martingale $M_{\bullet}$ on the filtered probability space $(\Omega, \mathcal{F},\mathbb{F},\mathbb{P})$.
There exists a stocastic process $\{f_t\}_{t\ge0}\in\mathcal{L}^2$ so that
\[
M_t = \mathbb{E}[M_0] + \int_0^tf_u\diff B_u.
\]
\end{theorem}
This theorem relies on an important auxiliary result:
\begin{quotation}
Let $\zeta\in L^2(\Omega,\mathcal{F}_T,\mathbb{P})$, then there exists $f_{\bullet}\in\mathcal{L}^2$ so that 
\[
\zeta = \mathbb{E}[\zeta] + \int_0^Tf_t\diff B_t.
\]
\end{quotation}
\begin{proof}[Proof for Theorem~\ref{The:12:7}]
Assume the claim is true, then for the martingale $M_{\bullet}$ with $M_T\in L^2(\Omega,\mathcal{F}_T,\mathbb{P})$, there exists $\{f_t\}\in\mathcal{L}^2$ so that 
\[
M_T = \mathbb{E}[M_T]  + \int_0^Tf_t\diff B_t = \mathbb{E}[M_0] + \int_0^Tf_t\diff B_t.
\]
Hence for any $t\in[0,T]$, by the martingability of $M_{\bullet}$, 
\[
M_t=\mathbb{E}[M_T\mid\mathcal{F}_t]=\mathbb{E}[M_0] + \int_0^tf_u\diff B_u.
\]
\end{proof}

This key auxiliary result can be shown by applying Ito's formula:
\begin{proof}[Proof on the Claim]
Let $T>0$ and define the stochastic exponential martingale for any $f\in\mathcal{L}^2([0,T])$:
\[
\mathcal{E}[f]_t=\exp\left\{
\int_0^tf(u)\diff B_u - \frac{1}{2}\int_0^tf^2(u)\diff u
\right\},\quad t\in[0,T].
\]
Then we can show that $\ell\equiv\text{span}\{\mathcal{E}[f]_t:~f\in\mathcal{L}^2([0,T])\}$ is dense in the space $L^2(\Omega,\mathcal{F}_T,\mathbb{P})$.
It remains to verify this claim when $\zeta=\mathbb{E}[f]_t$ for some $f\in \mathcal{L}^2([0,T])$.
Directly applying Ito's formula gives
\[
\diff\mathcal{E}[f]_t=f(t)\mathcal{E}[f]_t\diff B_t\implies
\mathbb{E}[f]_t = 1 + \int_0^Tf(t)\mathcal{E}[f]_t\diff B_t.
\]
\end{proof}






