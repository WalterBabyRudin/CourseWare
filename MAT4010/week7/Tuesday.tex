
\chapter{Week7}

\section{Tuesday}\index{week7_Tuesday_lecture}

\subsection{Reflection Principle}

Consider a Brownian motion $\{B_t\}_{t\ge0}$ on a probability space $(\Omega,\mathcal{F},\mathbb{P})$.
Let the filteration $\{\mathcal{F}_t\}_{t\ge0}$ be the natural filteration, i.e., 
$\mathcal{F}_t=\sigma(B_u:~u\le t)$. Suppose that $a>0$, define
\[
T_a\triangleq \inf\{t\ge0:~B_t=a\}.
\]
Then $T_a$ is the first time that Brownian motion hits level $a$.
By convention, $\inf\emptyset=+\infty$.

\begin{theorem}
The hitting time is finite almost surely:
\[
\mathbb{P}(T_a<\infty)=1.
\]
\end{theorem}
\begin{proof}
Based on $B_t$, define new stochastic process $\{Z_t^\theta\}$ with
\[
Z_t^\theta = \exp\left(
\theta B_t - \frac{\theta^2t}{2}
\right),\quad t\ge0.
\]
As a result,
\begin{itemize}
\item
Since $\mathbb{E}[e^{\theta X}] = e^{\theta^2\sigma^2/2}$ for $X\sim \mathcal{N}(0,\sigma^2)$,
\[
\mathbb{E}[|Z_t^\theta|] = \mathbb{E}\exp\left(
\theta B_t - \frac{\theta^2t}{2}
\right)=1,\forall t
\]
\item
For any $0\le u<t$, we have
\begin{align*}
\mathbb{E}[Z_t^\theta\mid\mathcal{F}_u]
&=
\mathbb{E}\left[
\exp\left(
\theta B_t - \frac{\theta^2}{t}
\right)
\middle|\mathcal{F}_u
\right]\\
&=
\mathbb{E}\left[
\exp\left(
\theta (B_t-B_u) - \frac{\theta^2(t-u)}{2}
\right)
\exp\left(
\theta B_u - \frac{\theta^2u}{2}
\right)
\middle|\mathcal{F}_u
\right]\\
&=
\exp\left(
\theta B_u - \frac{\theta^2u}{2}
\right)\cdot \mathbb{E}\left[
\exp\left(
\theta (B_t-B_u) - \frac{\theta^2(t-u)}{2}
\right)\middle|\mathcal{F}_u\right]\\
&=
\exp\left(
\theta B_u - \frac{\theta^2u}{2}
\right)\cdot \mathbb{E}\left[
\exp\left(
\theta B_{t-u} - \frac{\theta^2(t-u)}{2}
\right)\right]=Z_u^\theta
\end{align*}
\end{itemize}
Therefore, $\{Z_t^\theta\}$ is a martingale w.r.t. $\{\mathcal{F}_t\}_{t\ge0}$.
Now we compute $\lim_{t\to\infty}\mathbb{E}[
Z_{T_a}^\theta1\{T_a<\infty\}
]$ as the following:
\begin{itemize}
\item
Since i) $Z_{t\land T_a}^\theta1\{T_a<\infty\}\xrightarrow{a.s.} Z_{T_a}^\theta1\{T_a<\infty\}$ as $t\to\infty$,
ii) $|Z_{t\land T_a}^\theta|\le e^{\theta a}$ for any $t$,
by bounded convergence theorem,
\[
\lim_{t\to\infty}\mathbb{E}
[Z_{t\land T_a}^\theta1\{T_a<\infty\}]
=
\mathbb{E}
[
Z_{T_a}^\theta1\{T_a<\infty\}
]
\]
\item
Since $Z_{t\land T_a}^\theta1\{T_a=\infty\}=Z_{t}^\theta1\{T_a=\infty\}$ for any $t$, and
\[Z_{t}^\theta1\{T_a=\infty\}\le e^{\theta a-\theta^2t/2}\to0,\] by bounded convergence theorem,
\[
\lim_{t\to\infty}\mathbb{E}
[
Z_{t\land T_a}^\theta1\{T_a=\infty\}
]
=
\lim_{t\to\infty}\mathbb{E}
[
Z_{t}^\theta1\{T_a=\infty\}
]
=0.
\]


\end{itemize}
On the other hand, since the stopped process $\{Z_{t\land T_a}^\theta\}$ is a martingale,
\[
\mathbb{E}
[
Z_{t\land T_a}^\theta
]
=
\mathbb{E}
[
Z_{0}^\theta
]
=1,\quad\forall t.
\]
It follows that 
\begin{align*}
1&=
\lim_{t\to\infty}\mathbb{E}
[Z_{t\land T_a}^\theta1\{T_a<\infty\}]
+
\lim_{t\to\infty}\mathbb{E}
[
Z_{t\land T_a}^\theta1\{T_a=\infty\}
]\\
&=\mathbb{E}
[
Z_{T_a}^\theta1\{T_a<\infty\}
]
=
\mathbb{E}
[
e^{\theta a - \theta^2T_a/2}1\{T_a<\infty\}
].
\end{align*}
Therefore, $\mathbb{E}[e^{- \theta^2T_a/2}1\{T_a<\infty\}]=e^{-\theta a}$. Since $e^{- \theta^2T_a/2}1\{T_a<\infty\}$ is increasing, by monotone convergence theorem,
\[
1=\lim_{\theta\to0}e^{-\theta a}=\lim_{\theta\to0}\mathbb{E}[e^{- \theta^2T_a/2}1\{T_a<\infty\}]
=\mathbb{E}[1\{T_a<\infty\}]=\mathbb{P}[1\{T_a<\infty\}].
\]


\end{proof}

\begin{remark}
The stationary property shows that $\{B_{t+s}-B_s\}_{t\ge0}$ is also a Brownian motion for any $s>0$.
Given that $T_a$ is a stopping time and finite a.s., we can assert that $\{B_{t+T_a}-B_{T_a}\}_{t\ge0}$ is also a Brownian motion, independent of $\mathcal{F}_{T_a}$.
\end{remark}

\begin{theorem}\label{The:7:2}
Let $\{B_{t}\}_{t\ge0}$ be a standard Brownian motion.
Let $M_t=\sup_{0\le u\le t}B_u$ be the running maximal of Brownian motion.
For any $a\ge0$,
\[
\mathbb{P}(M_t\ge a) = 2\mathbb{P}(B_t\ge a) = \frac{2}{\sqrt{2\pi t}}\int_a^{\infty}e^{-\frac{x^2}{2t}}\diff x.
\]
\end{theorem}
\begin{proof}
Firstly simplify $\mathbb{P}(B_t\ge a)$ as the following:
\begin{align*}
\mathbb{P}(B_t\ge a)&=
\mathbb{P}(B_t\ge a, M_t\ge a) + \mathbb{P}(B_t\ge a, M_t< a)\\
&=\mathbb{P}(B_t\ge a, M_t\ge a)\\
&=\mathbb{P}(B_t\ge a\mid M_t\ge a)\mathbb{P}(M_t\ge a)\\
&=\mathbb{P}(B_t\ge a\mid T_a\le t)\mathbb{P}(M_t\ge a)\\
&=\mathbb{P}(B_t- B_{T_a}\ge 0\mid T_a\le t)\mathbb{P}(M_t\ge a)
\end{align*}
Since conditioned on $\{T_a\le t\}$, $B_t-B_{T_a}$ is normally distributed with mean $0$, we imply 
$\mathbb{P}(B_t- B_{T_a}\ge 0\mid T_a\le t)=\frac{1}{2}$.
Therefore,
\[
\mathbb{P}(B_t\ge a)=\frac{1}{2}\mathbb{P}(M_t\ge a)=\frac{1}{\sqrt{2\pi t}}\int_a^{\infty}e^{-\frac{x^2}{2t}}\diff x.
\]

\end{proof}

\subsection{Distributions of Brownian Motion}
\begin{theorem}
The joint distribution of the Brownian motion and the running maximal of Brownian motion is:
\[
\mathbb{P}(M_t\ge a, B_t\le a-y) = \mathbb{P}(B_t\ge a+y) = \frac{1}{\sqrt{2\pi t}}\int_{a+y}^\infty e^{-\frac{x^2}{2t}}\diff x.
\]
\end{theorem}
\begin{proof}
Simplify $\mathbb{P}(B_t\ge a+y)$ as the following:
\begin{align*}
\mathbb{P}(B_t\ge a+y)&= \mathbb{P}(B_t\ge a+y\mid M_t\ge a)\mathbb{P}(M_t\ge a)\\
&=\mathbb{P}(B_t\ge a+y\mid T_a\le t)\mathbb{P}(M_t\ge a)\\
&=\mathbb{P}(B_t-B_{T_a}\ge y\mid T_a\le t)\mathbb{P}(M_t\ge a)\\
&=\mathbb{P}(B_t-B_{T_a}\le -y\mid T_a\le t)\mathbb{P}(M_t\ge a)\\
&=\mathbb{P}(B_t\le a-y\mid M_t\ge a)\mathbb{P}(M_t\ge a).
\end{align*}
\end{proof}

\begin{theorem}
For any $\lambda>0$,
\[
\mathbb{E}[e^{-\lambda T_a}] = e^{-\sqrt{2\lambda}a}.
\]
\end{theorem}

\begin{proof}
By Theorem~\ref{The:7:2}, the density of $T_a$ is 
\[
f_{T_a}(t) = \frac{a}{t\sqrt{t}}\frac{1}{\sqrt{2\pi}}\exp\left(-\frac{a^2}{2t}\right).
\]
Computing the integral $\int_0^{\infty}f_{T_a}(t)e^{-\lambda t}\diff t$ gives the desired result.
\end{proof}
\begin{proof}[Another Quick Proof]
Since $\mathbb{P}(T_a<\infty)=1$, substituting $\theta$ with $\sqrt{2\lambda}$
for $\mathbb{E}[e^{- \theta^2T_a/2}1\{T_a<\infty\}]=e^{-\theta a}$ gives the desired result.
\end{proof}

\begin{theorem}
Consider the Brownian motion with drift:
\[
X_t\triangleq \mu t + \sigma B_t,
\]
where $\mu\ne0,\sigma>0$.
\begin{enumerate}
\item
For any $0\le s<t$, $X_t-X_s$ is normally distributed with mean $\mu(t-s)$ and variance $\sigma^2(t-s)$. Independent incremental property also holds.
\item
Time reversal: $\lim_{t\to\infty}\frac{X_t}{t}=\mu$.
\item
For $\mu<0$, define $M_\infty=\sup_{t\ge0}X_t$ as the running maximal of drifted Brownian motion to infinity. Then $M_\infty$ is exponentially distributed with parameter $\frac{2|\mu|}{\sigma^2}$:
\[
\mathbb{P}(M_\infty>y)=\exp\left(-\frac{2|\mu|}{\sigma^2}y\right),\quad y\ge0.
\]
\end{enumerate}
\end{theorem}
The first two parts are trivial, and we give a proof for the last part:
\begin{proof}
Choose some $\theta\ne0$ and define the random process $\{V_t^\theta\}_{t\ge0}$ such that
\[
V_t^\theta=\exp\left( \theta X_t - \mu\theta t - \frac{\sigma^2\theta^2}{2}t\right).
\]
It follows that $\{V_t^\theta\}_{t\ge0}$ is a martingale:
\begin{itemize}
\item
Since $X_t\sim\mathcal{N}(\mu t,\sigma^2 t)$, $\mathbb{E}[|V_t^\theta|]=1$;
\item
For any $0\le u<t$, we have
\begin{align*}
\mathbb{E}[V_t^\theta\mid\mathcal{F}_u]
&=
\mathbb{E}\left[\exp\left( \theta X_t - \mu\theta t - \frac{\sigma^2\theta^2}{2}t\right)\middle|\mathcal{F}_u\right]\\
&=
\mathbb{E}\left[
\exp\left( \theta(X_t-X_u) - (\mu\theta+\frac{\sigma^2\theta^2}{2})(t -u)\right)
e^{(
\theta X_u - (\mu\theta+\frac{\sigma^2\theta^2}{2})u
)}
\middle|\mathcal{F}_u
\right]\\
&=
e^{(
\theta X_u - (\mu\theta+\frac{\sigma^2\theta^2}{2})u
)}
\mathbb{E}\left[
\exp\left( \theta(X_t-X_u) - (\mu\theta+\frac{\sigma^2\theta^2}{2})(t -u)\right)\right]\\
&=V_u^\theta.
\end{align*}
\end{itemize}
For $a<0<b$, define $T_{a,b}=\inf\{t\ge0:~X_t=a\text{ or }X_t=b\}$.
Define $\theta = -\frac{2\mu}{\sigma^2}$ such that $\mu\theta+\frac{1}{2}\sigma^2\theta^2=0$.
Considering that $T_{a,b}\le T_a$ with $T_a$ being finite a.s., we imply $V_{t\land T_{a,b}}^\theta\xrightarrow{a.s.}V_{T_{a,b}}^\theta$. Moreover, $|V_{t\land T_{a,b}}^\theta|\le \max(e^{-\theta a},e^{\theta b})$. By dominated convergence theorem,
\[
\mathbb{E}[V_{T_{a,b}}^\theta]=\lim_{t\to\infty}
\mathbb{E}[V_{t\land T_{a,b}}^\theta]
=
\lim_{t\to\infty}\mathbb{E}[V_0^\theta]=1.
\]
Therefore,
\begin{align*}
1&=\mathbb{E}[V_{T_{a,b}}^\theta]=\mathbb{E}[V_{T_{a,b}}^\theta1\{T_{a,b}=a\}]
+\mathbb{E}[V_{T_{a,b}}^\theta1\{T_{a,b}=b\}]\\
&=e^{\theta a}\mathbb{P}(T_{a,b}=a) + e^{\theta b}\mathbb{P}(T_{a,b}=b).
\end{align*}
Together with the fact that $\mathbb{P}(T_{a,b}=a)+\mathbb{P}(T_{a,b}=b)=1$, we assert that 
\[\mathbb{P}(T_{a,b}=a)=\frac{e^{\theta b}-1}{e^{\theta b} - e^{\theta a}}.\]
Now define the set $A_a = \{T_{a,b}=T_a\}$, then the sequence $\{A_a\}$ is monotone in $a$:
\[
a_2<a_1<0\implies A_{a_2}\subseteq A_{a_1}.
\]
Define the set 
\begin{align*}
\Lambda&\triangleq\{M_\infty<b\}=\{\omega\in\Omega:~\text{$X_t(\omega)$ never hits level $b$}\}=\bigcap_{a<0,a\in\mathbb{Q}}A_a.
\end{align*}
Thus
\[
\mathbb{P}\bigg(
\bigcap_{a<0,a\in\mathbb{Q}}A_a
\bigg)=\lim_{a\to\infty}\mathbb{P}(A_a)=\lim_{a\to\infty}\mathbb{P}(T_{a,b}=a)=1-e^{-\theta b}.
\]
Then we conclude that 
\[
\mathbb{P}(M_\infty<b)=\mathbb{P}(\Lambda)=1-e^{-2|\mu|/\sigma^2}b.
\]
In particular, we imply $\mathbb{P}(M_\infty<\infty)=1$.
\end{proof}

\section{Thursday}
\subsection{Unbounded Variation of Brownian Motion}

\begin{definition}[Partition]
Consider a closed interval $[a,b]$. A sequence
\[
a=t_0<t_1<\cdots<t_n=b
\]
is called a partition of $[a,b]$, denoted as $\Pi = \Pi(t_0,t_1,\ldots,t_n)$.
\end{definition}

\begin{definition}[Total Variation]
Let $f$ be a continuous function on $[a,b]$:
\[
f:~[a,b]\mapsto\mathbb{R}.
\]
Then the total variation of $f$ is defined as 
\[
TV(f)[a,b] = \sup_{\Pi}\sum_k|f(t_k) - f(t_{k-1})|
\]
where the supremum is taken over all the possible partitions $\Pi$ on the interval $[a,b]$.
Since $\sum_k|f(t_k) - f(t_{k-1})|$ is increasing as the partition being smaller, 
\[
TV(f)[a,b] = \lim_{\|\Pi\|\to0}\sum_k|f(t_k) - f(t_{k-1})|
\]
with $\|\Pi\|=\max_k|t_k - t_{k-1}|$.
\end{definition}

\begin{remark}
The real analysis shows that a bounded variation function, i.e., the function whose total variation is bounded, is differentiable almost everywhere.
Then if the function is nowhere differentiable, it is not of bounded variation.
\end{remark}

\begin{theorem}\label{The:7:6}
Brownian motion is not of bounded variation almost surely, i.e.,
\[
\mathbb{P}(\{\omega\in\Omega:~
TV(B_{\cdot}(\omega))[0,t] = \infty
\})=1,\qquad \forall t>0.
\]
\end{theorem}
This result is based on the fact that the Brownian motion is nowhere differentiable alomst surely.
\begin{theorem}
Brownian motion is nowhere differentiable almost surely. In particular,
\[
\mathbb{P}\left(\left\{\omega\in\Omega:~
\limsup\limits_{n\to\infty}\bigg|
\frac{B_{t+h}(\omega) - B_t(\omega)}{h}
\bigg|
=\infty,
t\in[0,\infty)
\right\}\right)=1
\]
\end{theorem}
\begin{remark}
If a stochastic process $\{A_t\}_{t\ge0}$ has bounded variation, then the integral $\int_a^bf(t)\diff A_t(\omega)$ can be defined in Riemann integraion sense $\omega$-wisely.
However, Brownian motion is not of the bounded variaiton.
The stochastic integral $\int_a^bf(t)\diff B_t(\omega)$ shall be defined in a new manner.
\end{remark}
\begin{proof}
Choose any $T>0$ and $M>0$, define the set
\[
A^{(M)}\triangleq \bigg\{
\omega\in\Omega:~\exists t\in[0,T]\text{ such that }\limsup\limits_{h\to0}\left|
\frac{B_{t+h}(\omega) - B_t(\omega)}{h}
\right|\le M
\bigg\}.
\]
It suffices to show that $\mathbb{P}(A^{(M)})=0$.
If $\omega\in A^{(M)}$, there exists $t\in[0,T]$ and $n_0$ such that when $n\ge n_0$,
\[
\left|
\frac{B_{u}(\omega) - B_t(\omega)}{u-t}
\right|\le 2M,\qquad \forall u\in(t-2/n,t+2/n).
\]
Decompose $A^{(M)}$ into many smaller sets.
Define the set
\begin{equation}\label{Eq:7:1}
A^{(M)}_n\triangleq \bigg\{
\omega\in\Omega:~\exists t\in[0,T]\text{ such that }
|B_u(\omega) - B_t(\omega)|\le 2M|u-t|,\quad
\forall u\in(t-2/n,t+2/n)
\bigg\}. 
\end{equation}
Then i) $A^{(M)}\subseteq \cup_nA_n^{(M)}$, and ii) $\{A^{(M)}_n\}$ is monotone: $A_n\subseteq A_{n+1}$.

Suppose that $\omega\in A_n$ with $t$ having such a property in (\ref{Eq:7:1}).
Let $k=\sup\{j\in\mathbb{Z}:~j/n\le t\}$, which means $k$ is close enough to $t$. Define $Y_k$ as the maximal of three independent increments:
\[
Y_k = \max\left\{
\left|
B_{(k+2)/n} - B_{(k+1)/n} 
\right|,
\left|
B_{(k+1)/n} - B_{k/n} 
\right|,
\left|
B_{k/n} - B_{(k-1)/n} 
\right|
\right\}.
\]
We can show that $Y_k(\omega)\le 6M/n, \forall \omega\in A_n^{(M)}$ as follows.
Firstly, 
\begin{align*}
\left|
B_{(k+2)/n}(\omega) - B_{(k+1)/n}(\omega)
\right|
&\le
\left|
B_{(k+2)/n}(\omega) - B_{t}(\omega)
\right|
+
\left|
B_{t}(\omega) - B_{(k+1)/n}(\omega)
\right|\\
&\le 2M\left|
\frac{k+2}{n}-t
\right|
+
2M\left|
\frac{k+1}{n}-t
\right|\\&\le 2M\cdot\frac{2}{n}+2M\cdot\frac{1}{n}=\frac{6M}{n}
\end{align*}
where the last inequality is because that $k/n\le t<(k+1)/n$.
Following the similar technique, we can show that 
\[\left|
B_{(k+1)/n}(\omega) - B_{k/n}(\omega) 
\right|,
\left|
B_{k/n}(\omega) - B_{(k-1)/n}(\omega)
\right|\le \frac{6M}{n}
\implies
Y_k(\omega)\le \frac{6M}{n}
.
\]
Now define the new set based on the consequence of the claim about $A_n^{(M)}$:
\[
E_n^{(M)}\triangleq \left\{
\omega\in\Omega:~\exists j\in[1,T_n]\cap\mathbb{Z}\text{ such that }Y_j(\omega)\le\frac{6M}{n}
\right\}.
\]
with 
\[
Y_j = \max\left\{
\left|
B_{(j+2)/n} - B_{(j+1)/n} 
\right|,
\left|
B_{(j+1)/n} - B_{j/n} 
\right|,
\left|
B_{j/n} - B_{(j-1)/n} 
\right|
\right\}.
\]
Directly $A_n^{(M)}\subseteq E_n^{(M)}$ for each $n$.
Now we begin to upper bound $\mathbb{P}(E_n^{(M)})$:
\begin{subequations}
\begin{align}
\mathbb{P}(E_n^{(M)})&\le \sum_{1\le j\le T_n}\mathbb{P}(Y_j\le\frac{6M}{n} )\nonumber\\
&\le T_n\mathbb{P}\bigg(
\max\left\{
\left|
B_{(j+2)/n} - B_{(j+1)/n} 
\right|,
\left|
B_{(j+1)/n} - B_{j/n} 
\right|,
\left|
B_{j/n} - B_{(j-1)/n} 
\right|
\right\}\le\frac{6M}{n}
\bigg)\nonumber\\
&=T_n\cdot \prod_{i=j-1:j+1}\mathbb{P}\bigg(
\left|
B_{(i+1)/n} - B_{(i)/n} 
\right|\le\frac{6M}{n}
\bigg)\label{Eq:7:2:a}\\
&=T_n\cdot\left[
\mathbb{P}\bigg(
|B_{1/n}|\le\frac{6M}{n}
\bigg)
\right]^3,\label{Eq:7:2:b}
\end{align}
where (\ref{Eq:7:2:a}) is because of the independent increments of Brownian motion,
and (\ref{Eq:7:2:b}) is because of its stationary increment property.
In particular,
\begin{align}
\mathbb{P}\bigg(
|B_{1/n}|\le\frac{6M}{n}
\bigg)&=\mathbb{P}\bigg(
-\frac{6M}{n}\le B_{1/n}\le \frac{6M}{n}
\bigg)\nonumber\\
&=\mathbb{P}\bigg(
-\frac{6M}{\sqrt{n}}\le B_{1}\le \frac{6M}{\sqrt{n}}
\bigg)\label{Eq:7:2:c}\\
&=\frac{1}{\sqrt{2\pi}}\int_{-\frac{6M}{\sqrt{n}}}^{\frac{6M}{\sqrt{n}}}e^{-x^2/2}\diff x\nonumber\\
&\le \frac{2}{\sqrt{2\pi}}\frac{6M}{\sqrt{n}}\label{Eq:7:2:d}
\end{align}
where (\ref{Eq:7:2:c}) is by the scaling property, and (\ref{Eq:7:2:d}) is by upper bounding $e^{-x^2/2}\le 1$.
\end{subequations}
It follows that 
\[
\mathbb{P}(E_n^{(M)})\le T_n(\frac{2}{\sqrt{2\pi}}\frac{6M}{\sqrt{n}})^3\to0.
\]
Since $A_n^{(M)}\subseteq E_n^{(M)}$, $\mathbb{P}(A_n^{(M)})\to0$.
Since $A^{(M)}\subseteq\cup_nA_n^{(M)}$,
\[
\mathbb{P}(A^{(M)})\le \mathbb{P}(\cup_nA_n^{(M)})=\lim_{n\to\infty}\mathbb{P}(A_n^{(M)})=0.
\]
The proof is complete.
\end{proof}

\chapter{Weak 8}
\section{Thursday}
\subsection{Quadratic Variation}

\begin{definition}[Quadratic Variation]
Consider a partition $\Pi$ on the interval $[0,T]$.
The \emph{quadratic variation} of $\{B_t(\omega)\}_{0\le t\le T}$ over the partition $\Pi$ is defined as
\[
Q(\Pi,\omega) = \sum_k|B_{t_k}(\omega) - B_{t_{k-1}}(\omega)|^2.
\]
\end{definition}

\begin{theorem}\label{The:7:8}
Consider a sequence of partitions $\{\Pi^{(n)}\}$ with $\|\Pi^{(n)}\|\to0$, 
where $\|\Pi\|\triangleq \max_k|t_k - t_{k-1}|$.
Then 
\[
\lim_{n\to\infty}\mathbb{E}\left[
(Q(\Pi^{(n)}) - T)^2
\right]=0.
\]
\end{theorem}

\begin{proof}
Given a partition $\Pi$ on the interval $[0,T]$, define
\[
\theta_k = (B_{t_k} - B_{t_{k-1}})^2 - (t_k - t_{k-1})\implies
Q(\Pi)=T+\sum_k\theta_k
\]
We claim that $\theta_j,\theta_k$ are uncorrelated for $j\ne k$:
\begin{align*}
\mathbb{E}[\theta_j\theta_k]&=
\mathbb{E}\bigg[
\left((B_{t_j} - B_{t_{j-1}})^2 - (t_j - t_{j-1})\right)
\left((B_{t_k} - B_{t_{k-1}})^2 - (t_k - t_{k-1})\right)
\bigg]\\
&=
\mathbb{E}\left[
(B_{t_j} - B_{t_{j-1}})^2(B_{t_k} - B_{t_{k-1}})^2
\right]
-
(t_j - t_{j-1})\mathbb{E}\left[(B_{t_k} - B_{t_{k-1}})^2\right]\\
&\quad -(t_k - t_{k-1})\mathbb{E}\left[(B_{t_j} - B_{t_{j-1}})^2\right] + (t_j - t_{j-1})(t_k - t_{k-1})\\
&=
\mathbb{E}\left[
(B_{t_j} - B_{t_{j-1}})^2\right]\mathbb{E}\left[(B_{t_k} - B_{t_{k-1}})^2
\right]
-
(t_j - t_{j-1})\mathbb{E}\left[(B_{t_k} - B_{t_{k-1}})^2\right]\\
&\quad -(t_k - t_{k-1})\mathbb{E}\left[(B_{t_j} - B_{t_{j-1}})^2\right] + (t_j - t_{j-1})(t_k - t_{k-1})\\
&=(t_{j}-t_{j-1})(t_{k}-t_{k-1})-(t_j - t_{j-1})(t_{k}-t_{k-1})\\&\quad-(t_k - t_{k-1})(t_{j}-t_{j-1})+ (t_j - t_{j-1})(t_k - t_{k-1})=0.
\end{align*}
Then we begin to simplify $\mathbb{E}[
(Q(\Pi^{(n)}) - T)^2
]$:
\begin{align*}
\mathbb{E}\left[
(Q(\Pi^{(n)}) - T)^2
\right]
&=\mathbb{E}\left[
(\sum_k\theta_k)^2
\right]\\
&=\sum_k\mathbb{E}\left[
\theta_k^2
\right]+\sum_{j\ne k}\mathbb{E}\left[
\theta_j\theta_k
\right]\\
&=\sum_k\mathbb{E}\left[
\left((B_{t_k} - B_{t_{k-1}})^2 - (t_k - t_{k-1})\right)^2
\right]\\
&=
\sum_k\mathbb{E}\left[
(B_{t_k} - B_{t_{k-1}})^4
\right]\\&\quad-2\sum_k(t_k - t_{k-1})\mathbb{E}\left[
(B_{t_k} - B_{t_{k-1}})^2
\right]+\sum_k(t_k - t_{k-1})^2\\
&=3\sum_k(t_k - t_{k-1})^2 - 2\sum_k(t_k - t_{k-1})^2 + \sum_k(t_k - t_{k-1})^2\\
&=2\sum_k(t_k - t_{k-1})^2\le 2\|\Pi^{(n)}\|\sum_k(t_k - t_{k-1})\\
&=2T\cdot \|\Pi^{(n)}\|\to0.
\end{align*}
\end{proof}

\begin{remark}
The Theorem~\ref{The:7:8} shows that the quadratic variation of Brownian motion on the interval 
$[0,T]$ converges to $T$ in $L^2$ for any $T$.
This implies that $Q(\Pi^{(n)})\to T$ in probability as $\|\Pi^{(n)}\|\to0$.
Then there exists a subsequence of $\Pi^{(n)}$, such that $Q(\Pi^{(n)})\to T$ almost surely.
\end{remark}

\begin{theorem}
If $\|\Pi^{(n)}\|\to0$ faster than $1/n^2$, i.e.,
\[
\lim_{n\to\infty}n^2\cdot\|\Pi^{(n)}\|=0,
\]
then $Q(\Pi^{(n)})\to T$ almost surely.
\end{theorem}

\begin{proof}
Take $\delta_n\triangleq n^2\cdot\|\Pi^{(n)}\|$, then by Markov inequality,
\begin{align*}
\mathbb{P}\left(
(Q(\Pi^{(n)})- T)^2>2\delta_n
\right)&\le \frac{\mathbb{E}\left[(Q(\Pi^{(n)})- T)^2\right]}{2\delta_n}\le\frac{2T\|\Pi^{(n)}\|}{2\delta_n} = \frac{T}{n^2}.
\end{align*}
Considering that $\sum_n\frac{T}{n^2}<\infty$,
\[
\sum_n\mathbb{P}\left(
(Q(\Pi^{(n)})- T)^2>2\delta_n
\right)<\infty.
\]
By Borel-Cantelli Lemma,
\[
\mathbb{P}\left(
(Q(\Pi^{(n)})- T)^2>2\delta_n,\quad\text{infinitely often}
\right)=0.
\]
Therefore, for almost all $\omega\in\Omega$,
\[
|Q(\Pi^{(n)},\omega)- T|>\sqrt{2\delta_n},\quad\text{for finite $n$}\implies
|Q(\Pi^{(n)},\omega)- T|\le \sqrt{2\delta_n},~n\to\infty.
\]
By the assumption that $\delta_n\to0$ as $n\to\infty$, we conclude that 
\[
|Q(\Pi^{(n)},\omega)- T|\to0.
\]
The proof is complete.
\end{proof}

The Brownian motion is nowhere differentiable almost surely and does not have bounded variation.
However, it turns to have finite quadratic variation limit.
Using this quadratic variation property, we can go back to show that the Brownian motion does not have bounded variation.


\begin{proof}[A Direct Proof for Theorem~\ref{The:7:6}]
Since the Brownian motion is almost surely continuous, on the closed interval $[0,T]$,
$B_t(\omega)$ is uniformly continuous for almost all $\omega\in\Omega$:
\[
\max_k~|B_{t_k}(\omega) - B_{t_{k-1}}(\omega)|\to0,\quad\text{as }\max_k~|t_k - t_{k-1}|\to0
\]
Assume on the contrary that there exists $t>0$ such that 
\[
\mathbb{P}(TV(B)[0,t]=\infty)<1\implies
\mathbb{P}(TV(B)[0,t]<\infty)>0.
\]
Define the set $\Lambda = \{\omega\in\Omega:~TV(B)[0,t]<\infty\}$.
Then for any partition $\Pi$, if $\omega\in\Lambda$,
\[
\sum_k|B_{t_k}(\omega) - B_{t_{k-1}}(\omega)|\le TV(B_{\cdot}(\omega))[0,t]<\infty
\]
As a result, for $\omega\in\Lambda$, the quadratic variation converges to $0$ as $\|\Pi\|\to0$:
\[
Q(\Pi,\omega)=\sum_k(B_{t_k}(\omega) - B_{t_{k-1}}(\omega))^2\le \|\Pi\|\sum_k|B_{t_k}(\omega) - B_{t_{k-1}}(\omega)|\to0.
\]
Then $\mathbb{P}(\lim_{n\to\infty}Q(\Pi^{(n)})=0)\ge \mathbb{P}(\Lambda)>0$, where $\|\Pi^{(n)}\|\to0$.
Choose some $\epsilon\in(0,t)$.
If $\omega\in\{\lim_{n\to\infty}Q(\Pi^{(n)})=0\}$, there exists $n_0$ such that for $n\ge n_0$,
\[
\omega\in \{|Q(\Pi^{(n)})-t|>\epsilon\}\implies
\lim_{n\to\infty}\mathbb{P}(|Q(\Pi^{(n)})-t|>\epsilon)>0,
\]
which contradicts to the fact that $Q(\Pi^{(n)})\to t$ in probability.
\end{proof}

















