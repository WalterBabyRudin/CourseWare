


\section{Thursday}\index{week7_Tuesday_lecture}

\subsection{Introduction to SDE}
In this week, we will introduce some concept of stochastic differential equations~(SDEs).
We will talk about how to solve some simple SDEs.
We will also study some important SDEs in applications that cannot find explicit solutions, and establish the theorem of existence and uniqueness of solutions.

\paragraph{Motivation}
SDEs are usually regarded as an ODE plus a stocahstic perturbation driven by Brownian motion, which is also called ``noises''.
The following ODE characterizes the population growth in practice:
\[
\frac{\diff S_t}{\diff t}=r\diff t.
\]
This simple ODE admits a deterministic solution $S_t$. 
The corresponding SDE is $\frac{\diff S_t}{\diff t} = r\diff t + \sigma\diff B_t$, called Black-Sholes equation, in which the increasing rate is a constant plus a random perturbation.

In this lecture, we will conside the SDE of the Markovian type, i.e., the parameter $\mu$ and $\sigma$ at time index $t$ only depends on $t,X_t$ instead of $\{X_u\}_{u<t}$:
\begin{equation}\label{Eq:13:1}
\diff X_t = \mu(t, X_t)\diff t + \sigma(t, X_t)\diff B_t, 
\end{equation}
where $B_{\bullet}$ is a standard Brownian motion, and $X_{\bullet}$ is the unknown continuous process. The differential for SDE has no meaning, while only the integration has the meaning. 
Therefore, the above equation refers to the follwoing equation in integral form:
\begin{equation}\label{Eq:13:2}
X_t-X_0 = \int_0^t\mu(u,X_u)\diff u + \int_0^t\sigma(u,X_u)\diff B_u.
\end{equation}
where $\mu:~[0,\infty)\times\mathbb{R}\to\mathbb{R}$ and $\sigma:~[0,\infty)\times\mathbb{R}\to\mathbb{R}$.
A particular situation is when $\mu(t,X_t)=\mu(X_t)$ and $\sigma(t,X_t)=\sigma(X_t)$, i.e., $\mu:~[0,\infty)\to\mathbb{R}$ and $\sigma:~[0,\infty)\to\mathbb{R}$.
We call the SDE in this case the type of \emph{time-homogeneous Markovian} or \emph{Ito}.

\begin{definition}[Solution to SDE]
Given the functions $\mu,\sigma$ defined above, the solution to (\eqref{Eq:13:1}) is the process $(X,B)$ on a filtered probability space $(\Omega,\mathcal{F},\mathbb{F},\mathbb{P})$ satisfying
\begin{itemize}
\item
$B_{\bullet}$ is a standard $\mathbb{F}$-Brownian motion;
\item
The equation (\ref{Eq:13:2}) holds.
\end{itemize}
\end{definition}
\begin{remark}
In addition to the unknown process $X_{\bullet}$, the probability space and Brownian motion are also parts of the solution. Only the coefficients $\mu$ and $\sigma$ are given.
\end{remark}

There are two interpretations for the uniqueness of an SDE:
\begin{definition}[Pathwise Uniqueness]
The solution to (\eqref{Eq:13:1}) has pathwise uniqueness if given the initial value $X_0$ and the Brownian motion $\{B_t\}_{t\ge0}$ on a filtered probability space $(\Omega,\mathcal{F},\mathbb{F},\mathbb{P})$, there is a unique stochastic process $X_{\bullet}$, pathwisely, satisfying the equation.
In other words, if two solutions $(X,B)$ and $(X',B')$ have the same initial value, i.e., $X_0=X_0'$ and $B_0=B_0'$, then $X_t=X_t'$ almost surely for any $t\ge0$.
\end{definition}

\begin{definition}[Uniqueness in Law]
The solution to (\eqref{Eq:13:1}) has uniqueness in law if two solutions $X,X'$ with the same initial distribution are equivalent, i.e., $X$ and $X'$ have the same finite dimensional distribution.
\end{definition}

\begin{definition}[Strong/Weak Solution]
A solution $(X,B)$ to equation (\ref{Eq:13:1}) is called a \emph{strong solution} if $X$ is adapted to the filteration $\mathbb{F}^B$ that is generated by the Brownian motion $B_{\bullet}$ with completion.
A solution that is not a strong solution is called a \emph{weak solution}.
\end{definition}

\begin{remark}
If a strong solution exists, then the SDE always has a solution for any given probability space and Brownian motion.
In other words, when a strong solution exists, {\it any initial value and Brownian motion corresponds to at least one solution.}
If one has the pathwise uniqueness together with the existence of a strong solution, then {\it any initial value and Brownian motion will correspond to a unique solution.}
\end{remark}

The following example shows an important concept in SDE: sometimes the solution exists but there is no strong solution; sometimes we have no pathwise uniqueness but only uniqueness in law.

\begin{example}[The Tanaka Equation]
Consider the 1-dimensional equation
\begin{equation}\label{Eq:13:3}
X_t = \int_0^t\text{sign}(X_u)\diff B_u,\quad 0\le t<\infty,
\end{equation}
where $\text{sign}(X)=1\{X\ge0\} - 1\{X<0\}$ denotes the sign function.
It corresponds to the SDE
\begin{equation}\label{Eq:13:4}
\diff X_t= \text{sign}(X_u)\diff B_t,\quad X_0=0.
\end{equation}
We have the following conclusions on this SDE:
\begin{enumerate}
\item
The solution to (\ref{Eq:13:4}) has uniqueness in law:
Take $X$ be a standard Brownian motion. For any other solution $X'$ satisfying (\eqref{Eq:13:3}), notice that $X'$ is a continuouis local martingale and the quadratic variation $\langle X'\rangle_t=\int_0^t\text{sign}(X'_u)^2\diff u =\int_0^t\diff u=t$.
By the Levy's theorem, $X$ and $X'$ share the same distribution.
\item
A weak solution exists: 
choose $X$ be any Brownian motion $\{W_t\}_{t\ge0}$, and define $\tilde{B_t}$ by $\tilde{B}_t=\int_0^t\text{sign}(W_s)\diff W_s$, i.e., $\diff\tilde{B}_t=\text{sign}(X_t)\diff X_t$.
Then $\tilde{B}_{\bullet}$ is also a Brownian motion. Moreover, 
\[
\diff X_t = \text{sign}(X_t)\diff\tilde{B}_t.
\]
Hence, the pair $(W,\tilde{B})$ is a weak solution.
\item
Pathwise uniqueness does not hold:
when $(X,B)$ is a solution, then $(-X,B)$ is also a solution.
\item
There is no strong solution. %why?
\end{enumerate}
\end{example}

Next we will discuss how to solve some simple SDEs:
\begin{example}[Ornstein-Uhlenbeck Process]
Consider solving the SDE
\[
\diff X_t=-\alpha X_t\diff t + \sigma\diff B_t,\quad X_0=1,
\]
where $\alpha,\sigma$ are some non-negative constants.
\begin{itemize}
\item
Take $Y_t=e^{\alpha t}X_t$, where $e^{\alpha t}$ can be viewed as the \emph{integrating factor}.
Applying Ito's formula on $Y_t$ with $f(x,t)=e^{\alpha t}x$ gives
\begin{align*}
\diff Y_t&=\alpha e^{\alpha t}X_t\diff t + e^{\alpha t}\diff X_t\\
&=\alpha e^{\alpha t}X_t\diff t + e^{\alpha t}(-\alpha X_t)\diff t + e^{\alpha t}\sigma\diff B_t\\
&=\sigma e^{\alpha t}\diff B_t
\end{align*}
It follows that
\[
Y_t = Y_0 + \sigma\int_0^te^{\alpha u}\diff B_u\implies
X_t=e^{-\alpha t}Y_t=e^{-\alpha t}+\sigma\int_0^te^{\alpha(u-t)}\diff B_u.
\]
\end{itemize}
\end{example}

\begin{example}[Geometric Brownian Motion]
Consider the SDE
\[
\diff X_t = \mu X_t\diff t + \sigma X_t\diff B_t,\quad X_0=1,
\]
where $\mu,\sigma$ are constants. We claim that there is a unique strong solution $X_t=e^{(\mu-1/2\sigma^2)t + \sigma B_t}$.
\begin{itemize}
\item
To check it is indeed a solution, apply the Ito's formula on $X_t$ with $f(t,B)=e^{(\mu-1/2\sigma^2)t + \sigma B}$ gives
\begin{align*}
\diff X_t&=\left(
\mu - \frac{1}{2}\sigma^2
\right)e^{(\mu-1/2\sigma^2)t + \sigma B_t}\diff t +  
\sigma e^{(\mu-1/2\sigma^2)t + \sigma B_t}\diff B_t\\
&\qquad + \frac{1}{2}\sigma^2e^{(\mu-1/2\sigma^2)t + \sigma B_t}\diff t\\
&=\mu e^{(\mu-1/2\sigma^2)t + \sigma B_t}\diff t + \sigma e^{(\mu-1/2\sigma^2)t + \sigma B_t}\diff B_t\\
&=\mu X_t\diff t + \sigma X_t\diff B_t.
\end{align*}
\end{itemize}
\end{example}


\begin{example}
Consider the simple SDE
\[
\diff X_t = b(t,X_t)\diff t + \diff B_t,
\]
where $b(t,x):~[0,\infty)\times\mathbb{R}\to\mathbb{R}$ is a bounded Borel measurable function.
\begin{itemize}
\item
We can apply the change of probability measure trick to solve this SDE.
Let $\{W_t\}_{t\ge0}$ be a standard Brownian motion on a filtered probabiltiy space $(\Omega,\mathcal{F},\mathbb{F},\mathbb{P})$.
Define a new probability measure $Q$ with $\left.\frac{\diff Q}{\diff\mathbb{P}}\right|_{\mathcal{F}_t}=Z_t, t\ge0$, where $Z_t$ is the stochastic exponential of $N_t\equiv \int_0^tb(u,W_u)\diff W_u$:
\[
Z_t = \exp\left\{
\int_0^tb(u,W_u)\diff W_u - \frac{1}{2}\int_0^tb^2(u,W_u)\diff u
\right\}.
\]
Since $N_{\bullet}$ is a martingale w.r.t. $\mathbb{P}$, we can check that $Z_{\bullet}$ is also a martingale.
Applying Ito's formula on $Z_t$ with $f(\zeta)=e^{\zeta}$ and $\zeta=N_t - \frac{1}{2}\langle N\rangle_t$ gives
\begin{align*}
\diff Z_t&=\diff \exp\left(
N_t - \frac{1}{2}\langle N\rangle_t
\right)\\
&=e^{
N_t - \frac{1}{2}\langle N\rangle_t
}\diff N_t - \frac{1}{2}e^{
N_t - \frac{1}{2}\langle N\rangle_t
}\diff \langle N\rangle_t + \frac{1}{2}e^{
N_t - \frac{1}{2}\langle N\rangle_t
}\diff \langle N\rangle_t
% why it is \langle N\rangle_t?
=Z_t\diff N_t.
\end{align*}
Hence, the process $Z_t$ admits the integral equation
\[
Z_t = 1 + \int_0^tZ_u\diff N_u.
\]
\item
We recover the original solution by Girsanov theorem.
Define $\tilde{B}_t=W_t - W_0 - \int_0^t\frac{1}{Z_u}\diff\inp{W}{Z}_u$.
In particular,
\begin{align*}
\inp{W}{Z}_u&=\inp{\int_0^t\diff W_u}{\int_0^tZ_u\diff N_u}=\int_0^tZ_u\diff\inp{W}{N}_u,
\end{align*}
which implies that $\int_0^t\frac{1}{Z_u}\diff\inp{W}{Z}_u=\inp{W}{N}_t$.
Therefore, $\tilde{B}_t=W_t-W_0-\inp{W}{N}_t$ is a martingale w.r.t. the probability measure $Q$.
Moreover, $\langle \tilde{B}\rangle_t=\langle W\rangle_t=t$. By Levy's theorem, $\tilde{B}_{\bullet}$ is a standard Brownian motion w.r.t. $Q$.
Furthermore,
\[
\inp{W}{N}_t=\inp{\int_0^t\diff W_u}{\int_0^tb(u,W_u)\diff W_u}=\int_0^tb(u,W_u)\diff u.
\]
Substituting this form into $\tilde{B}_t$ yields
\[
W_t = W_0 + \int_0^tb(u,W_u)\diff u + \tilde{B}_t = \int_0^tb(u,W_u)\diff u + \tilde{B}_t.
\]
As a result, $(W,\tilde{B})$ is a solution on the probability space $(\Omega,\mathcal{F},Q)$.
This solution is a weak solution.
\end{itemize}

\end{example}


\chapter{Week13}
\section{Thursday}
\subsection{Fundamental Theorems in SDE}
The solution to an SDE could be either a strong solution or a weak solution.
If we concern about the Brownian motion to be \emph{pre-determined}, then we need to consider the strong solution; otherwise the weak solution is enough, if we only concern the distribution of a process or the construction of a process.
Only when we need to construct different solutions w.r.t. the same Brownian motion, the strong solution is needed. The first theorem indicates that the existence of solution, together with the pathwise uniqueness, implies the existence and uniqueness of a strong solution.

\begin{theorem}[Existence and Uniqueness for SDEs]
Suppose that the SDE in (\eqref{Eq:13:1}) has pathwise uniqueness, then
\begin{enumerate}
\item
It also has uniqueness in law;
\item
The existence of solution implies the existence of strong solution, i.e., there exists a functional $F:~\mathbb{R}\times\mathcal{C}[0,\infty)\to\mathcal{C}[0,\infty)$ so that $x=F(x_0,B)$.
\end{enumerate}
\end{theorem}

The pathwise uniqueness is difficult to verify. The following theorem gives a different way to check the existence and uniqueness:

\begin{theorem}
Suppose that the SDE in (\eqref{Eq:13:1}) satisfies
\begin{enumerate}
\item
Coefficients $\mu,\sigma$ are Lipschitz in $x$ uniformly in $t\in[0,T]$, i.e., there exists constant $L$ so that 
\[
|\mu(t,x) - \mu(t,y)|\le L|x-y|,\quad
|\sigma(t,x) - \sigma(t,x)|\le L|x-y|,~\forall x,y, t\in[0,T].
\]
\item
Coefficients $\mu,\sigma$ satisfy the linear growth condition in $x$ uniformly in $t$, i.e., there exists a constant $C$ so that
\[
|\mu(t,x)|\le C(1+|x|), |\sigma(t,x)|\le C(1+|x|),\quad \forall x, t\in[0,T].
\]
\item
Let $Z$ be a random variable on $(\Omega,\mathcal{F}_0,\mathbb{P})$, independent of the Brownian motion $\{B_t\}_{t\ge0}$, satisfying $\mathbb{E}[Z^2]<\infty$.
\end{enumerate}
Then the SDE with the initial condition $X_0=Z$ admits a strong solution $\{X_t\}_{t\ge0}$ adapted to the filtration $\{\mathcal{F}_t\}_{t\ge0}$ and $X\in\mathcal{L}^2$.
Furthermore, the solution has pathwise uniqueness.
\end{theorem}

\begin{proof}
The uniqueness is by Ito's isometry and Lipschitz condition.
The existence of strong solution can be shown by Picard iteration method used in ODE.
\end{proof}

We can use Ito's formula to solve the time-homogeneous SDE:
\begin{equation}\label{Eq:13:5}
\diff X_t = \mu(X_t)\diff t + \sigma(X_t)\diff B_t.
\end{equation}
Suppose that $\mu,\sigma\in\mathcal{C}^\infty(\mathbb{R})$ are smooth functions with bounded derivative.
Let $X_{\bullet}$ be the unique strong solution.
Take $f\in\mathcal{C}^2(\mathbb{R})$, and by Ito's formula,
\[
f(X_t) - f(X_0) = \int_0^t\frac{\partial f}{\partial X}(X_u)\diff X_u + \frac{1}{2}\int_0^t\frac{\partial^2 f}{\partial X^2}(X_u)\diff\langle X\rangle_u.
\]
The SDE in (\eqref{Eq:13:5}) can also be written as the integral form:
\[
X_t = X_0 + \int_0^t\mu(X_u)\diff u + \int_0^t\sigma(X_u)\diff B_u\implies
\langle X\rangle_t = \int_0^t\sigma^2(X_u)\diff u.
\]
It follows that
\[
f(X_t) - f(X_0) = \int_0^t\left[
\frac{\partial f}{\partial X}(X_u)\mu(X_u) + \frac{1}{2}\frac{\partial^2 f}{\partial X^2}(X_u)\sigma^2(X_u)
\right]\diff u + \int_0^t \frac{\partial f}{\partial X}(X_u)\sigma(X_u)\diff B_u.
\]

\begin{theorem}
Let $\mu,\sigma$ be bounded Borel-measurable functions, and assume there exists a constant $\lambda>0$ so that $\frac{1}{\lambda}<\sigma(\cdot)<\lambda$.
Define the operator 
\[
\mathcal{L}=\frac{1}{2}\sigma^2(x)\frac{\diff^2}{\diff x^2} + \mu(x)\frac{\diff}{\diff x}.
\]
If $X_{\bullet}$ is a continuous process on $(\Omega,\mathcal{F},\mathbb{P})$ such taht for any $f\in\mathcal{C}^2(\mathbb{R})$, the process
\[
W_t^f = f(X_t) - f(X_0) - \int_0^t(\mathcal{L}f)(X_s)\diff s
\]
is a continuous local martingale, then $X_{\bullet}$ solves the SDE (\eqref{Eq:13:5}) on the space $(\Omega,\mathcal{F},\mathbb{P})$.
\end{theorem}
% weak or strong:
\begin{proof}
In order to show $X_{\bullet}$ is a weak solution on $(\Omega,\mathcal{F},\mathbb{P})$, it suffices to construct a Brownian motion $B_{\bullet}$ so that 
\[
X_t= X_0 + \int_0^t\sigma(X_u)\diff B_u + \int_0^t\mu(X_u)\diff u.
\]
We set $f(x)=x$ and $B_t = \int_0^t\frac{1}{\sigma(X_u)}\diff W_u^f$.
Then we show it is a Brownian motion by Levy's characterization.
\end{proof}
















