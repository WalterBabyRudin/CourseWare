\section{Wednesday for MAT4002}\index{Monday_lecture}
\paragraph{Reviewing}
We can construct a continuous injection from $|K|$ to $|K'|$, where $K=(V,\Sigma)$ is a simplicial complex, and $K'=(V',\Sigma')$ is its subcomplex:

Let $D_{\Sigma}:=\coprod_{\sigma\in\Sigma}\sigma$ and $D_{\Sigma'}:=\coprod_{\sigma'\in\Sigma'}\sigma'$, then
$|K'| = D_{\Sigma'}/\sim_{\Sigma'}$ and $|K| = D_{\Sigma}/\sim_{\Sigma}$, which follows that
\[
f:D_{\Sigma'}\to D_{\Sigma}\xrightarrow{P}D_{\Sigma}/\sim_{\Sigma},\quad\text{$P$ denotes the canonical projection mapping}
\]
The whole mapping $f$ descends to a continuous mapping
\[
\tilde{f}:D_{\Sigma'}/\sim_{\Sigma'}\to D_{\Sigma}/\sim_{\Sigma}
\]
The $\tilde{f}$ is injective since
\begin{equation}\label{Eq:8:2}
x\sim_{\Sigma'}y\Longleftrightarrow
i(x)\sim_{\Sigma}i(y),\qquad
\forall x,y\in D_{\Sigma},
\end{equation}
where $i$ denotes the inclusion mapping.

Another way is to consider the inclusion $i:|K'|\to |K|$, which is continuous and injective as well.
Note that $i(|K'|)$ is closed in $|K|$.

\begin{proposition}
For each $K=(V,\Sigma)$, and finite $V$, 
there is a continuous injection 
$g:|K|\hookrightarrow\mathbb{R}^n$ for some $n$.
\end{proposition}

\begin{proof}
Consider $K^p:=(V,\Sigma^p)$, where $\Sigma^p$ is the power set of $V$.
Therefore, $|K^p| = \Delta^{|V|-1}\subseteq\mathbb{R}^{|V|}$, and $K$ is a simplicial subcomplex of $K^p$, which follows that
\[
l:|K'|\xrightarrow{i}|K^p|\xrightarrow{i}\mathbb{R}^{|V|}
\]
The whole mapping $l$ is an inclusion mapping from $|K'|$ to $\mathbb{R}^{|V|}$, which is continuous and injective. The proof is complete.
\end{proof}

\begin{proposition}[Hausdorff]
If $K=(V,\Sigma)$ with fintie $V$, then $|K|$ is Hausdorff.
\end{proposition}

\begin{proof}
Let $g:|K|\xrightarrow{l}\mathbb{R}^n$.
Consider the bijective $g:|K|\to g(|K|)$, which is continuous.
Sicne $|K|$ is compact, and $g(|K|)\subseteq \mathbb{R}^n$ is Hausdorff, we imply that $|K|$ and $g(|K|)$ are homeomorphic, i.e., $|K|$ is Hausdorff.
\end{proof}

\begin{definition}[Edge Path]
An \emph{edge path} of $K=(V,\Sigma)$ is a sequence of vertices $(v_1,\dots,v_n), v_i\in V$ such that $\{v_i,v_{i+1}\}\in\Sigma,\forall i$.
\end{definition}

\begin{proposition}[Connectedness]
Let $K=(V,\Sigma)$ be a simplicial complex. TFAE:
\begin{enumerate}
\item
$|K|$ is connected
\item
$|K|$ is path-connected
\item
Any 2 vertices in $(V,\Sigma)$ can be joined by an edge path, i.e., for $\forall u,v\in V$, there exists $v_1,\dots,v_k\in V$ such that $(u,v_1,\dots,v_k,v)$ is an edge path.
\end{enumerate}
\end{proposition}

\begin{proof}[Sketch of Proof (to be revised)]
\begin{enumerate}
\item
(3) implies (2):
For every $x,y\in|K|$, 
\[
\left\{
\begin{aligned}
x\in\Delta_{\sigma_1}\text{ for some $\sigma_1\in\Sigma$.}\\
y\in\Delta_{\sigma_2}\text{ for some $\sigma_2\in\Sigma$.}\\
\end{aligned}
\right.
\]
Take a path joining $x$ to a vertex $v_1\in\sigma_1$ and a path joining $y$ to a vertex $v_2\in\sigma_2$.

By (3), we have a path joninig $v_1$ and $v_2$.
\item
(1) implies (3):
Suppose on the contrary that there is a vertex $v$ not satisfying (3).
Take $V'$ as the set of vertexs that can be joined with $v$; and $V''$ as the set of vertexs that cannot be joinied with $v$.

Then $V',V''\ne\emptyset$.
Consider $K',K''$ be simplicial subcomplexes of $K$, spanned by $V'$ and $V''$.
Then $|K'|,|K''|$ are disjoint, closed in $|K|$.

$|K| = |K'|\cup|K''|$. 
If there exists $x\in |K|\setminus(|K'|\cup|K''|)$, then for any $\sigma\in\Sigma$ such that $x\in\Delta_{\sigma}$, we imply $\Delta_\sigma\not\subseteq |K'|$ or $|K''|$.

Therefore, $\sigma$ consists of vertices in both $V'$ and $V''$.
Then there is $v',v''\in\sigma$ joining $V'$ and $V''$.

Therefore, there is no such $x$ and hence $|K|=|K'|\cup|K''|$ is a disjoint union of two closed sets, i.e., not connected.
\end{enumerate}
\end{proof}

\subsection{Homotopy}
\paragraph{Yoneda's ``philosophy''}
To understand an object $X$ (in our focus, $X$ denotes topological space), we should understand functions
\[
\begin{array}{lll}
f:A\to X,
&
\text{ or }
&
g:X\to B
\end{array}
\]
One special example is to let $B=\mathbb{R}$.

There are many type of continuous mappings from $X$ to $Y$. We will group all these mappings into equivalence classes.

\begin{definition}[Homotopy]
A \emph{Homotopy} between two continuous maps $f,g:X\to Y$ is a continuous map
\[
H:X\times[0,1]\to Y
\]
such that 
\[
H(x,0)=f(x),\quad
H(x,1)=g(x)
\]
If such $H$ exists, we say $f$ and $g$ are \emph{homotopic}, denoted as $f\simeq g$.
\end{definition}

\begin{example}\label{exp:8:7}
Let $Y\subseteq\mathbb{R}^2$ be a convex subset.
Consider two continuous maps $f:X\to Y$ and $g:X\to Y$.
They are always homotopic since we can define the homotopy
\[
H(x,t) = tg(x) + (1-t)f(x)
\]
\end{example}

\begin{proposition}
Homotopy is an equivalent relation.
\end{proposition}
\begin{proof}
\begin{enumerate}
\item
Let $f:X\to Y$ be any continuous map. Then $f\simeq f$:
we can define a homotopy $H(x,t) = f(x),\forall 0\le t\le 1$.
\item
Suppose $f\simeq g$, i.e., $H$ is a homotopy between $f$ and $g$, then $g\simeq f$:
Define the mapping $H'(x,t) = H(x,1-t)$, 
then 
\[
\begin{array}{ll}
H'(x,0)=g(x),
&
H'(x,1)=f(x)
\end{array}
\]
\item
Let $f,g,h:X\to Y$ be three continuous maps.
If $f$ and $g$ are homotopic and $g$ and $h$ are homotopic, then $f$ and $h$ are homotopic:

Let $H:X\times[0,1]\to Y$ be a continuous map such that 
\[
H(x,0)=f(x),H(x,1)=g(x);
\]
$K:X\times[0,1]\to Y$ be a continuous map such that 
\[
K(x,0)=g(x),K(x,1)=h(x).
\]

Define a function $J:X\times[0,1]\to Y$ by
\[
J(x,t)=\left\{
\begin{aligned}
H(x,2t),&\quad 0\le t\le1/2\\
K(x,2t-1),&\quad 1/2\le t\le 1
\end{aligned}
\right.
\]
\begin{itemize}
\item
$J$ is continuous, since for all closed $V\subseteq Y$,
\[
J^{-1}(V)=(J^{-1}(V)\cap(X\times[0,1/2]))\cup(J^{-1}(V)\cap(X\times[1/2,1]))
=
H^{-1}(V)\cup K^{-1}(V),
\]
and the closedness of $H^{-1}(V)$ and $K^{-1}(V)$ implies the closedness of $J^{-1}(V)$
\item
Moreover, $J$ has the property that $J(x,0)=H(x,0)=f(x)$, while $J(x,1)=K(x,1)=h(x)$.
\end{itemize}
\end{enumerate}
\end{proof}




%
%
%\begin{enumerate}
%
%\item
%If $f\cong g$, then $g\cong f$:
%For homotopic from $f$ to $g$, say $H(x,t)$, construct
%\[
%H'(x,t):=H(x,1-t)
%\]
%Therefore, $H'(x,0)=g(x)$ and $H'(x,1)=f(x)$.
%\item
%If $f\cong g$ and $g\cong h$, then $f\cong h$:
%suppose $H:f\cong g$, and $K:g\cong h$.
%Consider
%\[
%J(x,t)=\left\{
%\begin{aligned}
%H(x,2t),&\quad 0\le t\le1/2\\
%K(x,2t-1),&\quad 1/2\le t\le 1
%\end{aligned}
%\right.
%\]
%Note that $J(x,1/2)$ are well-defined.
%Then $J$ is continuous, since for all closed $V\subseteq Y$,
%\[
%J^{-1}(V)=(J^{-1}(V)\cap(X\times[0,1/2]))\cup(J^{-1}(V)\cap(X\times[1/2,1]))
%=
%H^{-1}(V)\cup K^{-1}(V)
%\]
%Since $H^{-1}(V)$ and $K^{-1}(V)$ are both closed, we imply $J^{-1}(V)$ is closed.
%\end{enumerate}

\begin{remark}
There are only one equivalence class in example~(\ref{exp:8:7}). Actually, for given space $X$ and $Y$, if any two continuous mapping are homotopic, then we imply there is only one equivalence class.
\end{remark}
%
%Therefore, there is only one equivalence class in example~(1).
%This reflects the fact that $Y\subseteq\mathbb{R}^2$ is a ``simple'' object.
%
%\begin{proof}
%Tkae $y_0\in Y$.
%Consider $C_y:X\to Y$ by $C_y(x)=y_0,\forall x$.
%For all continuous maps $f:X\to Y$, $f\cong C_y$.
%
%Therefore, there is only one equivalence class since every continuous map is homotopic to $C_y$
%\end{proof}
%
%
%


















