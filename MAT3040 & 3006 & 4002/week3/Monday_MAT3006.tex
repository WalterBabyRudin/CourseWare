\section{Monday for MAT3006}\index{Monday_lecture}
\paragraph{Reviewing}
\begin{enumerate}
\item
Compactness/Sequential Compactness:
\begin{itemize}
\item
Equivalence for metric space
\item
Stronger than closed and bounded
\end{itemize}
\item
Completeness: 
\begin{itemize}
\item
The metric space $(E,d)$ is complete if every Cauchy sequence on $E$ is convergent.
\item
$\mathbb{P}[a,b]\subseteq\mathcal{C}[a,b]$ is not complete:
\[
f_N(x)=\sum_{n=0}^N(-1)^n\frac{x^{2n}}{(2n)!}\to\cos x,
\]
while $\cos x\notin\mathcal{P}[a,b]$.
\end{itemize}
\end{enumerate}


\subsection{Remarks on Completeness}

\begin{proposition}
Let $(X,d)$ be a metric space.
\begin{enumerate}
\item
If $X$ is complete and $E\subseteq X$ is closed, then $E$ is complete.
\item
If $E\subseteq X$ is complete, then $E$ is closed in $X$.
\item
If $E\subseteq X$ is compact, then $E$ is complete.
\end{enumerate}
\end{proposition}
\begin{proof}
\begin{enumerate}
\item
Every Cauchy sequence $\{e_n\}$ in $E\subseteq X$ is also a Cauchy sequence in $E$.

Therefore we imply $\{e_n\}\to x\in X$, due to the completeness of $X$.

Due to the closedness of $E$, the limit $x\in E$, i.e., $E$ is complete.
\item
Consider any convergent sequence $\{e_n\}$ in $E$, with some limit $x\in X$.

We imply $\{e_n\}$ is Cauchy and thus $\{e_n\}\to e\in E$, due to the completeness of $E$.

By the uniqueness of limits, we must have $x=z\in E$, i.e., $E$ is closed.
\item
Consider a Cauchy sequence $\{e_n\}$ in $E$.
There exists a subsequence $\{e_{n_j}\}\to e\in E$, 
due to the sequential compactness of $E$.

It follows that for large $n$ and $j$,
\[
d(e_n,e)
\makebox[1cm][c]{$\overset{\text{(a)}} \leq$}
 d(e_n,e_{n_j})+d(e_{n_j},e)
 \makebox[1cm][c]{$\overset{\text{(b)}} <$}
 \varepsilon
\]
where (a) is due to triangle inequality
and (b) is due to the Cauchy property of $\{e_n\}$ and the convergence of $\{e_{n_j}\}$. 

Therefore, we imply $\{e_n\}\to e\in E$, i.e., $E$ is complete.
\end{enumerate}
\end{proof}

\begin{remark}
Given any metric space that may not be necessarily complete, we can make the union of it with another space to make it complete, e.g., just like the completion from $\mathbb{Q}$ to $\mathbb{R}$.
\end{remark}


\subsection{Contraction Mapping Theorem}

The motivation of the contraction mapping theorem comes from solving an equation $f(x)$. More precisely, such a problem can be turned into a problem for fixed points, i.e., it suffices to find the fixed points for $g(x)$, with $g(x)=f(x)+x$.

\begin{definition}
Let $(X,d)$ be a metric space. 
A map $T:(X,d)\to (X,d)$ is a \emph{contraction} 
if there exists a constant $\tau\in(0,1)$ such that
\[
\begin{array}{ll}
d(T(x),T(y))<\tau\cdot d(x,y),
&
\forall x,y\in X
\end{array}
\]
A point $x$ is called a fixed point of $T$ if $T(x)=x$.
\end{definition}
\begin{remark}
All contractions are continuous: 
Given any convergence sequence $\{x_n\}\to x$, 
for $\varepsilon>0$, take $N$ such that 
$d(x_n,x)<\frac{\varepsilon}{\tau}$ for $n\ge N$. It suffices to show the convergence of $\{T(x_n)\}$:
\[
d(T(x_n),T(x))<\tau\cdot T(x_n,x)<\tau\cdot\frac{\varepsilon}{\tau}=\varepsilon.
\]
Therefore, the contraction is Lipschitz continuous with Lipschitz constant $\tau$.
\end{remark}

\begin{theorem}[Contraction Mapping Theorem / Banach Fixed Point Theorem]
Every contraction $T$ in a \emph{complete} metric space $X$ has a unique fixed point.
\end{theorem}

\begin{example}
\begin{enumerate}
\item
The mapping $f(x)=x+1$ is not a contraction in $X=\mathbb{R}$, and it has no fixed point.
\item
Consider an in-complete space $X=(0,1)$ and a contraction $f(x)=\frac{x+1}{2}$. It doesn't admit a fixed point on $X$ as well.
\end{enumerate}
\end{example}

\begin{proof}
Pick any $x_0\in X$, and define a sequence recursively by setting $x_{n+1}=T(x_n)$ for $n\ge0$.
\begin{enumerate}
\item
Firstly show that the sequence $\{x_n\}$ is Cauchy.

We can upper bound the term $d(T^n(x_0),T^{n-1}(x_0))$:
\begin{equation}\label{Eq:3:4}
d(T^n(x_0),T^{n-1}(x_0))\le \tau d(T^{n-1}(x_0),T^{n-2}(x_0))\le\cdots\le \tau^{n-1}d(T(x_0),x_0)
\end{equation}
Therefore for any $n\ge m$, where $m$ is going to be specified later,
\begin{subequations}
\begin{align}
d(x_n,x_m)&=d(T^n(x_0),T^m(x_0))\label{Eq:3:5:a}\\
&\le\tau d(T^{n-1}(x_0),T^{m-1}(x_0))\le\cdots\le\tau^md(T^{n-m}(x_0),x_0)\label{Eq:3:5:b}\\
&\le\tau^m\sum_{j=1}^{n-m}\tau^{n-m-j}d(T(x_0),x_0)\label{Eq:3:5:c}\\
&<\frac{\tau^{m}}{1-\tau}d(T(x_0),x_0)\label{Eq:3:5:d}\\
&\le\varepsilon\label{Eq:3:5:e}
\end{align}
\end{subequations}
where (\ref{Eq:3:5:b}) is by repeatedly applying contraction property of $d$; (\ref{Eq:3:5:c}) is by applying the triangle inequality and (\ref{Eq:3:4}); (\ref{Eq:3:5:e}) is by choosing sufficiently large $m$ such that $\frac{\tau^{m}}{1-\tau}d(T(x_0),x_0)<\varepsilon$.

Therefore, $\{x_n\}$ is Cauchy. By the completeness of $X$, we imply $\{x_n\}\to x\in X$.
\item
Therefore, we imply 
\[
x=\lim_{n\to\infty}x_{n+1}=\lim_{n\to\infty}T(x_n)=T(\lim_{n\to\infty}x_n)=T(x),
\]
i.e., $x$ is a fixed point of $T$.

Now we show the uniqueness of the fixed point. Suppose that there is another fixed point $y\in X$, then
\[
d(x,y)=d(T(x),T(y))<\tau\cdot d(x,y)\implies d(x,y)<\tau d(x,y),\quad\tau\in(0,1),
\]
and we conclude that $d(x,y)=0$, i.e., $x=y$.
\end{enumerate}
\end{proof}

\begin{example}[Convergence of Newton's Method]
The Newton's method aims to find the root of $f(x)$ by applying the iteration
\[
x_{n+1}=x_n-\frac{f(x_n)}{f'(x)}
\]
Suppose $r$ is a root for $f$, the pre-assumption for the convergence of Newton's method is:
\begin{enumerate}
\item
$f'(r)\ne0$
\item
$f\in\mathcal{C}^2$ on some neighborhood of $r$
\end{enumerate}
\begin{proof}
\begin{enumerate}
\item
We first show that there exists $[r-\varepsilon,r+\varepsilon]$ such that the mapping
\[
\begin{array}{ll}
T:&\mathbb{R}\to\mathbb{R},\\
\text{with}&
x\mapsto x-\frac{f(x)}{f'(x)}
\end{array}
\]
satisfies 
\begin{equation}\label{my:Eq:3:6}
|T'(x)|<1,\quad
\forall x\in[r-\varepsilon,r+\varepsilon].
\end{equation}

Note that $T'(x)=\frac{f(x)}{[f'(x)]^2}f''(x)$, and we define $h(x)=|T'(x)|$.

It's clear that $h(r)=0$ and $h(x)$ is continuous, which implies
\[
r\in h^{-1}((-1,1))\implies
B_\rho(r)\subseteq h^{-1}((-1,1))\ \text{for some $\rho>0$}.
\]
Or equivalently, $h((r-\rho,r+\rho))\subseteq(-1,1)$. Take $\varepsilon=\frac{\rho}{2}$, and the result is obvious.
\item
Moreover, for any $x,y\in [r-\varepsilon,r+\varepsilon]$,
\begin{subequations}
\begin{align}
d(T(x),T(y)):&=|T(x)-T(y)|\label{Eq:3:6:a}\\
&=|T'(\xi)||x-y|\label{Eq:3:6:b}\\
&\le\max_{\xi\in [r-\varepsilon,r+\varepsilon]}|T'(\xi)||x-y|\label{Eq:3:6:c}\\
&<m\cdot|x-y|\label{Eq:3:6:d}
\end{align}
\end{subequations}
where (\ref{Eq:3:6:b}) is by applying MVT, and $\xi$ is some point in $[r-\varepsilon,r+\varepsilon]$; we assume that $\max_{\xi\in [r-\varepsilon,r+\varepsilon]}|T'(\xi)|<m$ for some $m<1$ in (\ref{Eq:3:6:d}).
\item
Note that (2) is not enough to show that $T$ is a contraction.
We further need to show that $T(x)\in [r-\varepsilon,r+\varepsilon]$ provided that $x\in [r-\varepsilon,r+\varepsilon]$:
\[
|T(x)-r|=|T(x)-T(r)|=|T'(s)||x-r|\le\sup_{[r-\varepsilon,r+\varepsilon]}|T'(s)||x-r|<|x-r|.
\]
\end{enumerate}
Combining (2) and (3), we imply $T$ is a contraction on $[r-\varepsilon,r+\varepsilon]$.
By applying the contraction mapping theorem, there exists a unique fixed point near $[r-\varepsilon,r+\varepsilon]$:
\[
x-\frac{f(x)}{f'(x)}=x\implies\frac{f(x)}{f'(x)}=0\implies f(x)=0,
\]
i.e., we obtain a root $x=r$.
\end{proof}

Summary: if we use Newton's method on any point between $[r-\varepsilon,r+\varepsilon]$ where $f(r)=0$ and $\varepsilon$ is sufficiently small, then we will eventually get close to $r$.
\end{example}

\subsection{Picard Lindelof Theorem}
We will use Banach fixed point theorem to show the existence and uniqueness of the solution of ODE
\begin{equation}\label{Eq:3:7}
\left\{
\begin{aligned}
\frac{\diff y}{\diff x}&=f(x,y(x))\\
y(x_0)&=y_0
\end{aligned}
\right.\qquad
\textbf{Initial Value Problem, IVP}
\end{equation}
\begin{example}
Consider the IVP
\[
\left\{
\begin{aligned}
\frac{\diff y}{\diff x}&=x^{2}y^{1/5}\\
y(x_0)&=c>0
\end{aligned}
\right.
\implies 
y=\left(
\frac{4x^3}{15}+c^{4/5}
\right)^{5/4}
\]
which can be solved by the separation of variables:
\[
c>0\implies y=\left(
\frac{4x^3}{15}+c^{4/5}
\right)^{5/4}.
\]
However, when $c=0$, the ODE does not have a unique solution. One can verify that $y_1,y_2$ given below are both solutions of this ODE:
\[
y_1=(\frac{4x^3}{15})^{5/4},\qquad
y_2=0
\]
This example shows that even when $f$ is very nice, the IVP may not have unique solution. The Picard-Lindelof theorem will give a clean condition on $f$ ensuring the unique solvability of the IVP~(\ref{Eq:3:7}).
\end{example}


