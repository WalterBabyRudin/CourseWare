
\section{Wednesday for MAT3040}\index{Wednesday_lecture}
\paragraph{Reviewing}
\begin{itemize}
\item
Given the polynomial $f(x)\in\mathbb{F}[x]$, we extend it into the linear operator $f(T):V\to V$.
\item
The minimal polynomial $m_T(x)$ is defined to be the polynomial with least degree such that 
\[
m_T(T)=\bm0_{V\to V},
\]
i.e., $[m_T(T)]\bm v=0_{\bm V},\forall\bm v\in V$.
\item
The minimial polynomial of a vector $\bm v$ relative to $T$ is defined to be the polynomial $m_{T,\bm v}(x)$ with the least degree such that
\[
m_{T,\bm v}(T)(\bm v)=0
\]
\item
If $f(T) = \bm0_{V\to V}$, then we imply $m_T(x)\mid f(x)$.
If $[g(T)]({\bm w})=0_V$, following the similar argument, we imply $m_{T,\bm w}(x)\mid g(x)$.
\item
In particular, $m_T(T)\bm w=\bm0$, which implies $m_{T,\bm w}(x)\mid m_T(x)$.
\end{itemize}
\subsection{Cayley-Hamiton Theorem}
Let's raise an motivative example first:

\begin{example}
Consider the matrix and its induced mapping $\bm A=\begin{pmatrix}
1&0\\0&2
\end{pmatrix}$. It has the characteristic polynomial
\[
\mathcal{X}_{A}=(x-1)(x-2).
\]
\begin{itemize}
\item
Note that $m_A(x)$ cannot be with degree one, since otherwise $m_A(x)=x-k$ with some $k$, and
\[
m_A(\bm A) = \bm A-k\bm I=\begin{pmatrix}
1-k&0\\0&2-k
\end{pmatrix}\ne\bm0,\quad\forall k,
\]
which is a contradiction.
\item
However, one can verify that the $m_A(x)$ is with degree $2$:
\[
m_A(x) = (x-1)(x-2).
\]
\item
The minimial polynomial with eigenvectors can be with degree 1:
\[
\bm w=[0,1]\trans\implies
(A-2I)\bm w=\bm0\implies
m_{A,\bm w}(x) = x-2
\]
\end{itemize}
\end{example}

\begin{remark}
More generally, given an eigen-pair $(\lambda,\bm v)$, the minimal polynomial of an $\bm v$ has the explicit form
\[
m_{T,\bm v}(x) =  (x-\lambda)\implies
(x-\lambda)\mid m_T(x)
\]
Now we want to relate the characterstic polynomial $m_T(x)$ with $\mathcal{X}_T(x)$.
Suppose that
\begin{equation}\label{Eq:7:1}
\mathcal{X}_T(x) = (x-\lambda_1)^{e_1}\cdots(x-\lambda_k)^{e_k}\in\mathbb{F}[x].
\end{equation}
Then we imply
\begin{itemize}
\item
$\lambda_i$ is an eigenvalue of $T$;
\item
$(x-\lambda_i)\mid m_T(x)$;
\end{itemize}
which implies that $(x-\lambda_1)\cdots(x-\lambda_k)\mid m_T(x)$.

Furthermore, (a). does $m_T(x)$ possess other factors, e.g., does there exist $\mu\ne\lambda_i, i=1,\dots,k$ such that $(x-\mu)\mid m_T(x)$?
(b). does $(x-\lambda_i)^{f_i}\mid m_T(x)$ when $f_i>e_i$?

The answer is no for both question~(a) and (b).
\end{remark}
\begin{theorem}[Cayley-Hamilton]
$m_T(x)\mid\mathcal{X}_T(x)$. In particular, $\mathcal{X}_T(T)=\bm0$.
\end{theorem}
The nice equality in (\ref{Eq:7:1}) does not necessarily hold. 
Sometimes $\mathcal{X}_{T}(x)$ cannot be factorized into linear factors in $\mathbb{F}[x]$, e.g., $A=\begin{pmatrix}
0&-1\\1&0
\end{pmatrix}$ in $\mathbb{R}$.

However, for every $f(x)\in\mathbb{F}[x]$, we can extend $\mathbb{F}$ into the algebraically closed set $\overline{F}\supseteq\mathbb{F}$ such that
\[
f(x) = (x-\lambda_1)^{e_1}\cdots(x-\lambda_k)^{e_k}
\]
where $\lambda_i\in\overline{\mathbb{F}}$.

For example, for $f(x) = x^2+1\in\mathbb{R}[x]$, we can extend $\mathbb{R}$ into $\mathbb{C}$ to obtain
\[
f(x) = (x+i)(x-i).
\]

Therefore, the general proof outline for the Cayley-Hamilton Theorem is as follows:
\begin{itemize}
\item
Consider the case where $m_T(x),\mathcal{X}_T(x)$ are both in $\overline{F}[x]$
\item
Show that $m_T(x)\mid\mathcal{X}_T(x)$ under $\overline{F}[x]$.
\end{itemize}

Before the proof, let's study the invariant subspaces, which leads to the decomposition of charactersitc polynomial:
\paragraph{Assumption}
From now on, we assume that $V$ is finite dimensional by default.
\begin{definition}[Invariant Subspace]
An \emph{invariant subspace} of a linear operator $T:V\to V$ is a subspace $W\le V$ that is preserved by T, i.e., $T(W)\subseteq W$. We also call $W$ as $T$-invariant.
\end{definition}
\begin{remark}
If $W\leq V$ is $T$-invariant, then the restriction of the linear operator $T:V \to V$ induces the linear operator
\[
T\mid_W : W \to W.
\]
\end{remark}
\begin{example}
\begin{enumerate}
\item
$V$ itself is $T$-invariant.
\item
For the eigenvalue $\lambda$, the associated $\lambda$-eigenspace $U=\ker(T-\lambda I)$ is $T$-invariant.
\item
More generally, 
$U=\ker(g(T))$ is $T$-invariant for any polynomial $g$:

If $\bm v\in\ker(g(T))$, i.e., $g(T)\bm v=\bm0$, it suffices to show $T(\bm v)\in\ker(g(T))$:
\begin{align*}
g(T)[T(\bm v)] &= (a_mT^m+\cdots+a_0I)[T(\bm v)]\\
&=
(a_mT\circ T^m+\cdots+a_1T\circ T+a_0T\circ I)(\bm v)\\
&=
T[g(T)\bm v]=T(\bm 0)=\bm0
\end{align*}
\item
For $\bm v\in\ker(T-\lambda I)$, $U=\Span\{\bm v\}$ is $T$-invariant.
\end{enumerate}
\end{example}

\begin{proposition}
Suppose that $T:V\to V$ is a linear transformation and 
$W\le V$ is $T$-invariant, then we construct the subspace mapping and the recipe mapping
\begin{subequations}
\begin{equation}
\begin{array}{ll}
T\mid_W:&W\to W\\
\text{with}&\bm w\mapsto T(\bm w)
\end{array}
\end{equation}
\begin{equation}
\begin{array}{ll}
\tilde{T}:&V/W\to V/W\\
\text{with}&\bm v+W\mapsto T(\bm v)+W
\end{array}
\end{equation}
\end{subequations}
\[
\text{(Here the well-definedness of the recipe mapping $\tilde{T}$ is shown in Hw2, Exercise 4),}
\]
which leads to the decomposition of the charactersitic polynomial:
\[
\mathcal{X}_T(x) = \mathcal{X}_{T\mid_W}(x)\mathcal{X}_{\tilde{T}}(x).
\]
\end{proposition}
\begin{proof}
Suppose $\mathcal{C} = \{\bm v_1,\dots,\bm v_k\}$ is a basis of $W$, and extend it into the basis of $V$, denoted as 
\[
\mathcal{B}=\{\bm v_1,\dots,\bm v_k,\bm v_{k+1},\dots,\bm v_n\}
\]
Therefore, $\overline{\mathcal{B}} = \{\bm v_{k+1}+W,\dots,\bm v_n+W\}$ is a basis of $V/W$.
By Homework 2, Question 5, the representation $(T)_{\mathcal{B},\mathcal{B}}$ can be written as the block matrix
\[
(T)_{\mathcal{B},\mathcal{B}} = \begin{pmatrix}
(T|_{W})_{\mathcal{C},\mathcal{C}}&\times\\
\bm0&(\tilde{T})_{\overline{\mathcal{B}},\overline{\mathcal{B}}}
\end{pmatrix}_{(k+(n-k))\times(k+(n-k))}
\]
Therefore, the characteristic polynomial of $T$ can be calculated as:
\begin{align*}
\mathcal{X}_T(x)&= \det((T)_{\mathcal{B},\mathcal{B}} - xI)\\
&=\det((T|_{U})_{\mathcal{C},\mathcal{C}} - xI)\cdot\det((\tilde{T})_{\overline{\mathcal{B}},\overline{\mathcal{B}}} - xI)
\end{align*}
\end{proof}

\begin{proposition}\label{pro:7:12}
Suppose that 
\[
\mathcal{X}_T(x) = (x-\lambda_1)\cdots(x-\lambda_n)
\]
where $\lambda_i$'s are not necessarily distinct.
Then there exists a basis of $V$, say $\mathcal{A}$, such that
\[
(T)_{\mathcal{A},\mathcal{A}}=
\begin{pmatrix}
\lambda_1&\times&\times&\times\\
0&\lambda_2&\cdots&\times\\
0&\cdots&\ddots&\times\\
0&0&\cdots&\lambda_n
\end{pmatrix}
\]
\end{proposition}
%\begin{remark}
%This proposition is the generalization of the eigenvalue decomposition studied in Linear Algebra:
%\begin{definition}[Eigenvalue Decomposition]
%A matrix $\bm A\in\mathbb{R}^{n\times n}$ (or $\mathbb{C}^{n\times n}$) is said to admit an \emph{eigenvalue decomposition} if there exists a nonsingular $\bm V\in \mathbb{C}^{n\times n}$ and a collection of scalars $\lambda_1,\dots,\lambda_n\in\mathbb{C}$ such that
%\[
%\bm A=\bm V\bm\Lambda\bm V^{-1}
%\]
%where $\bm\Lambda=\diag(\lambda_1,\dots,\lambda_n)$.
%\end{definition}
%\end{remark}

\begin{proof}
The proof is by induction on $n$, i.e., suppose the results hold for all vector spaces with dimension no more than $n-1$, and we aim to show this result holds for dimension~$n$.
\begin{enumerate}
\item
\textbf{Step 1}:
Argue that there exists the associated eigenvector $\bm v$ of $\lambda_1$ under the linear operator $T$.

Consider any basis $\mathcal{M}$, by MAT2040, there exists associated eigenvector of $\lambda_1$, say $\bm y\in\mathbb{C}^n$ such that
\[
(T)_{\mathcal{M},\mathcal{M}}\cdot\bm y = \lambda_1\bm y
\]
Since the operator $(\cdot)_{\mathcal{M}}:V\to\mathbb{C}^n$ is an isomorphism, there exists $\bm v\in V\setminus\{\bm0\}$ such that $(\bm v)_{\mathcal{M}} = \bm y$. It follows that
\[
(T)_{\mathcal{M},\mathcal{M}}(\bm v)_{\mathcal{M}} = \lambda_1(\bm v)_{\mathcal{M}}\implies
(T\bm v)_{\mathcal{M}} = (\lambda_1\bm v)_{\mathcal{M}}\implies
T\bm v = \lambda_1\bm v
\]
\item
\textbf{Step 2}:
Dimensionality reduction of $\mathcal{X}_T(x)$:
Construct $W=\Span\{\bm v\}$, which is $T$-invariant.
By the proof of proposition~(\ref{pro:7:12}), we have 
%there is a basis of $V$,say $B:=\{\bm v,\bm h_2,\dots,\bm h_n\}$, such that
%\[
%(T)_{\mathcal{B},\mathcal{B}}=\begin{pmatrix}
%(T|_W)_{\{\bm v\}}&\times\\\bm0&(\tilde{T})_{\overline{\mathcal{B}},\overline{\mathcal{B}}}
%\end{pmatrix}
%=
%\begin{pmatrix}
%\lambda_1&\times\\\bm0&(\tilde{T})_{\overline{\mathcal{B}},\overline{\mathcal{B}}}
%\end{pmatrix}
%\]
%where 
$\tilde{T}:V/W\to V/W$ admits the characteristic polynomial
\[
\mathcal{X}_{\tilde{T}}(x) = (x-\lambda_2)\cdots(x-\lambda_n)
\]
\item
\textbf{Step 3:}
Applying the induction, 
there exists basis $\overline{\mathcal{C}}$ of $V/W$, i.e.,
\[
\overline{\mathcal{C}} = \{\bm w_2+W,\dots,\bm w_n+W\}
\]
such that
\[
(\tilde{T})_{\overline{\mathcal{C}},\overline{\mathcal{C}}}
=
\begin{pmatrix}
\lambda_2&\times&\times&\times\\
0&\lambda_3&\cdots&\times\\
0&\cdots&\ddots&\times\\
0&0&\cdots&\lambda_n
\end{pmatrix}
\]
\item
\textbf{Step 4:}
Therefore, we construct the set $\mathcal{A}:=
\{\bm v,\bm w_2,\dots,\bm w_n\}$.
We claim that 
\begin{itemize}
\item
$\mathcal{A}$ is a basis of $V$~(left as exercise in Hw2, Question 2)
\item
\[
(T)_{\mathcal{A},\mathcal{A}}=\begin{pmatrix}
\lambda_1&\times\\\bm0&(\tilde{T})_{\overline{\mathcal{C}},\overline{\mathcal{C}}}
\end{pmatrix}=\begin{pmatrix}
\lambda_1&\times&\times&\times\\
0&\lambda_2&\cdots&\times\\
0&\cdots&\ddots&\times\\
0&0&\cdots&\lambda_n
\end{pmatrix}
\]
(This statement is also left as exercise in Hw2, Question 5.)
\end{itemize}
\end{enumerate}
\end{proof}

\begin{proposition}\label{pro:7:13}
Suppose that $\mathcal{X}_T(x) = (x-\lambda_1)\cdots(x-\lambda_n)$, then $\mathcal{X}_T(T)=\bm0$.
\end{proposition}
\begin{remark}
One special case is that $\bm A = \diag(\lambda_1,\dots,\lambda_n)$.
The results for proposition~(\ref{pro:7:13}) gives
\[
(A-\lambda_1\bm I)\cdots(A-\lambda_n\bm I)\text{ is a zero matrix}
\]
\end{remark}

















