
\chapter{Week7}
\section{Field of Fractions}

An integral domain fails to be a field when there is a nonzero element with no multiplicative inverse. For example, the ring $\mathbb{Z}$ is not a field, since $2\in\mathbb{Z}$.

However, any nonzero $n\in\mathbb{Z}$ has a multiplicative inverse $\frac{1}{n}$ in $\mathbb{Q}$, which is a field.

Can we \emph{enlarge} a given integral domain to be a field, by formally adding multiplicative inverse to the ring?

\paragraph{Equivalence Relation}
Given an integral domain $R$ (commutative, with $1\ne0$), we consider the set $R\times R_{\ne0}:=\{(a,b):a,b\in R,b\ne0\}.$ Define a relation $\equiv$ on $R\times R_{\ne0}$:
\[
(a,b)\equiv(c,d)\mbox{ if }ad=bc.
\]
This equivalence relation is
\begin{enumerate}
\item
Reflexive: $(a,b)\equiv(a,b)$
\item
Symmetric: $(a,b)\equiv(c,d)$ implies $(c,d)\equiv(a,b)$
\item
Transitive: $(a,b)\equiv(c,d)$ and $(c,d)\equiv(e,f)$ implies $(a,b)\equiv(e,f)$
\end{enumerate}

Recall that the equivalence class of $a\in S$ is the set of all elements in $S$ which are equivalent to $a$. We use $[s]$ to label such a set. If $s\sim t$, then $[s]=[t]$. The equivalence class forms a disjoint union of $S$.

\begin{definition}[Quotient Set]
Given an equivalence relation on a set $S$, say $\sim$; the \emph{quotient set} $S/\sim$ is the set of all equivalence classes of $S$, w.r.t. $\sim$.
\end{definition}

Return to the $R\times R_{\ne0}$, we define the addition and multiplication as folllows:
\begin{align*}
(a,b)+(c,d)&:=(ad+bc,bd)\\
(a,b)\cdot(c,d)&:=(ac,bd)
\end{align*}


\begin{proposition}
Suppose $(a,b)\equiv(a',b')$ and $(c,d)\equiv(c',d')$, then
\begin{enumerate}
\item
$(a,b)+(c,d)\equiv(a',b')+(c',d')$
\item
$(a,b)\cdot(c,d)\equiv(a',b')\cdot(c',d')$
\end{enumerate}
\end{proposition}

Let $\mbox{Frac}(R)=(R\times R_{\ne0})/\equiv$, and define
\begin{align*}
[(a,b)]+[(c,d)]&=[(ad+bc,bd)]\\
[(a,b)]\cdot[(c,d)]&=[(ac,bd)]
\end{align*}

\begin{proposition}
The set $\mbox{Frac}(R)$ equipped with addition and multiplication defined above, forms a fieldm with additive identity $0:=[(0,1)]$ and mutliplicative identity $1:=[(1,1)].$ The multiplicative inverse of a nonzero element $[(a,b)]\in \mbox{Frac}(R)$ is $[(b,a)]$
\end{proposition}

\begin{definition}[Fraction Field]
The set $\mbox{Frac}(R)$ is called the \emph{fraction field} of $R$
\end{definition}
\begin{remark}
If we define $a/b\in\mathbb{Q}$ with $[(a,b)]\in\mbox{Frac}(\mathbb{Z})$, then $\mbox{Frac}(\mathbb{Z})=\mathbb{Q}$
\end{remark}

\subsection{Homomorphisms}
\begin{definition}[Unital Ring Homomorphism]
Let $(A,+_A,*_A)$ and $(B,+_B,*_B)$ be rings. A \emph{ring homomorphism} from $A$ to $B$ is a map $\phi: A\to B$ with the following properties:
\begin{enumerate}
\item
$\phi(1_A)=1_B$
\item
$\phi(a_1+_Aa_2) = \phi(a_1)+_B\phi(a_2)$, for $\forall a_1,a_2\in A$
\item
$\phi(a_1*_Aa_2) = \phi(a_1)*_B\phi(a_2)$ for $\forall a_1,a_2\in A$.
\end{enumerate}
\end{definition}
The ordinary ring homorphism is excluded the first condition.
\begin{remark}
If $\phi: A\to B$ is a homomorphism, then
\begin{enumerate}
\item
\[
1_B=\phi(1_A)=\phi(1_A+0_A)=\phi(1_A)+\phi(0_A)=1_B+\phi(0_A),
\]
which implies $\phi(0_A)=0_B$
\item
For $\forall a\in A$, $0_B=\phi(0_A)=\phi(a+(-a))=\phi(-a)+\phi(a)$, which implies $\phi(-a)=-\phi(a)$
\item
If $u$ is a unit in $A$, then $1=\phi(u\cdot u^{-1})=\phi(u)\cdot \phi(u^{-1})$; similarly, $1=\phi(u^{-1})\phi(u)$, which implies that $\phi(u)$ is a unit, with $\phi(u^{-1})=[\phi(u)]^{-1}$
\end{enumerate}
\end{remark}
\begin{example}
The map $\phi:\mathbb{Z}\to\mathbb{Q}$ defined by $\phi(n)=n$ is a homomorphism, since
\begin{enumerate}
\item
$\phi(1)=1$
\item
$\phi(n+_{\mathbb{Z}}m)=n+_{\mathbb{Q}}m$
\item
$\phi(n\cdot_{\mathbb{Z}}m)=n\cdot_{\mathbb{Q}}m$
\end{enumerate}
\end{example}
\begin{example}
For $\mathbb{Z}_m=\{0,1,\dots,m-1\}$ to be a ring,  with the additon law and multiplication law defined as:
\begin{align*}
s+_mt&=\overline{s+t}\\
s\times_mt&=\overline{s*t}
\end{align*}
Define the map $\phi:\mathbb{Z}\to\mathbb{Z}_m$ as follows:
\[
\phi(n)=\overline{n},\quad\forall n\in\mathbb{Z}
\]
then $\phi$ is a unital homomorphism.
\begin{enumerate}
\item
$\phi(1)=\overline{1}=1$
\item
$\phi(s+t)=\overline{s+t}=\overline{\bar s+\bar t}=\bar s+_m\bar t=\phi(s)+_m\phi(t)$
\item
$\phi(s\cdot t)=\overline{s\cdot t}=\overline{\bar s\cdot \bar t}=\bar s\cdot_m\bar t=\phi(s)\cdot_m\phi(t)$
\end{enumerate}
\end{example}
\begin{example}
For any ring $R$, define the map $\phi:\mathbb{Z}\to R$ as follows:
\begin{align*}
\phi(0)&=0_R\\
\phi(n)&=n\circ1_R=1_R+\cdots+1_R\\
\phi(-n)&=-n\circ 1_R=n\circ(-1_R)=(-1_R)+\cdots+(-1_R)
\end{align*}
Then $\phi$ is a homomorphism
\end{example}
Question.
\begin{example}
Let $R$ be a commutative ring. For each element $r\in R$, we may define the \emph{evalutaion map} $\phi_r:R[x]\to R$ as follows:
\[
\phi_r(\sum_{k=0}^na_kx^k)=\sum_{k=0}^na_kr^k
\]
Then the map $\phi$ is a homomorphism.
\end{example}
Question: unital.

\begin{definition}[Isomorphism]
If a ring homomorphism $\phi: A\to B$ is a bijective mapping, then we say that $\phi$ is an \emph{isomorphism}, and $A$ and $B$ are isomorphic as rings. We denote it as $A\cong B$.
\end{definition}
question: unital isomorphism.
\begin{proposition}
If $\phi: A\to B$ is an (unital) isomorphism, then $\phi^{-1}:B\to A$ is an (unital)isomorphism.
\end{proposition}
\begin{proof}
\begin{enumerate}
\item
Since $\phi(1_A)=1_B$, we have $\phi^{-1}(1_B)=\phi^{-1}(\phi(1_A))=1_A$
\item
The similar proof for the remaining condition.
\end{enumerate}
\end{proof}

The isomorphism keeps more algebraic structure. Two rings does not necessarily have isomorphism structure.
\begin{proposition}
$\mathbb{Z}$ and $\mathbb{Q}$ cannot be isomorphic.
\end{proposition}
\begin{proof}
Consider $2\in\mathbb{Q}$, which is a unit as well. Thus $\phi^{-1}(2)$ must be a unit of $\mathbb{Z}$.

The only units in $\mathbb{Z}$ are $\pm1$. Note that $\phi(1)=1\ne2$; and therefore $\phi(-1)=2$, and
\[
1=\phi((-1)\cdot(-1))=\phi(-1)\cdot\phi(-1)
\]
which is a contradiction.
\end{proof}
\begin{theorem}
The firlds $\mathbb{Q}$ and $\mbox{Frac}(\mathbb{Z})$ are (unital) \emph{isomorphic}.
\end{theorem}
\begin{proof}
Construct a mapping $\phi:\mathbb{Q}\to\mbox{Frac}(\mathbb{Z})$ as follows:
\[
\phi(a/b)=[(a,b)],\forall a/b\in\mathbb{Q},a,b\in\mathbb{Z},b\ne0
\]
\begin{enumerate}
\item
Firstly it is well-defined. For $a/b=c/d\in\mathbb{Q}$, we have $[(a,b)]=[(c,d)]$ since $ad=bc$.
\item
Then we show it is homomorphism.
\begin{enumerate}
\item
$\phi(1)=\phi(1/1)=[(1,1)],$ which is the multiplicative inverse of $\mbox{Frac}(\mathbb{Z})$
\item
$\phi(a/b+c/d)=\phi(a/b)+\phi(c/d)$
\item
$\phi(a/b\cdot c/d)=[(a,b)]\cdot[(c,d)]=\phi(a/b)\phi(c/d)$
\end{enumerate}
\item
Then we show it is bijective.
\begin{enumerate}
\item
For $\phi(a/b)=\phi(c/d)$, we have $[(a,b)]=[(c,d)]$, which implies $ad=bc$, i.e., $a/b=c/d$
\item
For $\forall[(a,b)]\in\mbox{Frac}(\mathbb{Z}),$ we have $a,b\in\mathbb{Z},b\ne0$. Then $a/b$ belongs to $\mathbb{Q}$, and $\phi(a/b)=[(a,b)]$.
\end{enumerate}
\end{enumerate}
\end{proof}
\begin{theorem}
If $F$ is a field, then $\mbox{Frac}(F)\cong F$
\end{theorem}
\begin{proof}
Define a mapping $\phi: F\to\mbox{Frac}(F)$ as follows:
\[
\phi(s)=[(s,1)],\quad \forall s\in F
\]
\begin{enumerate}
\item
For $a,b\in F$, we have
\[
\phi(a+b)=[(a+b,1)],\quad
\phi(a)+\phi(b)=[(a+b,1)]
\]
and
\[
\phi(a*b)=\phi(a)\phi(b)
\]
and
\[
\phi(1_F)=[(1_F,1_F)]
\]
\item
Also, $\phi$ is a bijective.
\end{enumerate}
\end{proof}
\begin{definition}[Kernel]
A \emph{kernel} of a ring homomorphism $\phi:A\to B$ is the set
\[
\mbox{ker}\phi=\{a\in A\mid\phi(a)=0\}
\]
The \emph{image} of $\phi$ is the set
\[
\mbox{im}\phi=\{b\in B\mid b=\phi(a),\mbox{ for some }a\in A\}
\]
\end{definition}
\begin{proposition}
A ring homomorphism $\phi:A\to B$ is one-to-one iff $\mbox{ker}\phi=\{0\}$
\end{proposition}
\begin{proof}
For the forward direction, since $\phi(0)=\phi(a)=0$ for $a\in\mbox{ker}\phi$, we have $a=0$;

For the reverse direction, if $\phi(a)=\phi(a')$, then $0=\phi(a)-\phi(a')=\phi(a-a')$, which implies $a-a'\in\mbox{ker}\phi$, i.e., $a-a'=0$.
\end{proof}

\begin{definition}[Ring of polynomials]
Let $R$ be a commutative ring, let $R[x,y]$ denote the ring of polynomials in $x,y$ with coefficients in $R$:
\[
R[x,y]=\left\{
\sum_{i=0}^m\sum_{j=0}^na_{ij}x^iy^j:m,n\in\mathbb{Z}^+,a_{ij}\in R
\right\}
\]
\end{definition}
\begin{proposition}
$R[x,y]$ is isomorphic to $R[x][y]$, where $R[x][y]$ is the ring of polynomials in $y$ with coefficients in the ring $R[x]$
\end{proposition}
\begin{proof}
Define a map $\phi:R[x,y]\to R[x][y]$ as follows:
\[
\phi\left(
\sum_{i=0}^m\sum_{j=0}^na_{ij}x^iy^j
\right)
=
\sum_{j=0}^n\left(
\sum_{i=0}^ma_{ij}x^i
\right)y^j
\]
It is clear that $\phi$ is homomorphism; then we show it is bijective.
\begin{enumerate}
\item
For $f=\sum_{i=0}^m\sum_{j=0}^na_{ij}x^iy^j\in\mbox{ker}\phi$, we have
\[
\phi(f)=\sum_{j=0}^n\left(
\sum_{i=0}^ma_{ij}x^i
\right)y^j=0_{R[x][y]}=\sum_{j=0}^n\left(
\sum_{i=0}^ma_{ij}x^i
\right)y^j=\sum_{j=0}^n0_{R[x]}y^j
\]
which follows that
\[
\sum_{i=0}^ma_{ij}x^i=0_{R[x]}=\sum_{i=0}^m0_Rx^i,0\le j\le n
\]
which implies $a_{ij}=0_R$ for $0\le i\le m$ and $0\le j\le n$. Therefore $\mbox{ker}\phi=\{0\}$, i.e., $\phi$ is one-to-one mapping.

Question: in exam do we need to show $0_{R[x][y]}$ is such that $a_{ij}=0$?
\item
Given $g=\sum_{j=0}^np_jy^j\in R[x][y]$, where $p_j\in R[x]$, suppose $m=\max_j\mbox{deg}(p_j)$, then
\[
g=\sum_{j=0}^n(\sum_{i=0}^ma_{ji}x^i)y^j
\]
It is clear that
\[
\phi(\sum_{j=0}^n\sum_{i=0}^ma_{ji}x^iy^j)=g
\]
\end{enumerate}
\end{proof}

\begin{definition}[Ideal]
An \emph{ideal} $I$ in a commutative ring $R$ is a subset of $R$ that satisfies
\begin{enumerate}
\item
$0\in I$
\item
If $a,b\in I$, then $a+b\in I$
\item
For all $a\in I$, we have $ar\in I$ for all $r\in R$.
\end{enumerate}
If an ideal $I$ is a proper subset of $R$, then we say it is \emph{proper ideal}.
\end{definition}
\begin{remark}
If an ideal $I$ contains $1$, then $r=1r\in I$ for $\forall r\in R$, i.e., $I=R$.
\end{remark}
\begin{definition}[Generalized ideal]
An \emph{ideal} $I$ in a commutative ring $R$ is a subset of $R$ that satisfies
\begin{enumerate}
\item
$(I,+)$ forms a group
\item
For any $r\in R$, we have $rI\subseteq I$ and $Ir\subseteq I$
\end{enumerate}
\end{definition}

 \begin{example}
For any commutative ring $R$, the set $\{0\}$ is an ideal, since $0+0=0$ and $0\cdot r=0$
 \end{example}
\begin{example}
\begin{enumerate}
\item
For $\forall m\in\mathbb{Z}$, the set $I=m\mathbb{Z}$ is an ideal:
\begin{enumerate}
\item
$(I,+)$ forms a group
\item
For any $r\in\mathbb{Z}$, we have $rI\subseteq I$ and $Ir\subseteq I$
\end{enumerate}
\item
Recall the ring homomorphism $\phi:\mathbb{Z}\to\mathbb{Z}_m$ defined by $\phi(n)=\bar n$, we claim that the kernel of $\phi$ is
\[
\mbox{ker}\phi=m\mathbb{Z}
\]
\end{enumerate}
\end{example}
\begin{proposition}
Let $A$ be a (commutative) ring. If $\phi: A\to B$ is a unital ring homomorphism, then $\mbox{ker}\phi$ is an ideal of $A$.
\end{proposition}
\begin{proposition}
A nonzero commutative ring $R$ is a field if and only if its only ideals are $\{0\}$ and $R$.
\end{proposition}
\begin{proof}
For the forward direction, show that $1\in I$ if $I$ is nonzero; for the reverse direction, show that any nonzero element of $R$ is a unit.
\end{proof}
\begin{proposition}
Let $k$ be a field, and $R$ a nonzero ring. Any unital ring homomorphism $\phi:k\to R$ is necessarily one-to-one.
\end{proposition}
\begin{proof}
Since $R$ is not a zero ring, it contains $1\ne0$, and $\phi(1)=1\ne0$, which implies $\mbox{ker}\phi$ is a proper ideal of $k$. Since $k$ is a field, we imply $\mbox{ker}\phi=\{0\}$. Thus $\phi$ is one-to-one.
\end{proof}

















