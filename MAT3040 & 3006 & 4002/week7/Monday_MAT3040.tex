\chapter{Week7}
\section{Monday for MAT3040}\index{Monday_lecture}
\paragraph{Reviewing}
Define the characteristic polynomial for an linear operator $T$:
\[
\mathcal{X}_T(x)= \det((T)_{\mathcal{A},\mathcal{A}} - x\bm I)
\]
We will use the notation ``$I/\bm I$'' in two different occasions:
\begin{enumerate}
\item
$I$ denotes the identity transformation from $V$ to $V$ with $I(\bm v)=\bm v,\forall\bm v\in V$
\item
$\bm I$ denotes the identity matrix $(I)_{\mathcal{A},\mathcal{A}}$, defined based on any basis $\mathcal{A}$.
\end{enumerate}

\subsection{Minimal Polynomial}
\begin{definition}[Linear Operator Induced From Polynomial]
Let $f(x):=a_mx^m+\cdots+a_0$ be a polynomial in $\mathbb{F}[x]$, and $T:V\to V$ be a linear operator.
Then the mapping
\[
f(T)=a_mT^m+\cdots+a_1T+a_0I:\quad
V\to V,
\]
is called a linear operator induced from the polynomial $f(x)$.
\end{definition}
\begin{remark}
\begin{enumerate}
\item
The composition of linear operators is not abelian, e.g., in general $S\circ T=T\circ S$ does not hold.
The reason follows similarly from the fact that square-matrix multiplication is not abelian in general.
\item
However, we always have $f(T)T=Tf(T)$, where $f(T)$ is a linear operator induced from the polynomial $f(x)$:
\begin{proof}
We can show that $T^nT=TT^n,\forall n$ by induction.
Suppose that $f(x)=\sum_ia_ix^i$, which follows that
\[
f(T)T=\sum_ia_iT^iT=\sum_ia_iTT^i=T\sum_ia_iT^i=Tf(T).
\]
\end{proof}
\item
We can generalize the statement in (2) into the fact that the composition of linear operators induced from polynomials is abelian, i.e., 
\[
f(T)g(T)=g(T)f(T)
\]
for any polynomials $f(x),g(x)$.
\end{enumerate}


\end{remark}






\begin{definition}[Minimal Polynomial]
Let $T:V\to V$ be a linear operator.
The \emph{minimal polynomial} $m_T(x)$ is a \emph{nonzero monic polynomial} 
of least (minimal) degree such that 
\[
m_T(T)=\bm0_{V\to V}.
\]
where $\bm0_{V\to V}$ denotes the zero vector in $\text{Hom}(V,V)$.
\end{definition}

\begin{example}
\begin{enumerate}
\item
Let $\bm A=\begin{pmatrix}
1&0\\0&1
\end{pmatrix}$, then $\bm A$ defines a linear operator:
\[
\begin{array}{ll}
A:&\mathbb{F}^2\to\mathbb{F}^2\\
\text{with}&\bm x\mapsto\bm A\bm x
\end{array}
\]
Here $\mathcal{X}_{ A}(x) = (x-1)^2$ and $\bm A-\bm I=\bm0$, which gives $m_A(x)=x-1$.
\item
Let $\bm B=\begin{pmatrix}
1&1\\0&1
\end{pmatrix}$, which implies
\[
\mathcal{X}_{ B}(x)=(x-1)^2,
\]
The question is that can we get the minimal polynomial with degree 1?

The answer is no, since $\bm B-k\bm I=\begin{pmatrix}
1-k&1\\0&1-k
\end{pmatrix}\ne\bm0$.

In fact, $m_B(x) = (x-1)^2$, since
\[
(\bm B-\bm I)^2=\begin{pmatrix}
0&1\\0&0
\end{pmatrix}^2=\begin{pmatrix}
0&0\\0&0
\end{pmatrix}.
\]
\end{enumerate}
\end{example}
%
%\begin{proposition}
%If $\bm A$ and $\bm B$ are similar, then
%\[
%m_{\bm A}(x)=m_{\bm B}(x).
%\]
%\end{proposition}
%\begin{proof}
%Left as exercise.
%\end{proof}
Two questions naturally arises:
\begin{enumerate}
\item
Does $m_T(x)$ exist? If exists, is it unique?
\item
What's the relationship between  $m_T(x)$ and $\mathcal{X}_T(x)$?
\end{enumerate}
Regarding to the first question, the minimal polynomial $m_T(x)$ may not exist, if $V$ has infinite dimension:
\begin{example}
Consider $V=\mathbb{R}[x]$ and the mapping
\[
\begin{array}{ll}
T:&V\to V\\
&p(x)\mapsto\int_0^xp(t)\diff t
\end{array}
\]
In particular, $T(x^n)=\frac{1}{n+1}x^{n+1}$.
Suppose $m_T(x)$ is with degree $n$, i.e., 
\[
m_T(x) = x^n +\cdots+a_1x+a_0,
\]
then 
\[
m_T(T)=T^n+\cdots+a_0I\ \text{is a zero linear transformation}
\]
It follows that
\[
[m_T(T)](x) = \frac{1}{n!}x^n+a_{n-1}\frac{1}{(n-1)!}x^{n-1}+\cdots+a_1x+a_0=0_{\mathbb{F}},
\]
which is a contradiction since
the coefficients of $x^k$ is nonzero on LHS for $k=1,\dots,n$, but zero on the RHS.
\end{example}

\begin{proposition}\label{pro:7:1}
The minimal polynomial $m_T(x)$ always exists for $\dim(V)=n<\infty$.
\end{proposition}
\begin{proof}
It's clear that $\{I,T,\dots,T^n,T^{n+1},\cdots,T^{n^2}\}\subseteq\text{Hom}(V,V).$
Since $\dim(\text{Hom}(V,V))=n^2$, we imply $\{I,T,\dots,T^n,T^{n+1},\cdots,T^{n^2}\}$ is linearly dependent, i.e., there exists $a_i$'s that are not all zero such that
\[
a_0I+a_1T+\cdots+a_{n^2}T^{n^2}=0
\]
i.e., there is a polynomial $g(x)$ of degree less than $n^2$ such that $g(T)=0$.

The proof is complete.
\end{proof}
%\paragraph{Assumption}
%We will asssume $V$ has finite dimension from now on.

\begin{proposition}\label{pro:7:2}
The minimal polynomial $m_T(x)$, if exists, then it exists uniquely.
\end{proposition}
\begin{proof}
Suppose $f_1,f_2$ are two distinct minimal polynomials with $\text{deg}(f_1)=\text{deg}(f_2)$.
It follows that
\begin{itemize}
\item
$\text{deg}(f_1-f_2)<\text{deg}(f_1)$.
\item
$f_1-f_2\ne0$
\item
$(f_1-f_2)(T) = f_1(T) - f_2(T)=0_{V\to V}$
\end{itemize}
By scaling $f_1-f_2$, there is a monic polynomial $g$ with lower degree satisfying $g(T)=0,$
which contradicts the definition for minimal polynomial.
\end{proof}

\begin{proposition}\label{pro:7:3}
Suppose $f(x)\in\mathbb{F}[x]$ satisfying $f(T)=\bm0$, then
\[
m_T(x)\mid f(x).
\]
\end{proposition}
\begin{proof}
It's clear that $\text{deg}(f)\ge\text{deg}(m_T)$.
The division algorithm gives 
\[
f(x)=q(x)m_T(x)+r(x).
\]
Therefore, for any $\bm v\in V$
\[
[r(T)](\bm v) = [f(T)](\bm v) - [q(T)m_T(T)](\bm v)=\bm0_V-q(T)\bm0_{V}=\bm0_V-\bm0_V=\bm0_{V}
\]
Therefore, $r(T) = \bm0_{V\to V}$.
By definition of minimal polynomial, we imply $r(x)\equiv0$.
\end{proof}
\begin{proposition}\label{pro:7:4}
If $\bm A,\bm B\in\mathbb{F}^{n\times n}$ are similar to each other, then $m_A(x) = m_B(x)$.
\end{proposition}
\begin{proof}
Suppose that $\bm B= \bm P^{-1}\bm A\bm P$, and that 
\[
\begin{array}{ll}
m_A(x)=x^k+\cdots+a_1x+a_0,
&
m_B(x)=x^{\ell}+\cdots+b_0.
\end{array}
\]
It follows that
\begin{align*}
m_A(\bm B)&=\bm B^k+\cdots+a_0I\\
&=\bm P^{-1}\bm A^k\bm P+\cdots+a_0\bm P^{-1}\bm P\\
&=\bm P^{-1}(\bm A^k+\cdots+a_0\bm I)\bm P\\
&=\bm P^{-1}(m_{A}(\bm A))\bm P
\end{align*}
Therefore, $m_A(\bm B)=\bm0$ since $m_{A}(\bm A)=\bm0$. By proposition~(\ref{pro:7:3}), we imply $m_{B}(x)\mid m_{A}(x)$. 
Similarly, $m_A(x)\mid m_B(x)$. 
Since $m_A(x)$ and $m_B(x)$ are monic, we imply $m_A(x)=m_B(x)$.
\end{proof}
\begin{remark}
Proposition~(\ref{pro:7:4}) claims that the minimal polynomial is \emph{similarity-invariant}; actually, the characteristic polynomial is \emph{similarity-invariant} as well.
\end{remark}
\paragraph{Assumption}
We will asssume $V$ has finite dimension from now on.
Now we study the vanishing of a single vector $\bm v\in V$.
\paragraph{Notation}
The $m_T(x)$ is a nonzero monic poylnomial of least degree such that 
\[
m_T(T)=\bm0_{V\to V}.
\]
\subsection{Minimal Polynomial of a vector}

\begin{definition}[Minimal Polynomial of a vector]
Similar to the minimal polynomial, we define the \emph{minimal polynomial of a vector $\bm v$ relative to $T$}, say $m_{T,\bm v}(x)$, as the monic polynomial of least degree such that 
\[
m_{T,\bm v}(T)(\bm v)=0
\]
\end{definition}
The existence of minimal polynomial of a vector is due to the existence of minimal polynomial; the uniqueness follows similarly as in proposition~(\ref{pro:7:2}).

\begin{proposition}
Let $T:V\to V$ be a linear operator and $\bm v\in V$.
The degree of the minimal polynomial of a vector is upper bounded by:
\[
\text{deg}(m_{T,\bm v}(x))\le \dim (V).
\]
\end{proposition}
\begin{proof}
It's clear that $\{\bm v,T\bm v,\dots,T^n\bm v\}\subseteq V$ and the proof follows similarly as in proposition~(\ref{pro:7:1}).
\end{proof}

Similar to the division property in proposition~(\ref{pro:7:3}), we have the division proprty for minimal polynomial of a vector:
\begin{proposition}\label{pro:7:6}
Suppose $f(x)\in\mathbb{F}[x]$ satisfying $f(T)(\bm v)=\bm0_{V}$, then
\[
m_{T,\bm v}(x)\mid f(x).
\]
In particular, $m_{T,\bm v}\mid m_T(x)$.
\end{proposition}

\begin{proof}
The proof follows similarly as in proposition~(\ref{pro:7:3}).
\end{proof}

\begin{proposition}
Suppose that $m_{T,\bm v}(x)= f_1(x)f_2(x)$, where $f_1,f_2$ are both monic. Let $\bm w = f_1(T)\bm v$, then
\[
m_{T,\bm w}(x) = f_2(x)
\]
\end{proposition}
\begin{proof}
\begin{enumerate}
\item
\[
f_2(T)\bm w = f_2(T)f_1(T)\bm v = m_{T,\bm v}(T)\bm v=\bm0
\]
By the proposition~(\ref{pro:7:3}), we imply 
$
m_{T,\bm w}|f_2
$.
\item
On the other hand,
\[
\bm0 = m_{T,\bm w}(T)(\bm w) = m_{T,\bm w}(T)f_1(T)\bm v= f_1(T)m_{T,\bm w}(T)\bm v,
\]
which implies that $m_{T,\bm v}(x)\mid f_1(x)m_{T,\bm w}(x),$, i.e.,
\[
f_1\cdot f_2\mid f_1\cdot m_{T,\bm w}\implies
f_2\mid m_{T,\bm w}.
\]
The proof is complete.
\end{enumerate}
\end{proof}















