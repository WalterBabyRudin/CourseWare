
\section{Wednesday for MAT3006}\index{Wednesday_lecture}

\begin{proposition}[Fatou's Lemma]
Suppose $\{f_n\}$ is a sequence of measurable, nonnegative functions.
\[
\lim_{n\to\infty}\inf \int f_n\diff m
\ge
\int
\lim_{n\to\infty}\inf(f_n)\diff m
\]
\end{proposition}

\begin{proof}
Define $g_n(x) := \inf_{k \geq n} f_k(x)$ and 
\[
f(x)=\lim_{n\to\infty}\inf f_n(x)
=\lim_{n\to\infty}\left(\inf_{k\ge n}f_k(x)\right):=\lim_{n\to\infty}g_n(x)
\]
To study the integral $\int f\diff m$, we will only focus on $f(x)$ on $E\subseteq\mathbb{R}$, where $f(x)>0,\forall x\in E$. 

It suffices to show that $\int_E\phi\diff m\le \lim_{n\to\infty}\inf\int_Ef_n\diff m$ for all simple $\phi$ satisfying $0\le \phi(x)\le f(x),\forall x\in E$.
(Then taking supremum both sides leads to the desired result.)

\begin{enumerate}
\item
Construct the simple function $\phi'$ on $E$ such that 
\[
\phi'(x)=\left\{
\begin{aligned}
\phi(x)-\varepsilon,&\quad\text{if $\phi(x)>0$}\\
0,&\quad\text{if $\phi(x)=0$}
\end{aligned}
\right.
\]
in which we pick $\varepsilon$ small enough such that $\phi(x)-\varepsilon\ge0$.

As a result, $\phi'<f,\forall x\in E$ (why?).
\item
Note that $g_n(x)$ is monotone increasing with $n$, and therefore convergent to $f(x)$.
Consider $A_n:=\{x\in E\mid \phi'(x)\le g_n(x)\}$, which follows that
\begin{itemize}
\item[(a)]
$A_n\subseteq A_{n+1}$
\item[(b)]
$\cup_{n=1}^\infty A_n=E$ (We do need $\phi'$ is strictly less than $f$ to obtain this condition).
\end{itemize}
Therefore, for any $k\ge n$,
\[
\int_{A_n}\phi'\diff m\le \int_{A_n}g_n\diff m\le 
\int_{A_n}f_k\diff m,
\]
which implies $\int_{A_n}\phi'\diff m\le \int_Ef_k\diff m$ since $f_k\mathcal{X}_{A_n}\le f_k\mathcal{X}_{E}$.
Or equivalently,
\begin{equation}\label{Eq:10:2}
\int_{A_n}\phi'\diff m\le\inf_{k\ge n}\int_Ef_k\diff m
\end{equation}
\item
Taking limits $n\to\infty$ both sides for (\ref{Eq:10:2}):
\begin{itemize}
\item
For LHS, suppose that $\phi'=\sum_i\alpha_i\mathcal{X}_{c_i}$, then $\int_{A_n}\phi'\diff m=\sum_i\alpha_i m(c_i\cap A_n)$, which follows that
\[
\lim_{n\to\infty}\int_{A_n}\phi'\diff m
=
\sum_i\alpha_i\lim_{n\to\infty}m(c_i\cap A_n)
=
\sum_i\alpha_im(c_i)=\int_E\phi'\diff m
\]
\item
The limit of RHS equals $\lim_{n\to\infty}\inf\int_Ef_n\diff m$, and therefore 
\[
\int_E\phi'\diff m\le \lim_{n\to\infty}\inf\int_Ef_n\diff m
\]
\end{itemize}
Note that the goal is to show $\int_E\phi\diff m\le \lim_{n\to\infty}\inf\int_Ef_n\diff m$, and therefore we need to evaluate $\phi'$ in terms of $\phi$.
\item
\begin{enumerate}
\item
Consider the case where $m(\phi^{-1}(0,\infty))=P<\infty$, then
\[
\int_E\phi'\diff m = \int_E\phi\diff m-\varepsilon\cdot P\le \lim_{n\to\infty}\inf\int_Ef_n\diff m,
\]
for all small $\varepsilon>0$. Then the desired result holds.
\item
Consider the case where $m(\phi^{-1}(0,\infty)) = \infty$, and we write the canonical form $\phi=\sum\alpha_i\mathcal{X}_{c_i}$ with $\alpha_i>0$.
Define $C=\cup_ic_i$ such that $m(c)=\infty$.

Construct the simple function $\phi'=a\mathcal{X}_C$, where $a:=\frac{1}{2}\min\{\alpha_i\}$, which implies
\begin{itemize}
\item
$\phi'\le\phi$
\item
$\int_E\phi'\diff m=am(c)=\infty$, which follows that $\int_E\phi\diff m=\infty$.
\end{itemize}
Our goal is to show $\lim_{n\to\infty}\inf\int_Ef_n\diff m=\infty$.

Consider $B_n=\{x\in E\mid g_n(x)>a\}$, then $\cup B_n=E, B_n\subseteq B_{n+1}$.

Observe the inequality
\[
\int_{C\cap B_n}a\diff m\le \int_{B_n}a\diff m\le \int_{B_n}g_n\diff m\le \inf_{k\ge n}\int_Ef_n\diff m
\]
Taking $n\to\infty$ both sides. For LHS, by definition of $B_n$, the limit equals $\int_Ca\diff m = \int\phi'\diff m=\infty$;
and the limit of RHS equals to $\lim_{n\to\infty}\inf\int_Ef_n\diff m$, i.e.,
\[
\lim_{n\to\infty}\inf\int_Ef_n\diff m=\infty
\]
\end{enumerate}

\end{enumerate}

\end{proof}

\begin{theorem}[Monotone Convergence Theorem I]
Let $\{f_n\}$ be a sequence of non-negative measurable functions, with 
\begin{itemize}
\item
$f_n(x)$ being monotone increasing
\item
$f_n(x)\to f(x)$ pointwisely
\end{itemize}
Then we have
\[
\lim_{n\to\infty}\int f_n\diff m = \int\left(\lim_{n\to\infty}f_n\right)\diff m:=\int f\diff m
\]
\end{theorem}
\begin{proof}
\begin{itemize}
\item
On the one hand, for all $n\in\mathbb{N}$, we have
\[
f_n\le f\implies \int f_n\diff m\le\int f\diff m\implies
\lim_{n\to\infty}\sup \int f_n\diff m\le \int f\diff m
\]
\item
On the other hand, applying the Fatou's lemma,
\[
\int f\diff m:=
\int\left(\lim_{n\to\infty}\inf f_n\right)\diff m\le \lim_{n\to\infty}\inf \int f_n\diff m
\]
Togehter with the previous inequality, we imply
\[
\lim_{n\to\infty}\sup \int f_n\diff m\le \int f\diff m
\le
 \lim_{n\to\infty}\inf \int f_n
\]
\end{itemize}
Therefore, all inequalities above are equalities, and the limit exists since limsup and liminf coincides. Moreover,
\[
\lim_{n\to\infty}\int f_n\diff m=
 \int f\diff m.
\]
\end{proof}

From MCT I, the Lebesgue integral $\int f\diff m$ can be computed as follows:
\begin{itemize}
\item
Construct simple functions $\phi_n\le\phi_{n+1}$ with $\phi_n\to f$
\item
Evaluate $\int\phi_n\diff m$ and then $\int f\diff m = \lim_{n\to\infty}\int\phi_n\diff m$
\end{itemize}



\subsection{Consequences of MCT}
\begin{proposition}
The Lebesgue integral is finitely addictive for measurable non-negative functions.
In other words, suppose $f,g$ are measurable and nonnegative, then
\[
\int f\diff m+\int g\diff m=\int(f+g)\diff m
\]
\end{proposition}
\begin{proof}
Suppose we have simple increasing functions $\{\phi_n\}$ and $\{\psi_n\}$ such that $\phi_n\to f$ and $\psi_n\to f$. Then
\begin{subequations}
\begin{align}
\int(f+g)\diff m&=\lim_{n\to\infty}\int(\phi_n+\psi_n)\diff m\label{Eq:10:3:a}\\
&=\lim_{n\to\infty}\int\phi_n\diff m+\lim_{n\to\infty}\int\psi_n\diff m\label{Eq:10:3:b}\\
&=\int f\diff m+\int g\diff m\label{Eq:10:3:c}
\end{align}
where (\ref{Eq:10:3:a}) and (\ref{Eq:10:3:c}) is by applying MCT I; and (\ref{Eq:10:3:b}) is by definition of simple function.
\end{subequations}
\end{proof}
\begin{corollary}
The Lebesgue integral is linear defined for measurable, nonnegative functions.
In other words, suppose $f,g$ are measurable and nonnegative, then
\[
\int(af+bg)\diff m=a\int f\diff m+b\int g\diff m,
\]
for any $a,b\ge0$.
\end{corollary}
\begin{proposition}\label{pro:10:14}
The Lebesgue integral for non-negative continuous function on a bounded closed interval coincides with the Riemann integral.
In other words, let $f$ be a non-negative continuous function on $[a,b]$. then
\[
\int_{[a,b]}f\diff m = \int_a^bf(x)\diff x.
\]
\end{proposition}
We will extend this result into all proper Riemann integrable functions on $[a,b]$ soon.

\begin{proof}
Let $\phi_n$ be the simple function giving the Riemann lower sum of $f(x)$ with $2^n$ equal subintervals:
\[
\phi_n(x)=\sum_{k=1}^{2^n}\left(\min_{y\in\bar{I}_k}f(y)\right)\mathcal{X}_{I_k},\ \text{where }I_k=[a+(b-a)\frac{k-1}{2^n}, a+(b-a)\frac{k}{2^n}]
\]
\begin{itemize}
\item
$\phi_n(x)\ge0$ is monotone increasing~(that's the reason we should divde intervals into $2^n$ pieces instead of $n$ pieces)
\item
$\phi_n(x)\to f(x)$ pointwisely:
for any $x\in[a,b]$ and $\varepsilon>0$, by (uniform) continuity of $f$, there exists $\delta>0$ such that
\[
|y-x|<\delta\implies |f(y)-f(x)|<\varepsilon.
\]

Therefore, for sufficiently large $n$, we imply for any $x\in I_{k,n}$, $|I_{k,n}|<\delta$.
As a result, 
\[
\left|\min_{y\in I_{k,n}}f(y) - f(x)\right|<\varepsilon.
\]
\end{itemize}
Therefore,
\begin{align*}
\int_{[a,b]}f\diff m&=\lim_{n\to\infty}\int\phi_n\diff m
\\&=
\lim_{n\to\infty}\left[\text{Riemann lower integral of $\int_a^bf(x)\diff x$}\right]\\&=\int_a^bf(x)\diff x
\end{align*}
\end{proof}

\begin{example}
The Lebesgue integral gives us an alternative way to compute improper integrals.
Suppose that we want to compute the integral
\[
\int_0^1(1-x)^{-1/2}\diff x.
\]
\begin{enumerate}
\item
The old method is that we know the integral
\[
\int_0^{1-1/n}(1-x)^{-1/2}\diff x\text{ exists for any $n$}.
\]
Then we extend the definition of Riemann integration by taking limit of $n$:
\[
\int_0^1f(x)\diff x=\lim_{n\to\infty}\int_0^{1-1/n}(1-x)^{-1/2}
=
\lim_{n\to\infty}2-2\sqrt{\frac{1}{n}}=2
\]
\item
The Lebesgue integration does not require us to extend the definition.
Consider
\[
f_n(x)=(1-x)^{-1/2}\mathcal{X}_{[0,1-1/n]}
\]
Then
\begin{itemize}
\item
$f_n(x)\to f(x)$ on $[0,1)$
\item
$f_n(x)$ is monotone increasing
\end{itemize}
Therefore, by applying MCT I,
\[
\int_{[0,1)}(1-x)^{-1/2}\diff x=\lim_{n\to\infty}\int f_n\diff m=\lim_{n\to\infty}\int_{[0,1-1/n]}(1-x)^{-1/2}\diff x.
\]
By Proposition~(\ref{pro:10:14}), $\int_{[0,1-1/n]}(1-x)^{-1/2}\diff x=\int_0^{1-1/n}(1-x)^{-1/2}\diff x$ for all $n$.
Therefore, we conclude that the Lebesgue integral is equal to the (improper) Riemann integral in this case.
\end{enumerate}
\end{example}





























