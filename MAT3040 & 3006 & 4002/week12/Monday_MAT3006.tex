
\section{Monday for MAT3006}\index{Monday_lecture}
\subsection{Remarks on MCT}
\begin{example}
The MCT can help us to compute the integral
\begin{align*}
\lim_{n\to\infty}\int_{[0,n\pi]}\cos\left(\frac{x}{2n}\right)xe^{-x^2}\diff x
\end{align*}

Construct $f_n(x) = \cos\left(\frac{x}{2n}\right)xe^{-x^2}\mathcal{X}_{[0,n\pi]}$.
\begin{itemize}
\item
Since $\cos(x/2n)<\cos(x/2(n+1))$ for any $x\in[0,n\pi]$, we imply $f_n$ is monotone increasing with $n$
\item
$f_n(x)$ is integrable for all $n$.
\item
$f_n$ converges pointwise to $xe^{-x^2}\mathcal{X}_{[0,\infty)}$
\end{itemize}
Therefore, MCT I applies and
\[
\lim_{n\to\infty}\int_{[0,n\pi]}\cos\left(\frac{x}{2n}\right)xe^{-x^2}\diff x
=
\int\left(\lim_{n\to\infty}f_n\right)\diff m
\]
with
\[
\lim_{n\to\infty}f_n = xe^{-x^2}\mathcal{X}_{[0,\infty)}.
\]
Moreover, 
\begin{subequations}
\begin{align}
\int\left(\lim_{n\to\infty}f_n\right)\diff m &= 
\lim_{m\to\infty}\int_{[0,m]}xe^{-x^2}\diff x\label{Eq:12:1}\\
&=\int_0^\infty xe^{-x^2}\diff x\\
&=\frac{1}{2}
\end{align}
where (\ref{Eq:12:1}) is by applying MCT I with $g_m(x) = xe^{-x^2}\mathcal{X}_{[0,m]}$ and proposition~(\ref{pro:10:14}) to compute a Lebesgue integral by evaluating a proper Riemann integral.
\end{subequations}
\end{example}

Then we discuss the Lebesgue integral for series:

\begin{corollary}[Lebesgue Series Theorem]
Let $\{f_n\}$ be a series of measurable functions such that
\[
\sum_{n=1}^\infty\int|f_n|\diff m<\infty,
\]
then $\sum_{n=1}^kf_n$ converges to an integrable function $f = \sum_{n=1}^\infty f_n$ a.e., with
\[
\int f\diff m = \sum_{n=1}^\infty\int f_n\diff m
\]
\end{corollary}
\begin{proof}
\begin{itemize}
\item
For each $f_n$, consider 
\[
f_n = f_n^+ - f_n^-,\ \text{where $f_n^+,f_n^-$ are nonnegative}.
\]
By proposition~(\ref{pro:11:6}), 
\[
\int\sum_{n=1}^\infty f_n^+\diff m = \sum_{n=1}^\infty\int f_n^+\diff m\le \sum_{n=1}^\infty\int |f_n|\diff m<\infty.
\]
Therefore, $f^+:=\sum_{n=1}^\infty f_n^+=\lim_{k\to\infty}\sum_{n=1}^kf_n^+$ is integrable.
The same follows by replacing $f^{+}$ with $f^{-}$.
By corollary~(\ref{cor:9:6}), $f^+(x),f^-(x)<\infty,\forall x\in U$, where $U^c$ is null.
\item
Therefore, construct 
\[
f(x)=\left\{
\begin{aligned}
f^+(x)-f^-(x),&\quad x\in U\\
0,&\quad x\in U^c
\end{aligned}
\right.
\]
Moreover, for $x\in U$, 
\begin{align*}
f(x)&=\left(\lim_{k\to\infty}\sum_{n=1}^kf_n^+(x)\right)-
\left(\lim_{k\to\infty}\sum_{n=1}^kf_n^-(x)\right)\\
&=\lim_{k\to\infty}
\left(
\sum_{n=1}^kf_n^+(x)
-
\sum_{n=1}^kf_n^-(x)
\right)\\
&=\lim_{k\to\infty}\left[\sum_{n=1}^k(f_n^+(x)-f_n^-(x))\right]
\\&=
\sum_{n=1}^\infty f_n(x)
\end{align*}
where the first equality is because that both terms are finite.
\item
It follows that
\begin{subequations}
\begin{align}
\int f\diff m&=\int f^+\diff m - \int f^-\diff m\label{Eq:12:2:a}\\
&=
\int\sum_{n=1}^\infty f_n^+\diff m -\int\sum_{n=1}^\infty f_n^-\diff m\\
&=\left(\sum_{n=1}^\infty\int f_n^+\diff m\right)
-
\left(\sum_{n=1}^\infty\int f_n^-\diff m\right)\label{Eq:12:2:c}\\
&=\sum_{n=1}^\infty
\left(
\int f_n^+\diff m -\int f_n^-\diff m
\right)\label{Eq:12:2:d}\\
&=\sum_{n=1}^\infty\int f_n\diff m\label{Eq:12:2:e}
\end{align}
where (\ref{Eq:12:2:a}),(\ref{Eq:12:2:d}) is because that summation/subtraction between series holds when these series are finite; (\ref{Eq:12:2:c}) is by proposition~(\ref{pro:11:6}); (\ref{Eq:12:2:e}) is by definition of $f_n$.
\end{subequations}
\end{itemize}
\end{proof}

\begin{example}
Compute the integral
\[
\int_{(0,1]}e^{-x}x^{\alpha-1}\diff x,\ \alpha>0.
\]
\begin{itemize}
\item
Construct $f_n(x) = (-1)^n\frac{x^{\alpha+n-1}}{n!}\mathcal{X}_{(0,1]}, n\ge0$, and
\[
\sum_{n=0}^N f_n(x)\to e^{-x}x^{\alpha-1}, \ \text{pointwisely}, x\in(0,1]. 
\]
By applying MCT I,
\[
\int|f_n|\diff m=\frac{1}{(\alpha+n)n!}
\]
Therefore, 
\[
\sum_{n=0}^\infty\int|f_n|\diff m=
\sum_{n=0}^\infty\frac{1}{(\alpha+n)n!}<\infty
\]
\item
Applying the Lebesgue Series Theorem,
\[
\int_{(0,1]}e^{-x}x^{\alpha-1}\diff x = 
\int_{(0,1]}(\sum_{n=0}^\infty f_n)\diff m
=
\sum_{n=0}^\infty\int f_n\diff m=
\sum_{n=0}^\infty\frac{(-1)^n}{(\alpha+n)n!}
\]
\end{itemize}
\end{example}

\begin{remark}
It's essential to have $\sum\int|f|\diff m<\infty$ rather than $\sum\int f_n\diff m<\infty$ in the Lebesgue Series Theorem.
For example, let
\[
f_n=\frac{(-1)^{n+1}}{(n+1)}\mathcal{X}_{[n,n+1)}
\implies
\sum_{n=1}^\infty\int f_n\diff m =\log(2)<\infty
\]
However, $f:=\sum f_n$ is not integrable.
\end{remark}

\subsection{Dominated Convergence Theorem}
\begin{theorem}
Let $\{f_n\}$ be a sequence of measruable functions such that $|f_n|\le g$ a.e., and $g$ is integrable.
Suppose that $\lim_{n\to\infty}f_n(x)=f(x)$ a.e., then
\begin{enumerate}
\item
$f$ is integrable,
\item
\[
\int f\diff m =\lim_{n\to\infty}\int f_n\diff m
\]
\end{enumerate}
\end{theorem}
\begin{proof}
\begin{itemize}
\item
Observe that 
\[
|f_n|\le g\implies
\lim_{n\to\infty}|f_n|\le g\implies |f|\le g
\]
By comparison test, $g$ is integrable implies $|f|$ is integrable, and further $f$ is integrable.
\item
Consider the sequence of non-negative functions
$\{g-f_n\}_{n\in\mathbb{N}}$ and $\{g+f_n\}_{n\in\mathbb{N}}$.

By Fatou's Lemma, 
\begin{align*}
\lim_{n\to\infty}\inf\int(g-f_n)\diff m&\ge \int \lim_{n\to\infty}\inf(g-f_n)\diff m\\
&=\int(g-f)\diff m\\
&=\int g\diff m - \int f\diff m
\end{align*}
which follows that
\[
\int g\diff m - \lim_{n\to\infty}\sup\int f_n\diff m
\ge
\int g\diff m - \int f\diff m
\]
i.e.,
\[
\int f\diff m\ge  \lim_{n\to\infty}\sup\int f_n\diff m
\]
\item
Similarly, 
\[
\lim_{n\to\infty}\inf(g+f_n)\diff m\ge \int\lim_{n\to\infty}\inf(g+f_n)\diff m
=
\int g\diff m + \int f\diff m
\]
which implies
\[
 \lim_{n\to\infty}\inf\int f_n\diff m\ge\int f\diff m
\]
\end{itemize}
As a result,
\[
 \lim_{n\to\infty}\sup\int f_n\diff m
 \le
 \int f\diff m\le  \lim_{n\to\infty}\inf\int f_n\diff m,
\]
which implies
\[
\int f\diff m = \lim_n\int f_n\diff m
\]
\end{proof}

\begin{corollary}[Bounded Convergence Theorem]
Suppose that $E\in\mathcal{M}$ be such that $m(E)<\infty$.
If
\begin{itemize}
\item
$|f_n(x)|\le K<\infty$ for any $x\in E,n\in\mathbb{N}$
\item
$f_n\to f$ a.e. in $E$,
\end{itemize}
then $f$ is integrable in $E$ with
\[
\int_Ef\diff m = \lim_{n\to\infty}\int f_n\diff m
\]
\end{corollary}
\begin{proof}
Take $g=K\mathcal{X}_E$ in DCT.
\end{proof}


\begin{proposition}
Every Riemann integrable function $f$ on $[a,b]$ is Lebesgue integrable, without the condition that $f$ is continuous a.e.
\end{proposition}
\begin{proof}
Since $f$ is Riemann integrable, we imply $f$ is bounded.
We construct the Riemann lower abd upper functions with $2^n$ equal intervals, denoted as $\{\phi_n\}$ and $\{\psi_n\}$, which follows that
\begin{itemize}
\item
$\phi_n$ is monotone increasing;
$\psi_n$ is monotone decreasing;
\item
$\phi_n\le f\le \psi_n$, and
\[
\lim_{n\to\infty}\int_{[a,b]}\phi_n=\int_a^bf(x)\diff x = \lim_{n\to\infty}\int_{[a,b]}\psi_n.
\]
\end{itemize}
Construct $g=\sup_n\phi_n$ and $h=\inf_n\psi_n$.
Now we can apply the bounded convergence theorem:
\begin{itemize}
\item
$\phi_n$ is bounded on $[a,b]$
\item
$\phi_n\to g$ on $[a,b]$
\end{itemize}
which implies
$g$ is Lebesgue integrable on $[a,b]$, with 
\[
\int_{[a,b]}g\diff m = \lim_{n\to\infty}\int_{[a,b]}\phi_n\diff m=\int_a^bf(x)\diff x.
\]
Similarly, $h$ is Lebesgue integrable, with
\[
\int_{[a,b]}h\diff m = \lim_{n\to\infty}\int_{[a,b]}\psi_n\diff m=\int_a^bf(x)\diff x.
\]
Moreover, $g\le f\le h$, and
\[
\int_{[a,b]}(h-g)\diff m = \int_{[a,b]}h\diff m - \int_{[a,b]}g\diff m=\int_a^bf(x)\diff x-\int_a^bf(x)\diff x=0,
\]
which implies $h=g$ a.e., and further $f=g$ a.e., which implies
\[
\int_{[a,b]} f\diff m = \int_{[a,b]} g\diff m= \int_a^bf(x)\diff x.
\]
\end{proof}
\begin{remark}
However, an improper Riemann integral does not necessarily has the corresponding Lebesgue integral:
\[
f(x)=\sum_{n=1}^\infty (-1)^nn\cdot\mathcal{X}_{(1/(n+1),1/n]},\ x\in[0,1]
\]
In this case, $f$ is Riemann integrable but not Lebesgue integrable.
\end{remark}













