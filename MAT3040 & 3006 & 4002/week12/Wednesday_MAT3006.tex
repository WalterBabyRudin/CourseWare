
\section{Wednesday for MAT3006}\index{Wednesday_lecture}
\subsection{Riemann Integration $\&$ Lebesgue Integration}
\begin{example}
Compute the integral
\[
L=\lim_{n\to\infty}\int_0^1\frac{nx\log(x)}{1+n^2x^2}\diff x.
\]
Let $f_n(x) = \frac{nx\log(x)}{1+n^2x^2}\mathcal{X}_{(0,1]}$, which is continuous on $[0,1]$, i.e., integrable on $[0,1]$.
The goal is to show $L=0$.
\begin{itemize}
\item
Note that $f_n(x)\to0,\forall x\in[0,1]$ pointwisely, as $n\to\infty$.
\item
Note that $t/(1+t^2)\le\frac{1}{2},\forall t\ge0$. Take $t=nx$, we imply
\[
|f_n(x)| \le\frac{1}{2}|\log(x)|\mathcal{X}_{(0,1]}
\]
We claim that $\frac{1}{2}|\log(x)|\mathcal{X}_{(0,1]}:=-\frac{1}{2}\log(x)\mathcal{X}_{(0,1]}$ is integrable:
by MCT I,
\[
\int-\frac{1}{2}\log(x)\mathcal{X}_{(0,1]}\diff m
=
\lim_{n\to\infty}\int_{1/n}^1-\frac{1}{2}\log(x)\diff x
=\frac{1}{2}<\infty.
\]
\end{itemize}
Therefore, the DCT applies, and
\[
\lim_{n\to\infty}\int_{(0,1]}\frac{nx\log(x)}{1+n^2x^2}\diff x
=
\int_{(0,1]}\lim_{n\to\infty}\frac{nx\log(x)}{1+n^2x^2}\diff x
=
\int_{(0,1]}0\diff x= 0
\]
However, $f_n(x)$ does not converge to $f(x)\equiv0$ uniformly on $[0,1]$:
\[
\sup_{0\le x\le1}|f_n(x)-0|
\ge
|f_n(1/n) - 0|
=
\frac{1}{2}\log(n)\to\infty, \text{as $n\to\infty$}
\]
Therefore, we cannot switch integral symbol and limit by using the tools in MAT2006.
\end{example}

\begin{proposition}
Suppose that $f(x)$ is a proper Riemann integrable function on $[a,b]$. Then $f(x)$ is Lebesgue integrable on $[a,b]$ with 
\[
\int_{[a,b]}f\diff m = \int_a^bf(x)\diff x.
\]
\end{proposition}
\begin{proof}
Since $f$ is properly Riemann inregrable, we imply $f(x)$ is bounded on $[a,b]$, i.e., $|f(x)|\le K,\forall x\in[a,b]$.
Construct the Riemann lower and upper functions with $2^n$ equal subintervals, denoted as $\phi_n,\psi_n$, which follows that
\begin{itemize}
\item
$\phi_n(x)\le f(x)\le\psi_n(x),\forall n$
\item
$\phi_n(x)$ is monotone increasing
\item
$\psi_n(x)$ is monotone decreasing
\end{itemize}
Now apply bounded convergence theorem on $\psi_n-\phi_n$:
\begin{itemize}
\item
$|\psi_n(x)-\phi_n(x)|\le 2K$ on $[a,b]$
\item
$\psi_n-\phi_n\to \psi-\phi$
\end{itemize}
which implies
\begin{align*}
\int|\psi-\phi|\diff m &= \int\psi-\phi\diff m\\ &= \lim_{n\to\infty}\int \psi_n-\phi_n\diff m
=\lim_{n\to\infty}\int \psi_n\diff m - \lim_{n\to\infty}\int \phi_n\diff m
\\&=\text{Riemann Upper Sum} - \text{Riemann Lower Sum}\\
&=0
\end{align*}
Therefore, $\int|\psi-\phi|\diff m=0$ implies $\psi(x)=\phi(x)$ a.e.
By sandwich theorem,
\[
\psi(x)=f(x)=\phi(x)\text{ a.e.}
\]
Therefore, 
\[
\int f\diff m = \int\phi\diff m = \lim_{n\to\infty}\int\phi_n\diff m=\int_a^bf(x)\diff x
\]
where the second equality is by MCT II.


\end{proof}


\begin{remark}
The improper Riemann integrable functions $f(x)$ is not necessarily Lebesgue integrable.
However, if we assume $f(x)\ge0$, then $f(x)$ is improper Riemann integrable implies $f(x)$ is Lebesgue integrable, with the same integral value.
\begin{proof}[Proof Outline]
Suppose $f(x)$ is improper Riemann integrable on $[a,b]$, where $a,b\in\mathbb{R}\cup\{\pm\infty\}$.
\begin{itemize}
\item
Construct $f_n=f\mathcal{X}_{[a_n,b_n]}$, with $[a_{n},b_n]\subseteq[a_{n+1},b_{n+1}]\subseteq\cdots\subseteq[a,b]$.
\item
By previous proposition, $f_n$ is proper Riemann integrable implies $f_n$ is Lebesgue integrable.
\item
Then we apply the MCT I to $\{f_n\}$.
\end{itemize}
\end{proof}

\end{remark}
\subsection{Continuous Parameter DCT}

\begin{theorem}[Continous parameter DCT]\label{The:12:4}
Let $I,J\subseteq\mathbb{R}$ be intervals, and 
$f:I\times J\to\mathbb{R}$ be such that 
\begin{enumerate}
\item
for fixed $y\in J$, the function $f(x):=f(x,y)$ is an integrable function over $I$.
\item
for fixed $y\in J$, 
\[
\lim_{y'\to y}f(x,y')=f(x,y)
\]
for almost all $x\in I$
\item
There exists integrable $g(x)$~(do not depend on $y$) such that for all $y\in J$,
\[
|f(x,y)|\le g(x)
\]
for almost all $x\in I$.
\end{enumerate}
As a result, 
\[
F(y) = \int_If(x,y)\diff x
\]
is a continuous function on $J$.
\end{theorem}
\begin{remark}
Note that the integrability of $f(x)$ in hypothesis~(1) can be weaken into the measurability of $f(x)$:
The measurability of $f(x)$ together with hypothesis~(3), and DCT implies the integrability of $f(x)$.
\end{remark}

\begin{proof}
Let $\{y_n\}$ be a sequence on $J$ such that $y_n\to y$.
It suffices to show $F(y_n)\to F(y)$.

Construct $f_n(x) = f(x,y_n)$, which follows that
\begin{itemize}
\item
$f_n(x)$ is integrable for all $n$ (by hypothesis~(1))~(why check integrable)
\item
$|f_n(x)|\le g(x)$ a.e. for all $n$, and $g(x)$ is integrable (by hypothesis~(3))
\item
By hypothesis~(2),
\[
\lim_{n\to\infty}f_n(x) = f(x,y)
\]
\end{itemize}

Therefore, the DCT applies, and
\[
\lim_{n\to\infty}\int_If_n(x,y_n)\diff m =\int\lim_{n\to\infty} f_n(x,y_n)\diff m=\int_If(x,y)\diff m
\]
Or equivalently,
\[
\lim_{n\to\infty}F(y_n)= F(y)
\]
\end{proof}

\begin{example}
Consider $f(x,y) = e^{-x}x^{y-1}$ with $I\times J = (0,\infty)\times[m,M]$, where $0<m<M<\infty$.
We will study the integral
\[
\Gamma(y) = \int_0^\infty e^{-x}x^{y-1}\diff x
\]
We check the hypothesis in the Theorem~(\ref{The:12:4}):
\begin{enumerate}
\item
For fixed $y\in[m,M]$, $f(x):=f(x,y)$ is indeed measurable on $(0,\infty)$, since $f(x)$ is continous on $(0,\infty)$.
%\[
%\left(e^{-x}x^{k-1}\right)\mathcal{X}_{(0,\infty)}\le 1\cdot x^{k-1}\mathcal{X}_{(0,K]}+
%10e^{-x/2}\mathcal{X}_{[K,\infty)}
%\]
%where $K$ is a sufficiently large number in $(0,\infty)$.
\item
The hypothesis~(2) follows directly from the contiuity of $f(x,y)$
\item
\begin{align*}
|f(x,y)|&\le e^{-x}x^{m-1}\mathcal{X}_{[0,1]}+e^{-x}x^{M-1}\mathcal{X}_{(1,\infty)}\\
&\le x^{m-1}\mathcal{X}_{[0,1]}+e^{-x}x^{M-1}\mathcal{X}_{(1,\infty)}
\end{align*}
Here $x^{m-1}\mathcal{X}_{[0,1]}$ is integrable. 
Following the similar argument in~(1), we imply $e^{-x}x^{M-1}\mathcal{X}_{(1,\infty)}$ is integrable as well.
\end{enumerate}
Therefore, $\Gamma(y)$ is continuous for any $m\le y\le M$.
Since the choice of $0<m<M<\infty$ is arbitrary, we imply $T(y)$ is continous on $(0,\infty)$.

In the next lecture we wish to show that
\[
F'(y) = \int_I\frac{\partial f}{\partial y}(x,y)\diff x
\]
\end{example}











