\subsection{Solution to Assignment Two}
\begin{enumerate}
\item
\begin{proof}[Proof.]
\begin{proof}[Sufficiency.]\qquad \\
If $\bm M$ is invertible, then there exists matrix $\bm N$ such that $\bm M\bm N = \bm N\bm M = \bm I$.
\[\implies
(\bm{ABC})\bm N = \bm I,\bm N(\bm{ABC})=\bm I.
\implies
\bm A(\bm{BCN}) = \bm I,(\bm{NAB})\bm C = \bm I.
\]
$\implies \bm{BCN}$ is the right inverse of $\bm A$, $\bm{NAB}$ is the left inverse of $\bm C$. 
\\Hence $\bm A$ and $\bm C$ is invertible.\\
Moreover, $(\bm{ABC})\bm N = \bm I \implies (\bm{AB})\bm{CN} = \bm I$. Hence $\bm{CN}$ is the right inverse of $\bm{AB}$. 
\\Hence $\bm{AB}$ is invertible. Hence there exists $(\bm{AB})^{-1}$ such that $((\bm{AB})^{-1})(\bm{AB}) = \bm I$.\\$\implies ((\bm{AB})^{-1}\bm A)\bm B = \bm I$. Hence $(\bm{AB})^{-1}\bm A$ is the left inverse of $\bm B$. \\Hence $\bm B$ is invertible.
\begin{proof}[Necessity.]\qquad \\
If $\bm A,\bm B,\bm C$ is invertible, then there exist $\bm A^{-1},\bm B^{-1},\bm C^{-1}$ such that \[\bm A\bm A^{-1} = \bm I,\bm B\bm B^{-1} = \bm I,\bm C\bm C^{-1} = \bm I.\]
\[
\begin{split}
\implies \bm{ABC}(\bm C^{-1}\bm B^{-1}\bm A^{-1}) &=\bm{AB}(\bm C\bm C^{-1})(\bm B^{-1}\bm A^{-1}) = \bm{AB}\bm I(\bm B^{-1}\bm A^{-1}) 
\\&= \bm{AB}(\bm B^{-1}\bm A^{-1}) = \bm A(\bm B\bm B^{-1})\bm A^{-1} = \bm A\bm I\bm A^{-1} \\
&= \bm A\bm A^{-1} = \bm I.
\end{split}
\]
Hence $\bm C^{-1}\bm B^{-1}\bm A^{-1}$ is the right inverse of $\bm{ABC}$. Hence $\bm{ABC}$ is invertible.
\end{proof}
\end{proof}
\end{proof}
\item
\begin{proof}[Solution.]
The inverse are respectively given by 
\[
\begin{array}{lll}
\begin{bmatrix}
\bm I&\bm 0\\-\bm C&\bm I
\end{bmatrix},
&
\begin{bmatrix}
\bm A^{-1}&\bm 0\\ -\bm D^{-1}\bm C\bm A^{-1}&\bm D^{-1}
\end{bmatrix},
&
\begin{bmatrix}
-\bm D&\bm I\\\bm I&\bm 0
\end{bmatrix}.
\end{array}
\]
\begin{itemize}
\item
\[
\begin{bmatrix}
\bm I&\bm 0\\\bm C&\bm I
\end{bmatrix}
\begin{bmatrix}
\bm I&\bm 0\\-\bm C&\bm I
\end{bmatrix} = \begin{bmatrix}
\bm I\bm I+\bm 0(-\bm C)&\bm I\bm 0+\bm 0\bm I\\
\bm C\bm I+\bm I(-\bm C)&\bm C\bm 0+\bm I\bm I
\end{bmatrix}= \begin{bmatrix}
\bm I&\bm 0\\\bm 0&\bm I
\end{bmatrix}
\]
Hence $\begin{bmatrix}
\bm I&\bm 0\\-\bm C&\bm I
\end{bmatrix}$ is the right inverse of $\begin{bmatrix}
\bm I&\bm 0\\\bm C&\bm I
\end{bmatrix}$, hence $\begin{bmatrix}
\bm I&\bm 0\\-\bm C&\bm I
\end{bmatrix}$ is the inverse of $\begin{bmatrix}
\bm I&\bm 0\\\bm C&\bm I
\end{bmatrix}$.


\item
\[
\begin{bmatrix}
\bm A&\bm 0\\\bm C&\bm D
\end{bmatrix}\begin{bmatrix}
\bm A^{-1}&\bm 0\\ -\bm D^{-1}\bm C\bm A^{-1}&\bm D^{-1}
\end{bmatrix}
 = \begin{bmatrix}
\bm A\bm A^{-1}+\bm 0(-\bm D^{-1}\bm C\bm A^{-1})&\bm A\bm 0+\bm 0\bm D^{-1}\\
\bm C\bm A^{-1}+\bm D(-\bm D^{-1}\bm C\bm A^{-1})&\bm C\bm 0+\bm D\bm D^{-1}
\end{bmatrix}
=\begin{bmatrix}
\bm I&\bm 0\\\bm 0&\bm I
\end{bmatrix}
\]
Hence $\begin{bmatrix}
\bm A^{-1}&\bm 0\\ -\bm D^{-1}\bm C\bm A^{-1}&\bm D^{-1}
\end{bmatrix}$ is the right inverse of $\begin{bmatrix}
\bm A&\bm 0\\\bm C&\bm D
\end{bmatrix}$, hence  $\begin{bmatrix}
\bm A^{-1}&\bm 0\\ -\bm D^{-1}\bm C\bm A^{-1}&\bm D^{-1}
\end{bmatrix}$ is the inverse of $\begin{bmatrix}
\bm A&\bm 0\\\bm C&\bm D
\end{bmatrix}$.
\item
\[
\begin{bmatrix}
\bm 0&\bm I\\\bm I&\bm D
\end{bmatrix}\begin{bmatrix}
-\bm D&\bm I\\\bm I&\bm 0
\end{bmatrix} = 
\begin{bmatrix}
\bm 0(-\bm D)+\bm I\bm I & \bm 0\bm I+\bm I\bm 0\\
\bm I(-\bm D)+\bm D\bm I & \bm I\bm I+\bm D\bm 0
\end{bmatrix} = \begin{bmatrix}
\bm I&\bm 0\\\bm 0&\bm I
\end{bmatrix}
\]
Hence $\begin{bmatrix}
-\bm D&\bm I\\\bm I&\bm 0
\end{bmatrix}$ is the right inverse of $\begin{bmatrix}
\bm 0&\bm I\\\bm I&\bm D
\end{bmatrix}$, hence $\begin{bmatrix}
-\bm D&\bm I\\\bm I&\bm 0
\end{bmatrix}$ is the inverse of $\begin{bmatrix}
\bm 0&\bm I\\\bm I&\bm D
\end{bmatrix}$.
\end{itemize}
\end{proof}

\item
\begin{proof}[Solution.]
Firstly, we do Elimination for this matrix:
\[
\begin{bmatrix}
2&c&c\\c&c&c\\8&7&c
\end{bmatrix}
\xLongrightarrow[\bm E_{21} = \begin{bmatrix}
1&0&0\\0&1&0\\-\frac{c}{2}&0&1
\end{bmatrix}]{\bm E_{31} = \begin{bmatrix}
1&0&0\\-4&1&0\\0&0&1
\end{bmatrix}}
\begin{bmatrix}
2&c&c\\0&c-\frac{c^2}{2}&c-\frac{c^2}{2}\\0&7-4c&-3c
\end{bmatrix}
\]
Notice that $c-\frac{c^2}{2}\ne 0$, otherwise the second row has no nonzero entries, the Gaussian Elimination cannot continue.
\[
\xLongrightarrow{\bm E_{32} = \begin{bmatrix}
1&0&0\\0&1&0\\0&\frac{4c-7}{c-c^2/2}&1\end{bmatrix}}
\begin{bmatrix}
2&c&c\\0&c-\frac{c^2}{2}&c-\frac{c^2}{2}\\0&0&c-7
\end{bmatrix}
\]
In order to continue the Gaussian Elimination, we have to let three pivots not equal to zero, hence we have $c-\frac{c^2}{2}\ne0, c-7\ne 0.$\\
Hence $c\ne 0, c\ne 2, c\ne 7.$
\end{proof}
\item
\begin{proof}[Solution.]\qquad 
\begin{enumerate}
\item
True, because if the whole row has no nonzero entries, the pivot in this row doesn't exist, the Gaussian Elimination cannot continue, hence there doesn't exist the inverse.
\item
False, for example, for matrix $\bm A = \begin{bmatrix}
1&1\\1&1
\end{bmatrix}$, if we do elimination, we obtain \[\begin{bmatrix}
1&1\\1&1
\end{bmatrix}\implies \begin{bmatrix}
1&1\\0&0
\end{bmatrix}\]so we cannot continue Gaussian Elimination as the second row has no pivot, hence $\bm A$ is not invertible.
\item
True, if $\bm A$ is invertible, we have $\bm A\bm A^{-1} = \bm I$. Hence $\bm A$ is the left inverse of $\bm A^{-1}$. Hence $\bm A$ is the inverse of $\bm A^{-1}$.
\item
True, if $\bm A\trans$ is invertible, there exists $\bm B$ such that $\bm B\bm A\trans = \bm I$.
\[
\implies (\bm B\bm A\trans)\trans = (\bm A\trans)\trans(\bm B)\trans = \bm A\bm B\trans = \bm I
\]
Hence $\bm B\trans$ is the right inverse of $\bm A$. Hence $\bm B$ is the inverse of $\bm A$.
\end{enumerate}
\end{proof}
\end{enumerate}