
\section{Wednesday for MAT3040}\index{Wednesday_lecture}
\subsection{Tensor Product for Linear Transformations}

\begin{proposition}
Suppose that $T:V\to V'$ and $S:W\to W'$ are linear transformations, then there exists an unique linear transformation 
\[
\begin{array}{ll}
T\otimes S:&V\otimes W\to V'\otimes W'\\
\text{satisfying}&(T\otimes S)(v\otimes w) = T(v)\otimes S(w)
\end{array}
\]
\end{proposition}

\begin{proof}
We construct the mapping
\[
\begin{array}{ll}
T\times S:&V\times W\to V'\otimes W'\\
\text{with}&(T\times S)(v,w) = T(v)\otimes S(w)
\end{array}
\]
This mapping is indeed bilinear: for instance, we can show that 
\[
(T\times S)(av_1+bv_2,w) = a(T\times S)(v_1,w)+b(T\times S)(v_2,w)
\]
Therefore, $T\times S\in\text{Obj}$. Since the tensor product satisfies the universal property, we imply there exists an unique linear transformation
\[
\begin{array}{ll}
T\otimes S&V\otimes W\to V'\otimes W'\\
\text{satisfying}&(T\otimes S)(v\otimes w)=T(v)\otimes S(w)
\end{array}
\]


\end{proof}
\paragraph{Notation Warning}
Does the notion $T\otimes S$ really form a tensor product, i.e., do we obtain the addictive rules for tensor product such as 
\[
(aT_1+bT_2)\otimes S = a(T_1\otimes S)+b(T_2\otimes S)?
\]


\begin{example}\label{exp:13:2}
Let $V=V'=\mathbb{F}^2$ and $W=W'=\mathbb{F}^3$.
Define the matrix-multiply mappings:
\[\left\{
\begin{array}{ll}
T:&V\to V\\
\text{with}&\bm v\mapsto\bm A\bm v\\
&\bm A=\begin{pmatrix}
a&b\\c&d
\end{pmatrix}
\end{array}\right.\qquad
\left\{
\begin{array}{ll}
S:&W\to W\\
\text{with}&\bm w\mapsto\bm B\bm w\\
&\bm B=\begin{pmatrix}
p&q&r\\
s&t&u\\
v&w&x
\end{pmatrix}
\end{array}
\right.
\]
How does $T\otimes S:V\otimes W\to V\otimes W$ look like?
\begin{itemize}
\item
Suppose $\{e_1,e_2\},\{f_1,f_2,f_3\}$ are usual basis of $V,W$, respectively.
Then the basis of $V\otimes W$ is given by:
\[
\mathcal{C}=\{e_1\otimes f_1,e_1\otimes f_2,e_1\otimes f_3,e_2\otimes f_1,e_2\otimes f_2,e_2\otimes f_3\}.
\]
\item
As a result, we can compute $(T\otimes S)(e_i\otimes f_j)$ for $i=1,2$ and $j=1,2,3$. For instance,
\begin{align*}
(T\otimes S)(e_1\otimes e_1)&=T(e_1)\otimes S(e_1)\\
&=(ae_1+ce_2)\otimes(pe_1+se_2+ve_3)\\
&=
(ap)e_1\otimes e_1+(as)e_1\otimes e_2+(av)e_1\otimes e_3+(cp)e_2\otimes e_1+(cs)e_2\otimes e_2+(cv)e_2\otimes e_3
\end{align*}
\item
Therefore, we obtain a matrix representation for the linear transformation $(T\otimes S)$:

\end{itemize}
We want a matrix representation for $(T\otimes S)$:
\[
(T\otimes S)_{\mathcal{C},\mathcal{C}}
=
\begin{pmatrix}
aB&bB\\
cB&dB
\end{pmatrix},
\]
which is a large matrix formed by taking all possible products between the elements of $\bm A$ and those of $\bm B$.
This operation is called the \emph{Kronecker Tensor Product}, see the command \textit{kron} in MATLAB for detail.


\end{example}
\begin{proposition}
More generally, given the linear operator $T:V\to V$ and $S:W\to W$, 
let $\mathcal{A}=\{v_1,\dots,v_n\},\mathcal{B}=\{w_1,\dots,w_m\}$ be a basis of $V,W$ respectively, with
\[
\begin{array}{ll}
(T)_{\mathcal{A},\mathcal{A}}=(a_{ij})
&
(S_{\mathcal{B},\mathcal{B}})=(b_{ij}):=B
\end{array}
\]
As a result, $(T\otimes S)_{\mathcal{C},\mathcal{C}}=A\otimes B$, where 
$\mathcal{C}=\{v_1\otimes w_1,\dots, v_n\otimes w_m\}$, and $A\otimes B$ denotes the Kronecker tensor product, defined as the matrix
\[
\begin{pmatrix}
a_{1,1}B&\cdots&a_{1,n}B\\
\vdots&\ddots&\vdots\\
a_{n,1}B&\cdots&a_{n,n}B
\end{pmatrix}.
\]
\end{proposition}
\begin{proof}
Following the similar procedure as in Example~(\ref{exp:13:2}) and applying the relation
\begin{align*}
(T\otimes S)(v_i\otimes w_j)&=T(v_i)\otimes S(w_j)\\
&=\left(
\sum_{k=1}^na_{ki}v_k
\right)
\otimes
\left(
\sum_{\ell=1}^mb_{\ell j}w_\ell
\right)\\
&=\sum_{k=1}^n\sum_{\ell=1}^m(a_{ki}b_{\ell j})v_k\otimes w_{\ell}
\end{align*}
\end{proof}

\begin{proposition}
The operation $T\otimes S$ satisfies all the properties of tensor product.
For example,
\begin{align*}
(aT_1+bT_2)\otimes S &= a(T_1\otimes S)+b(T_2\otimes S)\\
T\otimes(cS_1+dS_2) &= c(T\otimes S_1)+d(T\otimes S_2)
\end{align*}
Therefore, the usage of the notion ``$\otimes$'' is justified for the definition of $T\otimes S$.
\end{proposition}
\begin{proof}[Proof using matrix multiplication]
For instance, consider the operation $(T+T')\otimes S$, with $(T)_{\mathcal{A},\mathcal{A}}=(a_{ij})$, $(T')_{\mathcal{A},\mathcal{A}}=(c_{ij}), (S)_{\mathcal{B},\mathcal{B}}=B$.

We compute its matrix representation directly:
\begin{align*}
((T+T')\otimes S)_{\mathcal{C},\mathcal{C}}
&=
(T+T')_{\mathcal{A},\mathcal{A}}\otimes (S)_{\mathcal{B},\mathcal{B}}\\
&=
[(T)_{\mathcal{A},\mathcal{A}}+(T')_{\mathcal{A},\mathcal{A}}]\otimes (S)_{\mathcal{B},\mathcal{B}}\\
&=
(T)_{\mathcal{A},\mathcal{A}}\otimes (S)_{\mathcal{B},\mathcal{B}}
+
(T')_{\mathcal{A},\mathcal{A}}\otimes (S)_{\mathcal{B},\mathcal{B}}
\end{align*}
where the last equality is by the addictive rule for kronecker product for matrices.
Therefore,
\[
((T+T')\otimes S)_{\mathcal{C},\mathcal{C}}=
(T\otimes S)_{\mathcal{C},\mathcal{C}} + 
(T'\otimes S)_{\mathcal{C},\mathcal{C}}
\implies
(T+T')\otimes S
=
T\otimes S+T'\otimes S
\]
\end{proof}
\begin{proof}[Proof using basis of $T\otimes S$]
Another way of the proof is by computing 
\[
((T+T')\otimes S)(v_i\otimes w_j),
\] 
where $\{v_i\otimes w_j\mid 1\le i\le n,1\le j\le m\}$ forms a basis of $(T+T')\otimes S$:
\begin{align*}
((T+T')\otimes S)(v_i\otimes w_j)
&=(T+T')(v_i)\otimes S(w_j)\\
&=(T(v_i)+T'(v_i))\otimes S(w_j)\\
&=T(v_i)\otimes S(w_j)+T'(v_i)\otimes S(w_j)\\
&=(T\otimes S)(v_i\otimes w_j)+(T'\otimes S)(v_i\otimes w_j)
\end{align*}
Since $((T+T')\otimes S)(v_i\otimes w_j)$ coincides with $(T\otimes S + T'\otimes S)(v_i\otimes w_j)$ for all basis vectors $v_i\otimes w_j\in\mathcal{C}$, we imply
\[
(T+T')\otimes S = T\otimes S+T'\otimes S
\]
\end{proof}


\begin{proposition}
Let $A,C$ be linear operators from $V$ to $V$, and $B,D$ be linear operators from $W$ to $W$, then
\[
(A\otimes B)\circ(C\otimes D)=(AC)\otimes(BD)
\]
\end{proposition}

\begin{proposition}
Define linear operators $A:V\to V$ and $B:W\to W$ with $\dim(V),\dim(W)<\infty$.
Then
\[
\det(A\otimes B) = (\det(A))^{\dim(W)}(\det(B))^{\dim(V)}
\]
\end{proposition}

\begin{corollary}
There exists a linear transformation 
\[
\begin{array}{ll}
\Phi:&
\text{Hom}(V,V)\otimes\text{Hom}(W,W)\to
\text{Hom}(V\otimes W,V\otimes W)\\
\text{with}&A\otimes B\mapsto A\otimes B
\end{array}
\]
where the input of $\Phi$ is the tensor product of linear transformations, and the output is the linear transformation.
\end{corollary}
\begin{proof}
Construct the mapping
\[
\begin{array}{ll}
\Phi&:
\text{Hom}(V,V)\times\text{Hom}(W,W)\to
\text{Hom}(V\otimes W,V\otimes W)\\
\text{with}&\Phi(A,B)=A\otimes B
\end{array}
\]
The $\Phi$ is indeed bilinear: for instance, 
\begin{align*}
\Phi(pA+qC,B)&=(pA+qC)\otimes B\\
&=p(A\otimes B)+q(C\otimes B)\\
&=p\Phi(A,B)+q\Phi(C,B)
\end{align*}
This corollary follows from the universal property of tensor product.
\end{proof}
\begin{remark}
If assuming that $\dim(V),\dim(W)<\infty$, we imply
\begin{align*}
\dim(\text{Input space of $\Phi$})&=\dim(\text{Hom}(V,V))\dim(\text{Hom}(W,W))\\
&=
[\dim(V)\dim(V)]
\cdot
[\dim(W)\dim(W)]
=
[\dim(V)\dim(W)]^2\\
&=[\dim(V\otimes W)]^2\\
&=\dim(\text{Hom}(V\otimes W,V\otimes W))\\
&=\dim(\text{Output space of $\Phi$})
\end{align*}
Therefore, is $\Phi$ is an isomorphism?
If so, then every linear operator $\alpha:V\otimes W\to V\otimes W$ can be expressed as
\[
\alpha = A_1\otimes B_1+\cdots+A_k\otimes B_k
\]
where $A_i:V\to V$ and $B_j:W\to W$.
\end{remark}










