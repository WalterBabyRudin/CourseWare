\section{Thursday}\index{week7_Thursday_lecture}



\subsection{Principal Ideal Domainas}
For a fixed finite set of elements $a_1,\dots,a_n$ in a commutative ring $R$, let $\gen{a_1,\dots,a_n}$ denote the subset:
\[
\{r_1a_1+\cdots+r_na_n\mid r_i\in R\}
\]
\begin{proposition}
The set $\gen{a_1,\dots,a_n}$ is an ideal of $R$.
\end{proposition}
\begin{proof}
\begin{enumerate}
\item
It forms a group.
\item
Given any $\sum_ir_ia_i\in I$, for any $r\in R$, we have
\[
r\sum_{i}r_ia_i=\sum_i(rr_i)a_i\in I.
\]
\end{enumerate}
\end{proof}
\begin{definition}
We call $\gen{a_1,\dots,a_n}$ the ideal \emph{generated} by $a_1,\dots,a_n$. An ideal $\gen{a}=\{ar\mid r\in R\}$ generated by one element $a\in R$ is called the \emph{principal ideal}.
\end{definition}
\begin{remark}
Note that $R=\gen{1}$ and $\{0\}:=\gen{0}$ are both principal ideals.
\end{remark}
\begin{theorem}
Every ideal in the ring $\mathbb{Z}$ is a principal ideal.
\end{theorem}
\begin{proof}
w.l.o.g., suppose $I$ contains nonzero element, say $a$. Then $-1\in\mathbb{Z}$ implies that $-a\in I$, and therefore $I$ contains at least one positive integer. Suppose $I$ contains a positive integer $d$ that is smaller than any other elements that is positive in $I$. We claim $I=\gen{d}$.

For any $a\in I$, we have $a=dp+r$ for $0\le r<d$, which implies that $r=a-dp$ lies in $I$, since $I$ is an ideal, which implies $d=0$, i.e., $a=dq$. Thus $I\subseteq\gen{d}$.

On the other hand, we have $dr\in I$ for any $r\in\mathbb{Z}$, i.e., $\gen{d}\subseteq\mathbb{Z}$
\end{proof}
\begin{proposition}
Given $a,b$ in a commutative ring $R$. If $b=au$ for some unit $u\in R$, then $\gen{a}=\gen{b}$. If $R$ is an integral domain and $\gen{a}=\gen{b}$, then $b=au$ fo some unit $u\in R$. 
\end{proposition}
\begin{proof}
For the case $b=0$, we imply $a=0$ and the result is trivial.

For $b\ne0$, there exists $u,v\in R$ such that $b=au$ and $a=bv$. Thus
\[
b=buv\implies b(1-uv)=0
\]
Since $R$ is an integral domain, and $b\ne0$, we have $1-uv=0$, which implies $uv=1$, o.e., $u$ is a unit.
\end{proof}

\begin{definition}[PID]
If $R$ is an integral domain in which every ideal is principal, we say that $R$ is a \emph{principal integral domain}.
\end{definition}
We claim that for any field $k$, the ring of polynomails $k[x]$ is also a PID.
\begin{proposition}
Let $R$ be a commutative ring. For $\forall d,f\in R[x]$ such that the leading coefficient of $d$ is a unit in $R$, then there exists $q,r\in R[x]$ such that
\[
f=qd+r,
\]
with $\mbox{deg} r<\mbox{deg} d$.
\end{proposition}
\begin{proof}
We prove this theorem by induction.

If $\mbox{deg} f<\mbox{deg} d$, take $r=f$ and $q=0$

Let $d=\sum_{i=0}^na_ix^i\in R[x]$ be fixed, where $a_n$ is a unit of $R$. For any given $f=\sum_{i=0}^mb_ix^i\in R[x]$, $m\ge n$, suppose the claim holds for any $f'$ with $\mbox{deg}f'<\mbox{deg}f$.

Construct $f'=f - a_n^{-1}b_mx^{m-n}d$, thus there exists $q',r'\in R[x]$ with $\mbox{deg}r'<\mbox{deg}d$ such that
\[
f - a_n^{-1}b_mx^{m-n}d = q'd+r'
\]
which implies 
\[
f = (q'+a_n^{-1}b_mx^{m-n})d+r'
\]
\end{proof}
\begin{theorem}
Let $k$ be a field, then $k[x]$ is a PID.
\end{theorem}
\begin{proof}
Let $I$ be an ideal of $k[x]$. Let $d$ be a nonzero polynomial in $I$ with the least leading degree. The existence of this polynomial is because the leading degree of a polynomail is a non-negative integer. I is clear that $\gen{d}\subseteq I$. It suffices to show $I\subseteq\gen{d}$

For $\forall f\in I$, we have $f=qd+r$ for some $q,r\in k[x]$ such that $\mbox{deg}(r)<\mbox{deg}(d)$. Then $r=f-qd$ lines in $I$. Since $d$ has the least degree, we imply $r=0$. Thus $f=qd$, which implies $f\in\gen{d}$. Thus $I\subseteq\gen{d}$.
\end{proof}

\subsection{Qotient Ring}
Let $R$ be a commutative ring. Let $I$ be an ideal of $R$. Define a relation $\sim$ on $R$ as follows:
\[
a\sim b,\mbox{ if }b-a\in I
\]
\begin{definition}[Congruent modulo]
If $a\sim b$, we say that $a$ is congruent modulo $I$ to $b$, and write
\[
a\equiv b(\bmod I)
\]
\end{definition}
\begin{proposition}
Congruence modulo $I$ is an equivalence relation.
\end{proposition}
\begin{proof}
\begin{enumerate}
\item
$a-a=0\in I$
\item
$a-b\in I$ implies $b-a=(-1)(a-b)\in I$
\item
$a-b,b-c\in I$ implies $(a-b)+(b-c)\in I$
\end{enumerate}
\end{proof}
\begin{definition}[Residue]
Let $R/I$ be the set of equivalence classes of $R$ w.r.t. the relation $\sim$. Each element in $R/I$ has the form
\[
\bar{r}=r+I=\{r+a\mid a\in I\},\qquad r\in R
\]
We call $\bar{r}$ as the \emph{residue} of $r$ in $R/I$. Note that $r\in I$ implies $\bar{r}=\bar{0}$.
\end{definition}
Observe that
\begin{align*}
(r+a)+(r'+a')&\in(r+r')+I=\overline{r+r'}\\
(r+a)(r'+a')\in rr'+I=\overline{rr'}
\end{align*}
Thus we define binary operation on $R/I$:
\begin{align*}
\bar{r}+\bar{r'}&=\overline{r+r'}\\
\bar r\cdot\bar{r'}&=\overline{rr'}
\end{align*}
\begin{proposition}
The set $R/I$ equipped with the addition and multiplicaiton defined above, is a \emph{commutative ring}.
\end{proposition}
\begin{proposition}
The mapping $\pi: R\to R/I$, defined by
\[
\pi(r)=\bar{r},\quad \forall r\in R
\]
 is a surjective ring homomorphism with the kernel $\mbox{ker}(\pi)=I$.
\end{proposition}

Let $m$ be a natural number. The set 
\[
m\mathbb{Z}=\{mn\mid n\in\mathbb{Z}\}
\]
is an ideal of $\mathbb{Z}$.
\begin{proposition}
The quotient ring $\mathbb{Z}/m\mathbb{Z}$ is isomorphic to $\mathbb{Z}_m$.
\end{proposition}
\begin{proof}
Define $r_m$ to be the remainder of the division of $r$ by $m$.

It is clear that $\bar{r}=\bar{r_m}$. We define a mapping $\phi:\mathbb{Z}_m\to\mathbb{Z}/m\mathbb{Z}$:
\[
\phi(r)=\bar{r},\qquad\forall r\in\mathbb{Z}_m
\]
We claim it is a homomorphism:
\begin{itemize}
\item
$\phi(1)=\bar{1}=1_{\mathbb{Z}/m\mathbb{Z}}$
\item
$\phi(r+_mr')=\overline{r+_mr'}=\overline{(r+r')_m}=\overline{r+r'}=\phi(r)+\phi(r')$
\item
$\phi(r\cdot_mr')=\phi(r)\phi(r')$
\end{itemize}

Then we show that $\phi$ is bijective:

For any $\bar{r}$ in $\mathbb{Z}/m\mathbb{Z}$, we have $\phi(r_m)=\bar{r}$

Suppose $\phi(r)=\bar r=0$ in $\mathbb{Z}/m\mathbb{Z}$, then $r\in m\mathbb{Z}$, which implies $r=0$.

\end{proof}
\begin{proposition}
Let $\phi:R\to R'$ be a ring homomorphism, then the image of $\phi$
\[
\mbox{im}\phi=\{r'\in R'\mid r'=\phi(r)\mbox{ for some }r\in R\}
\]
is a ring.
\end{proposition}

\begin{theorem}[First Isomorphism Theorem]
Let $R$ be a commutative ring, let $\phi:R\to R'$ be a ring homomorphism, then
\[
R/\mbox{ker}\phi\cong\mbox{im}\phi
\]
\end{theorem}
\begin{corollary}
If the ring homomorphism is surjective, $\phi:R\to R'$, then
\[
R'\cong R/\mbox{ker}\phi
\]
\end{corollary}
\begin{example}
For the map $\phi:\mathbb{Z}\to\mathbb{Z}_m$ defined by $\phi(n)=n_m$ for $\forall n\in\mathbb{Z}$, it is clear that $\phi$ is a surjective ring homomorphism, and $\mbox{ker}\phi=m\mathbb{Z}$. Thus
\[
\mathbb{Z}_m\cong\mathbb{Z}/m\mathbb{Z}
\]
\end{example}
question
\begin{example}
The ring $\mathbb{Z}[i]/(1+3i)$ is isomorphic to $\mathbb{Z}/10\mathbb{Z}$.

Define a mpap $\phi:\mathbb{Z}\to\mathbb{Z}[i]/(1+3i)$:
\[
\phi(n)=\bar n
\]
Show that $\mbox{ker}\phi=10\mathbb{Z}$, and therefore
\[
\mathbb{Z}/10\mathbb{Z}\cong \mathbb{Z}/10\mathbb{Z}
\]
\end{example}
\begin{example}
The rings $\mathbb{R}[x]/(x^2+1)$ and $\mathbb{C}$ are isomorphic.

Define a map from $\mathbb{R}[x]$ to $\mathbb{C}$:
\[
\phi(\sum_{k=0}^n a_kx^k)=\sum_{k=0}^na_ki^k
\]

Question: PID of $\mathbb{R}[x]$ implies $\mbox{ker}\phi=\gen{p}$ for some $p\in\mathbb{R}[x]$. Then show that $\mbox{ker}\phi=\gen{x^2+1}$.

By isomorphism theorem, $\mathbb{R}[x]/(x^2+1)\cong\mathbb{C}$.

\end{example}














