
\section{Monday for MAT3006}\index{Monday_lecture}
\subsection{Tonelli's and Fubini's Theorem}

\begin{proposition}\label{pro:14:1}
For all $E\in\mathcal{M}\otimes\mathcal{M}$,we have
\begin{equation}\label{Eq:14:1}
\int m_Y(E_x)\diff x =\int m_X(E_y)\diff y=\pi(E),
\end{equation}
where $\pi(\cdot)$ is a measure on $\mathcal{M}\otimes\mathcal{M}.$
\end{proposition}
Here note that 
\begin{align*}
m_X(E_y):&=\int(\mathcal{X}_E)_y(x)\diff x\\
m_Y(E_x):&=\int(\mathcal{X}_E)_x(y)\diff y
\end{align*}

\begin{proof}
Construct 
\[
\mathcal{A}=\left\{E\in\mathcal{M}\otimes\mathcal{M}\middle| 
\begin{array}{l}
x\mapsto m_Y(E_x)\text{ measurable}\\
y\mapsto m_X(E_y)\text{ measurable}\\
(\ref{Eq:14:1})\text{ holds for $E$}
\end{array}\right\}
\]
Following the proof given in the last lecture, it suffices to show $\mathcal{A}$ is a monotone class:
\begin{itemize}
\item
Construct 
\[
\mathcal{A}_k=\mathcal{A}\cap \{E\in\mathcal{M}\otimes\mathcal{M}\mid E\subseteq[-k,k]\times[-k,k]\}.
\]
We first show that $\mathcal{A}_k$ is a monotone class for all $k\in\mathbb{N}$:
\begin{enumerate}
\item
Suppose that $E_n\subseteq E_{n+1},\forall n$ and $E_n\in \mathcal{A}_k$, and we aim to show $E:=\cup_{n=1}^\infty E_n\in\mathcal{A}_k$.
Consider the function $f_n(x)=m_Y((E_n)_x)$, which is measurable for all $n$, and $f_n(x)\le f_{n+1}(x)$ for all $n$, since $E_n\subseteq E_{n+1}$.

The MCT I implies that $f(x)=m_Y(E_x)$ is measurable with
\[
\int m_Y(E_x)\diff x
=
\lim_{n\to\infty}\int m_Y((E_n)_x)\diff x
\stackrel{(a)}{=}
\lim_{n\to\infty}\pi(E_n)
\stackrel{(b)}{=}
\pi(E)
\]
where (a) is because that $E_n\in\mathcal{A}$; and (b) is due to the exercise in Hw3.
Similarly, $y\mapsto m_X(E_y)$ is measurable, with $\int m_X(E_y)\diff y=\pi(E)$.
Therefore, $E\in\mathcal{A}$, i.e., $E\in\mathcal{A}_k$ as well.
\item
Suppose that $F_i\in\mathcal{A}_k, F_i\supseteq F_{i+1}$, and we aim to show $F:=\cap_{i=1}^\infty F_i\in\mathcal{A}_k$.
Construct the measurable function $g_n(x)=m_Y((F_n)_x)$, and $g_n(x)\ge g_{n+1}(x)$;
$|g_n(x)|\le g_1(x)$, with $g_1(x)$ integrable. (You may see the bounded rectangle in $\mathcal{A}_k$ matters here)

The DCT implies that $g(x)=m_Y(F_x)$ is measurable, with 
\[
\int m_Y(F_x)\diff x = \lim_{n\to\infty}\int g_n\diff x=\lim_{n\to\infty}\pi(F_n)=\pi(F).
\]
Similarly, $y\mapsto m_X(F_y)$ is measurable, with $\int m_X(F_y)\diff y = \pi(F)$.
Therefore, $F\in\mathcal{A}_k$.
\end{enumerate}
Together with the results from last lecture, we conclude that claim~(1) and (2) holds for $\mathcal{A}_k$.
Following the similar idea of the results obtained from last lecture, we conclude that $\mathcal{A}_k = \{E\in\mathcal{M}\otimes\mathcal{M}\mid E\subseteq[-k,k]\times[-k,k]\}$.
\item
Then we show $\mathcal{A}$ is a monotone class, i.e., closed under countable decreasing intersections.

Suppose that $F_i\in\mathcal{A}, F_i\supseteq F_{i+1}$, we aim to show that $F:=\cap F_i\in\mathcal{A}$.

Construct 
\[
F_i^{(k)} = F_i\cap ([-k,k]\times[-k,k]),
\] 
which follows that $F_i^{(k)}\supseteq F_{i+1}^{(k)}$, and $F_i^{(k)}\in\mathcal{A}_k$ since 
$F_i^{(k)}\in\mathcal{M}\otimes\mathcal{M}$ and $F_i^{(k)}\subseteq[-k,k]\times[-k,k]$.
We denote $F^{(k)}=\cap_{i=1}^\infty F_i^{(k)}$.
The previous result implies that $F^{(k)}\in\mathcal{A}_k$, i.e.,
\[
\int m_Y((F^{(k)})_x)\diff x =\pi(F^{(k)})
\]
Now note that $F^{(1)}\subseteq F^{(2)}\subseteq\cdots$, and $F=\cup_{k\in\mathbb{N}}F^{(k)}$.
Therefore, applying MCT gives
\[
\int m_Y(F_x)\diff x = \lim_{k\to\infty}\int m_Y((F^{(k)})_x)\diff x = \lim_{k\to\infty}\pi(F^{(k)}) = \pi(F).
\]
Therefore, $F$ satisfies (\ref{Eq:14:1}), i.e., $F\in\mathcal{A}$


\end{itemize}
\end{proof}

\begin{theorem}[Tonelli's Theorem]
Let $F:\mathbb{R}^2\to[0,\infty]$ be measurable under the space $(\mathbb{R}^2,\mathcal{M}\otimes\mathcal{M},\pi)$.
Then
\[
\left\{
\begin{array}{l}
x\mapsto \int F(x,y)\diff y\\
y\mapsto \int F(x,y)\diff x
\end{array}
\right.\text{ is measurable},
\]
and 
\[
\int F\diff\pi = \int\left(\int F(x,y)\diff x\right)\diff y = \int\left(\int F(x,y)\diff y\right)\diff x
\]
\end{theorem}

\begin{proof}
Let
\[
\phi_n(x,y)=\sum_{k=0}^{4^n}(k\cdot 2^{-n})\mathcal{X}_{F^{-1}([k\cdot 2^{-n},(k+1)\cdot 2^{-n}])}
+
2^n\mathcal{X}_{F^{-1}(2^n,\infty]}
\]
We just re-write the terms above as $\sum_k\alpha_k\mathcal{X}_{E_k}$.
Our constructed $\phi_n(x,y)$ is a monotone increasing simple function such that $\phi_n\to F$ pointwise.
It follows that
\begin{subequations}
\begin{align}
\int F\diff\pi
&=
\lim_{n\to\infty}\int\phi_n\diff\pi\label{Eq:14:2:a}\\
&=
\lim_{n\to\infty}\int\left(
\sum_k\alpha_k\mathcal{X}_{E_k}
\right)\diff\pi\\
&=
\lim_{n\to\infty}\sum_k\alpha_k\int\mathcal{X}_{E_k}\diff\pi\label{Eq:14:2:c}=\lim_{n\to\infty}
\sum_k\alpha_k\pi(E_k)\\&
=
\lim_{n\to\infty}\sum_k\alpha_k\int
\left(
\int\mathcal{X}_{E_k}(x,y)\diff x
\right)
\diff y\label{Eq:14:2:d}\\
&=\lim_{n\to\infty}
\int\int\left(\sum_k
\alpha_k\mathcal{X}_{E_k}(x,y)
\right)\diff x\diff y\label{Eq:14:2:e}\\
&=\lim_{n\to\infty}
\int\left(
\int\phi_n(x,y)\diff x
\right)\diff y\\
&=\int \lim_{n\to\infty}
\left(
\int
\phi_n(x,y)\diff x
\right)\diff y\label{Eq:14:2:g}\\
&=\int\int \lim_{n\to\infty}\phi_n(x,y)\diff x\diff y\label{Eq:14:2:h}\\
&=\int \int F(x,y)\diff x\diff y\label{Eq:14:2:i}
\end{align}
\end{subequations}
where (\ref{Eq:14:2:a}) is by the MCT I on $\phi_n$;
(\ref{Eq:14:2:c}) is by the linearity of integral;
(\ref{Eq:14:2:d}) is by proposition~(\ref{pro:14:1})
(\ref{Eq:14:2:e}) is by the linearity of integral;
(\ref{Eq:14:2:g}) is by the MCT I on $f_n(y)=\int \phi_n(x,y)\diff x$;
(\ref{Eq:14:2:h}) is by the MCT I on $g_n(x)=\phi_n(x,y)$;
(\ref{Eq:14:2:i}) is because that $\phi_n(x,y)\to F(x,y)$.
\end{proof}

\begin{theorem}[Fubini's Theorem]
Suppose that $F:\mathbb{R}^2\to[-\infty,\infty]$ is integrable, then
\[
\int F\diff\pi
=
\int\left(
\int F(x,y)\diff x
\right)\diff y
=
\int\left(
\int F(x,y)\diff y
\right)\diff x
\]
\end{theorem}
\begin{proof}
Suppose $F=F^+-F^-$, where $F^{\pm}$ are both integrable.
Applying Tonell's theorem on both $F^-$ and $F^+$ and the linearity of integrals gives the desired result.
\end{proof}











