
\section{Wednesday for MAT3006}\index{Wednesday_lecture}
\subsection{Fubini's and Tonell's Theorem}


\paragraph{Easiest Goal}:
$E\in A\times B\in\mathcal{M}\otimes\mu$, $A\in\mathcal{M}_X,B\in\mathcal{M}_Y$, i.e.,
\[
\pi(A\times B) = m_X(A)m_Y(B)
\]
see graph
\[
\int m_X((A\times B)_y)\diff y \diff y = \int m_X(A)\mathcal{X}_(y)\diff y=m_X(A)\int\mathcal{X}_B(y)\diff y=
m_X(A)m_Y(B)
\]
Similarly,
\[
\int m_Y((A\times B)_x)\diff x = m_X(A)m_Y(B)
\]
Therefore, the easier goal holds for $E=A\times B$.

From easier goal to the real goal requires simply MCT.
Now  the question is that how can we prove ``easier goal holds for $E\in A\times B$ implies easier goal holds for all $E\in\mathcal{M}\otimes\mathcal{M}$ ''

\begin{definition}[Monotone Class]
Let $X$ be any set. A \emph{monotone class} $\mathcal{T}$ is a collection of subsets of $X$ satisfying 
\begin{enumerate}
\item
If $E_i\in\mathcal{T} (i\in\mathbb{N})$ and $E_i\subseteq E_{i+1},\forall i$, then
\[
\bigcup_{i=1}^\infty E_i\in\mathcal{T}
\]
\item
If $F_i\in\mathcal{T} (i\in\mathbb{N})$ with $F_i\supseteq F_{i+1},\forall i$, then
\[
\bigcap_{i=1}^\infty F_i\in\mathcal{T}
\]
\end{enumerate}
\end{definition}


\begin{example}
Let $X=\mathbb{R}$, then $\mathcal{M}$ (all Lebesgue measurable subsets) and $\mathcal{B}$ (all Borel measurable subsets) are monotone classes.
More generally, all $\sigma$-algebra are monotone classes.
\end{example}

\begin{definition}[Smallest Monotone Class]
For all colletion of subsets $S$ in $X$ (e.g., $S$ denotes all intervals in $X\in\mathbb{R}$),
let $\mathcal{M}(S)$be the smallest monotone class containing $S$
\end{definition}
\begin{remark}
As a result,
\[
\mathcal{M}(S)\subseteq\sigma(S)
\]
where $\sigma(S)$ is the smallest $\sigma$-algebra containing $S$.

Question: when do we have $\mathcal{M}(S)=\sigma(S)$?
\end{remark}

\begin{theorem}[Monotone Class Theorem]
Let $X$ be any set.
If a family of subsets $S$ of $X$ forms an \emph{algebra}, i.e.,
\[
E_1,E_2\in S\implies E_1\cup E_2\in S,E_1\cap E_2\in S,E_1^c\in S,
\]
then $\mathcal{M}(S)=\sigma(S)$.
\end{theorem}

\begin{example}
\begin{enumerate}
\item
Let $X=\mathbb{R}$, and $S=\{\text{all intervals}\}$.
is \emph{not} an algebram e.g.,
\[
[1,2]\in S^1\implies [1,2]^c = (-\infty,1)\cup(2,\infty)\notin S^1.
\]
However, $S=\{\text{finite disjoint union of intervals}\}$ is an algebra.
As a result, $\mu(S)=\sigma(S)=\mathcal{B}$ (borel $\sigma$-algebra)
\item
Let $X=\mathbb{R}^2$, and 
\[
S=\{\text{finite disjoint union of measurable rectangles }\cup_{i=1}^k(A_i\times B_i), A_i,B_i\in\mathcal{M}\}
\]
Then $S$ is an algebra, e.g., (see figure)
\[
(A\times B)^c = (A^c\times \mathbb{R})\cup(A\times B^c)
\]
is a disjoint union of 2 measurable rectangles.
Therefore, $\mathcal{M}(S)=\sigma(S)=\mathcal{M}\otimes\mathcal{M}$
\end{enumerate}
\end{example}

\begin{proposition}
For all $E\in\mathcal{M}\otimes\mathcal{M}$, we have
\[
\pi(E) = \int m_Y(E_x)\diff x
=
\int m_X(E_y)\diff y\quad(*)
\]
\end{proposition}
\begin{proof}
Let 
\[
\mathcal{A} = \left\{E\in\mathcal{M}\otimes\mathcal{M}
\middle|
\begin{array}{l}
x\mapsto m_Y(E_x)\text{ is a measurable function of $x$}\\
y\mapsto m_X(E_y)\text{ is a measurable function of $y$}\\
\text{(*) holds}
\end{array}
\right\}
\]
\begin{itemize}
\item
Claim 1: $\mathcal{A}$ is a monotone class
\item
Claim 2: All finite disjoint union of measurable rectangles 
\[
\bigcup_{i=1}^k(A_i\times B_i)\in\mathcal{A}
\]
\end{itemize}
If claim~(1),(2) holds, then $\mathcal{A}$ is a monotone class containing all elements in 
\[
S=\{\text{finite disjoint union of measurable rectangles}\}
\]
Since $\mathcal{M}(S)$ is the smallest monotone class containing $S$, we imply
\[
\mathcal{M}(S)\subseteq\mathcal{A}
\]
By monotone class theorem, $\sigma(S)=\mathcal{M}(S)\subseteq\mathcal{A}$, i.e.,
\[
\mathcal{M}\times\mathcal{M} = \sigma(S)=\mathcal{M}(S)\subseteq\mathcal{A}
\]
Therefore, (*) holds for all $E\in\mathcal{A}=\mathcal{M}\times\mathcal{M}$.
\end{proof}
\begin{proof}[Proof of Claim 2]
For any $E=\cup_{i=1}^k(A_i\times B_i)$, 
\[
m_Y(E_x)=\sum_{i=1}^k m_Y(B_i)\mathcal{X}_{A_i}(x)
\]
is a simple function on $x$, and therefore measurable.

Similarly, 
\[
m_X(E_y) = \sum_{i=1}^km_X(A_i)\mathcal{X}_{B_i}(y)
\]
is also measurable.

Therefore, (*) also holds for $E$ by the ``easiest goal''.
\end{proof}




















