
\section{Wednesday for MAT3040}\index{Wednesday_lecture}
\paragraph{Reviewing}
\begin{itemize}
\item
Basis, Dimension
\item
Basis Extension
\item
$W_1\bigcap W_2=\emptyset$ implies $W_1\oplus W_2=W_1+W_2$ (Direct Sum).
\end{itemize}

\subsection{Remark on Direct Sum}

\begin{proposition}
The set $W_1+W_2=W_1\oplus W_2$ iff 
any $\bm w\in W_1+W_2$ can be uniquely expressed as 
\[
\bm w=\bm w_1+\bm w_2,
\]
where $\bm w_i\in W_i$ for $i=1,2$.
\end{proposition}

\begin{remark}
We can also define addiction among finite set of vector spaces $\{W_1,\dots,W_k\}$.

If $\bm w_1+\cdots+\bm w_k=\bm0$ implies $\bm w_i=0,\forall i$, then we can write $W_1+\cdots+W_k$ as
\[
W_1\oplus\cdots\oplus W_k
\]
\end{remark}
\begin{proposition}[Complementation]\label{Pro:2:14}
Let $W\le V$ be a vector subspace of a 
fintie dimension vector space $V$. 
Then there exists $W'\le V$ such that
\[
W\oplus W'=V.
\]
\end{proposition}

\begin{proof}
It's clear that $\dim(W):=k\le n:=\dim(V)$. Suppose $\{\bm v_1,\dots,\bm v_k\}$ is a basis of $W$. 

By the basis extension proposition, we can extend it into $\{\bm v_1,\dots,\bm v_k,\bm v_{k+1},\dots,\bm v_n\}$, which is a basis of $V$.

Therefore, we take $W'=\Span\{\bm v_{k+1},\dots,\bm v_n\}$, which follows that
\begin{enumerate}
\item
$W+W'=V$: $\forall\bm v \in V$ has the form 
\[
\bm v=
\left(
\alpha_1\bm v_1+\cdots+\alpha_k\bm v_k
\right)+\left(
\alpha_{k+1}\bm v_{k+1}+\cdots+\alpha_n\bm v_n
\right),
\]
where $\alpha_1\bm v_1+\cdots+\alpha_k\bm v_k\in W$ and $\alpha_{k+1}\bm v_{k+1}+\cdots+\alpha_n\bm v_n\in W'$.
\item
$W\bigcap W'=\{\bm0\}$: Suppose $\bm v\in W\bigcap W'$, i.e., 
\begin{align*}
\bm v&=
\left(
\beta_1\bm v_1+\cdots+\beta_k\bm v_k
\right)+
(0\bm v_{k+1}+\cdots+0\bm v_n)\in W\\
&=(0\bm v_1+\cdots+0\bm v_k)+\left(\beta_{k+1}\bm v_{k+1}+\cdots+\beta_n\bm v_n\right)\in W'.
\end{align*}
By the uniqueness of coordinates, we imply $\beta_1=\cdots=\beta_n=0$, i.e., $\bm v=\bm0$.
\end{enumerate}
Therefore, we conclude that $W\oplus W'=V$.
\end{proof}


\subsection{Linear Transformation}

\begin{definition}[Linear Transformation]
Let $V,W$ be vector spaces. 
Then $T:V\to W$ is a \emph{linear transformation} if 
\[
T(\alpha\bm v_1+\beta\bm v_2)=\alpha  T(\bm v_1)+\beta T(\bm v_2),
\]
for $
\forall\alpha,\beta\in\mathbb{F}$ and $\bm v_1,\bm v_2\in V$.
\end{definition}

\begin{proposition}
\begin{enumerate}
\item
Suppose that $S:V\to W$ and $T:W\to U$ are linear transformations, then
so is $T\circ S:V\to U$.
\item
For any linear transformation $T:V\to W$, we have
\[
T(\bm 0_V)=\bm 0_W
\]
\end{enumerate}
\end{proposition}
\begin{proof}
Simply apply the definition of the linear transformation.
\end{proof}



\begin{example}
\begin{enumerate}
\item
The transformation $T:\mathbb{R}^n\to\mathbb{R}^m$ defined as $\bm x\mapsto\bm{Ax}$ (where $\bm A\in\mathbb{R}^{m\times n}$) is a linear transformation.

\item
The transformation $T:\mathbb{R}[x]\to\mathbb{R}[x]$ defined as
\[
\begin{array}{ll}
p(x)\mapsto T(p(x))=p'(x),
&
p(x)\mapsto T(p(x))=\int_0^xp(t)\diff t
\end{array}
\]
is a linear transformation

\item
The transformation $T: M_{n\times n}(\mathbb{R})\to\mathbb{R}$ defined as
\[
\bm A\mapsto\trace(\bm A):=\sum_{i=1}^na_{ii}
\]
is a linear transformation.

However, the transformation
\[
\bm A\mapsto\det(\bm A)
\]
is not a linear transformation.
\end{enumerate}
\end{example}

\begin{definition}[Kernel/Image]
Let $T:V\to W$ be a linear transfomation. 
\begin{enumerate}
\item
The \emph{kernel} of $T$ is
\[
\ker(T)=T^{-1}(\bm0)=\{\bm v\in V\mid T(\bm v)=\bm0\}
\]
\item
The \emph{image} (or range) of $T$ is
\[
\text{Im}(T)=T(\bm v)=\{T(\bm v)\in W\mid\bm v\in V\}
\]
\end{enumerate}
\end{definition}

\begin{example}
\begin{enumerate}
\item
Let $T:\mathbb{R}^n\to\mathbb{R}^n$ be a linear transformation with $T(\bm x)=\bm{Ax}$, then
\[
\ker(T)=\{\bm x\in\mathbb{R}^n\mid\bm{Ax}=\bm0\}
=
\text{Null}(\bm A)\qquad\mbox{Null Space}
\]
and
\[
\text{Im}(T)=\{\bm{Ax}\mid\bm x\in\mathbb{R}^n\}
=\text{Col}(\bm A)
=\Span\{\text{columns of $\bm A$}\}
\qquad\mbox{Column Space}
\]
\item
For $T(p(x))=p'(x)$, $\ker(T)=\{\text{constant polynomials}\}$ and $\text{Im}(T)=\mathbb{R}[x]$.
\end{enumerate}
\end{example}

\begin{proposition}
The kernel or image for a linear transformation $T:V\to W$ also forms a vector subspace:
\[
\begin{array}{ll}
\ker(T)\le V,
&
\text{Im}(T)\le W
\end{array}
\]
\end{proposition}

\begin{proof}
For $\bm v_1,\bm v_2\in\ker(T)$, we imply
\[
T(\alpha\bm v_1+\beta\bm v_2)=\bm0,
\]
which implies $\alpha\bm v_1+\beta\bm v_2\in\ker(T)$.

The remaining proof follows similarly.
\end{proof}

\begin{definition}[Rank/Nullity]
Let $V,W$ be finite dimensional vector spaces and 
$T:V\to W$ a linear transformation. 
Then we define
\begin{align*}
\rank(T)&=\dim(\text{im}(T))\\
\text{nullity}(T)&=\dim(\ker(T))
\end{align*}
\end{definition}

\begin{remark}
Let
\[
\text{Hom}_{\mathbb{F}}(V,W)=\{\text{all linear transformations }T:V\to W\},
\]
and we can define the addiction and scalar multiplication to make it a vector space:
\begin{enumerate}
\item
For $T,S\in\text{Hom}_{\mathbb{F}}(V,W)$, define
\[
(T+S)(\bm v)=T(\bm v)+S(\bm v),
\]
which implies $T+S\in\text{Hom}_{\mathbb{F}}(V,W)$.
\item
Also, define
\[
(\gamma T)(\bm v)=\gamma T(\bm v),\qquad
\text{for }\forall\gamma\in\mathbb{F},
\]
which implies $\gamma T\in\text{Hom}_{\mathbb{F}}(V,W)$.
\end{enumerate}
In particular, if $V=\mathbb{R}^n,W=\mathbb{R}^m$, then
\[
\text{Hom}_{\mathbb{F}}(V,W)=M_{m\times n}(\mathbb{R}).
\]
\end{remark}

\begin{proposition}
If $\dim(V)=n,\dim(W)=m$, then $\dim(\text{Hom}_{\mathbb{F}}(V,W))=mn$.
\end{proposition}

\begin{proposition}

There are anternative characterizations for the injectivity and surjectivity of lienar transformation $T$:
\begin{enumerate}
\item
The linear transformation $T$ is injective if and only if
\[
\ker(T)=0,
\Longleftrightarrow
\mbox{nullity}(T)=0.
\]
\item
The linear transformation $T$ is surjective if and only if
\[
\text{im}(T)=W,
\Longleftrightarrow
\rank(T)=\dim(W).
\]
\item
If $T$ is bijective, then $T^{-1}$ is a linear transformation.
\end{enumerate}
\end{proposition}

\begin{proof}
\begin{enumerate}
\item
\begin{enumerate}
\item

For the forward direction of (1),
\[
\bm x\in\ker(T)\implies
T(\bm x)=0=T(\bm0)\implies
\bm x=\bm0
\]
\item

For the reverse direction of (1),
\[
T(\bm x)=T(\bm y)\implies
T(\bm x-\bm y)=\bm0\implies
\bm x-\bm y\in\ker(T)=\bm0
\implies
\bm x=\bm y
\]

\end{enumerate}

\item
The proof follows similar idea in (1).

\item
Let $T^{-1}:W\to V$. 
For all $\bm w_1,\bm w_2\in W$, 
there exists $\bm v_1,\bm v_2\in V$ such that 
$T(\bm v_i)=\bm w_i$, i.e., $T^{-1}(\bm w_i)=\bm v_i$ $i=1,2$.

Consider the mapping
\begin{align*}
T(\alpha\bm v_1+\beta\bm v_2)&=\alpha T(\bm v_1)+\beta T(\bm v_2)\\
&=\alpha\bm w_1+\beta\bm w_2,
\end{align*}
which implies $\alpha\bm v_1+\beta\bm v_2=T^{-1}(\alpha\bm w_1+\beta\bm w_2)$, i.e.,
\[
\alpha T^{-1}(\bm w_1)+\beta T^{-1}(\bm w_2)
=
T^{-1}(\alpha\bm w_1+\beta\bm w_2).
\]
\end{enumerate}
\end{proof}



\begin{definition}[isomorphism]
We say that the vector subspaces $V$ and $W$ are isomorphic 
if there exists a bijective linear transfomation $T:V\to W$. ($V\cong W$)

This mapping $T$ is called an \emph{isomorphism} from $V$ to $W$.
\end{definition}

\begin{remark}
If $\dim(V)=\dim(W)=n<\infty$, then $V\cong W$:

Take $\{\bm v_1,\dots,\bm v_n\},\{\bm w_1,\dots,\bm w_n\}$ as basis of $V$ and $W$, respectively. Then one can construct $T:V\to W$ satisfying $T(\bm v_i)=\bm w_i$ for $\forall i$ as follows:
\[
T(\alpha_1\bm v_1+\cdots+\alpha_n\bm v_n)
=
\alpha_n\bm w_1+\cdots+\alpha_n\bm w_n\
\forall\alpha_i\in\mathbb{F}
\]
It's clear that our constructed $T$ is a linear transformation.
\end{remark}

\begin{remark}
$V\cong W$ doesn't imply any linear transformations $T:V\to W$ is an isomorphism. e.g., 
$T(\bm v)=\bm0$ is not an isomorphic if $W\ne\{\bm0\}$.
\end{remark}

\begin{theorem}[Rank-Nullity Theorem]\label{The:2:3}
Let $T:V\to W$ be a linear transformation with $\dim(V)<\infty$. Then
\[
\rank(T)+\text{nullity}(T)=\dim(V).
\]
\end{theorem}

\begin{proof}
Since $\ker(T)\le V$, by proposition~(\ref{Pro:2:14}), there exists  $V_1\le V$ such that 
\[
V=\ker(T)\oplus V_1.
\]
\begin{enumerate}
\item
Consider the transformation $T\mid_{V_1}:V_1\to T(V_1)$, which is an isomorphism, since:
\begin{itemize}
\item
Surjectivity is immediate
\item
For $\bm v\in\ker(T\mid_{V_1})$,
\[
T(\bm v)=\bm0\implies \bm v\in\ker(T),
\]
which implies $\bm v=\bm0$ since $\bm v\in\ker(T)\cap V_1=0$, i.e., the injectivity follows.
\end{itemize}
Therefore, $\dim(V_1)=\dim(T(V_1))$.
\item
Secondly, given an isomorphism $T$ from $X$ to $Y$ with $\dim(X)<\infty$, then $\dim(X)=\dim(T(X))$. The reason follows from assignment 1 questions (8-9):
\[
\{\bm v_1,\dots,\bm v_k\}\text{ is a basis of $X$}\implies
\{T(\bm v_1),\dots,T(\bm v_k)\}\text{ is a basis of $Y$}
\]
\item
Note that $T(V_1)=T(V)=\text{im}(T)$, since:
\begin{itemize}
\item
for $\forall\bm v\in V$, $\bm v=\bm v_k+\bm v_1$, where $\bm v_k\in\ker(T),\bm v_1\in V_1$, which implies
\[
T(\bm v)=T(\bm v_k)+T(\bm v_1)=\bm0+T(\bm v_1),
\]
i.e., $T(V)\subseteq T(V_1)\subseteq T(V)$, i.e., $T(V)=T(V_1)$.
\end{itemize}
\item
We can show that $\dim(V)=\dim(\ker(T))+\dim(V_1)$:
Let $\{v_1,\dots,v_k\}$ be a basis of $\ker(T)$,
and $\{v_{k+1},\dots,v_n\}$ be a basis of $V_1$,
then by the proof of complementation proposition~(\ref{Pro:2:14}),
we imply $\{v_1,\dots,v_n\}$ is a basis of $V$, i.e., 
$\dim(V)=n=k+(n-k)=\dim(\ker(T))+\dim(V_1)$.

Therefore, we imply
\begin{align*}
\dim(V)&=\dim(\ker(T))+\dim(V_1)\\
&=\text{nullity}(T)+\dim(T(V_1))\\
&=\text{nullity}(T)+\dim(T(V))\\
&=\text{nullity}(T)+\dim(\text{im}(T))\\
&=\text{nullity}(T)+\rank(T).
\end{align*}
\end{enumerate}
\end{proof}













