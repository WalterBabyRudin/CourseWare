
\chapter{Week2}
\section{Monday for MAT3040}\index{Monday_lecture}
\paragraph{Reviewing}
\begin{enumerate}
\item
Linear Combination and Span
\item
Linear Independence
\item
Basis: a set of vectors $\{\bm v_1,\dots,\bm v_k\}$ is called a \emph{basis} for $V$ if $\{\bm v_1,\dots,\bm v_k\}$ is linearly independent, and $V=\Span\{\bm v_1,\dots,\bm v_k\}$.

Lemma: Given $V=\Span\{\bm v_1,\dots,\bm v_k\}$, 
we can find a basis for this set. 
Here $V$ is said to be \emph{finitely generated}.
\item
Lemma: The vector $\bm w\in\Span\{\bm v_1,\dots,\bm v_n\}\setminus\Span\{\bm v_2,\dots,\bm v_n\}$ implies that 
\[
\bm v_1\in\Span\{\bm w,\bm v_2,\dots,\bm v_n\}\setminus\Span\{\bm v_2,\dots,\bm v_n\}
\]
\end{enumerate}
\subsection{Basis and Dimension}
\begin{theorem}
Let $V$ be a finitely generated vector space. Suppose $\{\bm v_1,\dots,\bm v_m\}$ and $\{\bm w_1,\dots,\bm w_n\}$ are two basis of $V$. Then $m=n$. (where $m$ is called the \emph{dimension})
\end{theorem}
\begin{proof}
Suppose on the contrary that $m\ne n$. Without loss of generality (w.l.o.g.), assume that $m<n$. 
Let 
$
\bm v_1=\alpha_1\bm w_1+\cdots+\alpha_n\bm w_n,
$
with some $\alpha_i\ne0$. w.l.o.g., assume $\alpha_1\ne0$. Therefore, 
\begin{equation}\label{Eq:2:1}
\bm v_1\in\Span\{\bm w_1,\bm w_2,\dots,\bm w_n\}\setminus\Span\{\bm w_2,\dots,\bm w_n\}
\end{equation}
which implies that $\bm w_1\in\Span\{\bm v_1,\bm w_2,\dots,\bm w_n\}\setminus\Span\{\bm w_2,\dots,\bm w_n\}$.

Then we claim that $\{\bm v_1,\bm w_2,\dots,\bm w_n\}$ is a basis of $V$:
\begin{enumerate}
\item
Note that $\{\bm v_1,\bm w_2,\dots,\bm w_n\}$ is a spannning set:
\begin{align*}
\bm w_1&\in\Span\{\bm v_1,\bm w_2,\dots,\bm w_n\}
\implies
\{\bm w_1,\bm w_2,\dots,\bm w_n\}\subseteq
\Span\{\bm v_1,\bm w_2,\dots,\bm w_n\}\\
&\implies
\Span\{\bm w_1,\bm w_2,\dots,\bm w_n\}\subseteq\Span\{\Span\{\bm v_1,\bm w_2,\dots,\bm w_n\}\}\subseteq\Span\{\bm v_1,\bm w_2,\dots,\bm w_n\}
\end{align*}

Since $V=\Span\{\bm w_1,\bm w_2,\dots,\bm w_n\}$, we have $\Span\{\bm v_1,\bm w_2,\dots,\bm w_n\}=V$.

\item
Then we show the linear independence of $\{\bm v_1,\bm w_2,\dots,\bm w_n\}$.
Consider the equation 
\[
\beta_1\bm v_1+\beta_2\bm v_2+\cdots+\beta_n\bm w_n=\bm0
\]
\begin{enumerate}
\item
When $\beta_1\ne0$, we imply
\[
\bm v_1=
\left(
-\frac{\beta_2}{\beta_1}
\right)
\bm w_2
+\cdots+
\left(
-\frac{\beta_n}{\beta_1}
\right)
\bm w_n
\in\Span\{\bm w_2,\dots,\bm w_n\},
\]
which contradicts (\ref{Eq:2:1}).
\item
When $\beta_1=0$, then $\beta_2\bm w_2+\cdots+\beta_n\bm w_n=\bm0$, which implies $\beta_2=\cdots=\beta_n=0$, due to the independence of $\{\bm w_2,\dots,\bm w_n\}$.
\end{enumerate}
\end{enumerate}

Therefore, $\bm v_2\in\Span\{\bm v_1,\bm w_2,\dots,\bm w_n\}$, i.e.,
\[
\bm v_2=\gamma_1\bm v_1+\cdots+\gamma_n\bm v_n,
\]
where $\gamma_2,\dots,\gamma_n$ cannot be all zeros, since otherwise $\{\bm v_1,\bm v_2\}$ are linearly dependent, i.e., $\{\bm v_1,\dots,\bm v_m\}$ cannot form a basis. w.l.o.g., assume $\gamma_2\ne0$, which implies
\[
\bm w_2\in\Span\{\bm v_1,\bm v_2,\bm w_3,\dots,\bm w_n\}\setminus\Span\{\bm v_1,\bm w_3,\dots,\bm w_n\}.
\]
Following the simlar argument above, $\{\bm v_1,\bm v_2,\bm w_3,\dots,\bm w_n\}$ forms a basis of $V$.

Continuing the argument above, we imply $\{\bm v_1,\dots,\bm v_m,\bm w_{m+1},\dots,\bm w_n\}$ is a basis of $V$. Since $\{\bm v_1,\dots,\bm v_m\}$ is a basis as well, we imply
\[
\bm w_{m+1}=\delta_1\bm v_1+\cdots+\delta_m\bm v_m
\]
 for some $\delta_i\in\mathbb{F}$, i.e., $\{\bm v_1,\dots,\bm v_m,\bm w_{m+1}\}$ is linearly dependent, which is a contradction.
\end{proof}
\begin{example}
A vector space may have more than one basis.

Suppose $V=\mathbb{F}^n$, it is clear that $\dim(V)=n$, and
\[
\text{$\{\bm e_1,\dots,\bm e_n\}$ is a basis of $V$, where $\bm e_i$ denotes a unit vector.}
\]

There could be other basis of $V$, such as
\[
\left\{
\begin{pmatrix}
1\\0\\\vdots\\0
\end{pmatrix},
\begin{pmatrix}
1\\1\\\vdots\\0
\end{pmatrix},
\cdots,
\begin{pmatrix}
1\\1\\\vdots\\1
\end{pmatrix},
\right\}
\]

Actually, the columns of any invertible $n\times n$ matrix forms a basis of $V$. 
\end{example}

\begin{example}
Suppose $V=M_{m\times n}(\mathbb{R})$, we claim that $\dim(V)=mn$:
\[
\left\{
E_{ij}\middle|
\begin{aligned}
1\le i\le m\\
1\le j\le n
\end{aligned}
\right\}
\text{ is a basis of $V$},
\]
where $E_{ij}$ is $m\times n$ matrix with $1$ at $(i,j)$-th entry, and $0$s at the remaining entries.
\end{example}

\begin{example}
Suppose $V=\{\text{all polynomials of degree $\le$ n}\}$, then $\dim(V)=n+1$.
\end{example}

\begin{example}
Supppose $V=\{\bm A\in M_{n\times n}(\mathbb{R})\mid\bm A\trans=\bm A\}$, then 
$\dim(V)=\frac{n(n+1)}{2}$.
\end{example}

\begin{example}
Let $W=\{\bm B\in M_{n\times n}(\mathbb{R})\mid \bm B\trans=-\bm B\}$, then 
$\dim(V)=\frac{n(n-1)}{2}$.
\end{example}


\begin{remark}
Sometimes it should be classified the field $\mathbb{F}$ for the scalar multiplication to define a vector space. Conside the example below:
\begin{enumerate}
\item
Let $V=\mathbb{C}$, then $\dim(\mathbb{C})=1$ for the scalar multiplication defined under the field $\mathbb{C}$.
\item
Let $V=\Span\{1,i\}=\mathbb{C}$, then $\dim(\mathbb{C})=2$ for the scalar multiplication defined under the field $\mathbb{R}$, since all $z\in V$ can be written as $z=a+bi$, $\forall a,b\in\mathbb{R}$.
\item
Therefore, to aviod confusion, it is safe to write
\[
\begin{array}{ll}
\dim_{\mathbb{C}}(\mathbb{C})=1,
&
\dim_{\mathbb{R}}(\mathbb{C})=2.
\end{array}
\]
\end{enumerate}
\end{remark}

\subsection{Operations on a vector space}
Note that the basis for a vector space is characterized as the \emph{maximal linearly independent set}.

\begin{theorem}[Basis Extension]
Let $V$ be a finite dimensional vector space, and $\{\bm v_1,\dots,\bm v_k\}$ be a linearly independent set on $V$, Then we can extend it to the basis 
$\{\bm v_1,\dots,\bm v_k,\bm v_{k+1},\dots,\bm v_n\}$ of $V$.
\end{theorem}
\begin{proof}
\begin{itemize}
\item
Suppose $\dim(V)=n>k$, and $\{\bm w_1,\dots,\bm w_n\}$ is a basis of $V$. 
Consider the set 
$\{
\bm w_1,\dots,\bm w_n
\}
\bigcup
\{
\bm v_1,\dots,\bm v_k
\}$,
which is linearly dependent, i.e., 
\[
\alpha_1\bm w_1+\cdots+\alpha_n\bm w_n
+
\beta_1\bm v_1+\cdots+\beta_k\bm v_k=\bm0,
\]
with some $\alpha_i\ne0$, since otherwise this equation will only have trivial solution. w.l.o.g., assume $\alpha_1\ne0$. 
\item
Therefore, consider the set 
$\{
\bm w_2,\dots,\bm w_n
\}
\bigcup
\{
\bm v_1,\dots,\bm v_k
\}$. 
We keep removing elements from $\{
\bm w_2,\dots,\bm w_n
\}$ until we first get the set 
\[
S\bigcup\{\bm v_1,\dots,\bm v_k\},
\]
with $S\subseteq\{
\bm w_1,\bm w_2,\dots,\bm w_n
\}$
and $S\bigcup\{\bm v_1,\dots,\bm v_k\}$ 
is linearly independent, i.e., 
$S$ is a maximal subset of 
$\{\bm w_1,\dots,\bm w_n\}$ such that 
$S\bigcup\{\bm v_1,\dots,\bm v_k\}$
 is linearly independent.
 \item
Rewrite $S=\{\bm v_{k+1},\dots,\bm v_m\}$ and therefore $S'=\{\bm v_1,\dots,\bm v_k,\bm v_{k+1},\dots,\bm v_m\}$ are linearly independent. It suffices to show $S'$ spans $V$.
\begin{itemize}
\item
Indeed, for all $\bm w_i\in\{\bm w_1,\dots,\bm w_n\}$, $\bm w_i\in\Span(S')$, since otherwise the equation 
 \[
 \alpha\bm w_i+\beta_1\bm v_1+\cdots+\beta_m\bm v_m=\bm0\implies\alpha=0,
 \]
 which implies that $\beta_1\bm v_1+\cdots+\beta_m\bm v_m=\bm0$ admits only trivial solution, i.e., 
 \[
 \{\bm w_i\}\bigcup S'=\{\bm w_i\}\bigcup S\bigcup\{\bm v_1,\dots,\bm v_k\}\text{ is linearly independent},
 \]
 which violetes the maximality of $S$.
\end{itemize}
 Therefore, all $\{\bm w_1,\dots,\bm w_n\}\subseteq\Span(S')$, which implies $\Span(S')=V$.
 
  Therefore, $S'$ is a basis of $V$.
\end{itemize}
\end{proof}
\begin{remark}
Start with a spanning set, we keep removing something to form a basis; 
start with independent set, we keep adding something to form a basis.

In other words, the basis is both the minimal spanning set, and the maximal linearly independent set.
\end{remark}

\begin{definition}[Direct Sum]
Let $W_1,W_2$ be two vector subspaces of $V$, then
\begin{enumerate}
\item
$W_1\bigcap W_2:=\{\bm w\in V\mid\bm w\in W_1,\text{ and }\bm w\in W_2\}$
\item
$W_1+W_2:=\{\bm w_1+\bm w_2\mid\bm w_i\in W_i\}$
\item
If furthermore that $W_1\bigcap W_2=\{\bm0\}$, then $W_1+W_2$ is denoted as $W_1\oplus W_2$, which is called \emph{direct sum}.
\end{enumerate}
\end{definition}
\begin{proposition}
$W_1\bigcap W_2$ and $W_1+W_2$ are vector subspaces of $V$.
\end{proposition}



















