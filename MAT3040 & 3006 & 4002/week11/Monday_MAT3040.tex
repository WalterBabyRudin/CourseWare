\chapter{Week11}
\section{Monday for MAT3040}\index{Monday_lecture}
\paragraph{Reviewing}
Adjoint Operator: $\inp{T'(\bm v)}{\bm w}=\inp{\bm v}{T(\bm w)}$.

\subsection{Self-Adjoint Operator}
\begin{definition}[Self-Adjoint]
Let $V$ be an inner product space and $T:V\to V$ be a linear operator.
Then $T$ is \emph{self-adjoint} if $T'=T$.
\end{definition}

\begin{example}
Let $V=\mathbb{C}^n$, and $\mathcal{B}=\{\bm e_1,\dots,\bm e_n\}$ be a orthonormal basis.
Let $T:V\to V$ be given by
\[
T(\bm v)=\bm A\bm v,\quad
\text{where $A\in M_{n\times n}(\mathbb{C})$.}
\]
Or equivalently, there exists basis $\mathcal{B}$ such that $(T)_{\mathcal{B},\mathcal{B}}=\bm A$.

In such case, $T$ is self-adjoint if and only if 
$(T')_{\mathcal{B},\mathcal{B}}=(T)_{\mathcal{B},\mathcal{B}}$, i.e.,
$\overline{(T)_{\mathcal{B},\mathcal{B}}\trans}=(T)_{\mathcal{B},\mathcal{B}}$, i.e.,
$\bm A\Her=\bm A$.

Therefore, $T(\bm v)=\bm A\bm v$ is self-adjoint if and only if $\bm A\Her=\bm A$.

Moreover, if $\mathbb{C}$ is replaced by $\mathbb{R}$, then $T$ is seld-adjoint if and only if $\bm A$ is symmetric.
\end{example}
\begin{remark}
The notion of self-adjoint for linear operator is essentially the generalized notion of Hermitian for matrix that we have stuided in MAT2040.
\end{remark}

We also have some nice properties for self-adjoint, and the proof for which are essentially the same for the proof in the case of Hermitian matrices.
\begin{proposition}\label{pro:11:1}
If $\lambda$ is an eigenvalue of a self-adjoint operator $T$, then $\lambda\in\mathbb{R}$. 
\end{proposition}
\begin{proof}
Suppose there is an eigen-pair $(\lambda,\bm w)$ for $\bm w\ne\bm0$, then
\begin{align*}
\lambda\inp{\bm w}{\bm w}&=\inp{\bm w}{\lambda\bm w}\\
&=\inp{\bm w}{T(\bm w)}
=\inp{T'(\bm w)}{\bm w}\\
&=\inp{T(\bm w)}{\bm w}
=\inp{\lambda\bm w}{\bm w}\\
&=\bar{\lambda}\inp{\bm w}{\bm w}
\end{align*}
Since $\inp{\bm w}{\bm w}\ne0$ by non-degeneracy property, we have $\lambda=\bar{\lambda}$, i.e., $\lambda\in\mathbb{R}$.
\end{proof}

\begin{proposition}
If $U\le V$ is $T$-invariant over the self-adjoint operator $T$, then so is $U^\perp$.
\end{proposition}
\begin{proof}
It suffices to show $T(\bm v)\in U^\perp,\forall\bm v\in U^\perp$, i.e., for any $\bm u\in U$, check that
\[
\inp{\bm u}{T(\bm v)}=\inp{T'(\bm u)}{\bm v}=\inp{T(\bm u)}{\bm v}=0,
\]
where the last equality is because that $T(\bm u)\in U$ and $\bm v\in U^\perp$.
Therefore, $T(\bm v)\in U^\perp$.
\end{proof}

\begin{theorem}\label{the:11:1}
If $T:V\to V$ is self-adjoint, and $\dim(V)<\infty$, then there exists an orthonormal basis of eigenvectors of $T$, i.e., an orthonormal basis of $V$ such that any element from this basis is an eigenvector of $T$.
\end{theorem}
\begin{proof}
We use the induction on $\dim(V)$:
\begin{itemize}
\item
The result is trival for $\dim(V)=1$.
\item
Suppose that this theorem holds for all vector spaces $V$ with $\dim(V)\le k$, then we want to show the theorem holds when $\dim(V)=k+1$:

Suppose that $T:V\to V$ is self-adjoint with $\dim(V)=k+1$, then consider 
\[
\mathcal{X}_T(x)=x^{k+1}+\cdots+a_1x+a_0,\quad a_i\in\mathbb{K},\ \text{where $\mathbb{K}$ denotes $\mathbb{R}$ or $\mathbb{C}$.}
\]
\begin{itemize}
\item
If $\mathbb{K}=\mathbb{C}$, then $\mathcal{X}_T(x)$ can be decomposed as
\[
\mathcal{X}_T(x)=(x-\lambda_1)\cdots(x-\lambda_{k+1})
\]
In paricular, we obtain the eigen-pair $(\lambda_1,\bm v)$
\item
If $\mathbb{K}=\mathbb{R}$, i.e., we treat real number as scalars, then
\[
\mathcal{X}_T(x)=(x-\lambda_1)\cdots(x-\lambda_{k+1}),\
\text{where $\lambda_i\in\mathbb{C}$}.
\]
By proposition~(\ref{pro:11:1}), we imply all $\lambda_i$'s are in $\mathbb{R}$.
Moreover, we also obtain the eigen-pair $(\lambda_1,\bm v)$
\end{itemize}

Consider $U=\Span\{\bm v\}$, then
\begin{itemize}
\item
$U$ is $T$-invariant
\item
$V=U\oplus U^\perp$, since $V$ is finite dimensional
\item
$U^\perp$ is $T$-invariant.
\end{itemize}
Consider $T\mid_{U^\perp}$, which is a self-adjoint operator on $U^\perp$, with $\dim(U^\perp)=k$.

By induction, there exists an orthonormal basis $\{\bm e_2,\dots,\bm e_{k+1}\}$ of eigenvectors of $T\mid_{U^\perp}$.

Consider the basis $\mathcal{B}=\{\bm v'=\bm v/\|\bm v\|,\bm e_2,\dots,\bm e_{k+1}\}$. As a result,
\begin{enumerate}
\item
$\mathcal{B}$ forms a basis of $V$
\item
All $\bm v',\bm e_i$ are of norm 1 eigenvectors of $T$.
\item
$\mathcal{B}$ is an orthonormal set, e.g., $\inp{\bm v'}{\bm e_i}=0$, where $\bm v'\in U$ and $\bm e_i\in U^\perp$.
\end{enumerate}
Therefore, $\mathcal{B}$ is a basis of orthonormal eigenvectors of $V$.
\end{itemize}
\end{proof}

\begin{corollary}
If $\dim(V)<\infty$, and $T:V\to V$ is self-adjoint, then there exists orthonormal basis $\mathcal{B}$ such that
\[
(T)_{\mathcal{B},\mathcal{B}}=\diag(\lambda_1,\dots,\lambda_n)
\]
In paticular, for all real symmtric matrix $\bm A\in\mathbb{S}^n$, there exists orthogonal matrix $P$ ($P\trans P=\bm I_n$) such that
\[
P^{-1}AP=\diag(\lambda_1,\dots,\lambda_n)
\]
\end{corollary}
\begin{proof}
\begin{enumerate}
\item
By applying theorem~(\ref{the:11:1}), there exists orthonormal basis of $V$, say $\mathcal{B}=\{\bm v_1,\dots,\bm v_n\}$ such that $T(\bm v_i)=\lambda_i\bm v_i$. Directly writing the basis representation gives
\[
(T)_{\mathcal{B},\mathcal{B}}=\diag(\lambda_1,\dots,\lambda_n).
\]
\item
For the second part, consider $T:\mathbb{R}^n\to\mathbb{R}^n$ by $T(\bm v)=\bm A\bm v$.
Since $\bm A\trans=\bm A$, we imply $T$ is self-adjoint.
There exists orthonormal basis $\mathcal{B}=\{\bm v_1,\dots,\bm v_n\}$ such that
\[
(T)_{\mathcal{B},\mathcal{B}}=\diag(\lambda_1,\dots,\lambda_n).
\]
In particular, if $\mathcal{A}=\{\bm e_1,\dots,\bm e_n\}$, then $(T)_{\mathcal{A},\mathcal{A}}=\bm A$.
We construct $P:=\mathcal{C}_{\mathcal{A},\mathcal{B}}$, which is the change of basis matrix from $\mathcal{B}$ to $\mathcal{A}$, then
\[
P=\begin{pmatrix}
\bm v_1&\cdots&\bm v_n
\end{pmatrix}
\]
and 
\[
P^{-1}(T)_{\mathcal{A},\mathcal{A}}P=(T)_{\mathcal{B},\mathcal{B}}
\]
Or equivalently, $P^{-1}AP=\diag(\lambda_1,\dots,\lambda_n)$, with 
\[
P\trans P=\begin{pmatrix}
\bm v_1\trans\\\vdots\\\bm v_n\trans
\end{pmatrix}\begin{pmatrix}
\bm v_1&\cdots&\bm v_n
\end{pmatrix}=\bm I
\]
\end{enumerate}
\end{proof}
\subsection{Orthononal/Unitary Operators}
\begin{definition}
A linear operator $T:V\to V$ over $\mathbb{K}$ with $\inp{T(\bm w)}{T(\bm v)}=\inp{\bm w}{\bm v},\forall \bm v,\bm w\in V$, is called
\begin{enumerate}
\item
\emph{Orthogonal} if $\mathbb{K}=\mathbb{R}$
\item
\emph{Unitary} if $\mathbb{K}=\mathbb{C}$
\end{enumerate}
\end{definition}
\begin{proposition}
$T$ is orthogonal / unitary if and only if $T'\circ T=I$
\end{proposition}
\begin{proof}
The reverse direction is by directly checking that 
\[
\inp{T(\bm w)}{T(\bm v)}=\inp{T'\circ T(\bm w)}{\bm v}=\inp{\bm w}{\bm v}
\]
The forward direction is by checking $T'\circ T(\bm w)=\bm w,\forall \bm w\in V$:
\[
\inp{T'\circ T(\bm w)}{\bm v}=\inp{T(\bm w)}{T(\bm v)}=\inp{\bm w}{\bm v}
\implies
\inp{T'\circ T(\bm w)-\bm w}{\bm v}=0,\forall \bm v\in V
\]
By non-degeneracy, $T'\circ T(\bm w)-\bm w=0$, i.e.,
$T'\circ T(\bm w)=\bm w,\forall \bm w\in V$.
\end{proof}

\begin{example}
Let $T:\mathbb{K}^n\to\mathbb{K}^n$ be given by $T(\bm v)=A\bm v$.
Then $T$ is orthogonal implies $(T')_{\mathcal{B},\mathcal{B}}(T)_{\mathcal{B},\mathcal{B}}=I$.

(Orthogonal) When $\mathbb{K}=\mathbb{R}$, then $A\trans A=I$

(Unitary) When $\mathbb{K}=\mathbb{C}$, then $A\Her A=I$.
\end{example}

\begin{definition}[Orthogonal/Unitary Group]
\[\text{Orthognoal Group}:
O(n,\mathbb{R})=\{A\in M_{n\times n}(\mathbb{R})\mid A\trans A=I\}
\]
\[\text{Unitary Group}:
U(n,\mathbb{C})=\{A\in M_{n\times n}(\mathbb{C})\mid A\Her A=I\}
\]
\end{definition}














