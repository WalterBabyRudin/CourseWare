
\section{Wednesday for MAT3006}\index{Wednesday_lecture}
\begin{proposition}[Linearity]
If $f,g$ are integrable, then $f+g$ and $\alpha f$ are integrable with
\begin{itemize}
\item
$\int(f+g)\diff m = \int f\diff m+\int g\diff m$
\item
$\int\alpha f\diff m = \alpha\int f\diff m$ ($\alpha\in\mathbb{R}$)
\end{itemize}
\end{proposition}
\begin{proof}
\begin{enumerate}
\item
Consider 
\[
(f+g)^+ - (f+g)^- = f+g
=
(f^+ - f^-) + (g^+ - g^-)
\]
It follows that
\[
(f+g)^+ + f^- + g^- = (f+g)^- + f^++g^+
\]
Therefore, both sides of the equality above are non-negative, which implies
\[
\int((f+g)^+ + f^- + g^- )\diff m
=
\int((f+g)^- + f^++g^+ )\diff m,
\]
i.e., 
\[
\int (f+g)^+\diff m +\int f^-\diff m+\int g^-\diff m
=
\int(f+g)^-\diff m+\int f^+\diff m+\int g^+\diff m
\]
i.e.,
\[
\int(f+g)\diff m = \int f\diff m+\int g\diff m
\]
\item
Assume $\alpha<0$. Then
\begin{align*}
\int(\alpha f)\diff m&:=\int(\alpha f)^+\diff m - \int(\alpha f)^-\diff m\\
&=\int(-\alpha)f^-\diff m - \int(-\alpha)f^+\diff m\\
&=(-\alpha)\int f^-\diff m - (-\alpha)\int f^+\diff m\\
&=\alpha\left(\int f^+\diff m - \int f^-\diff m\right)\\
&=\alpha \int f\diff m
\end{align*}

\end{enumerate}
\end{proof}


\begin{corollary}
Suppose that $f,g$ are integrable, then 
\begin{enumerate}
\item
If $f\le g$, then $\int f\diff m\le \int g\diff m$
\item
If $f=g$ a.e., then $\int f\diff m = -\int g\diff m$
\end{enumerate}
\end{corollary}
\begin{proof}
\begin{enumerate}
\item
Since $g-f\ge0$, then
\[
\int(g-f)\diff m\ge\int0\diff m=0
\]
By linearity, $\int g\diff m-\int f\diff m\ge0$, i.e.,
\[
\int g\diff m\ge \int f\diff m
\]
\item
Same proof as before:
let $U=\{x\mid f(x)=g(x)\}$, then 
\begin{align*}
\int f\diff m&=\int f\mathcal{X}_U\diff m
+
\int f\mathcal{X}_{U^c}\diff m\\
&=\int g\mathcal{X}_U\diff m\\
&=\int g\mathcal{X}_U\diff m+\int g\mathcal{X}_{U^c}\diff m\\
&=\int g\diff m
\end{align*}


\end{enumerate}
\end{proof}
By corollary of Markov inequality, $\int f^{\pm}\mathcal{X}_{U^c}\diff m=0$.


\begin{remark}
Let $\mathcal{T}=\{f:\mathbb{R}\to[-\infty,\infty],\ \text{integrable}\}$, which $0:=\text{zero function}$ is a vector space.

Define a ``norm'' on $f\in\mathcal{T}$ by
\[
\|f\|=\int|f|\diff m
\]
then $\|\alpha f\|=|\alpha|\|f\|$ and $\|f+g\|\le\|f\|+\|g\|$.

However, $\exists f\ne0$ s.t. $\|f\|=0$, e.g., $f=\mathcal{X}_{\mathbb{Q}}$.
\end{remark}
To remedy this, consider the equivalence relation on $\mathcal{T}$: $f\sim g$ if $f=g$ a.e.

Consider the equivalence classes of $\mathcal{T}$ under $\sim$, $L^1(\mathbb{R}) = \mathcal{T}/\sim$, e.g., $[f]\in\mathcal{T}/\sim$.

Now define a norm on $L^1(\mathbb{R})$ by $\|[f]\| = \int|f|\diff m$, then $L^1(\mathbb{R})$ has a vector space structure
\begin{align*}
[f]+[g]&=[f+g]\\
\alpha[f]&=[\alpha f]
\end{align*}

Consider a vector subspace $\mathcal{N}$ of $\mathcal{T}$ defined by
\[
\mathcal{N}:=\{g\in\mathcal{T}\mid g=0\text{ a.e.}\}
\]
then $\mathcal{T}/\sim=\mathcal{T}/\mathcal{N}$.

Therefore, we get an normed space:
\begin{align*}
\|\alpha[f]\|&=|\alpha|\|[f]\|\\
\|[f]+[g]\|&\le \|[f]\|+\|[g]\|\\
\|[f]\|&=0
\Longleftrightarrow
\int|f|\diff m=0
\Longleftrightarrow
f=0\text{ a.e.}
\Longleftrightarrow
[f]=0_{L^1(\mathbb{R})}
\end{align*}
Similarly, we can study $L^2(\mathbb{R}),\dots,L^p(\mathbb{R})$.

For example, $L^2(\mathbb{R}) = \{f:\mathbb{R}\to[-\infty,\infty]\mid \int |f|^2\diff m<\infty\}/\mathcal{N}$ and $\|f\|_2=(\int|f|^2\diff m)^{1/2}$

\begin{example}
Consider $f = \sum_{k=0}^\infty \frac{(-1)^k}{k+1}\mathcal{X}_{[k,k+1)}$

Then the improper Riemann integral gives 
\[
\int_0^\infty f(x)\diff x=\log(2) = 1-\frac{1}{2}+\frac{1}{3}-\cdots
\]

However, $f$ is not Lebesgue integrable.
Suppose on the contrary that it is , then $|f|$ is integrable:
\[
|f| = \sum_{k=0}^\infty\frac{1}{k+1}\mathcal{X}_{[k,k+1)}
\]
Consequences:
\begin{align*}
\int|f|\diff m&=\lim_{n\to\infty}\sum_{k=0}^n\int\left(\frac{1}{k+1}\mathcal{X}_{[k,k+1)}\right)\diff m\\
&=\sum_{k=0}^\infty\frac{1}{k+1}=\infty
\end{align*}

Concluson:
there are improper Riemann integrable functions that are not Lebesgue integrable.
However, all the proper Riemann integrable functions are Lebesgue integrable (and the integrals have the same value). (Prove it in two classes.)
\end{example}

\begin{theorem}[MCT II]
Let $\{f_n\}$ be a sequence of integrable functions such that
\begin{enumerate}
\item
$f_n\le f_{n+1}$ a.e.
\item
$\sup_n\int f_n\diff m<\infty$
\end{enumerate}
Then $f_n$ converges to an integrable function $f$ a.e., and
\[
\int f\diff m = \lim_{n\to\infty}\int f_n\diff m
\]
\end{theorem}
\begin{proof}
Re-define $f_n$ by changing its values on a null set such that
\begin{enumerate}
\item
$f_n(x)\in\mathbb{R}$, for any $x\in\mathbb{R}$
\item
$f_n(x)\le f_{n+1}(x)$ for any $n\in\mathbb{R},x\in\mathbb{R}$
\end{enumerate}
Let $f(x) = \lim_{n\to\infty}f(x)$.
Consider the sequence of functions $\{f_n - f_1\}_{n\in\mathbb{N}}$, then
\begin{enumerate}
\item
$f_n-f_1\ge0$
\item
$f_n-f_1$ is monotone increasing, integrable
\item
$f_n-f_1\to f-f_1$
\end{enumerate}
Then we apply the MCT II and get
\[
\int(f-f_1)\diff m =\lim_{n\to\infty}\int(f_n-f_1)\diff m
\]
which implies
\[
\int(f-f_1)\diff m+\int f_1\diff m
=
\lim_{n\to\infty}\int(f_n-f_1)\diff m+\int f_1\diff m
=
\lim_{n\to\infty}\int f_n\diff m 
\]
note that $\lim_{n\to\infty}\int f_n\diff m $ exists as $\lim_{n\to\infty}\int f_n\diff m =\sup_n\int f_n\diff m<\infty$.

Note that $\int(f-f_1)\diff m+\int f_1\diff m$ is integrable as it equals to $\lim_{n\to\infty}\int f_n\diff m<\infty$.

The proof is complete.
\end{proof}

















