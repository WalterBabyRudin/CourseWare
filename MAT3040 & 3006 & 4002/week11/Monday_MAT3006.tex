
\section{Monday for MAT3006}\index{Monday_lecture}

\paragraph{Reviewing}
Compute the integration
\[
\int_0^1(1-x)^{-1/2}\diff x
\]
Take $g_n(x) = (1-x)^{-1/2}\mathcal{X}_{[0,1-1/n]}$, then $g_n(x)\to (1-x)^{-1/2}\mathcal{X}_{[0,1)}$.

The MCT says that
\[
\int_{[0,1)}(1-x)^{-1/2}\diff x=\lim_{n\to\infty}\int g_n\diff m
=
2.
\]

Question: how about $\int_{[0,1]}(1-x)^{-1/2}\diff x$?

Answer: 
\[
(1-x)^{-1/2}\mathcal{X}_{[0,1)}=\left\{
\begin{aligned}
(1-x)^{-1/2},&\quad\text{if $0\le x<1$}\\
\infty,&\quad x=1
\end{aligned}
\right.\ \text{a.e.}
\]
\begin{proposition}
If $f,g$ are measurable non-negative functions, and $f=g$ a.e., then
\[
\int f\diff m=\int g\diff m
\]
\end{proposition}
\begin{proof}
Let $U=\{x\in\mathbb{R}\mid f(x) = g(x)\}$, then
\[
f=f\cdot \mathcal{X}_{U}+f\cdot \mathcal{X}_{U^c}
\]
Therefore,
\begin{align*}
\int f\diff m &= \int f \mathcal{X}_{U} + \int f\mathcal{X}_{U^c}=\int g \mathcal{X}_{U}+0\\
&=\int g\mathcal{X}_{U} + \int g\mathcal{X}_{U^c}\\
&=\int g\diff m
\end{align*}
where the third equality is because $g\cdot\mathcal{X}_{U^c}=0$ a.e.
\end{proof}
\begin{remark}
Consequence of Markov inequality:
integration of non-negative measurable function is zero iff if this function equals to 0 a.e..
\end{remark}

\begin{proposition}
Slight generalization of MCT: 
Suppose that $f_n(x)$ is nonnegative measurable and monotone increasing a.e., ($f_n(x)\le f_{n+1}(x)$ for almost all $x$) with $f_n(x)\to f(x)$ a.e., then
\[
\lim_{n\to\infty}\int f_n\diff m = \int f\diff m
\]
\end{proposition}

\begin{proof}
For each $n$, replace the value of $f_n(x)$ on a null set such that 
\[
f_n(x)\le f_{n+1}(x), \ \forall x\in\mathbb{R}
\]
Do this for each $n$, such that $f_n(x)\le f_{n+1}(x),\forall x\in\mathbb{R},\forall n\in\mathbb{N}$.

The difference between the odd $f_n(x)$ and the new $f_n(x)$ occurs only on a countable union of null set, which is still null.

So there new $f_n(x)\to g(x)$, where $g(x)=f(x)$ a.e.

Apply (3) and the old MCT to get 
\[
\lim_{n\to\infty}\int\tilde{f}_n\diff m=\int g\diff m = \int f\diff m=\lim_{n\to\infty}\int f_n\diff m
\]
\end{proof}

\begin{proposition}
Let $\{f_k\}$ be non-negative measurable and 
\[
f:=\sum_{k=1}^\infty f_k,
\]
then 
\[
\int f\diff m = \sum_{k=1}^\infty \int f_k\diff m
\]
\end{proposition}
Note that $f$ is measurable since
\[
f=\lim_{n\to\infty}\sum_{k=1}^nf_k
\]
\begin{proof}
Take $g_n = \sum_{k=1}^nf_k$ and therefore $g_n$ is monotone increasing. Then apply the MCT.
\end{proof}
\begin{example}
Consider
\[
(1-x)^{-1/2} = \sum_{n=0}^\infty\frac{(2n)!}{4^n(n!)^2}x^n
\]
Take $f_k = \frac{(2k)!}{4^k(k!)^2}x^k$ and we get
\[
\int_{[0,1]}(1-x)^{-1/2}\diff x=\sum_{n=0}^\infty\int_0^1\frac{(2n)!}{4^n\cdot (n!)^2}x^n\diff x
\]
Or equivalently,
\[
2 = \sum_{n=0}^\infty\frac{(2n)!}{4^n(n!)(n+1)!}
\]

\end{example}
\subsection{MCT II}
We now extend our study to all measurable functions

\begin{definition}
Let $f$ be a measurable function, then let
\[
f^+(x) = \left\{
\begin{aligned}
f(x),&\quad\text{if $f(x)>0$}\\
0,&\quad\text{if $f(x)\le0$}
\end{aligned}
\right.
=
f(x)\mathcal{X}_{f^{-1}((0,\infty])}
\]
and
\[
f^-(x) = \left\{
\begin{aligned}
-f(x),&\quad\text{if $f(x)\le0$}\\
0,&\quad\text{if $f(x)>0$}
\end{aligned}
\right.
=
-f(x)\mathcal{X}_{f^{-1}([-\infty,0])}
\]
As a result, $f^+$ and $f^-$ are both measurable.

Note that
\begin{itemize}
\item
$f(x) = f^+(x) - f^-(x)$
\item
$|f|(x) = f^+(x)+f^-(x)$
\end{itemize}
Now we define the Lebesgue integral of $f$ as
\[
\int f\diff m = \int f^+\diff m - \int f^-\diff m
\]
We say $f$ is \emph{Lebesgue integrable} if both $f^+$ and $f^-$ are interable, i.e., $\int f^{\pm}\diff m<\infty$
\end{definition}

\begin{proposition}
\begin{enumerate}
\item
If $f$ is measurable, then $f$ is integrable if and only if $|f|$ is integrable
\item
If $f$ is measurable, and $|f|\le g$ with $g$ integrable, then $f$ is also integrable
\end{enumerate}
\end{proposition}
\begin{proof}
\begin{enumerate}
\item
If $f$ is integrable, then $\int f^+\diff m,\int f^-\diff m<\infty$.
As a result,
\[
\int|f|\diff m = \int(f^++f^-)\diff m=\int f^+\diff m+\int f^-\diff m
<\infty
\]

For the reverse direction, if $|f|$ is integrable, then 
\[
\int|f| = \int f^++\int f^-
\]
therefore $\int f^{\pm}<\infty$, and hence $f$ is interable.
\item
Suppose $(0\le )|f|\le g$, then $\int|f|\diff m\le \int g\diff m<\infty$.

Therefore, $\int|f|\diff m<\infty$, and hence $|f|$ is integrable, which implies $f$ is integrable.
\end{enumerate}
\end{proof}

\begin{remark}
If $|f|\le g$, and $\int|f|\diff m=\infty$, then it also implies $\int g\diff m=\infty$.
\end{remark}












