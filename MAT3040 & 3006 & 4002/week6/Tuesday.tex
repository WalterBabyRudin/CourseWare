
\chapter{Week6}

\section{Ring}\index{week6_Tuesday_lecture}
\begin{definition}[Ring]
A \emph{ring} $R=(R,+,*)$ is a set equipped with \emph{two} binary operations:
\[
+,*:R\times R\to R,
\]
\begin{enumerate}
\item
$(R,+)$ is an \emph{abelian} group with an \emph{additive identity} $0$
\item
The multiplication $*$ is \emph{associative}, i.e.,
\[
(a*b)*c = a*(b*c),\quad
\forall a,b,c\in R
\]
\item
$R$ satisfies the \emph{distributive laws}: for $\forall a,b,c\in R$, we have
\begin{enumerate}
\item
$a*(b+c) = a*b + a*c$
\item
$(a+b)*c = a*c + b*c$
\end{enumerate}
\end{enumerate}
Moreover, if $R$ has a \emph{multiplicative identity} $1\in R$ such that
\[
1*a=a*1=a,\forall a\in R,
\]
then $R$ is called a \emph{unital ring}.
\end{definition}
Question for ring: Does the ring contain the additive inverse?
\begin{example}
\begin{enumerate}
\item
$\mathbb{Z},\mathbb{Q},\mathbb{R},\mathbb{C}$ are unital rings; $2\mathbb{Z}$ is a ring but not unital: since $1\notin2\mathbb{Z}$
\item
The set of polynomials $\mathbb{Z}[x],\mathbb{Q}[x],\mathbb{R}[x],\mathbb{C}[x]$ are \emph{unital rings}
\item
$\mathbb{Q}[\sqrt{2}] = \{a+b\sqrt{2}\mid a,b\in\mathbb{Q}\}$ is a unital ring
\item
$\bm M_2(\mathbb{Z})$ is a unital ring; $\{\begin{bmatrix}
a&a\\a&a
\end{bmatrix}\mid a\in\mathbb{Z}\}$ is also a unital ring (Question?)
\item
$\mathcal{C}[a,b]$ is a unital ring
\item
$(\mathbb{N},+,*)$ is \emph{not} a ring. (Question? Is the set $\mathbb{N}$ not containing $0$?)
\end{enumerate}
\end{example}
\begin{remark}
\begin{enumerate}
\item
We write $ab$ for $a*b$
\item
The additive identity $0$ for $R$ is \emph{unique}
\item
The additive inverse for any $r\in R$ is \emph{unique} (we pre-assume its existence)
\item
\emph{Commutativity} is required for addition but not necessarily for multiplication
\item
Each element in $R$ has an additive inverse, but not necessarily a multiplicative inverse, i.e., $\exists a\in R$ such that $ab\ne1$ for $\forall b\in R$.
\end{enumerate}
\end{remark}
\begin{proposition}
Each \emph{unital} ring $R$ contains a unique additive identity and a unique multiplicative identity
\end{proposition}
\begin{proof}
It suffices to show the uniqueness of multiplicative identity. Suppose $r_1,r_2$ are two multiplicative identity of $R$, then $r_1=r_1r_2=r_2$.
\end{proof}
\begin{proposition}
If $r\in R$ has a multiplicative inverse $r^{-1}$, then $r^{-1}$ is unique.
\end{proposition}
\begin{proof}
Suppose $r_1^{-1},r_2^{-1}$ are two multiplicative inverse of $r$, then $rr_1^{-1}=rr_2^{-1}=1$, which follows that
\[
r_1^{-1} = r_1^{-1} (rr_2^{-1})=(r_1^{-1}r)r_2^{-1}=r_2^{-1}
\]
\end{proof}
\begin{proposition}
For each $r\in R$, we have $0r=r0=0$.
\end{proposition}
\begin{proof}
By distributive laws,
\[
0r=(0+0)r=0r+0r,
\]
which follows that
\[
0=(0r+0r)+(-0r) = 0r+(0r+(-0r))=0r+0=0r.
\]
Question: Why not left-adding the term $0r$?

Similarly, we have $r0=0$.
\end{proof}

\begin{proposition}
For each $r\in R$, we have $(-1)(-r) = (-r)(-1)=r$.
\end{proposition}
\begin{proof}
Consider the equation
\[
0 = 0(-r) = (1+(-1))(-r) = -r + (-1)(-r),
\]
which follows that $(-1)(-r)=r$.

Similarly, $(-r)(-1)=r$.
\end{proof}
\begin{proposition}
For each $r\in R$, we have $(-1)r=r(-1)=-r$.
\end{proposition}
\begin{proof}
Consider the equation
\[
0=0r=(1+(-1))r=r+(-1)r
\]
which follows that $-r=(-1)r$
\end{proof}

\begin{proposition}
If a ring $R$ contains only a single element, then $R=\{0\}$. We call such $R$ a \emph{zero} ring.
\end{proposition}
\begin{proposition}
Let $R$ be a set with binary operations $+$ and $*$ such that $(R,+)$ is a group; $(R,*)$ is a monoid (i.e., associative and identity); $(R,+,*)$ satisfies the distributive laws. Then $+$ is commutative.
\end{proposition}
\begin{proof}
Note that
\begin{align*}
(1+1)(x+y)&=(x+y)+(x+y)=x+y+x+y\\
&=(1+1)x+(1+1)y=x+x+y+y
\end{align*}
\end{proof}

\begin{definition}[Commutative]
A ring $R$ is \emph{commutative} if its multiplication is \emph{commutative}:
\[
ab = ba,\forall ab\in R
\]
\end{definition}
\begin{example}
\begin{enumerate}
\item
$\mathbb{Z},\mathbb{Q},\mathbb{R},\mathbb{C}$ are commutative rings, and so are $\mathbb{Z}[x],\mathbb{Q}[x],\mathbb{R}[x],\mathbb{C}[x]$.
\item
The ring $\bm M_n(\mathbb{Z})$ is not commutative for $n>2$ (Question: is $n=2$ ok?)
\end{enumerate}
\end{example}
\subsection{Modular Arithmetic}
\begin{definition}[Congruent modulo]
Let $m\in\mathbb{Z}^+$. Then for $\forall a,b\in\mathbb{Z}$, we say they are \emph{congruent modulo} $m$ if $m\mid(a-b)$, i.e., $a\equiv b(\bmod m)$.
\end{definition}
This modular congruent defines an equivalence relation on $\mathbb{Z}$.
\begin{remark}
\begin{enumerate}
\item
Consider the set $\mathbb{Z}_m=\{0,1,2,\dots,m-1\}$. For each $n\in\mathbb{Z}$, let $\bar n$ denote the remainder of $n$ divided by $m$, and therefore $\bar n\in\mathbb{Z}_m$. Here $\mathbb{Z}_m$ can be viewed as a collection of equivalence class representatives, i.e., for $\forall a\in\mathbb{Z}$, it congruent modulo $m$ to unique one element in $\mathbb{Z}_m$
\item
Define the operations
\begin{align*}
\bar{a}+\bar{b}&=\overline{a+b}\\
\bar{a}*\bar{b}&=\overline{a*b}
\end{align*}
We can verify these operations are well-defined. Note that $(\mathbb{Z}_m,+)$ is a group; but $(\mathbb{Z}_m,*)$ is not necessarily a group, since the inverse of some element does not exist.
\item
Unless otherwise mentioned,
\begin{itemize}
\item
$(\mathbb{Z}_m,+)$ denotes a group
\item
$(\mathbb{Z}_m,+,*)$ denotes a ring.
\end{itemize}
\item
The modular congruence classes corresponds to the \emph{cosets} of $m\mathbb{Z}$ of $\mathbb{Z}$, and therefore $\mathbb{Z}/m\mathbb{Z}\cong\mathbb{Z}_m$.
\end{enumerate}
\end{remark}

\begin{proposition}
$(\mathbb{Z}_m,+,*)$ is a unital commutative ring.
\end{proposition}
\begin{proof}
We have shown $(\mathbb{Z}_m,+)$ is a group. It suffices to show $(\mathbb{Z}_m,*)$ is a commutative monoid, and the distributive laws:
\begin{enumerate}
\item
The associativity of multiplication is clear; the multiplication is commutative is easy to verify; the multiplicative identity is $1$
\item
\[
\bar a*(\bar b+\bar c)=\bar a*\overline{b+c}=\overline{a*(b+c)}=\bar{ab}+\bar{ac}=\bar{a}*\bar{b}+\bar{a}*\bar{c}
\]
The commutativity gives another distributive law.

\end{enumerate}


\end{proof}
\begin{proposition}
Let $m\in\mathbb{Z}^+$, suppose $a\equiv c(\bmod m)$ and $b\equiv d(\bmod m)$, then
\[
\begin{array}{ll}
a+b\equiv c+d(\bmod m),
&
ab\equiv cd(\mod m)
\end{array}
\]
\end{proposition}
\begin{proof}
Since $x\equiv x'(\bmod m)$ iff $\bar{x}=\bar{x'}$; immediately we have
\[
\overline{a+b}=\bar a+\bar b=\bar c+\bar d=\overline{c+d};
\]
the other equality follows similarly.
\end{proof}

\subsection{Rings of Polynomials}
\begin{definition}[polynomial over rings]
Let $R$ be a \emph{commutative} ring. A \emph{polynomial} (in a variable $x$) over $R$ is a formal sum
\[
f(x)=\sum_{i=0}^na_ix^i
\]
with $a_i\in R$ and $n=0$ or the leading coefficient $a_n\ne0$. 
\begin{enumerate}
\item
Here the degree of $f(x)$ is $\mbox{deg}(f)=n$
\item
$R[x]$ denotes the set of all polynomials over $R$.
\item
The addition and multiplication for any two elements $f:=\sum_{i=0}^ma_ix^i,g:=\sum_{i=1}^nb_ix^i$ in $R$ is given by:
\begin{align*}
f+g&:=\sum_{i=0}^{\max\{m,n\}}(a_i+b_i)x^i\\
fg&:=\sum_{i=0}^{m+n}(\sum_{j+k=i}a_jb_k)x^i
\end{align*}
\end{enumerate}. 
\end{definition}
\begin{proposition}
With $R$ defined above, ($R[x],+,*$) is a commutative ring.
\end{proposition}
\begin{proof}
Note that $(R[x],+)$ forms an abelian group. 
\begin{enumerate}
\item
The multiplication is associative
\item
$(R[x],*)$ has an identity element $f:=1$
\item
The multiplication is commutative
\item
The distributive laws are satisfied
\end{enumerate}

\end{proof}

\begin{remark}
A polynomial $f$ defines a function $f:R\to R$ by $a\mapsto f(a),$ but $f$ may not be determined by $f:R\to R$, e.g.,
\[
f(x)=1+x+x^2,g(x)=1,
\]
with the argument defined on $\mathbb{Z}_2$.
\end{remark}
\begin{proposition}
Find a nonzero function $f(x)\in\mathbb{Z}_6[x]$ such that $f(x)\equiv0$, i.e., $f(x)=0$ for all $x\in\mathbb{Z}_6$.
\end{proposition}
\begin{proof}
\[
f(x) = x(x-1)(x-2)(x-3)(x-4)(x-5)
\]
\end{proof}
Question: abuse of notation for $\mathbb{Z}_6[x]$.

\subsection{Integral Domains and Fields}
\begin{definition}[Integral Domain]
Let $D$ be a ring. A \emph{nonzero} $r\in D$ is called a \emph{zero divisor} if there exists a non-zero $s\in D$ such that $rs=0$ or $sr=0$. If $D$ has no zero divisors, then $D$ is called a \emph{domain}. A \emph{domain} that is a \emph{commutative ring} is an \emph{integral domain}.
\end{definition}
\begin{example}
\begin{enumerate}
\item
$\mathbb{Z},\mathbb{Q},\mathbb{R},\mathbb{C}$ are all integral domains, and so are $\mathbb{Z}[x],\mathbb{Q}[x],\mathbb{R}[x],\mathbb{C}[x]$. $R$ is an integral domain iff $R[x]$ is an integral domain.
\item
$\mathbb{Z}_6$ is \emph{not} an integral domain since $2*3\equiv0(\bmod 6)$. Thus $\mathbb{Z}_m$ is an integral domain iff $m$ is a prime.
\item
Let $R=C[-1,1]$, then $R$ is not a integral domain. Consider piecewise function.

\end{enumerate}
\end{example}
\begin{proposition}
Let $D$ be a commutative ring, then the followings are equivalent:
\begin{enumerate}
\item
$D$ is an integral domain
\item
For $\forall$ nonzero $a,b\in D$, we have $ab\ne0$
\item
$D$ satisfies the \emph{cancellation law}:
\[
ca=cb,c\ne0\implies 
a=b
\]
\end{enumerate}
\end{proposition}
\begin{proof}
It is clear that (1) is equivalent to (2);

For (1) implies (3): If $ca=cb$, then by distributive laws:
\[
c[a+(-b)]=ca+c(-b)=cb+c(-b)=c[b+(-b)]=0,
\]
which implies $c=0$ or $a+(-b)=0$ by applying the definition of integral domain, which implies $a=b$.

For (3) implies (1): suppose there exists nonzero $a,b\in D$ such that $ab=0$. Note that $0=a0$, which implies
\[
ab=a0\implies b=0,
\]
which is a contradiction.
\end{proof}
\begin{remark}
The proposition above can be generalized into non-commutative rings. Question.
\end{remark}
\begin{definition}
Let $R$ be a ring, then an element $a\in R$ is called a \emph{unit} if it has a \emph{multiplicative inverse} $a^{-1}\in R$ such that $aa^{-1}=a^{-1}a=1$.
\end{definition}
Question: Does such a ring \emph{unital}?
\begin{example}
\begin{enumerate}
\item
The only units of $\mathbb{Z}$ are $\pm1$
\item
Let $R:=\mathcal{F}(\mathbb{R})$, then a function $f\in R$ is a unit iff
\[
f(x)\ne0,\forall x\in\mathbb{R}
\]
\item
Let $R:=\mathcal{C}(\mathbb{R})$, then $f\in R$ is a unit iff it is either \emph{strictly positive} or \emph{strictly negative}.

\end{enumerate}
\end{example}
\begin{proposition}
The only units of $\mathbb{Q}[x]$ are \emph{nonzero constants}.
\end{proposition}
\begin{proof}
Take $f\in\mathbb{Q}[x]$ with $\mbox{deg}(f)\ge1$, aruge that $f$ cannot be unit. Then argue $f=0$ can not. For nonzero constant $f$, construct $g=1/f$ to be the inverse.
\end{proof}
\begin{definition}[Division Ring]
A \emph{division ring} $R$ is a ring that all its nonzero elements are units; furthermore, if $R$ is also commutative, then $R$ is a field.
\end{definition}
\begin{example}
\begin{enumerate}
\item
$\mathbb{Q},\mathbb{R},\mathbb{C}$ are fields, but $\mathbb{Z}$ is not
\item
$\mathbb{Q}[x],\mathbb{R}[x],\mathbb{C}[x]$ are not division rings.
\item
The \emph{quaternions}
\[
\mathbb{H}:=\{a+bi+cj+dk\mid a,b,c,d\in\mathbb{R},i^2=j^2=k^2=ijk=-1\}
\]
is a division ring with the usual addition and multiplication, but not a field.
\end{enumerate}
\end{example}
\begin{proposition}
A field is an integral domain.
\end{proposition}
\begin{proof}
Assume not, then $rs=0$ implies $r^{-1}rss^{-1}=0$, which is a contradiction.
\end{proof}
\begin{proposition}
Let $m\in\mathbb{Z}^+,k\in\mathbb{Z}_m^{\#}:=\mathbb{Z}_m\setminus\{0\}$. Let $d=\mbox{gcd}(k,m)$
\begin{enumerate}
\item
If $d=1$, then $k$ is a unit.
\item
If $d>1$, then $k$ is a zero divisor.
\end{enumerate}
\end{proposition}
\begin{proof}
\begin{enumerate}
\item
If $d=1$, there exists $a,b\in\mathbb{Z}$ such that
\[
ak+bm=1\implies \bar{a}\cdot k=1\implies\mbox{$k$ is a unit}
\]
If $d>1$, then $k=hd$ for some $h\in\mathbb{Z}_m$, which implies
\[
k\cdot(m/d)=hm=0,
\]
where $m/d\in\mathbb{Z}_m$
\end{enumerate}
\end{proof}
The results are summarized as follows:
\begin{align*}
\{\mbox{zero divisors in }\mathbb{Z}_m\}&=\{k\in\mathbb{Z}_m^{\#}\mid\mbox{gcd}(k,m)>1\}\\
\{\mbox{units in }\mathbb{Z}_m\}&:=\mathbb{Z}_m^*=\{k\in\mathbb{Z}_m^{\#}\mid\mbox{gcd}(k,m)=1\}
\end{align*}
\begin{proposition}
$(\mathbb{Z}_m^*,\cdot)$ forms a group, called the group of units in $\mathbb{Z}_m$.
\end{proposition}

\begin{corollary}
$\mathbb{Z}_m$ is a field iff $m$ is prime.
\end{corollary}
For each prime $p$, the field $\mathbb{Z}_p$ can be written as $\mathbb{F}_q$
\begin{definition}[Euler's phi function]
$\phi(n):=|\mathbb{Z}_n^*|$ is called the Euler's phi function, which denotes the number of units in the ring $\mathbb{Z}_m$.
\end{definition}
\begin{theorem}[Euler's Theorem]
Let $n\in\mathbb{Z}^+,a\in\mathbb{Z}_n^*$ be such that $\mbox{gcd}(a,n)=1$. Then $a^{\phi(n)}\equiv1(\bmod n)$.
\end{theorem}
\begin{proof}
It's clear that $\bar{a}\in\mathbb{Z}_n^*$. Suppose $\mathbb{Z}_n^*=\{u_1,\dots,u_{\phi(n)}\}$, and therefore
\[
u_{1}\bar a\cdots u_{\phi(n)}\bar a\equiv u_1\cdots u_{\phi(n)}(\bmod n),
\]
which implies $a^{\phi(n)}\equiv (u_1\cdots u_{\phi(n)})^2\equiv1(\bmod n)$
\end{proof}
\begin{proposition}
Let $F=\mathbb{Q}[\sqrt{2}]=\{a+b\sqrt{2}\mid a,b\in\mathbb{Q}\}$, then $F$ is a field.
\end{proposition}
Question: A field $F$ can be equivalent to:
\begin{enumerate}
\item
closed addition and multiplication
\item
Identity and inverse for addition and multiplication
\item
Associativity of $*$
\item
Distributive law
\end{enumerate}
\begin{proof}
For the multiplicative inverse,
\[
(a+b\sqrt{2})=\frac{a}{a^2-2b^2}-\frac{b}{a^2-2b^2}\sqrt{2}
\]
\end{proof}
\begin{proposition}
Every finite integral domain $D$ with $1_D$ is a field
\end{proposition}
\begin{proof}
Consider $aa_i=aa_j$ iff $i=j$ by applying the distributive law
\end{proof}

\begin{proposition}
Every finite integral domain $D$ contains a multiplicative identity $1_D$
\end{proposition}
\begin{proof}
\[
xa^n=xa^m\implies xa^{n-m}=x
\]
\end{proof}
\begin{definition}[Characteristic]
Let $R$ be a ring. For each $n\in\mathbb{N},a\in R$, define
\[
\begin{array}{ll}
n\circ a=\underbrace{a+\cdots+a}_{\text{$n$ terms}},
&
0\circ a = 0_R
\end{array}
\]
then $n$ is called the \emph{characteristic} of the ring $R$; if such $n$ does not exist, then $R$ is of characteristic $0$. The characteristic of $R$ is denoted as $\mbox{char}(R)$. If $R=F$ is a field, then $\mbox{char}(F)$ is the \emph{characteristic} of the field $F$.
\end{definition}
\begin{example}
\[
\mbox{char}(\mathbb{Z}_n)=n
\]
\[
\mbox{char}(\mathbb{Z})=\mbox{char}(\mathbb{Q})=\mbox{char}(\mathbb{R})=\mbox{char}(\mathbb{C})=0
\]
\end{example}
\begin{proposition}
The characteristic of an integral domain is either 0 or a prime,
\end{proposition}
\begin{proof}
Consider 
\[
(m\circ a)*(n\circ a)=(m*n)\circ a=0
\]
which implies $k\circ a=0$ or $l\circ a=0$.
\end{proof}
Question: multiplication?


\begin{theorem}
Let $R$ be a \emph{unital} ring. If there exists a smallest $n\in\mathbb{Z}^+$ such that $n\circ 1=0$, then $\mbox{char}(R)=0$, otherwise $\mbox{char}(R)=0$
\end{theorem}
\begin{proof}
Suppose there exists, then $\mbox{char}(R)\ge n$.
\[
n\circ a=a(1+\cdots+1)=a*(n\circ1)=0
\]
thus $\mbox{char}(R)\le n$.
\end{proof}

\subsection{Field of fractions}
To make up a integral domain to be a field, we need to add some extra elements.
\paragraph{Equivalence relation}
Let $R$ be an integral domain and $S:=\{(a,b)\mid a,b\in R,b\ne0\}$
\[
(a,b)\sim(c,d)\mbox{ iff }ad=bc
\]
Define
\[
(a,b)+(c,d)=(ad+bc,bd);(a,b)*(c,d)=(ac,bd)
\]

\begin{proposition}
Suppose $(a,b)\sim(a',b')$ and $(c,d)\sim(c',d')$, then
\[
(a,b)+(c,d)\sim(a',b')+(c',d'),\qquad
(a,b)*(c,d)\sim(a',b')*(c',d')
\]
\end{proposition}
\begin{definition}[Quotient set]
Equipped with $(S,\sim)$, we define \emph{quotient} set $S/\sim$ to be the set of all equivalence classes of $S$ w.r.t. $\sim$
\end{definition}


\begin{example}
For $\sim$ on $\mathbb{Z}$ s.t. $a\sim b$ iff $a\equiv b(\bmod 2)$, we have
\[
\mathbb{Z}/\sim  = \{2\mathbb{Z},2\mathbb{Z}+1\}
\]
\end{example}
\begin{definition}[Fraction field]
Equipped with $(S,\sim)$, where $S=\{(a,b)\mid a,b\in R,b\ne0\}$, we define \emph{fraction field} of $R$ to be the set $\mbox{Frac}(R):=S/\sim$, with the operation
\begin{align*}
[(a,b)]+[(c,d)]&=[(ad+bc,bd)]\\
[(a,b)]*[(c,d)]&=[(ac,bd)]
\end{align*}
\end{definition}


\begin{proposition}
Let $R$ be an integral domain, then $\mbox{Frac}(R)$ forms a field with additive identity $0=[(0,1)]$ and the multiplicative identity $1=[(1,1)]$. The multiplicative inverse of a non-zero element $[(a,b)]\in\mbox{Frac}(R)$ is $[(b,a)]$
\end{proposition}
\begin{remark}
When $R=\mathbb{Z}$, we find $[(a,b)]\in\mbox{Frac}(\mathbb{Z})$ since $a/b\in\mathbb{Q}$, and therefore $\mbox{Frac}(\mathbb{Z})\cong\mathbb{Q}$
\end{remark}




















