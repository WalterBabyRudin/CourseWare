
\chapter{Week1}
\section{Monday for MAT3040}\index{Monday_lecture}

\subsection{Introduction to Advanced Linear Algebra}
Advanced Linear Algebra is one of the most important course in MATH major, with pre-request MAT2040. This course will offer the really linear algebra knowledge.
\paragraph{What the content will be covered?}
\begin{itemize}
\item
In MAT2040 we have studied the space $\mathbb{R}^n$; while in MAT3040 we will study the general vector space $V$. 
\item
In MAT2040 we have studied the \textit{linear transformation} between Euclidean spaces, i.e., $T:\mathbb{R}^n\to\mathbb{R}^m$; while in MAT3040 we will study the linear transformation from vector spaces to vector spaces: $T:V\to W$
\item
In MAT2040 we have studied the eigenvalues of $n\times n$ matrix $\bm A$; while in MAT3040 we will study the eigenvalues of a \emph{linear operator} $T:V\to V$.
\item
In MAT2040 we have studied the dot product $\bm x\cdot \bm y=\sum_{i=1}^nx_iy_i$; while in MAT3040 we will study the \emph{inner product} $\inp{\bm v_1}{\bm v_2}$.
\end{itemize}
\paragraph{Why do we do the generalization?}
We are studying many other spaces, e.g., $\mathcal{C}(\mathbb{R})$ is called the space of all functions on $\mathbb{R}$, $\mathcal{C}^\infty(\mathbb{R})$ is called the space of all infinitely differentiable functions on $\mathbb{R}$, $\mathbb{R}[x]$ is the space of polynomials of one-variable.
\begin{example}
\begin{enumerate}
\item
Consider the Laplace equation $\Delta f=0$ with linear operator $\Delta$:
\[
\begin{array}{ll}
\Delta:\mathcal{C}^\infty(\mathbb{R}^3)\to\mathcal{C}^\infty(\mathbb{R}^3)
&
f\mapsto(\frac{\partial^2}{\partial x^2}+\frac{\partial^2}{\partial y^2}+\frac{\partial^2}{\partial z^2})f
\end{array}
\]
The solution to the PDE $\Delta f=0$ is the 0-eigenspace of $\Delta$.
\item
Consider the Schrödinger equation
$
\hat Hf = Ef
$
with the linear operator 
\[
\begin{array}{ll}
\hat H:\mathcal{C}^\infty(\mathbb{R}^3)\to\mathcal{C}^\infty(\mathbb{R}^3),
&
f\to
\left[{\frac
 {-\hbar ^{2}}{2\mu }}\nabla ^{2}+V(x,y,z )\right]f
\end{array}
\]
Solving the equation $\hat Hf=Ef$ is equivalent to finding the eigenvectors of $\hat H$. In fact, the eigenvalues of $\hat H$ are \emph{discrete}.
\end{enumerate}
\end{example}

\subsection{Vector Spaces}
\begin{definition}[Vector Space]
A \emph{vector space} over a field $\mathbb{F}$ (in particular, $\mathbb{F}=\mathbb{R}$ or $\mathbb{C}$) is a set of objects $V$ equipped with vector addiction and scalar multiplication such that 
\begin{enumerate}
\item
the vector addiction $+$ is closed with the rules:
\begin{enumerate}
\item
\textbf{Commutativity}: $\forall\bm v_1,\bm v_2\in V$, $\bm v_1+\bm v_2=\bm v_2+\bm v_1$.
\item
\textbf{Associativity}: 
$\bm v_1+(\bm v_2+\bm v_3)=(\bm v_1+\bm v_2)+\bm v_3$.
\item
\textbf{Addictive Identity}: $\exists\bm0\in V$ such that $\bm0+\bm v=\bm v$, $\forall\bm v\in V$.
\end{enumerate}
\item
the \emph{scalar multiplication} is closed with the rules:
\begin{enumerate}
\item
\textbf{Distributive}:
$\alpha(\bm v_1+\bm v_2) = \alpha\bm v_1+\alpha\bm v_2$,$\forall\alpha\in\mathbb{F}$ and $\bm v_1,\bm v_2\in V$
\item
\textbf{Distributive}:
$(\alpha_1+\alpha_2)\bm v=\alpha_1\bm v+\alpha_2\bm v$
\item
\textbf{Compatibility}:
$a(b\bm v) = (ab)\bm v$ for $\forall a,b\in\mathbb{F}$ and $\bm b\in V$.
\item
$0\bm v=\bm0$, $1\bm v=\bm v$.
\end{enumerate}
\end{enumerate}
\end{definition}
Here we study several examples of vector spaces:
\begin{example}
For $V=\mathbb{F}^n$, we can define
\begin{enumerate}
\item
Addictive Identity:
\[
\bm0=\begin{pmatrix}
0\\\vdots\\0
\end{pmatrix}
\]
\item
Scalar Multiplication:
\[
\alpha\begin{pmatrix}
x_1\\\vdots\\x_n
\end{pmatrix}=
\begin{pmatrix}
\alpha x_1\\\vdots\\\alpha x_n
\end{pmatrix}
\]
\item
Vector Addiction:
\[
\begin{pmatrix}
x_1\\\vdots\\x_n
\end{pmatrix}
+
\begin{pmatrix}
y_1\\\vdots\\y_n
\end{pmatrix}
=
\begin{pmatrix}
x_1+y_1\\\vdots\\x_n+y_n
\end{pmatrix}
\]
\end{enumerate}
\end{example}
\begin{example}
\begin{enumerate}
\item
It is clear that the set $V=M_{n\times n}(\mathbb{F})$ (the set of all $m\times n$ matrices) is a vector space as well.
\item
The set $V=\mathcal{C}(\mathbb{R})$ is a vector space:
\begin{enumerate}
\item
Vector Addiction:
\[
(f+g)(x)=f(x)+g(x),\forall f,g\in V
\]
\item
Scalar Multiplication:
\[
(\alpha f)(x)=\alpha f(x),\forall\alpha\in\mathbb{R},f\in V
\]
\item
Addictive Identity is a zero function, i.e., $\bm0(x)=0$ for all $x\in\mathbb{R}$.
\end{enumerate}
\end{enumerate}
\end{example}

\begin{definition}
A sub-collection $W\subseteq V$ of a vector space $V$ is called a \emph{vector subspace of $V$} if $W$ itself forms a vector space, denoted by $W\le V$.
\end{definition}
\begin{example}
\begin{enumerate}
\item
For $V=\mathbb{R}^3$, we claim that $W=\{(x,y,0)\mid x,y\in\mathbb{R}\}\le V$
\item
$W=\{(x,y,1)\mid x,y\in\mathbb{R}\}$ is not the vector subspace of $V$.
\end{enumerate}
\end{example}
\begin{proposition}
$W\subseteq V$ is a \emph{vector subspace} of $V$ iff for $\forall\bm w_1,\bm w_2\in W$, we have $\alpha\bm w_1+\beta\bm w_2\in W$, for $\forall\alpha,\beta\in\mathbb{F}$.
\end{proposition}
\begin{example}
\begin{enumerate}
\item
For $V=M_{n\times n}(\mathbb{F})$, the subspace $W=\{A\in V\mid \bm A\trans=\bm A\}\le V$
\item
For $V=\mathcal{C}^\infty(\mathbb{R})$, define $W=\{f\in V\mid\frac{\diff^2}{\diff x^2}f+f=0\}\le V$. For $f,g\in W$, we have
\[
(\alpha f+\beta g)''=\alpha f''+\beta g''=\alpha(-f)+\beta(-g)=-(\alpha f+\beta g),
\]
which implies $(\alpha f+\beta g)''+(\alpha f+\beta g)=0$.
\end{enumerate}
\end{example}





%
%\[
%\left[
%\begin{array}{@{}cc|c@{}}
%1 & 2 & 5 \\
%4 & 5 & 14
%\end{array}
%\right]
%\implies
%\left[
%\begin{array}{@{}cc|c@{}}
%1 & 2 & 5 \\
%0& -3 & -6
%\end{array}
%\right]
%\implies
%\left[
%\begin{array}{@{}cc|c@{}}
%1 & 2 & 5 \\
%0& 1 & 2
%\end{array}
%\right]
%\implies
%\left[
%\begin{array}{@{}cc|c@{}}
%1 & 0 & 1 \\
%0& 1 & 2
%\end{array}
%\right].
%\]













