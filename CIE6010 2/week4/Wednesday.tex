
\chapter{Week4}

\section{Wednesday}\index{week4_Wednesday_lecture}
\subsection{Comments for MATLAB Project}
You need to take care of several things during your assignment:
\paragraph{Do Not Repeat Computation} For example, 
enter
\[
|f(x^{k+1}) - f(x^k)|\le 10^{-\varepsilon}|f(x^k)|,
\]
is very bad, since you evaluated this function three times.
\paragraph{Arrange Computation Properly} For example, compute
\[
\bm x\bm x\trans\bm A
\]
is very expensive, but $\bm x(\bm x\trans\bm A)$ is not. Be aware of the size of matrices or vectors.
\paragraph{Appreciate sparsity} For example, when faced with high-dimensional matrices, compute $\bm A\bm D\bm A\trans$ is bad for using $\mathrm{D = diag(d)}$, while using $\mathrm{D = sparse(1:n,1:n,d)}$ is better.
\paragraph{Grading Criteria}
Your code should be at least faster than the testing script, while the error should be smaller than the testing script.

\subsection{Local Convergence Rate}
The study of the rate of convergence is often the dominant criteria for selecting appropirate algorithms. In this lecture we only focus on the local behaviour of the method in a neighborhood of an optimal solution.
\begin{definition}[$Q_1$ Factor]
Restrict the attention to a convergent sequence $\{\bm x^k\}$ with limit $\bm x^*$. Define an error function $e_k=\|\bm x^k-\bm x^*\|\to0$. The $Q_1$ factor of $\{\bm x^k\}$ is given as:
\[
Q_1 = \lim_{k\to\infty}\sup\frac{e_{k+1}}{e_k}
\]
\end{definition}
Here we want to study the performance of $e_k$. In our case, we compare $\{e_k\}$ with the geometric prograssion
\[
\begin{array}{ll}
\beta^k,
&
k=0,1,\dots
\end{array}
\]
\begin{itemize}
\item
If there exists $\beta\in(0,1)$ such that
\[
Q_1\le \beta,
\]
then we can show $e_k\le q\beta^k$ for some $q>0$. In this case $\{e_k\}$ is said to be \emph{Q-linear convergent}.
\item
If $Q_1=0$, then we say $\{e_k\}$ is \emph{Q-super-linear convergent}
\item
If $Q_1=1$, then we say $\{e_k\}$ is \emph{Q-sub-linear convergent}
\end{itemize}

\begin{remark}
\begin{enumerate}
\item
Q-linear convergence is not always so good, e.g., $\beta=.999$ may require 20000 iterations to meet satisfaction, while $\beta=.1$ may only require 20.
\item
Linear is always better than sublinear; super linear is always btetter than linear.
\end{enumerate}
\end{remark}

We have a faster type of convergence:
\begin{definition}[$Q_2$ Factor]
\[
Q_2 = \lim\sup_{k\to\infty}\frac{e^{k+1}}{(e^k)^2}
\]
If $Q_2=M<+\infty$,i.e., $e^{k+1}=O((e^k)^2)$, then $\{x^k\}\to x^*$ $Q$-quadratically.
\end{definition}
Newton's method generally gives us the quadratic convergence.
\subsection{Newton's Method}

The newton's method requires to solve a non-linear system of equations $\nabla f(x)=0$.

We don't know how to solve non-linear system in general. Fortunately, Newton gives us the remediation: 
in order to search for $\bm d$ to make $\nabla f(\bm x+\bm d)=0$, do the linearization first.
\begin{equation}\label{Eq:4:1}
\nabla f(\bm x+\bm d)\approx \nabla f(\bm x)+\inp{\nabla^2 f(\bm x)}{\bm d}=0
\end{equation}

To get the optimal solution, it suffices to solve (\ref{Eq:4:1}) for $\bm d$:
\[
\bm d=-(\nabla^2 f(\bm x))^{-1}\nabla f(\bm x),
\]
and hence update the solution to be $\bm x\leftarrow \bm x+\alpha\bm d$.

\paragraph{Interpretation}
In order to minimize a strictly convex function $f(x)$, we find
\[
f(\bm x+\bm d) \approx f(\bm x)+\inp{\nabla f(\bm x)}{\bm d}+\frac{1}{2}\bm d\trans\nabla^2 f(\bm x)\bm d:=q(\bm d)
\]
It suffices to minimize $q(\bm d)$:
\[
\nabla q(\bm d)=0\Longleftrightarrow
\nabla f(\bm x)+\nabla^2f(\bm x)\bm d=0
\]
How to guarntee $\bm d$ is the descent direction? Not necessarily we are able to do that. It becomes an art.

\paragraph{Convergence of Rate}
\begin{proposition}
Newton's method gurantees the $Q$-quadratic convergence.
\end{proposition}
\begin{proof}
Given a nonlinear system $F(x)=0$, suppose the sequence $\{\bm x^k\}$ is generated by Newton with limit $\bm x^*$ and $F(\bm x^*)=0$. By Newton's iteration,
\begin{equation}
\bm x^{k+1} =\bm x^k - [F'(\bm x^k)]^{-1}F(\bm x^k),
\end{equation}
which follows that
\begin{equation}
\begin{aligned}
\bm x^{k+1} - \bm x^* &=  \bm x^k - \bm x^* -  [F'(\bm x^k)]^{-1}\left(F(\bm x^k) - F(\bm x^*)\right)\\
&=[F'(\bm x^k)]^{-1}\left(F(\bm x^*) - F(\bm x^k) - F'(\bm x^k)(\bm x^* - \bm x^k)\right)
\end{aligned}
\end{equation}

Note that $F(\bm x^*) =F(\bm x^k) +F'(\bm x^k)(\bm x^* - \bm x^k)+O(\|\bm x^k-\bm x^*\|^2)$, which implies
\[
\|\bm x^{k+1} - \bm x^*\|\le \|[F'(x^k)]^{-1}\|O(\|\bm x^k - \bm x^*\|^2)
=
O(\|\bm x^k - \bm x^*\|^2).
\]

\end{proof}
\begin{remark}
During the proof we assume two things:
\begin{enumerate}
\item
Step-size is $1$!
\item
The limit exists.
\end{enumerate}
\begin{itemize}
\item
Hence, in practice, to implement Newton's method, $\bm 1$ is the first choice; but gredient descent method is not.
\item
Newton's method is good for nice problems, i.e., the funciton is convex, and the inverse of gredient is easy to solve.
\item
In machine learning most time we implement the gredient descent method, since Newton's method is expensive for computing the inverse.
\end{itemize}
\end{remark}
\subsection{Tutorial: Introduction to Convexity}
First we discuss some exercises:
\begin{enumerate}
\item
Given a sequence of convex functions $f_i$, the maximum over all functions
\[
\max\{f_1(x),f_2(x),\dots,f_n(x)\}:=f(x)
\]
is also convex. (Proof using Epi-Graph)
\item
Given a sequence of convex functions $f_i$, the combination $\sum_i\lambda_if_i$ is also convex
\item
Composition function may not be convex, .e.g., if $h=x^2$ and $g = -x$, we find $g\circ h$ is concave.
\item
However, if $g$ is convex and $h$ is affine, then $g\circ h$ is convex; if $g,h$ is convex, and $g$ is non-decreasing, then $g\circ h$ is convex.
\end{enumerate}
\paragraph{Convexity Examples}
Given the problem
\[
\begin{array}{ll}
\min&\quad \|\bm x\|_0\\
\mbox{s.t.}&\quad\bm{Ax}=\bm b,
\end{array}
\]
which is difficult to solve. Alternatively, we relax it and solve
\[
\begin{array}{ll}
\min&\quad \|\bm x\|_1\\
\mbox{s.t.}&\quad\bm{Ax}=\bm b
\end{array}
\]

Also, solving the problem
\[
\begin{array}{ll}
\min&\quad f(X)\\
\mbox{s.t.}&\quad\rank(X)=k
\end{array}
\]
is hard, we relax it and solve
\[
\begin{array}{ll}
\min&\quad f(X)\\
\mbox{s.t.}&\quad\|X\|_*\le k
\end{array}
\]
\paragraph{Announcement}
Learn and implement these things by yourself:
\begin{itemize}
\item
Luo's note: Lecture $\#4$, P7,Nesterov’s optimal 1st-order (also called acceleration) method
\item
Textbook P67-P72.
\end{itemize}
