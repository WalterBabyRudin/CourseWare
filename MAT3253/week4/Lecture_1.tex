
\chapter{Week4}

\section{Convergence}\index{week4_Friday_lecture}

\begin{definition}[Convergent]
An infinite sequence $\{z_n\}$ of complex numbers has a limit $z_0$, if for $\forall$ $\varepsilon>0$, there exists a positive integer $n_0$ such that
\[
\begin{array}{ll}
|z_n-z|<\varepsilon,
&
\mbox{whenever }n>n_0
\end{array}
\]
We say the sequence $z_n$ converges to $z$ and write as
\[
\lim_{n\to\infty}z_n=z
\]
When the sequence does not have a limit, then it diverges.
\end{definition}
The uniqueness of limit of a seuqnece is guaranteed.

\begin{proposition}
For $z_n=x_n+iy_n$, we have 
\[
\lim_{n\to\infty}z_n=x+iy
\]
if and only if
\[
\begin{array}{lll}
\lim_{n\to\infty}x_n=x,
&
\mbox{and}
&
\lim_{n\to\infty}y_n=y
\end{array}
\]
\end{proposition}

\begin{definition}[Convergent Series]
An infinite \emph{series} $\sum_{n=1}^\infty z_n$ of complex numbers converges to the sum $S$ if the partial sum sequences
\[
S_N=\sum_{n=1}^Nz_n
\]
converges to $S$, then we write
\[
\sum_{n=1}^\infty z_n=S.
\]
\end{definition}

\begin{proposition}
For $z_n=x_n+iy_n$, we have 
\[
\sum_{n=1}^\infty z_n=X+iY
\]
if and only if
\[
\begin{array}{lll}
\sum_{n=1}^\infty x_n=X
&
\mbox{and}
&
\sum_{n=1}^\infty y_n=Y
\end{array}
\]
\end{proposition}

\begin{proposition}
The series $\sum_{n=1}^\infty z_n$ converges implies that $\lim_{n\to\infty}z_n=0$.
\end{proposition}

\begin{definition}[Absolute Convergence]
The series $\sum_{n=1}^\infty z_n$ is said to be \emph{absolutely convergent} if
\[
\sum_{n=1}^\infty|z_n|
\]
converges, i.e., $\sum_{n=1}^\infty|x_n|$ and $\sum_{n=1}^\infty|y_n|$ converge.
\end{definition}
\begin{proposition}
Absolute convergence implies convergence
\end{proposition}

\begin{definition}[Remainder]
The \emph{remainder} $\rho_N$ of a series after $N$ terms is defined by:
\[
\rho_N=\sum_{n=N+1}^\infty S_n
\]
\end{definition}
\begin{proposition}
A series converges to a number $S$ iff the sequence of remainders tends to zero.
\end{proposition}

It's easy to verifty that
\[
\sum_{n=0}^\infty z_n=\frac{1}{1-z},\qquad\mbox{whenever }|z|<1
\]
with the aid of partial sums and remainders.

\subsection{Taylor Series}
\begin{definition}[Power Series]
The power series has the form
\[
\sum_{n=0}^\infty a_n(z-z_0)^n
\]
\end{definition}

\begin{theorem}[Convergent of Taylor Series]
Suppose $f$ is \emph{analytic} on $|z-z_0|<R$, then 
\begin{equation}
f(z)=\sum_{n=0}^\infty \frac{f^{(n)}(z_0)}{n!}(z-z_0)^n,
\end{equation}
for $|z-z_0|<R$, i.e., $f(z)$ admits its Taylor expansion at $z=z_0$ in this rigion.
\end{theorem}
\begin{remark}
Typically, when $z_0=0$, we say this series is the \emph{Maclaurin series}.
\end{remark}
\begin{proof}
\textbf{Step 1: Applying Cauchy Integral Formula.}
For fixed $z$, let $r:=|z-z_0|<R$ and take $r_0$ such that $r<r_0<R$. Construct a contour $C_0:\{z\in\mathbb{C}\mid |z-z_0|=r_0\}$ in the positive sense, which follows that
\begin{subequations}
\begin{align}
f(z)&=\frac{1}{2\pi i}\int_{C_0}\frac{f(s)}{s-z}\diff s\label{Eq:4:2:b}
\end{align}

\textbf{Step 2: Expand $1/(s-z)$.} With some calculation, we obtain
\begin{align}\label{Eq:4:2}
\frac{1}{s-z}&=\frac{1}{s-z_0}\cdot\frac{1}{1-\frac{z-z_0}{s-z_0}}\\
&=\frac{1}{s-z_0}\left\{1+\frac{z-z_0}{s-z_0}+\cdots+(\frac{z-z_0}{s-z_0})^{N-1}+\frac{(\frac{z-z_0}{s-z_0})^{N}}{1 - \frac{z-z_0}{s-z_0}}\right\}\label{Eq:4:2:d}\\
&=\frac{1}{s-z_0}+\frac{z-z_0}{(s-z_0)^2}+\cdots+\frac{(z-z_0)^{N-1}}{(s-z_0)^N}+\frac{(z-z_0)^N}{(s-z)(s-z_0)^N}\label{Eq:4:2:e}
\end{align}
where $(\ref{Eq:4:2:d})$ is because that
\[
\frac{1}{1-c}=1+c+c^2+\cdots+c^{N-1}+\frac{c^N}{1-c}. 
\]
Substituting (\ref{Eq:4:2:e}) into (\ref{Eq:4:2:b}), we obtain
\begin{align*}
f(z)&=\frac{1}{2\pi i}\int_{C_0}\left\{\frac{f(s)}{(s-z_0)}+\frac{f(s)(z-z_0)}{(s-z_0)^2}+\cdots+\frac{f(s)(z-z_0)^{N-1}}{(s-z_0)^N}+\frac{f(s)(z-z_0)^N}{(s-z)(s-z_0)^N}
\right\}\\
&=f(z_0)+f'(z_0)(z-z_0)+\cdots+\frac{f^{(N-1)}(z_0)}{(N-1)!}(z-z_0)^{N-1}+\rho_N(z)
\end{align*}
with
\begin{equation}\label{Eq:4:2:e}
\rho_N(z)=\frac{(z-z_0)^B}{2\pi i}\int_{C_0}\frac{f(s)}{(s-z)(s-z_0)^N}\diff s
\end{equation}

\textbf{Step 3: Show that $\rho_N(z)$ is convergent.}
\begin{align}
|\rho_N(z)|&\le\frac{|z-z_0|^N}{2\pi}\int_{C_0}\frac{|f(s)|}{|s-z||s-z_0|^N}|\diff s|\\
&\le\frac{r^N}{2\pi}\int_{C_0}\frac{M}{(r_0-r)r_0^N}|\diff s|\\
&=\frac{Mr_0}{r_0-r}\left(\frac{r}{r_0}\right)^N
\end{align}
where we suppose $|f(s)|\le M$ on $C_0$; and $|s-z|\ge|s-z_0| - |z-z_0| = r_0-r$. Therefore,
\[
\rho_N(z)\to0
\]
since $r<r_0$ and $(r/r_0)^N\to0$.
\end{subequations}




\end{proof}

\begin{example}
\begin{enumerate}
\item
For $f(z)=e^z$, which is analytic for $|z-0|<\infty$, thus we have
\[
e^z=\sum_{n=0}^\infty\frac{z^n}{n!},\qquad |z|<\infty
\]
\item
For $f(z)=\sin z=\frac{e^{iz} - e^{-i}}{2i}$, which is analytic for $|z-0|<\infty$, thus we have $f^{(2n)}(0)=0; f^{(2n+1)}(0)=(-1)^n$, and therefore
\[
\sin z=\sum_{n=0}^\infty\frac{(-1)^n}{(2n+1)!}z^{2n+1},\qquad |z|<\infty
\]
\item
For $f(z)=\frac{1}{1-z}$, which is analytic for $|z-0|<1$, we have $f^{(0)}=n!$, and therefore
\[
\frac{1}{1-z}=\sum_{n=0}^\infty\frac{n!}{n!}z^n=\sum_{n=0}^\infty z^n,\qquad |z|<1
\]
\item
For $f(z)=\frac{1}{z}\cdot\frac{1}{1+z}$, we have
\[
\frac{1}{1+z}=\sum_{n=0}^\infty(-z)^n,\qquad |z|<1,
\]
and therefore
\[
\frac{1}{z+z^2}=\frac{1}{z}+\sum_{n=1}^\infty(-1)^nz^{n-1},\qquad 0<|z|<1
\]
\end{enumerate}
\end{example}
\subsection{Laurent Series}
We cannot apply Taylor expansion at a non-analytic point. Fortunately, we can find another series representation for $f(z)$ that involving positive and negative powers of $(z-z_0)$
\begin{definition}[Laurent Series]
The \emph{Laurent series} has the form
\[
\sum_{n=0}^\infty a_n(z-z_0)^n+\sum_{n=1}^\infty\frac{b_n}{(z-z_0)^n}
\]
\end{definition}
\begin{theorem}
Suppose $f$ is analytic throughout an \emph{annular domain} $R_1<|z-z_0|<R_2$. Let $C$ be any \emph{positively oriented simple closed contour} around $z_0$ and lying in that domain. Then
\begin{subequations}
\begin{equation}
f(z)=\sum_{n=0}^\infty a_n(z-z_0)^n+\sum_{n=1}^\infty\frac{b_n}{(z-z_0)^n},\qquad
R_1<|z-z_0|<R_2
\end{equation}
with
\begin{align}
a_n&=\frac{1}{2\pi i}\int_C\frac{f(z)}{(z-z_0)^{n+1}}\diff z,\qquad n=0,1,2,\dots\\
b_n&=\frac{1}{2\pi i}\int_C\frac{f(z)}{(z-z_0)^{-n+1}}\diff z,\qquad n=1,2,\dots
\end{align}
\end{subequations}

\end{theorem}
\begin{remark}
The Laurent series is often written as the form
\[
f(z)=\sum_{n=-\infty}^\infty c_n(z-z_0)^n,\qquad R_1<|z-z_0|<R_2,
\]
where
\[
c_n=\frac{1}{2\pi i}\int_C\frac{f(z)}{(z-z_0)^{n+1}}\diff z,\qquad n=0,\pm1,\pm2,\dots
\]
When $f$ is analytic on $|z-z_0|<R_2$, we have $b_n=0,a_n=\frac{f^{(n)}(z_0)}{n!}$, i.e., the Laurent series reduces to the Taylor series.
\end{remark}
\begin{proof}
\begin{itemize}
\item
For fixed $z$ in the domain, let $r=|z-z_0|$, and construct two positively oriented contours $C_i:\{z\in\mathbb{C}\mid|z-z_0|=r_i\},i=1,2$ such that $R_1<r_1<r<r_2<R_2$. (The reaon why we don't use the boundary is that the function is not analytic on the boundary but only interior to)
\item
Construct a circle $\gamma:\{s\in\mathbb{C}\mid s = z + \delta e^{i\theta},0\le\theta\le2\pi\}$, where the $\delta$ is picked such that $\gamma$ is contained in the interior between $C_1,C_2$. By Cauchy Integral Formula,
\begin{equation}
\int_{C_2}\frac{f(s)\diff s}{s-z}=\int_{C_1}\frac{f(s)\diff s}{s-z}+\int_{\gamma}\frac{f(s)\diff s}{s-z}
\end{equation}
Or equivalently,
\[
f(z)=\frac{1}{2\pi i}\int_{C_2}\frac{f(s)\diff s}{s-z}+\frac{1}{2\pi i}\int_{C_1}\frac{f(s)\diff s}{z-s}
\]
\item
By applying the same trick as (\ref{Eq:4:2}), we have
\begin{subequations}
\begin{equation}
f(z)=\sum_{n=0}^{N-1}a_n(z-z_0)^n+\rho_N(z)+\sum_{n=1}^N\frac{b_n}{(z-z_0)^n}+\sigma_N(z)
\end{equation}
with
\begin{align}
a_n&=\frac{1}{2\pi i}\int_{C_2}\frac{f(s)\diff s}{(s-z_0)^{n+1}}\\
b_n&=\frac{1}{2\pi i}\int_{C_1}\frac{f(s)\diff s}{(s-z_0)^{-n+1}}\\
\rho_N(z)&=\frac{(z-z_0)^N}{2\pi i}\int_{C_2}\frac{f(s)\diff s}{(s-z)(s-z_0)^N}\\
\sigma_N(z)&=\frac{(z-z_0)^{-N}}{2\pi i}\int_{C_1}\frac{f(s)\diff s}{(z-s)(s-z_0)^{-N}}
\end{align}
\end{subequations}
\item
Then we bound the term $\rho_N(z)$ and $\sigma_N(z)$. Suppose $|f(s)|\le M$ on $C_1,C_2$, and note that $|s-z|\ge r_2-r$ for $s\in C_2$; $|z-s|\ge r-r_1$ for $s\in C_1$:
\begin{align*}
\rho_N(z)&\le\frac{Mr_2}{r_2-r}\left(\frac{r}{r_2}\right)^N\\
\sigma_N(z)&\le\frac{Mr_1}{r-r_1}\left(\frac{r_1}{r}\right)^N
\end{align*}
\item
Finally, note that
\begin{align*}
a_n&=\frac{1}{2\pi i}\int_{C}\frac{f(s)\diff s}{(s-z_0)^{n+1}}\\
b_n&=\frac{1}{2\pi i}\int_{C}\frac{f(s)\diff s}{(s-z_0)^{-n+1}}\\
\end{align*}





\end{itemize}
\end{proof}

\begin{example}
\begin{enumerate}
\item
\[
f(z)=\frac{1}{z-1} - \frac{1}{z-2}
\]
This function has two singular points $1,2$. We take $z_0=0$.
\begin{itemize}
\item
When $z\in D_1=\{z:|z|<1\}$, we obtain the Taylor expansion:
\[
f(z)=-\sum_{n=0}^\infty z^n+\sum_{n=0}^\infty\frac{z^n}{2^{n+1}}=\sum_{n=0}^\infty(2^{-n-1}-1)z^n
\]
\item
When $z\in D_2=\{z:1<|z|<2\}$, we obtain the Laurent series:
\[
f(z)=\sum_{n=0}^\infty\frac{z^n}{2^{n+1}}+\sum_{n=1}^\infty\frac{1}{z^n}
\]
\item
When $z\in D_3=\{z:|z|>2\}$, we obtain
\[
f(z)=\sum_{n=1}^\infty\frac{1-2^{n-1}}{z^n}
\]
\end{itemize}
\item
Expand $f(z)=\frac{-z}{(z-1)(z-3)}$ near $z_0=1$, and find the domain of expansion.

The expansion should be the laurent series with domain of expansion $0<|z-1|<2$.
\[
f(z)=\frac{1/2}{z-1}-\frac{3/2}{z-3}=\frac{1/2}{z-1}+\sum_{n=0}^\infty\frac{3}{2^{n+2}}(z-1)^n
\]





\end{enumerate}
\end{example}
\subsection{Power Series}
For power series
\begin{equation}\label{Eq:4:6}
\sum_{n=0}^\infty a_n(z-z_0)^n
\end{equation}
, first we study the range of its convergence.
\begin{theorem}\label{The:4:3}
If the power series (\ref{Eq:4:6}) converges at $z=z_1(\ne z_0)$, then it is \emph{absolutely convergent} at each point in the disk $|z-z_0|<|z_1-z_0|$.
\end{theorem}
\begin{proof}
For any point $z$ around the disk, we have, we have
\[
\left|\frac{z-z_0}{z_1-z_0}\right|:=q<1,
\]
which follows that
\[
|a_n(z-z_0)^n|=|a_n(z_1-z_0)^n|\left|\frac{z-z_0}{z_1-z_0}\right|^n\le Mq^n,
\]
where $|a_n(z_1-z_0)^n|\le M$ since $z_1$ makes the series convergent. By Comparison test, we conclude (\ref{Eq:4:6}) is absolutely convergent.
\end{proof}
\begin{definition}[Uniform convergence]
The series (\ref{Eq:4:6}) is said to be \emph{uniformly convergent} for $|z-z_0|<R$ if as $N\to\infty$,
\[
\sup_{|z-z_0|<R}|\rho_N(z)|\to0
\]
\end{definition}

\begin{theorem}
If the power series (\ref{Eq:4:6}) converges at $z=z_1(\ne z_0)$, then it must be uniformly convergent for any closed circle $|z-z_0|\le \rho$ ($\rho<|z_1-z_0|$).
\end{theorem}
\begin{proof}
Notice that for any $\rho<|z_1-z_0|$, for any $z$ in that closed circle, we have
\[
|a_n(z-z_0)^n|\le |a_n\rho^n|
\]
Due to the conclusion in Theorem(\ref{The:4:3}), we conclude $\sum_{n=1}^\infty |a_n|\rho^n$ is convergent, and therefore $(\ref{Eq:4:6})$ is uniformly convergent.

\end{proof}
Now we are curious about whether the power series is analytic. First we show under which condition does the power series is continuous.
\begin{theorem}
The series (\ref{Eq:4:6}) represents a continuous function at each point inside the circle of convergence.
\end{theorem}

\begin{theorem}
The sum $S(z)$ of power series is analytic at each point $z$ interior to the circle convergence of that series.
\end{theorem}





















