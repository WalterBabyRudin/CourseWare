
\chapter{Week2}
\section{Error}\index{week2_Tuesday_lecture}
\begin{definition}[Decimal floating-point number
]
\[
y=0.d_1d_2\cdots d_k\times 10^n,\quad 1\le d_1\le 9, 0\le d_2,\dots,d_k\le 9
\]
\end{definition}
\begin{definition}[Rounding]
Adds $5\times10^{n-(k+1)}$ to $y$ and then chps the result to obtain a number of form
\[
fl(y)=0.\delta_1\cdots\delta_k\times 10^n
\]
\end{definition}
Consider a decimal number $x$ with rounding to $\tilde x$ with $n$ digits.
\begin{itemize}
\item
If $(n+1)$th digit of $x$ is $5,\dots,9$, $\tilde x=\hat x+10^{-n}$, where $\hat x$ is a number with the same n digits as $x$ and all the other digits beyond the nth are 0.
\item
If $(n+1)$th digit of $x$ is $0,\dots,4$, then $x=\tilde x+\varepsilon$ with $\varepsilon<\frac{1}{2}\times10^{-n}.$
\end{itemize}
Thus $|x-\tilde x|\le\frac{1}{2}*10^{-n}$.
\begin{definition}[erros]
$p^*$ is the approximation to $p$, actual error is $p-p^*$; absolute error is $|p-p^*|$; relative error is $\frac{|p-p^*|}{|p|}$; $p^*$ is said to approximate $p$ to $k$ significant digits if $k$ is the largest non-negative integer for which the relative error is no more than $5\times 10^{-k}$.
\end{definition}

\subsection{Bisection}
\begin{theorem}
If $f$ is continuous in the interval $[a,b]$, and $f(a)f(b)<0$, then there exists at least one solution $x^*\in(a,b)$ such that $f(x^*)=0$.
\end{theorem}
\begin{example}[Bisection Algorithm]
Input: $a,b$, $\varepsilon,\delta$. Assume $f(a)<0<f(b)$
\begin{itemize}
\item
Set $a_0=a$, $b_0=b$; $p_0=\frac{a_0+b_0}{2}$
\item
If $f(p_0)>0$, set $a_{k+1}=a_k;b_{k+1}=p_k$;
\item
If $f(\frac{a_k+b_k}{2})<0$, set $a_{k+1}=p_k;b_{k+1}=b_k$;
\item
If $\frac{|p_k-p_{k-1}|}{|p_k|}<\varepsilon$, terminate and output $p_k$.
\end{itemize}
\end{example}
\begin{theorem}[Convergence Rate of bisection]
\[
|x^k-x^*|\le\frac{1}{2}(b_k-a_k)=2^{-(k+1)}(b-a)
\]
\end{theorem}
\begin{proof}
The zero point $x^*$ satisfies $x^*=\lim_{k\to\infty}a_k=\lim_{k\to\infty}b_k$ (use $f(a_k)f(b_k)<0$).

Let $x_k=\frac{a_k+b_k}{2}$, which implies $|x_k-x^*|\le\frac{1}{2}(b_k-a_k)$.
\end{proof}
\begin{theorem}[Existence of fixed point]
Let $g\in\mathcal{C}[a,b]$ and $g\in[a,b]$ for all $x\in[a,b]$, then $g$ has a fixed point.
\end{theorem}
\begin{proof}
Define $h=g(x)-x$ and consider $h(a)h(b)$
\end{proof}
\begin{theorem}[Uniqueness]
If $g'$ exists on $[a,b]$ and $|g'(x)|\le k<1$, then $g$ has a unique fixed point. The sequence $p_n=g(p_{m-1})$ will converge to unique fixed point $p$. The convergence rate is
\[
|p_n-p|\le\frac{k^n}{1-k}|p_1-p_0|
\]
\end{theorem}
\begin{proof}
\begin{itemize}
\item
For two fixed points $p,q$, we have $|p-q|=|g'(\xi)||p-q|<|p-q|$.
\item
Note that $|p_n-p|\le k|p_{n-1}-p|\le k^n|p_0-p|$
\item
Since $|p_{n+1}-p_n|\le k^n|p-1-p_0|$ and thus 
\[
|p_m-p_n|\le k^n(1+k+k^2+\cdots+k^{m-n-1})|p_1-p_0|,
\]
taking $m\to\infty$.
\end{itemize}
\end{proof}
\begin{remark}
If $\phi'(x)=\phi''(x)=\cdots=\phi^{(p-1)}(x)=0$, then
\[
x_{k+1}=\phi(x_k)=\phi(x^*)+\cdots+\frac{\phi^{(p)}(\xi_k)}{p!}(x_l-x^*)^p\implies
x_{k+1}-x^*=\frac{\phi^{(p)}(\xi_k)}{p!}(x_k-x^*)^p
\]

\end{remark}
\begin{definition}[Newton's method]
Consider $f(p)=0$ with $f(p)=f(p_0)+(p-p_0)f'(p_0)+\frac{f''(\xi_p)}{2!}(p-p_0)^2$, we derive
\[
p=p_0-f(p_0)/f'(p_0)
\]
stop criteria: $\frac{|p_N-p_{N-1}|}{|p_N|}<\varepsilon$
\end{definition}
\begin{theorem}[convergence rate of Newton's method]
There exists a $\delta >0$ s.t. the sequence $\{p_n\}$ converges to $p$ for any initial guess $p_0\in[p-\delta,p+\delta]$.
\end{theorem}
\begin{proof}
Define $x_{k+1}=\phi(x_k)$ and therefore $e_{k+1}=\phi'(\xi_k)e_k$. It suffices to show $|\phi'(\xi_k)|\le\frac{1}{2}$, it is true since 
\[
\phi'(x)=\frac{f(x)f''(x)}{[f'(x)]^2},\quad
\phi'(x^*)=0.
\]
\end{proof}
\begin{proof}
We apply Taylor's expansion
\[
f(x^*)=f(x_k)+f'(x_k)(x^*-x_k)+\frac{1}{2}f''(\xi_k)(x^*-x_k)^2
\]
and therefore
\[
x_{k+1}-x^*=\frac{(x_k-x^*)f'(x_k)-f(x_k)}{f'(x_k)}=\frac{f''(\xi_k)(x^*-x_k)^2}{2f'(x_k)}
\]
and therefore
\[
\lim_{k\to\infty}\frac{e_{k+1}}{e_k^2}=\frac{f''(x^*)}{2f'(x^*)}
\]

\end{proof}
\begin{definition}[Secant Method]
\[
p_n=p_{n-1}-\frac{f(p_{n-1})(p_{n-1}-p_{n-2})}{f(p_{n-1}) - f(p_{n-2})},
\]
\end{definition}
\begin{example}[Secant with False Position Method]
$q_0=f(p_0);q_1=f(p_1)$. 

Secant Method
\begin{itemize}
\item
Set $p=p_1-\frac{q_1(p_1-p_0)}{q_1-q_0}$
\item
Set $p_0=p_1$; $q_0=q_1$; $p_1=p$; $q_1=f(p)$
\end{itemize}
False Position Method
\begin{itemize}
\item
Set $p=p_1-\frac{q_1(p_1-p_0)}{q_1-q_0}$
\item
$q=f(p)$. If $q\cdot q_1<0$, then set $p_0=p_1$; $q_0=q_1$
\item
Set $p_1=p$; $q_1=q$.
\end{itemize}
\end{example}
\begin{theorem}[Weierstrass Approximation Theorem]
$f$ continuous on $[a,b]$. For $\forall\epsilon>0$, $\exists$ polynomial $P(x)$s.t. $|f(x)-P(x)|<\varepsilon$, $\forall x\in[a,b]$
\end{theorem}
\begin{theorem}[Mean Value Theorem]
Suppose $f\in\mathcal{C}^{n}[a,b]$ and $a=x_0<x_1<\cdots<x_n=b$. There exists some $\xi\in(a,b)$ such that 
\[
f[x_0,\dots,x_n]=\frac{f^{(n)}(\xi)}{n!}
\]
\end{theorem}
\begin{proof}
Let $P(x)$ be the newton form interpolation and $g=f(x)-P(x)$, which has $n+1$ distinct zeros. By roll's theorem, there exists $\xi$ such that $p^{(n)}(\xi)=n!f[x_0,\dots,x_n]=f^{(n)}(\xi)$,
\end{proof}

\begin{definition}[Lagrange interpolation]
Choose basis $\{L_{n,k}(x)\mid k=0,\dots,n\}$ such that
\[
\begin{array}{ll}
L_{n,k}(x_k)=1,
&
L_{n,k}(x_l)=0,\forall l\ne k
\end{array}
\]
Lagrange interpolation: $P(x)=\sum_{k=0}^nf(x_k)L_{n,k}(x)$ with
\[
L_{n,k}(x)=\frac{(x-x_0)(x-x_1)\cdots(x-x_n)}{(x_k-x_0)\cdots(x_k-x_n)}
\]
\end{definition}

\begin{theorem}[Error Bound for Lagrange polynomial]
For $f\in\mathcal{C}^{n+1}[a,b]$, we have
\begin{itemize}
\item
\[
f(x)=P(x)+\frac{f^{(n+1)}(\xi_x)}{(n+1)!}(x-x_0)(x-x_1)\cdots(x-x_n)
\]
with $P(x)=\sum_{k=0}^nf(x_k)L_{n,k}(x)$ is the interpolation of $f$ at $\{x_0=a,x_1,\dots,x_n=b\}.$
\item
Hermite's interpolation error:
\[
f(x)-H(x)=\frac{f^{(2n+2)}(\xi_x)}{(2n+2)!}(x-x_0)^2(x-x_1)^2\cdots(x-x_n)^2
\]
\end{itemize}
\end{theorem}
\begin{proof}
For fixed $x\notin\{x_0,\dots,x_n\}$, introduce
\[
\phi(t) = f(t) - P(t) - \lambda(t-x_0)(t-x_1)\cdots(t-x_n)
\]
Choose $\lambda$ s.t. $\phi(x)=0$, then $\phi(x)=0,\phi(x_0)=\cdots=\phi(x_n)=0$.

By Rolle's theorem, $\phi^{(n+1)}(t)$ has at least one zero, say $\xi_x\in(a,b)$. Therefore,
\[
0=\phi^{(n+1)}(\xi_x)=f^{(n+1)}(\xi_x)-\underbrace{p^{(n+1)}(\xi_x)}_{0}-\lambda(n+1)!\implies
\lambda=\frac{f^{(n+1)}(\xi_x)}{(n+1)!}
\]
Note that $\phi(x)=0$ for such $\lambda$, the proof is complete.
\end{proof}

\begin{definition}[Cubic spline interpolations]
The cubic spline interpolation on $\{x_0,x_1,\dots,x_n\}$ has the form
\[
S(x)=S_k(x)=a_k+b_k(x-x_k)+c_k(x-x_k)^2+d_k(x-x_k)^3,\qquad
x\in[x_k,x_{k+1}],\quad k=0,1,\dots,n-1
\]
\begin{enumerate}
\item
$S(x_i)=f_i$ for $i=0,1,\dots,n$; which gives 2 condition on each $[x_k,x_{k+1}]$, total $2n$
\item
$S'(x)$ and $S''(x)$ must be continuous on $[x_0,x_n]$, i.e., $S_{i+1}'(x_{i+1})=S'_i(x_{i+1})$ for $i=0,1,\dots,n-1$.
\item
Boundary conditions:
\begin{itemize}
\item
$S''(x_0)=S''(x_n)=0$  (natural cubic spline, free )
\item
$S'(x_0)=f(x_0)$ and $S'(x_n) = f(x_n)$ (clamped boundary)
\end{itemize}
\end{enumerate}
\end{definition}

\clearpage

\begin{definition}[Constructing natural spline interpolant]
First sovle system $\bm{Ax}=\bm b$ with $c_n=0$
\[
\bm A=\begin{bmatrix}
1&0&0&\cdots&\cdots&0\\
h_0&2(h_0+h_1)&h_1&\ddots&\ddots&\vdots\\
0&h_1&2(h_1+h_2)&h_2&\ddots&\vdots\\
\vdots&\ddots&\ddots&\ddots&\ddots&0\\
\vdots&\ddots&\ddots&h_{n-2}&2(h_{n-2}+h_{n-1})&h_{n-1}\\
0&\cdots&\cdots&0&0&1
\end{bmatrix},\qquad
\]
and
\[
\bm b=\begin{pmatrix}
0\\\frac{3}{h_1}(a_2-a_1)-\frac{3}{h_0}(a_1-a_0)\\
\vdots\\
\frac{3}{h_{n-1}}(a_{n}-a_{n-1})-\frac{3}{h_{n-2}}(a_{n-1}-a_{n-2})\\0
\end{pmatrix},\qquad
\bm x=\begin{bmatrix}
c_0\\c_1\\\vdots\\c_n
\end{bmatrix}
\]
Then solve for $b_j,d_j$, i.e., for $j=0,1,\dots,n-1$
\begin{align*}
b_j&=\frac{a_{j+1}-a_j}{h_j}-h_j\frac{(c_{j+1}+2c_j)}{3}\\
d_j&=\frac{c_{j+1}-c_j}{3h_j}
\end{align*}
\end{definition}

\paragraph{Equally spaced integral}
\[
\int_0^{nh}S(x)\diff x
=\sum_{j=0}^{n-1} a_jh_j+\frac{1}{2}b_jh_j^2+\frac{1}{3}c_jh_j^3+\frac{1}{4}d_jh_j^4
\]
\begin{definition}[Clamped Spline Interpolant: Linear system $\bm{Ax}=\bm b$]
\[
\bm A=\begin{bmatrix}
2h_0&h_0&0&\cdots&\cdots&0\\
h_0&2(h_0+h_1)&h_1&\ddots&\ddots&\vdots\\
0&h_1&2(h_1+h_2)&h_2&\ddots&\vdots\\
\vdots&\ddots&\ddots&\ddots&\ddots&0\\
\vdots&\ddots&\ddots&h_{n-2}&2(h_{n-2}+h_{n-1})&h_{n-1}\\
0&\cdots&\cdots&0&h_{n-1}&2h_{n-1}
\end{bmatrix},\qquad
\]
\[
\bm b=\begin{pmatrix}
\frac{3}{h_0}(a_1-a_0)-3f'(a)
\\\frac{3}{h_1}(a_2-a_1)-\frac{3}{h_0}(a_1-a_0)\\
\vdots\\
\frac{3}{h_{n-1}}(a_{n}-a_{n-1})-\frac{3}{h_{n-2}}(a_{n-1}-a_{n-2})\\
3f'(b)-\frac{3}{h_{n-1}}(a_n-a_{n-1})
\end{pmatrix}
\]
\end{definition}

\begin{definition}[Divided difference $\&$ Newtpn form]
\[
f[x_0,\dots,x_k]=\frac{f[x_1,\dots,x_k]-f[x_0,\dots,x_{k-1}]}{x_k-x_0}
\]
Therefore
\[
P(x)=f[x_0]+f[x_0,x_1](x-x_0)+\cdots+f[x_0,x_1,\dots,x_n](x-x_0)\cdots(x-x_{n-1})
\]
Hermite polynomial:
\[
H_{2n+1}(x)=f[z_0]+\sum_{k=1}^{2n+1}f[z_0,\dots,z_k](x-z_0)(x-z_1)\cdots(x-z_{k-1})
\]
\end{definition}

\begin{theorem}[Spline Error]
For $f\in\mathcal{C}^4[a,b]$ with $\max_{a\le x\le b}|f^{(4)}(x)|=M$, we have
\[
|f(x)-S_{\text{clamped}}(x)|\le\frac{5M}{384}\max_{0\le j\le n-1}(x_{j+1}-x_j)^4
\]
\end{theorem}





