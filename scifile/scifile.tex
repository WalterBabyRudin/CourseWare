% Use only LaTeX2e, calling the article.cls class and 12-point type.

\documentclass[12pt]{article}

% Users of the {thebibliography} environment or BibTeX should use the
% scicite.sty package, downloadable from *Science* at
% www.sciencemag.org/about/authors/prep/TeX_help/ .
% This package should properly format in-text
% reference calls and reference-list numbers.

\usepackage{scicite}
\usepackage{bm}
\usepackage{mathtools}
\usepackage{amsmath}
\usepackage{amssymb}
\usepackage{amsthm}
	\usepackage[utf8]{inputenc}
\usepackage[english]{babel}
 \DeclarePairedDelimiterX{\inp}[2]{\langle}{\rangle}{#1, #2}
 \newcommand{\trans}{^{\mathrm T}}
\newtheorem{theorem}{Theorem}[section]
\newtheorem{corollary}{Corollary}[theorem]
\newtheorem{lemma}[theorem]{Lemma}

% Use times if you have the font installed; otherwise, comment out the
% following line.

\usepackage{times}

% The preamble here sets up a lot of new/revised commands and
% environments.  It's annoying, but please do *not* try to strip these
% out into a separate .sty file (which could lead to the loss of some
% information when we convert the file to other formats).  Instead, keep
% them in the preamble of your main LaTeX source file.


% The following parameters seem to provide a reasonable page setup.

\topmargin 0.0cm
\oddsidemargin 0.2cm
\textwidth 16cm 
\textheight 21cm
\footskip 1cm


%The next command sets up an environment for the abstract to your paper.

\newenvironment{sciabstract}{%
\begin{quote} \bf}
{\end{quote}}


% If your reference list includes text notes as well as references,
% include the following line; otherwise, comment it out.

\renewcommand\refname{References and Notes}

% The following lines set up an environment for the last note in the
% reference list, which commonly includes acknowledgments of funding,
% help, etc.  It's intended for users of BibTeX or the {thebibliography}
% environment.  Users who are hand-coding their references at the end
% using a list environment such as {enumerate} can simply add another
% item at the end, and it will be numbered automatically.

\newcounter{lastnote}
\newenvironment{scilastnote}{%
\setcounter{lastnote}{\value{enumiv}}%
\addtocounter{lastnote}{+1}%
\begin{list}%
{\arabic{lastnote}.}
{\setlength{\leftmargin}{.22in}}
{\setlength{\labelsep}{.5em}}}
{\end{list}}


% Include your paper's title here

\title{An Unified Proof for the Theorems of Alternatives} 


% Place the author information here.  Please hand-code the contact
% information and notecalls; do *not* use \footnote commands.  Let the
% author contact information appear immediately below the author names
% as shown.  We would also prefer that you don't change the type-size
% settings shown here.

\author
{Jie Wang\\
\\
\normalsize{School of Science and Engineering}\\
\normalsize{The Chinese University of Hong Kong, Shenzhen}
}

% Include the date command, but leave its argument blank.

\date{}



%%%%%%%%%%%%%%%%% END OF PREAMBLE %%%%%%%%%%%%%%%%



\begin{document} 

% Double-space the manuscript.

\baselineskip22pt

% Make the title.

\maketitle 



% Place your abstract within the special {sciabstract} environment.

\begin{abstract}
Theorems of alternatives are very useful in applied mathematics in various ways, which is also a main focus of the mid-term exam for MAT3320.
The purpose of the present paper is to prove theorems of alternatives in a unified way by directly making use of the Farkas' Lemma.
\end{abstract}



% In setting up this template for *Science* papers, we've used both
% the \section* command and the \paragraph* command for topical
% divisions.  Which you use will of course depend on the type of paper
% you're writing.  Review Articles tend to have displayed headings, for
% which \section* is more appropriate; Research Articles, when they have
% formal topical divisions at all, tend to signal them with bold text
% that runs into the paragraph, for which \paragraph* is the right
% choice.  Either way, use the asterisk (*) modifier, as shown, to
% suppress numbering.

\section{Introduction}
Farkas' Lemma is one of the theorems of alternatives for determining the existence of solutions for linear systems.
The purpose of this reflective journal is to prove other theorems of alternatives by directly making use of the Farkas' Lemma.




%
%In this file, we present some tips and sample mark-up to assure your
%\LaTeX\ file of the smoothest possible journey from review manuscript
%to published {\it Science\/} paper.  We focus here particularly on
%issues related to style files, citation, and math, tables, and
%figures, as those tend to be the biggest sticking points.  Please use
%the source file for this document, \texttt{scifile.tex}, as a template
%for your manuscript, cutting and pasting your content into the file at
%the appropriate places.
%
%{\it Science\/}'s publication workflow relies on Microsoft Word.  To
%translate \LaTeX\ files into Word, we use an intermediate MS-DOS
%routine \cite{tth} that converts the \TeX\ source into HTML\@.  The
%routine is generally robust, but it works best if the source document
%is clean \LaTeX\ without a significant freight of local macros or
%\texttt{.sty} files.  Use of the source file \texttt{scifile.tex} as a
%template, and calling {\it only\/} the \texttt{.sty} and \texttt{.bst}
%files specifically mentioned here, will generate a manuscript that
%should be eminently reviewable, and yet will allow your paper to
%proceed quickly into our production flow upon acceptance \cite{use2e}.


\section{Notations}
We denote $\bm e$ as the column vector whose components are all ones. For given two vectors $\bm x,\bm y$, we define
\begin{align*}
\bm x&\ge\bm y,\quad\text{if $x_i\ge y_i,\forall i$};\\
\bm x&\gvertneqq\bm y,\quad\text{if $x_i\ge y_i,\forall i, \bm x\ne\bm y$};\\
\bm x&>\bm y,\quad\text{if $x_i> y_i,\forall i$}.
\end{align*}
Moreover, we use ($\bar{\text{I}}$) or ($\bar{\text{II}}$) to denote the \emph{negation} of the statement (I) or (II), respectively.
%
%
%\begin{quote}
%\begin{verbatim}
%However, this record of the solar nebula may have been
%partly erased by the complex history of the meteorite
%parent bodies, which includes collision-induced shock,
%thermal metamorphism, and aqueous alteration
%({\it 1, 2, 5--7\/}).
%\end{verbatim}
%\end{quote}
%
%
%\noindent Compiled, the last two lines of the code above, of course, would give notecalls in {\it Science\/} style:
%
%\begin{quote}
%\ldots thermal metamorphism, and aqueous alteration ({\it 1, 2, 5--7\/}).
%\end{quote}
%
%Under the same logic, the author could set up his or her reference list as a simple enumeration,
%
%\begin{quote}
%\begin{verbatim}
%{\bf References and Notes}
%
%\begin{enumerate}
%\item G. Gamow, {\it The Constitution of Atomic Nuclei
%and Radioactivity\/} (Oxford Univ. Press, New York, 1931).
%\item W. Heisenberg and W. Pauli, {\it Zeitschr.\ f.\ 
%Physik\/} {\bf 56}, 1 (1929).
%\end{enumerate}
%\end{verbatim}
%\end{quote}
%
%\noindent yielding
%
%\begin{quote}
%{\bf References and Notes}
%
%\begin{enumerate}
%\item G. Gamow, {\it The Constitution of Atomic Nuclei and
%Radioactivity\/} (Oxford Univ. Press, New York, 1931).
%\item W. Heisenberg and W. Pauli, {\it Zeitschr.\ f.\ Physik} {\bf 56},
%1 (1929).
%\end{enumerate}
%\end{quote}
%
%That's not a solution that's likely to appeal to everyone, however ---
%especially not to users of B{\small{IB}}\TeX\ \cite{inclme}.  If you
%are a B{\small{IB}}\TeX\ user, we suggest that you use the
%\texttt{Science.bst} bibliography style file and the
%\texttt{scicite.sty} package, both of which we are downloadable from our author help site
%(http://www.sciencemag.org/about/authors/prep/TeX\_help/).  You can also
%generate your reference lists by using the list environment
%\texttt{\{thebibliography\}} at the end of your source document; here
%again, you may find the \texttt{scicite.sty} file useful.
%
%Whether you use B{\small{IB}}\TeX\ or \texttt{\{thebibliography\}}, be
%very careful about how you set up your in-text reference calls and
%notecalls.  In particular, observe the following requirements:
%
%\begin{enumerate}
%\item Please follow the style for references outlined at our author
%  help site and embodied in recent issues of {\it Science}.  Each
%  citation number should refer to a single reference; please do not
%  concatenate several references under a single number.
%\item Please cite your references and notes in text {\it only\/} using
%  the standard \LaTeX\ \verb+\cite+ command, not another command
%  driven by outside macros.
%\item Please separate multiple citations within a single \verb+\cite+
%  command using commas only; there should be {\it no space\/}
%  between reference keynames.  That is, if you are citing two
%  papers whose bibliography keys are \texttt{keyname1} and
%  \texttt{keyname2}, the in-text cite should read
%  \verb+\cite{keyname1,keyname2}+, {\it not\/}
%  \verb+\cite{keyname1, keyname2}+.
%\end{enumerate}
%
%\noindent Failure to follow these guidelines could lead
%to the omission of the references in an accepted paper when the source
%file is translated to Word via HTML.
%
%\section*{Handling Math, Tables, and Figures}
%
%Following are a few things to keep in mind in coding equations,
%tables, and figures for submission to {\it Science}.
%
%\paragraph*{In-line math.}  The utility that we use for converting
%from \LaTeX\ to HTML handles in-line math relatively well.  It is best
%to avoid using built-up fractions in in-line equations, and going for
%the more boring ``slash'' presentation whenever possible --- that is,
%for \verb+$a/b$+ (which comes out as $a/b$) rather than
%\verb+$\frac{a}{b}$+ (which compiles as $\frac{a}{b}$).  Likewise,
%HTML isn't tooled to handle certain overaccented special characters
%in-line; for $\hat{\alpha}$ (coded \verb+$\hat{\alpha}$+), for
%example, the HTML translation code will return [\^{}$(\alpha)$].
%Don't drive yourself crazy --- but if it's possible to avoid such
%constructs, please do so.  Please do not code arrays or matrices as
%in-line math; display them instead.  And please keep your coding as
%\TeX-y as possible --- avoid using specialized math macro packages
%like \texttt{amstex.sty}.
%
%\paragraph*{Displayed math.} Our HTML converter sets up \TeX\
%displayed equations using nested HTML tables.  That works well for an
%HTML presentation, but Word chokes when it comes across a nested
%table in an HTML file.  We surmount that problem by simply cutting the
%displayed equations out of the HTML before it's imported into Word,
%and then replacing them in the Word document using either images or
%equations generated by a Word equation editor.  Strictly speaking,
%this procedure doesn't bear on how you should prepare your manuscript
%--- although, for reasons best consigned to a note \cite{nattex}, we'd
%prefer that you use native \TeX\ commands within displayed-math
%environments, rather than \LaTeX\ sub-environments.
%
%\paragraph*{Tables.}  The HTML converter that we use seems to handle
%reasonably well simple tables generated using the \LaTeX\
%\texttt{\{tabular\}} environment.  For very complicated tables, you
%may want to consider generating them in a word processing program and
%including them as a separate file.
%
%\paragraph*{Figures.}  Figure callouts within the text should not be
%in the form of \LaTeX\ references, but should simply be typed in ---
%that is, \verb+(Fig. 1)+ rather than \verb+\ref{fig1}+.  For the
%figures themselves, treatment can differ depending on whether the
%manuscript is an initial submission or a final revision for acceptance
%and publication.  For an initial submission and review copy, you can
%use the \LaTeX\ \verb+{figure}+ environment and the
%\verb+\includegraphics+ command to include your PostScript figures at
%the end of the compiled PostScript file.  For the final revision,
%however, the \verb+{figure}+ environment should {\it not\/} be used;
%instead, the figure captions themselves should be typed in as regular
%text at the end of the source file (an example is included here), and
%the figures should be uploaded separately according to the Art
%Department's instructions.


\section{Theorems of Alternatives}

\begin{theorem}[Farkas' Lemma]
Either
\begin{enumerate}
\item[(I)]
$\bm A\bm x=\bm b,\bm x\ge\bm0$ has a solution $\bm x$,

or
\item[(II)]
$\bm A\trans\bm y\ge\bm0,\bm b\trans\bm y<0$ has a solution $\bm y$,
\end{enumerate}
but never both.
\end{theorem}

We will prove the following theorems, by applying the Farkas' Lemma directly.

\begin{theorem}[Gordan's Theorem]
Either
\begin{enumerate}
\item[(I)]
$\bm A\bm x>\bm0$ has a solution $\bm x$,

or
\item[(II)]
$\bm A\trans\bm y=\bm0,\bm y\gvertneqq\bm0$ has a solution $\bm y$,
\end{enumerate}
but never both.
\end{theorem}
\begin{proof}
\begin{itemize}
\item
(I) implies ($\bar{\text{II}}$):
If (I) holds for $\bm x$, and suppose on the contrary that (II) holds for $\bm y$. Then we imply
\[
\bm0=\bm x\trans(\bm A\trans\bm y)=(\bm A\bm x)\trans\bm y.
\]
Since $\bm{Ax}\ge\bm0,\bm y\ge\bm0$, the equality above holds if and only if $\bm y=\bm0$, which is a contradiction.
\item
($\bar{\text{I}}$) implies (II):
If (I) does not hold, then the linear system below does not have a solution as well:
\[
\left\{
\begin{aligned}
\bm A\bm x-\theta\bm e&\ge\bm0\\
\theta&>0
\end{aligned}
\right.
\Longleftrightarrow
\left\{
\begin{aligned}
\begin{bmatrix}
\bm A&-\bm e
\end{bmatrix}\begin{pmatrix}
\bm x\\\theta
\end{pmatrix}&\ge\bm0\\
\begin{bmatrix}
\bm 0\trans&-1
\end{bmatrix}\begin{pmatrix}
\bm x\\\theta
\end{pmatrix}&<0
\end{aligned}
\right.
\]
By applying the reverse direction of Farkas' Lemma, we imply the linear system below has a solution:
\[
\left\{
\begin{aligned}
\bm A\trans\bm x&=\bm0\\
\bm e\trans\bm x&=1\\
\bm x&\ge\bm0
\end{aligned}
\right.
\]
Therefore, the statement (II) holds.
\end{itemize}
\end{proof}

\begin{theorem}[Stiemke's Theorem]
Either
\begin{enumerate}
\item[(I)]
$\bm A\bm x\gvertneqq\bm0$ has a solution $\bm x$,

or
\item[(II)]
$\bm A\trans\bm y=\bm0,\bm y>\bm0$ has a solution $\bm y$,
\end{enumerate}
but never both.
\end{theorem}
\begin{proof}
\begin{itemize}
\item
(II) implies ($\bar{\text{I}}$):
If (II) holds for $\bm y$, and suppose on the contrary that (I) holds for $\bm x$. Then we imply
\[
\bm0=\bm x\trans(\bm A\trans\bm y)=(\bm A\bm x)\trans\bm y.
\]
Since $\bm{Ax}\ge\bm0,\bm y>\bm0$, the equality above holds if and only if $\bm{Ax}=\bm0$, which is a contradiction.
\item
($\bar{\text{II}}$) implies (I):
If (II) does not hold, then the linear system below does not have a solution as well:
\[
\left\{
\begin{aligned}
\bm A\trans\bm y&\ge\bm0\\
-\bm A\trans\bm y&\ge\bm0\\
\bm y-\theta\bm e&\ge\bm0\\
\theta&>0
\end{aligned}
\right.
\Longleftrightarrow
\left\{
\begin{aligned}
\begin{bmatrix}
\bm A\trans&\bm0\\-\bm A\trans&\bm0\\\bm I&-\bm e
\end{bmatrix}\begin{pmatrix}
\bm y\\\theta
\end{pmatrix}&\ge\bm0\\
\begin{bmatrix}
\bm 0\trans&-1
\end{bmatrix}\begin{pmatrix}
\bm y\\\theta
\end{pmatrix}&<0
\end{aligned}
\right.
\]
By applying the reverse direction of Farkas' Lemma, we imply the linear system below has a solution:
\[
\left\{
\begin{aligned}
\begin{bmatrix}
\bm A&-\bm A&\bm I\\\bm0\trans&\bm0\trans&-\bm e\trans
\end{bmatrix}\begin{pmatrix}
\bm x_1\\\bm x_2\\\bm x_3
\end{pmatrix}&=\begin{pmatrix}
\bm 0\\-1
\end{pmatrix}\\
\bm x_1,\bm x_2,\bm x_3\ge\bm0
\end{aligned}
\right.
\implies
\bm A(\bm x_2-\bm x_1)=\bm x_3\gvertneqq\bm0,
\]
i.e., the statement (I) holds for $\bm x_2-\bm x_1$.

\end{itemize}
\end{proof}


\begin{theorem}[Gale's Theorem]
Assuming $\bm A\bm x\le\bm b$ is feasible.
Either
\begin{enumerate}
\item[(I)]
$\bm A\bm x\lvertneqq\bm b$ has a solution $\bm x$,

or
\item[(II)]
$\bm A\trans\bm y=\bm0,\bm b\trans\bm y=\bm0,\bm y>\bm0$ has a solution $\bm y$,
\end{enumerate}
but never both.
\end{theorem}
\begin{proof}
\begin{itemize}
\item
(II) implies ($\bar{\text{I}}$):
If (II) holds for $\bm y$, and suppose on the contrary that (I) holds for $\bm x$. Then we imply
\[
\bm0=\bm x\trans(\bm A\trans\bm y)=(\bm A\bm x)\trans\bm y< \bm b\trans\bm y=\bm0,
\]
where the inequality is strict since $\bm y>\bm0$ and $\bm A\bm x\lvertneqq\bm b$. Therefore, we derive a contradiction.
\item
($\bar{\text{II}}$) implies (I):
If (II) does not hold, then the linear system below does not have a solution as well:
\[
\left\{
\begin{aligned}
\bm A\trans\bm y&\ge\bm0\\
-\bm A\trans\bm y&\ge\bm0\\
\bm b\trans\bm y&\ge\bm0\\
-\bm b\trans\bm y&\ge\bm0\\
\bm y-\theta\bm e&\ge0\\
\theta&>0
\end{aligned}
\right.
\Longleftrightarrow
\left\{
\begin{aligned}
\begin{bmatrix}
\bm A\trans&\bm0\\
-\bm A\trans&\bm 0\\
\bm b\trans&\bm0\\
-\bm b\trans&\bm 0\\
\bm I&-\bm e
\end{bmatrix}\begin{pmatrix}
\bm y\\\theta
\end{pmatrix}&\ge\bm0\\
\begin{bmatrix}
\bm 0\trans&-1
\end{bmatrix}\begin{pmatrix}
\bm y\\\theta
\end{pmatrix}&<0
\end{aligned}
\right.
\]
By applying the reverse direction of Farkas' Lemma, we imply the linear system below has a solution:
\[
\left\{
\begin{aligned}
\begin{bmatrix}
\bm A&-\bm A&\bm b&-\bm b&\bm I
\\\bm0\trans&\bm0\trans&\bm0\trans&\bm0\trans&-\bm e\trans
\end{bmatrix}\begin{pmatrix}
\bm x_1\\\bm x_2\\\bm x_3\\\bm x_4\\\bm x_5
\end{pmatrix}&=\begin{pmatrix}
\bm 0\\-1
\end{pmatrix}\\
\bm x_1,\bm x_2,\bm x_3,\bm x_4,\bm x_5\ge\bm0
\end{aligned}
\right.
\implies
\bm A(\bm x_1-\bm x_2)+\bm b(x_3-x_4)=-\bm x_5\lvertneqq\bm0,
\]
Suppose we have a feasible solution $\bm x^0$ such that $\bm A\bm x^0\le\bm b$, which implies for any $N>0$,
\[
\bm A(\bm x_1-\bm x_2+N\bm x^0)+\bm b(x_3-x_4-N)\lvertneqq\bm0
\]
Therefore, we take sufficient large $N$ to make $x_3-x_4-N<0$, which follows that
\[
\bm A\left(\frac{1}{N-x_3-x_4}(\bm x_1-\bm x_2+N\bm x^0)\right)-\bm b\lvertneqq\bm0,
\]
i.e., the statement (I) holds for $(\bm x_1-\bm x_2+N\bm x^0)/(N-x_3-x_4)$.
\end{itemize}
\end{proof}

\begin{theorem}[Tucker's Theorem]
Assuming that $\bm A\ne\bm0$.
Either
\begin{enumerate}
\item[(I)]
$\bm A\bm x\gvertneqq\bm 0,\bm B\bm x\ge0,\bm C\bm x=\bm0$ has a solution $\bm x$,

or
\item[(II)]
$\bm A\trans\bm u+\bm B\trans\bm v+\bm C\trans\bm w=\bm0,\bm u>\bm0,\bm v\ge\bm0$ has a solution $(\bm u,\bm v,\bm w)$,
\end{enumerate}
but never both.
\end{theorem}
\begin{proof}
\begin{itemize}
\item
(II) implies ($\bar{\text{I}}$):
If (II) holds for $(\bm u,\bm v,\bm w)$, and suppose on the contrary that (I) holds for $\bm x$. Then we imply
\[
\bm0=\bm x\trans(\bm A\trans\bm u+\bm B\trans\bm v+\bm C\trans\bm w)
=
(\bm A\bm x)\trans\bm u+(\bm B\bm x)\trans\bm v+(\bm C\bm x)\trans\bm w>\bm0
\]
where the inequality is strict 
since $\bm A\bm x\gvertneqq\bm0$ and $\bm u>\bm0$. Therefore, we derive a contradiction.
\item
($\bar{\text{II}}$) implies (I):
If (II) does not hold, then the linear system below does not have a solution as well:
\[
\left\{
\begin{aligned}
\bm A\trans\bm u+\bm B\trans\bm v+\bm C\trans\bm w&\ge0\\
-\bm A\trans\bm u-\bm B\trans\bm v-\bm C\trans\bm w&\ge0\\
\bm v&\ge0\\
\bm u-\theta\bm e&\ge0\\
\theta&>0
\end{aligned}
\right.
\Longleftrightarrow
\left\{
\begin{aligned}
\begin{bmatrix}
\bm A\trans&\bm B\trans&\bm C\trans&0\\
-\bm A\trans&-\bm B\trans&-\bm C\trans&0\\
\bm0&\bm I&\bm0&0\\
\bm I&\bm0&\bm0&-\bm e
\end{bmatrix}\begin{pmatrix}
\bm u\\\bm v\\\bm w\\\theta
\end{pmatrix}&\ge\bm0\\
\begin{bmatrix}
\bm0\trans&\bm0\trans&\bm0\trans&-1
\end{bmatrix}\begin{pmatrix}
\bm u\\\bm v\\\bm w\\\theta
\end{pmatrix}&<0
\end{aligned}
\right.
\]
By applying the reverse direction of Farkas' Lemma, we imply the linear system below has a solution:
\[
\left\{
\begin{aligned}
\begin{bmatrix}
\bm A&-\bm A&\bm0&\bm I\\
\bm B&-\bm B&\bm I&\bm 0\\
\bm C&-\bm C&\bm 0&\bm 0\\
0&0&0&-\bm e\trans
\end{bmatrix}\begin{pmatrix}
\bm x_1\\\bm x_2\\\bm x_3\\\bm x_4
\end{pmatrix}&=\begin{pmatrix}
\bm0\\\bm0\\\bm0\\-1
\end{pmatrix}\\
\bm x_1,\bm x_2,\bm x_3,\bm x_4\ge\bm0
\end{aligned}
\right.
\implies
\left\{
\begin{aligned}
\bm A(\bm x_2-\bm x_1)&=\bm x_4\gvertneqq\bm0\\
\bm B(\bm x_2-\bm x_1)&=\bm x_2\ge\bm0\\
\bm C(\bm x_2-\bm x_1)&=\bm0
\end{aligned}
\right.,
\]
i.e., the statement (I) holds for $\bm x_2-\bm x_1$.
\end{itemize}
\end{proof}


\begin{theorem}[Motzkin's Theorem]
Assuming  that $\bm A\ne\bm0$.
Either
\begin{enumerate}
\item[(I)]
$\bm A\bm x>0,\bm B\bm x\ge0,\bm C\bm x=\bm0$ has a solution $\bm x$,

or
\item[(II)]
$\bm A\trans\bm u+\bm B\trans\bm v+\bm C\trans\bm w=\bm0,\bm u\gvertneqq\bm0,\bm v\ge\bm0$ has a solution $(\bm u,\bm v,\bm w)$,
\end{enumerate}
but never both.
\end{theorem}

\begin{proof}
\begin{itemize}
\item
(I) implies ($\bar{\text{II}}$):
If (I) holds for $\bm x$, and suppose on the contrary that (II) holds for $(\bm u,\bm v,\bm w)$. Then we imply
\[
\bm0=\bm x\trans(\bm A\trans\bm u+\bm B\trans\bm v+\bm C\trans\bm w)
=
(\bm A\bm x)\trans\bm u+(\bm B\bm x)\trans\bm v+(\bm C\bm x)\trans\bm w>\bm0
\]
where the inequality is strict 
since $\bm A\bm x>\bm0$ and $\bm u\gvertneqq0$. Therefore, we derive a contradiction.
\item
($\bar{\text{I}}$) implies (II):
If (I) does not hold, then the linear system below does not have a solution as well:
\[
\left\{
\begin{aligned}
\bm A\bm x-\theta\bm e&\ge0\\
\bm B\bm x&\ge\bm0\\
\bm C\bm x&\ge\bm0\\
-\bm C\bm x&\ge\bm0\\
\theta&>0
\end{aligned}
\right.
\Longleftrightarrow
\left\{
\begin{aligned}
\begin{bmatrix}
\bm A&-\bm e\\
\bm B&\bm0\\
\bm C&\bm0\\
-\bm C&\bm0\\
\end{bmatrix}\begin{pmatrix}
\bm x\\\theta
\end{pmatrix}&\ge\bm0\\
\begin{bmatrix}
\bm0\trans&-1
\end{bmatrix}\begin{pmatrix}
\bm x\\\theta
\end{pmatrix}&<0
\end{aligned}
\right.
\]
By applying the reverse direction of Farkas' Lemma, we imply the linear system below has a solution:
\[
\left\{
\begin{aligned}
\begin{bmatrix}
\bm A\trans&\bm B\trans&\bm C\trans&-\bm C\trans\\
-\bm e\trans&\bm0\trans&\bm0\trans&\bm0\trans
\end{bmatrix}\begin{pmatrix}
\bm x_1\\\bm x_2\\\bm x_3\\\bm x_4
\end{pmatrix}&=\begin{pmatrix}
\bm0\\-1
\end{pmatrix}\\
\bm x_1,\bm x_2,\bm x_3,\bm x_4\ge\bm0
\end{aligned}
\right.
\implies
\left\{
\begin{aligned}
\bm A\trans\bm x_1+\bm B\trans\bm x_2+\bm C\trans(\bm x_3-\bm x_4)&=0\\
\bm x_1\gvertneqq\bm0, 
\bm x_2&\ge\bm0
\end{aligned}
\right.,
\]
i.e., the statement (II) holds for $(\bm x_1,\bm x_2,\bm x_3-\bm x_4)$.
\end{itemize}
\end{proof}

\section{Conclusion}
Note that the equivalence of these theorems of alternatives can be shown by strong duality theorem \cite{R_1}.
The purpose of this paper is not on the equivalence result, but presenting an alternative approach for showing these theorems of alternatives.


\medskip
 
\begin{thebibliography}{9}
\bibitem{R_1} 
Perng, Cherng-tiao. (2017). On a class of theorems equivalent to Farkas's lemma. \textit{Applied Mathematical Sciences}. 11. 2175-2184. 



\end{thebibliography}






\end{document}




















