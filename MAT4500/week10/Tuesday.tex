
\chapter{Week10}

\section{Tuesday}\index{week7_Tuesday_lecture}

Recall the property 3) in proposition~\ref{pro:9:1}, in which we show that when $\{X_u\}_u$ is a simple process, $\left\{\int_0^tX_u\diff B_u\right\}_t$ is a square integrable, almost surely continuous martingale.
In this lecture, we first review some basic ideas about square integrability.
\subsection{Square Integrable Process}
\begin{definition}[Square Integrable Martingales]
\begin{enumerate}
\item
A stochastic process $\{X_t\}_{t\ge0}$ is said to be square integrable if 
$\mathbb{E}[X_t^2]<\infty,\forall t\ge0$.
\item
Let $\mathcal{U}^2$ be the class of square integrable, right-continuous, with left limit exist martingales.
\item
Let $\mathcal{U}_c^2$ be the class of square integrable, almost surely continuous martingales.
\end{enumerate}
\end{definition}
In particular, we denote $\mathcal{L}^2$ as the set of square integrable random variables.
\begin{definition}[Norm on $\mathcal{U}^2$]
For given $T>0$, define a norm $\|\cdot\|$ on $\mathcal{U}^2$:
\[
\|X\|\triangleq \left(
\mathbb{E}[X_T^2]
\right)^{1/2},\quad \{X_t\}_{t\ge0}\in\mathcal{U}^2.
\]
\end{definition}

\begin{theorem}[Completeness of Square Integrable Martingales]
With respect to the norm $\|\cdot\|$, 
\begin{enumerate}
\item
$\mathcal{U}^2$ is a complete metric space;
\item
$\mathcal{U}_c^2$ is a closed subspace of $\mathcal{U}^2$.
\end{enumerate}
\end{theorem}
\begin{proof}
\begin{enumerate}
\item
It is easy to see that $(\mathcal{U}^2,\|\cdot\|)$ is a metric space.
To show the completeness, it suffices to show that for any Cauchy sequence $\{X_t^{(n)}\}_{t\ge0}\in \mathcal{U}^2$ where $n=1,2,\ldots$, there exists a process $\{X_t\}_{t\ge0}\in\mathcal{U}^2$ such that 
\[
\|X^{(n)}-X\|\to0,\quad \text{as }n\to\infty.
\]
We first give a construction on such $\{X_t\}_{t\ge0}$:
\begin{itemize}
\item
Since $\{X^{(n)}\}_n$ is a Cauchy sequence, for any $\varepsilon>0$, there exists $N$ such that as long as $m,n>N$,
\[
\|X^{(m)}-X^{(n)}\|<\varepsilon.
\]
Since $\{X^{(m)}_t-X^{(n)}_t\}_{t\ge0}$ is a martingale, by convexity of quadratic function,
$\{(X^{(m)}_t-X^{(n)}_t)^2\}_{t\ge0}$ is a sub-martingale, which implies
\[
\forall t\le T,\quad
\mathbb{E}[(X^{(m)}_t-X^{(n)}_t)^2]\le \mathbb{E}[(X^{(m)}_T-X^{(n)}_T)^2]=\|X^{(m)}-X^{(n)}\|^2<\varepsilon^2.
\]
This means that for fixed $t\le T$, $\{X_t^{(n)}\}_{n}$ is a Cauchy sequence in $\mathcal{L}^2$, with the metric defined as $\|X\|_{\mathcal{L}^2}\triangleq\left(\mathbb{E}[X^2]\right)^{1/2}$.
By the completeness of $\mathcal{L}^2$, there exists a random variable $X_t\in\mathcal{L}^2$ such that $X_t^{(n)}\xrightarrow{L^2}X_t$ as $n\to\infty$.
\end{itemize}
Next, we show that $\{X_t\}_{t\ge0}\in\mathcal{U}^2$:
\begin{enumerate}
\item
In order to show $\{X_t\}_{t\ge0}$ is $\{\mathcal{F}_t\}$-adapted, we need the almost sure convergence of some process into $\{X_t\}$, as the definition of almost sure convergence is based on sample-path. The details are given as follows.
The $L^2$ convergence implies $X_t^{(n)}\xrightarrow{P}X_t$, and thus there exist s a subsequence $\{X_t^{(n_k)}\}_k$ such that \[X_t^{(n_k)}\xrightarrow{a.s.}X_t \quad\text{as $k\to\infty$}.\]
Then define the sample-path-based set
\[
\Lambda=\bigg\{
\omega\in\Omega:~\lim_{k\to\infty}X_t^{(n_k)}(\omega)=X_t(\omega)
\bigg\}.
\]
Re-define $X_t(\omega)=0$ for $\omega\in\Lambda^c$.
Note that $\Lambda^c\in\mathcal{F}_t$ since we assume $\{\mathcal{F}_t\}$ satisfies the usual condition~(see Definition~\ref{def:5:4}).
It follows that for $a<0$,
\[
\bigg\{
\omega\in\Omega:~X_t(\omega)\le a
\bigg\}=\bigcap_j\bigcup_m\bigcup_{k>m}\bigg\{
\omega\in\Omega:~X_t^{(n_k)}(\omega)<a+\frac{1}{j}
\bigg\}\in\mathcal{F}_t.
\]
For $a\ge0$, since $\Lambda^c\in\mathcal{F}_t$,
\[
\bigg\{
\omega\in\Omega:~X_t(\omega)\le a
\bigg\}=\Lambda^c\bigcup\left[\bigcap_j\bigcup_m\bigcup_{k>m}\bigg\{
\omega\in\Omega:~X_t^{(n_k)}(\omega)<a+\frac{1}{j}
\bigg\}\right]\in\mathcal{F}_t.
\]
\item
The integrability of $X_t$ is because $X_t\in\mathcal{L}^2$.
Then we show $\{X_t\}_{t\ge0}$ satisfies the martingale property, i.e, for fixed $0\le s<t$, we need to show $\int_AX_t\diff\mathbb{P}=\int_AX_s\diff\mathbb{P},~\forall A\in\mathcal{F}_s$.
Direct calculation, together with the martingability of $\{X_t^{(n)}\}_{t\ge0}$ gives
\[
\int_AX_t\diff\mathbb{P}-\int_AX_s\diff\mathbb{P}=\int_A(X_t-X_t^{(n)})\diff\mathbb{P}-\int_A(X_s-X_s^{(n)})\diff\mathbb{P},\quad\forall n.
\]
By the $L^2$ convergence of $X_t^{(n)}$,
\[
\int_A|X_t-X_t^{(n)}|\diff\mathbb{P}\le \mathbb{E}[|X_t-X_t^{(n)}|]\le \left(\mathbb{E}[(X_t-X_t^{(n)})^2]\right)^{1/2}\to0,\quad n\to\infty.
\]
Similarly, $\int_A|X_t-X_t^{(n)}|\diff\mathbb{P}\to0$. The martingability of $\{X_t\}_{t\ge0}$ follows since
\begin{align*}
\left|
\int_AX_t\diff\mathbb{P}-\int_AX_s\diff\mathbb{P}
\right|&=\lim_{n\to\infty}\left|
\int_A(X_t-X_t^{(n)})\diff\mathbb{P}-\int_A(X_s-X_s^{(n)})\diff\mathbb{P}
\right|\\
&\le \lim_{n\to\infty}\int_A|X_t-X_t^{(n)}|\diff\mathbb{P}+\int_A|X_t-X_t^{(n)}|\diff\mathbb{P}=0.
\end{align*}
\item
To make $\{X_t\}_{t\ge0}$ right-continuous with left limit exists, apply part 1) in Theorem~\ref{The:con:mart}.
\end{enumerate}
\item
Consider a sequence $\{X_t^{(n)}\}_t\in\mathcal{U}_c^2$ for $n=1,2,\ldots$.
By result in part 1), there exists $\{X_t\}_t\in\mathcal{U}^2$ as a limt.
It suffices to show that $\{X_t\}_t\in\mathcal{U}_c^2$.
In order to show the continuity result, we first construct the (almost sure) uniform convergence of some sequence in $\mathcal{U}_c^2$ with the limit $\{X_t\}_t$.
\begin{itemize}
\item
Note that $\{X_t^{(n)}-X_t\}_{t}$ is a martingale for all $n$, then by Doob's inequality,
\[
\mathbb{P}\left(
\sup_{t\le T}|X_t^{(n)}-X_t|>\varepsilon
\right)\le\frac{1}{\varepsilon^2}\mathbb{E}\left[
(X_T^{(n)}-X_T)^2
\right]\to0,\quad\text{as }n\to\infty.
\]
Since $\sup_{t\le T}|X_t^{(n)}-X_t|\xrightarrow{P}0$, there exists a subsequence $\{\sup_{t\le T}|X_t^{(n_k)}-X_t|\}_k$ that converges to $0$ almost surely.
Define 
\begin{align*}
W_1&=\bigcap_k\bigg\{
\omega\in\Omega:~X_t^{(n_k)}(\omega)\text{ is continuous}
\bigg\}\\
% why intersection is still probability one?
W_2&=\bigg\{
\omega\in\Omega:~\lim_{k\to\infty}\sup_{t\le T}|X_t^{(n_k)}-X_t|=0
\bigg\}\\
W&=W_1\cap W_2\implies \mathbb{P}(W)=1.
\end{align*}
Take $\omega\in W$, then for any $\varepsilon>0$, there exists $N>0$ such that as long as $k>N$ and $t\le T$,
\[
|X_t^{(n_k)}(\omega)-X_t(\omega)|\le \sup_{t\le T}|X_t^{(n_k)}(\omega)-X_t(\omega)|<\frac{\varepsilon}{3}.
\]
In other words, $X_t^{(n_k)}(\omega)$ uniformly converges to $X_t(\omega)$ for $\omega\in W$.
\end{itemize}
The continuity of $X_t^{(n_k)}(\omega)$ (and therefore uniform continuity) implies there exists $\delta$ such that as long as $h<\delta$,
\[
|X_{t+h}^{(n_k)}(\omega)-X_{t}^{(n_k)}(\omega)|<\frac{\varepsilon}{3},\quad\forall t\le T.
\]
Hence we conclude the continuity of $X_t(\omega)$ for $\omega\in W$:
\begin{align*}
|X_{t+h}(\omega)-X_t(\omega)|&\le |X_{t+h}(\omega)-X_{t+h}^{(n_k)}(\omega)|
+|X_{t+h}^{(n_k)}(\omega)-X_{t}^{(n_k)}(\omega)|\\&+|X_{t}^{(n_k)}(\omega)-X_t(\omega)|<\varepsilon.
\end{align*}
The proof is completed.


\end{enumerate}
\end{proof}
\section{Thursday}


\subsection{Introduction to Ito Integral}
By the conclusion in Section~\ref{sec:9:2:1}, for any process $\{X_t\}_{t\ge0}\in\mathcal{L}^2$,
when fixing $T>0$, there exists a sequence of simple processes $\{X_t^{(n)}\}_{t\ge0}, n=1,2,\ldots$ such that
\begin{equation}\label{Eq:10:2}
\lim_{n\to\infty}\mathbb{E}\left[
\int_0^T(X_t^{(n)}-X_t)^2\diff t
\right]=0.
\end{equation}
The Ito integral for a simple process $\{\tilde{X}_t\}_{t\ge0}$ is well-defined as in (\ref{Eq:9:3}):
\[
\int_0^T\tilde{X}_t(\omega)\diff B_t(\omega) = \sum_{k=0}^{n-1}\tilde{X}_{t_k}(\omega)\cdot[B_{t_{k+1}}(\omega) -B_{t_k}(\omega)] + \tilde{X}_{t_n}(\omega)\cdot[B_{T}(\omega) -B_{t_n}(\omega)]
\]
Denote $I_t(\tilde{X})\triangleq \{\int_0^t\tilde{X}_s(\omega)\diff B_s(\omega)\}_{\omega\in\Omega}$.

\begin{theorem}\label{The:10:2}
Given a process  $\{X_t\}_{t\ge0}\in\mathcal{L}^2$ and fixing $T>0$,
there exists an \emph{unique} process $\{Z_t\}_{t\ge0}\in\mathcal{U}_c^2$ such that 
\begin{equation}\label{Eq:10:2:2}
\lim_{n\to\infty}\mathbb{E}\left[
(I_t(X^{(n)})-Z_t)^2
\right]=0,\quad\forall t\in[0,T].
\end{equation}
\end{theorem}

\begin{remark}
The process $\{Z_t\}_{t\ge0}$ is unique in the following sense:
if there is another sequence of simple process, $\{\tilde{X}_t^{(n)}\}_{t\ge0}$, approximating $\{X_t\}_{t\ge0}$, and there exists $\{\tilde{Z}_t\}_{t\ge0}\in\mathcal{U}_c^2$ such that 
\[
\lim_{n\to\infty}\mathbb{E}\left[
(I_t(\tilde{X}^{(n)})-\tilde{Z}_t)^2
\right]=0,\quad\forall t\in[0,T],
\]
then
\[
\mathbb{P}\bigg(
\mbox{$Z_t=\tilde{Z}_t$ for all $t\in[0,T]$}
\bigg)=1.
\]
\end{remark}
\begin{proof}
The existence of $\{Z_t\}_{t\ge0}$ can be argued if we show $\{I_t(X^{(n)})\}_{t\ge0}, n=1,2,\ldots$ is a Cauchy sequence:
\begin{itemize}
\item
We first show that $I_T(X^{(n)}), n=1,2,\ldots$ is Cauchy (with respect to $L^2$ measure):
\begin{subequations}
\begin{align}
\mathbb{E}\left[
(I_T(X^{(j)}) - I_T(X^{(k)}))^2
\right]&=\mathbb{E}\left[
\left(\int_0^TX^{(j)}_t\diff B_t - \int_0^TX^{(k)}_t\diff B_t\right)^2
\right]\\
&=\mathbb{E}\left[
\int_0^T(X^{(j)}_t-X^{(k)}_t)^2\diff t
\right]\label{Eq:10:3:b}\\
&=\mathbb{E}\left[
\int_0^T(X^{(j)}_t-X_t+X_t-X^{(k)}_t)^2\diff t
\right]\label{Eq:10:3:c}\\
&\le 2\mathbb{E}\left[
\int_0^T(X^{(j)}_t-X_t)^2\diff t
\right]+2\mathbb{E}\left[
\int_0^T(X_t-X^{(k)}_t)^2\diff t
\right]\label{Eq:10:3:d}
\end{align}
where (\eqref{Eq:10:3:b}) is by the linearity and isometry property in Proposition~\ref{pro:9:1};
(\eqref{Eq:10:3:d}) is by the inequality $(a+b)^2\le 2a^2+2b^2$.
\end{subequations}

Recall (\eqref{Eq:10:2}), i.e., for any $\varepsilon>0$, there exists $N$ such that as long as $j,k>N$,
\[
\mathbb{E}\left[
\int_0^T(X^{(j)}_t-X_t)^2\diff t
\right]<\frac{\varepsilon}{4},\quad
\mathbb{E}\left[
\int_0^T(X^{(k)}_t-X_t)^2\diff t
\right]<\frac{\varepsilon}{4}.
\]
Thus $\mathbb{E}\left[
(I_T(X^{(j)}) - I_T(X^{(k)}))^2
\right]<\varepsilon$ for large $j,k$.
\item
Then we show that $I_t(X^{(n)}), n=1,2,\ldots$ is Cauchy for $t<T$.
Since $\{I_t(X^{(j)})\}_{t\ge0}$ and $\{I_t(X^{(k)})\}_{t\ge0}$ are martingales, together with the convexity of quadratic function, we can assert that $\{(I_t(X^{(j)}) - I_t(X^{(k)}))^2\}_{t\ge0}$ is a sub-martingale, which means for large $j,k$,
\[
\mathbb{E}\left[
(I_t(X^{(j)}) - I_t(X^{(k)}))^2
\right]\le \mathbb{E}\left[
(I_T(X^{(j)}) - I_T(X^{(k)}))^2
\right]<\varepsilon,\forall t<T.
\]
\end{itemize}
By part 3) in Proposition~\ref{pro:9:1}, the process $\{I_t(X^{(n)})\}_{t\ge0}\in\mathcal{U}_c^2$ for each $n$.
By the Cauchy property for $\{I_t(X^{(n)})\}_{t\ge0}, n=1,2,\ldots$ and the closedness of $\mathcal{U}_c^2$ with respect to the norm $\|X\|=(\mathbb{E}[X_{\color{red}T}^2])^{1/2}$, there exists a limit $\{Z_t\}_{t\ge0}\in\mathcal{U}_c^2$ such that
\[
\lim_{n\to\infty}\mathbb{E}\left[
(I_T(X^{(n)})-Z_T)^2
\right]=0.
\]
We can further show (\eqref{Eq:10:2:2}) holds by the sub-martingability of $\{(I_t(X^{(n)})-Z_t)^2\}_{t\ge0}$.

Now we begin to show the uniqueness of $\{Z_t\}_{t\ge0}$.
Suppose there is another simple process $\{\tilde{X}_t^{(n)}\}_{t\ge0}$ approximating $\{X_t\}_{t\ge0}$, and we have $\{\tilde{Z}_t\}_{t\ge0}$ such that $\lim_{n\to\infty}\mathbb{E}\left[
(I_t(\tilde{X}_t^{(n)})-\tilde{Z}_t)^2
\right]=0,\forall t\in[0,T]$.
Observe that $Z_T-\tilde{Z}_T\in\mathcal{L}^2$. By doob's inequality,
\[
\forall\varepsilon,\quad
\mathbb{P}\left(
\sup_{t\le T}|Z_t - \tilde{Z}_t|>\varepsilon
\right)\le \frac{1}{\varepsilon^2}\mathbb{E}[(Z_T - \tilde{Z}_T)^2].
\]
This suggests that in order to show that $\{\tilde{Z}_t\}_{t\ge0}$ is a version of $\{Z_t\}_{t\ge0}$,
we can start with bounding their $L^2$ distance:
\begin{align*}
\mathbb{E}[(Z_T - \tilde{Z}_T)^2]&=\mathbb{E}[(Z_T - I_T({X}_t^{(n)}) + I_T({X}_t^{(n)})-I_T(\tilde{X}_t^{(n)})+I_T(\tilde{X}_t^{(n)})-\tilde{Z}_T)^2]\\
&\le 3\mathbb{E}\left[(Z_T - I_T({X}_t^{(n)}))^2\right]+3\mathbb{E}\left[(I_T({X}_t^{(n)})-I_T(\tilde{X}_t^{(n)}))^2\right]\\&+3\mathbb{E}\left[(I_T(\tilde{X}_t^{(n)})-\tilde{Z}_T)^2\right]
\end{align*}
where the inequality is by applying $(\sum_{i=1}^na_i)^2\le n\sum_{i=1}^na_i^2$.
The first and last term vanishes as $n\to\infty$.
Moreover, the second term can be upper bounded as
\begin{align*}
\mathbb{E}\left[(I_T({X}_t^{(n)})-I_T(\tilde{X}_t^{(n)}))^2\right]
&=\mathbb{E}\left[(\int_0^T({X}_t^{(n)}-\tilde{X}_t^{(n)})\diff B_t)^2\right]\\
&=\mathbb{E}\left[\int_0^T({X}_t^{(n)}-\tilde{X}_t^{(n)})^2\diff t\right]\\
&\le 2\mathbb{E}\left[\int_0^T({X}_t^{(n)}-X_t)^2\diff t\right]
+2\mathbb{E}\left[\int_0^T(\tilde{X}_t^{(n)}-X_t)^2\diff t\right]
\end{align*}
and thus $\mathbb{E}\left[(I_T({X}_t^{(n)})-I_T(\tilde{X}_t^{(n)}))^2\right]\to0$ as $n\to\infty$.
Put things together, we can assert that
\[
\mathbb{E}[(Z_T - \tilde{Z}_T)^2]=0\implies
\mathbb{P}\left(
\sup_{t\le T}|Z_t - \tilde{Z}_t|>\varepsilon
\right)=0,\forall\varepsilon>0.
\]
Define $\Lambda=\{\omega:~Z_t(\omega)-\tilde{Z}_t(\omega)\ne0\text{ for some }t\in[0,T]\}$, then
\[
\Lambda\subseteq\bigcup_{n=1}^\infty\left(
\sup_{t\le T}|Z_t - \tilde{Z}_t|>\frac{1}{n}
\right),
\]
which implies $\mathbb{P}(\Lambda)=0$, i.e., 
\[
\mathbb{P}\bigg(
\mbox{$Z_t=\tilde{Z}_t$ for all $t\in[0,T]$}
\bigg)=\mathbb{P}(\Lambda^c)=1.
\]
\end{proof}









\begin{definition}[Ito Integral for Square Integrable Process]
For any process $\{X_t\}_{t\ge0}\in\mathcal{L}^2$, define the Ito integral
\begin{equation}
I_t(X)=Z_t,\quad t\in[0,T],
\end{equation}
where $\{Z_t\}_{t\ge0}$ is defined in Theorem~\ref{The:10:2}.
\end{definition}

\begin{proposition}
We find the Ito integral
\[
\int_0^TB_t\diff B_t=\frac{1}{2}\left[
B_T^2-T
\right].
\]
\end{proposition}
\begin{proof}
\begin{enumerate}
Before the computation, we need to show $\{B_t\}_{t\ge0}$ satisfies the assumption for Ito integral, i.e., $\{B_t\}_{t\ge0}\in\mathcal{L}^2$, which is trivial.
\item
Firstly, we need to figure out the simple process $\{B_t^{(n)}\}_{t\ge0}$ that approximating the argument in the integral, say $\{B_t\}_{t\ge0}$.
Define $\Pi^{(n)}=\{0,t_1^{(n)},t_2^{(n)},\ldots,T\}$ the partition on $[0,T]$, and  construct
\[
B_t^{(n)}=B_{t_j^{(n)}},\quad\forall t\in[t_j^{(n)},t_{j+1}^{(n)}).
\]
As a consequence,
\begin{align*}
\mathbb{E}\left[
\int_0^T(B_t^{(n)}-B_t)^2\diff t
\right]&=
\mathbb{E}\left[\sum_j
\int_{t_j^{(n)}}^{t_{j+1}^{(n)}}(B_t^{(n)}-B_t)^2\diff t
\right]=\sum_i\int_{t_j^{(n)}}^{t_{j+1}^{(n)}}\mathbb{E}\left[
(B_t^{(n)}-B_t)^2
\right]\diff t\\
&=\sum_j\int_{t_j^{(n)}}^{t_{j+1}^{(n)}}\mathbb{E}\left[
(B_{t_j^{(n)}}-B_t)^2
\right]\diff t=\sum_j\int_{t_j^{(n)}}^{t_{j+1}^{(n)}}(t-t_j^{(n)})\diff t\\
&=\frac{1}{2}\sum_j(t_{j+1}^{(n)}-t_j^{(n)})^2\le \frac{1}{2}\|\Pi^{(n)}\|\sum_j(t_{j+1}^{(n)}-t_j^{(n)})
=\frac{1}{2}\|\Pi^{(n)}\| T,
\end{align*}
which indicates that as long as we construct $\{B_t^{(n)}\}_{t\ge0}$ such that $\|\Pi^{(n)}\|\to0$, 
$\mathbb{E}\left[
\int_0^T(B_t^{(n)}-B_t)^2\diff t
\right]\to0$ as $n\to\infty$.
\item
The Ito integral of the simple process $\{B_t^{(n)}\}_{t\ge0}$ is 
\[
I_T(B^{(n)})=\sum_jB_{t_j^{(n)}}(B_{t_{j+1}^{(n)}}-B_{t_j^{(n)}})
\]
We will show this term converges to $\frac{1}{2}[B_T^2-T]$ in $L^2$.
Observe that
\begin{align*}
B_{t_j^{(n)}}(B_{t_{j+1}^{(n)}}-B_{t_j^{(n)}})=
\frac{1}{2}\left[
B_{t_{j+1}^{(n)}}^2-B_{t_j^{(n)}}^2-\left(
B_{t_{j+1}^{(n)}}-B_{t_j^{(n)}}
\right)^2
\right]
\end{align*}
which implies
\begin{align*}
I_T(B^{(n)})&=\frac{1}{2}\left[
\sum_j(B_{t_{j+1}^{(n)}}^2-B_{t_j^{(n)}}^2)-\sum_j\left(
B_{t_{j+1}^{(n)}}-B_{t_j^{(n)}}
\right)^2
\right]\\
&=\frac{1}{2}\left[
(B_T^2-B_0^2)-Q(\Pi^{(n)})
\right]=\frac{1}{2}\left[
B_T^2-Q(\Pi^{(n)})
\right]
\end{align*}
where $Q(\Pi^{(n)})$ is the quadratic variation of Brownian motion $\{B_t\}_{t\ge0}$ over the partition $\Pi^{(n)}$.
Recall that $Q(\Pi^{(n)})\xrightarrow{L^2}T$, which implies $I_T(B^{(n)})\xrightarrow{L^2}\frac{1}{2}\left[
B_T^2-T
\right]$.
\end{enumerate}
So we conclude that
\[
I_T(B)\triangleq \int_0^TB_t\diff B_t=\frac{1}{2}\left[
B_T^2-T
\right].
\]
\end{proof}

\subsection{Properties of Ito Integral}

\begin{proposition}
For any $\{X_t\}_{t\ge0}\in\mathcal{L}^2$, the Ito integral $\int_0^TX_t\diff B_t$ has the following properties:
\begin{enumerate}
\item
Linearity: let $\{X_t\}_{t\ge0},\{Y_t\}_{t\ge0}\in\mathcal{L}^2$, for any $\alpha,\beta$,
\[
\int_0^T(\alpha X_t + \beta Y_t)\diff B_t = \alpha \int_0^TX_t\diff B_t+\beta \int_0^TY_t\diff B_t.
\]
\item
Ito isometry: for any $0\le s<t<T$,
\[
\mathbb{E}\left[
\left(
\int_s^tX_u\diff B_u
\right)^2\middle|\mathcal{F}_s
\right]=\mathbb{E}\left[
\int_s^tX_u^2\diff u
\middle|\mathcal{F}_s
\right].
\]
\item
$\{I_t(X)\}_{t\ge0}\in\mathcal{U}_c^2$.
\end{enumerate}
\end{proposition}

\begin{proof}
The linearity property is trivial to show, and the third property comes from Theorem~\ref{The:10:2}.
It suffices to show the second propperty.
Let $\{X_t^{(n)}\}_{t\ge0}, n=1,2,\ldots$ be the sequence of simple processes approximating $\{X_t\}_{t\ge0}$, and $A\in\mathcal{F}_S$.
It follows that
\begin{align*}
\mathbb{E}\left[
\left(
\int_s^tX_u\diff B_u
\right)^21_A
\right]&=\mathbb{E}\left[
\left(
I_t(X)-I_s(X)
\right)^21_A
\right]\\
&=\mathbb{E}\left[
\left(
I_t(X)-I_t(X^{(n)})+I_t(X^{(n)})-I_s(X^{(n)})+I_s(X^{(n)})-I_s(X)
\right)^21_A
\right]\\
&=\underbrace{\mathbb{E}\left[
\left(
I_t(X)-I_t(X^{(n)})
\right)^21_A
\right]}_{(a)}+
\underbrace{\mathbb{E}\left[
\left(
I_t(X^{(n)})-I_s(X^{(n)})
\right)^21_A
\right]}_{(b)}
\\&+
\underbrace{\mathbb{E}\left[
\left(
I_s(X^{(n)})-I_s(X)
\right)^21_A
\right]}_{(c)}\\
&+\underbrace{2\mathbb{E}\left[
\left(
I_t(X)-I_t(X^{(n)})
\right)\left(
I_t(X^{(n)})-I_s(X^{(n)})
\right)1_A
\right]}_{(d)}\\
&+\underbrace{2\mathbb{E}\left[
\left(
I_t(X)-I_t(X^{(n)})
\right)\left(
I_s(X^{(n)})-I_s(X)
\right)1_A
\right]}_{(e)}\\
&+\underbrace{2\mathbb{E}\left[
\left(
I_t(X^{(n)})-I_s(X^{(n)})
\right)\left(
I_s(X^{(n)})-I_s(X)
\right)1_A
\right]}_{(f)}
\end{align*}
It is easy to show that $(a),(c)$ vanishes as $n\to\infty$:
\[
\mathbb{E}\left[
\left(
I_t(X)-I_t(X^{(n)})
\right)^21_A
\right]\le 
\mathbb{E}\left[
\left(
I_t(X)-I_t(X^{(n)})
\right)^2
\right]\to0.
\]
It is also easy to show that $(d),(e),(f)$ vanishes as $n\to\infty$. For instance,
\begin{align*}
&\mathbb{E}\left[
\left(
I_t(X)-I_t(X^{(n)})
\right)\left(
I_t(X^{(n)})-I_s(X^{(n)})
\right)1_A
\right]
\le \mathbb{E}\left[
\left|
I_t(X)-I_t(X^{(n)})
\right|\left|
I_t(X^{(n)})-I_s(X^{(n)})
\right|
\right]\\
\le& \left(
 \mathbb{E}\left[
(
I_t(X)-I_t(X^{(n)})
)^2
\right]
\right)^{1/2}
\left(
 \mathbb{E}\left[
(
I_t(X^{(n)})-I_s(X^{(n)})
)^2
\right]
\right)^{1/2}\to0,
\end{align*}
as $ \mathbb{E}\left[
(
I_t(X)-I_t(X^{(n)})
)^2
\right]\to0$ as $n\to\infty$.

It remains to show that the term (b) in fact converges to $\mathbb{E}\left[
\left(\int_s^tX_u^2\diff u\right)
1_A
\right]$ as $n\to\infty$:
\begin{align*}
\mathbb{E}\left[
\left(
I_t(X^{(n)})-I_s(X^{(n)})
\right)^21_A
\right]=\mathbb{E}\left[
\left(
\int_s^tX_u^{(n)}\diff B_u
\right)^21_A
\right]=\mathbb{E}\left[
\left(
\int_s^t(X_u^{(n)})^2\diff u
\right)1_A
\right],
\end{align*}
and
\begin{align*}
&\left|
\mathbb{E}\left[
\left(
\int_s^t(X_u^{(n)})^2\diff u
\right)1_A
\right]
-
\mathbb{E}\left[
\left(
\int_s^t(X_u)^2\diff u
\right)1_A
\right]
\right|\\
=&\left|
\mathbb{E}\left[
\left(
\int_s^t(X_u^{(n)}+X_u)(X_u^{(n)}-X_u)\diff u
\right)1_A
\right]
\right|\\
\le&\left|
\mathbb{E}\left[
\left(
\int_s^t|X_u^{(n)}+X_u||X_u^{(n)}-X_u|\diff u
\right)
\right]
\right|\\
\le&\mathbb{E}\left[
\int_s^t(X_u^{(n)}+X_u)^2\diff u
\right]^{1/2}\mathbb{E}\left[
\int_s^t(X_u^{(n)}-X_u)^2\diff u
\right]^{1/2}
\end{align*}
Note that the first term is bounded:
\begin{align*}
\mathbb{E}\left[
\int_s^t(X_u^{(n)}+X_u)^2\diff u
\right]&=
\mathbb{E}\left[
\int_s^t(X_u^{(n)}-X_u+2X_u)^2\diff u
\right]\\
&\le 2\mathbb{E}\left[
\int_s^t(X_u^{(n)}-X_u)^2\diff u
\right]+2\mathbb{E}\left[
\int_s^t(2X_u)^2\diff u
\right]<\infty.
\end{align*}
Together with the fact that the second term vanishes as $n\to\infty$, we assert that
\[
\left|
\mathbb{E}\left[
\left(
\int_s^t(X_u^{(n)})^2\diff u
\right)1_A
\right]
-
\mathbb{E}\left[
\left(
\int_s^t(X_u)^2\diff u
\right)1_A
\right]
\right|\to0.
\]
Or equivalently,
\begin{align*}
\lim_{n\to\infty}\mathbb{E}\left[
\left(
I_t(X^{(n)})-I_s(X^{(n)})
\right)^21_A
\right]=\lim_{n\to\infty}\mathbb{E}\left[
\left(
\int_s^t(X_u^{(n)})^2\diff u
\right)1_A
\right]=\mathbb{E}\left[
\left(
\int_s^t(X_u)^2\diff u
\right)1_A
\right].
\end{align*}
The proof is completed.

\end{proof}






















