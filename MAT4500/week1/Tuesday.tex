
\chapter{Week1}

\section{Tuesday}\index{Tuesday_lecture}
\subsection{Difference between ODE and SDE}
We first discuss the difference between deterministic differential equations and stochastic ones by considering several real-life problems.
\paragraph{Problem 1: Population Growth Model}
Consider the first-order ODE
\[
\left\{
\begin{aligned}
\frac{\diff N(t)}{\diff t} &= a(t)N(t)\\
N(0)&=N_0
\end{aligned}
\right.
\]
where $N(t)$ denotes the \emph{size} of the population at time $t$; $a(t)$ is the given (deterministic) function describing the \emph{rate} of growth of population at time $t$;
and $N_0$ is a given constant.

If $a(t)$ is not completely known, e.g.,
\[
a(t) = r(t)\cdot\mbox{noise},
\mbox{ or }
r(t)+\mbox{noise},
\]
with $r(t)$ being a deterministic function of $t$, and the ``noise'' term models something random.
The question arises: How to \textit{rigorously} describe the ``noise'' term and solve it?

\paragraph{Problem 2: Electric Circuit}
Let $Q(t)$ denote the charge at time $t$ in an electrical circuit, which admits the following ODE:
\[
\left\{
\begin{aligned}
LQ''(t)+RQ'(t)+\frac{1}{C}Q(t)&=F(t),\\
Q(0)=Q_0,\quad Q'(0)&=Q_0'
\end{aligned}
\right.
\]
where $L$ denotes the inductance, $R$ denotes the resistance, $C$ denotes the {capacity}, and 
$F(t)$ denotes the potential source.

Now consider the scenario where $F(t)$ is not completely known, e.g.,
\[
F(t)=G(t)+\mbox{noise}
\]
where $G(t)$ is deterministic. 
The question is how to solve the problem.

\begin{remark}
The differential equations above involving non-deterministic coefficients are called the \emph{stochastic differential equations}~(SDEs).
Clearly, the solution to an SDE should involve the randomness.
\end{remark}


\subsection{Applications of SDE}
Now we discuss some applications of SDE shown in the finance area.
\paragraph{Problem 3: Optimal Stopping Problem} Suppose someone holds an asset 
(e.g., stock, house). 
He plans to sell it at some future time. 
Denote $X(t)$ as the price of the asset at time $t$, satisfying the following dynamics:
\[
\frac{\diff X(t)}{\diff t}=rX(t)+\alpha X(t)\cdot\mbox{noise}
\]
where $r,\alpha$ are given constants. 
The goal of this person is to maximize the expected selling price:
\[
\begin{array}{ll}
\sup\limits_{\tau\ge0}
&
\mathbb{E}[X(\tau)]
\end{array}
\]
where the optimal solution $\tau^*$ is the optimal stopping time.

\paragraph{Problem 4: Portfolio Selection Problem}
Suppose a person is interested in two types of assets:
\begin{itemize}
\item
A risk-free asset which generates a deterministic return $\rho$, whose price $X_1(t)$ follows a deterministic dynamics
\[
\frac{\diff X_1(t)}{\diff t}=\rho X_1(t),
\]
\item
A risky asset whose price $X_2(t)$ satisfies the following SDE:
\[
\frac{\diff X_2(t)}{\diff t}=\mu X_2(t)+\sigma X_2(t)\cdot\mbox{noise}
\]
where $\mu,\sigma>0$ are given constants.
\end{itemize}
The policy of the investment is as follows.
The wealth at time $t$ is denoted as $v(t)$. 
This person decides to invest the fraction $u(t)$ of his wealth into the risky asset, with the remaining $1-u(t)$ part to be invested into the safe asset. 
Suppose that the utility function for this person is $U(\cdot)$, and his goal is to maximize the expected total wealth at the terminal time $T$:
\[
\max_{u(t),0\le t\le T}\mathbb{E}[U(v^u(T))]
\]
where the decision variable is the portfolio function $u(t)$ along whole horizon $[0,T]$.


%If we impose no-shot selling constant, we further require
%\[
%0\le u(t)\le 1,\forall t\in[0,T]
%\]

\paragraph{Problem 5: Option Pricing Problem}
The financial derivates are products in the market whose value depends on the underlying asset.
The European call option is a typical financial derivative.
Suppose that the underlying asset is stock $A$, whose price at time $t$ is $X(t)$.
Then the call option gives the option holder the right (not the obligation) to buy one unit of stock A at a specified price (strike price) $K$ at maturity date $T$.
The task is to inference the fair price of the option at the current time.
The formula for the price of the option is the following:
\[
c_0 = \mathbb{E}[
(X(T) - K)^+
]
\]
which is the famous Black-Sholes-Merton Formula.

%Suppose at time $0$, a person in the long position in an European call option has the right to buy the asset at a specified price $K$ at some future time $t$. How much the person should pay to the short position for the option? We can model this problem by Black-Sholes Formula.

\subsection{Reviewing for Probability Space}
Firstly, we review some basic concepts in real analysis.
\begin{definition}[$\sigma$-Algebra]
A set $\mathcal{F}$ containing subsets of $\Omega$ is called a $\sigma$-algebra if:
\begin{enumerate}
\item
$\Omega\in\mathcal{F}$;
\item
$\mathcal{F}$ is closed under complement, i.e., 
$A\in\mathcal{F}$ implies $\Omega\setminus A\in\mathcal{F}$;
\item
$\mathcal{F}$ is closed under countably union operation, i.e.,
$A_i\in\mathcal{F},i\ge1$ implies $\cup_{i=1}^\infty A_i\in\mathcal{F}$.
\end{enumerate}
\end{definition}

\begin{definition}[Probability Measure]
A function $\mathbb{P}:\mathcal{F}\to\mathbb{R}$ is called a \emph{probability measure} on $(\Omega,\mathcal{F})$ if
\begin{itemize}
\item
$\mathbb{P}(\Omega)=1$;
\item
$\mathbb{P}(A)\ge0,\forall A\in\mathcal{F}$;
\item
$\mathbb{P}$ is $\sigma$-additive, i.e., when $A_i\in\mathcal{F},i\ge1$ and $A_i\cap A_j=\emptyset,\forall i\ne j$, 
\[
\mathbb{P}\bigg(
\bigcup_{i=1}^\infty A_i
\bigg)
=
\sum_{i=1}^\infty\mathbb{P}(A_i).
\]
\end{itemize}
where $\mathbb{P}(A)$ is called the \emph{probability of the event} $A$.
\end{definition}


\begin{definition}[Probability Space]
A probability space is a triplet $(\Omega,\mathcal{F},\mathbb{P})$ defined as follows:
\begin{enumerate}
\item
$\Omega$ denotes the \emph{sample space}, 
and a point $\omega\in\Omega$ is called a sample point;
\item
$\mathcal{F}$ is a $\sigma$-algebra of $\Omega$, 
which is a collection of subsets in $\Omega$.
The element $A\in\mathcal{F}$ is called an ``event''; and
\item
$\mathbb{P}$ is a probability measure defined in the space $(\Omega,\mathcal{F})$.
\end{enumerate}
\end{definition}

\begin{definition}[Almost Surely True]
A statement $S$ is said to be \emph{almost surely~(a.s.) true} or \emph{true with probability 1}, if
\begin{itemize}
\item
$\mathfrak{B}:=\{w: S(w)\mbox{ is true}\}\in\mathcal{F}$
\item
$\mathbb{P}(F)=1$.
\end{itemize}
\end{definition}
\begin{definition}[Topological Space]
A \emph{topological space} $(X,\mathcal{T})$ consists of a (non-empty) set $X$, and a family of subsets of $X$ (``open sets'' $\mathcal{T}$) such that
\begin{enumerate}
\item
$\emptyset,X\in\mathcal{T}$
\item
$U,V\in\mathcal{T}$ implies $U\bigcap V\in \mathcal{T}$
\item
If $U_\alpha\in\mathcal{T}$ for all $\alpha\in\mathcal{A}$, then $\bigcup_{\alpha\in\mathcal{A}}U_\alpha\in\mathcal{T}$.
\end{enumerate}
When $A\in\mathcal{T}$, $A$ is called the open subset of $X$. The $\mathcal{T}$ is called a \emph{topology} on $X$.
\end{definition}





\begin{definition}[Borel $\sigma$-Algebra]
Consider a topological space $\Omega$, with $\mathcal{U}$ being the topology of $\Omega$.
The \emph{Borel $\sigma$-Algebra} $\mathcal{B}(\Omega)$ on $\Omega$ is defined to be 
the \textit{minimal} $\sigma$-algebra containing $\mathcal{U}$:
\[
\mathcal{B}(\Omega)\triangleq \sigma(\mathcal{U}).
\]
Any element $B\in\mathcal{B}(\Omega)$ is called the \emph{Borel set}.
\end{definition}

\begin{definition}[$\mathcal{F}$-Measurable / Random Variable]
\begin{enumerate}
\item
A function $f:(\Omega,\mathcal{F})\to(\mathbb{R}^n,\mathcal{B}(\mathbb{R}^n))$ is called \emph{$\mathcal{F}$-measurable} if
\[
f^{-1}(\bm B)=\{w\mid f(w)\in\mathcal{B}\}\in\mathcal{F},
\]
\mbox{for any $\bm B\in\mathcal{B}(\mathbb{R}^n)$.}
\item
A random variable $X$ is a function $X:(\Omega,\mathcal{F})\to(\mathbb{R}^n,\mathcal{B}(\mathbb{R}^n))$ and is $\mathcal{F}$-measurable.
\end{enumerate}
\end{definition}

\begin{definition}[Generated $\sigma$-Algebra]
Suppose $X$ is a random variable on $(\Omega,\mathcal{F},\mathbb{P})$. 
Then the $\sigma$-algebra generated by $X$, say $\mathcal{H}_X$ is defined to be 
the \emph{minimal $\sigma$-algebra} on $\Omega$ to make $X$ measurable.
\end{definition}
\begin{proposition}
$
\mathcal{H}_X = \{X^{-1}(\bm B):~\bm B\in\mathcal{B}(\mathbb{R}^n)\}.
$
\end{proposition}

\begin{proof}
Since $X$ is $\mathcal{H}_X$-measurable, for any $\bm B\in\mathcal{B}(\mathbb{R}^n)$, $X^{-1}(\bm B)\in\mathcal{H}_X$. Thus $\mathcal{H}_X \supseteq \{X^{-1}(\bm B):~\bm B\in\mathcal{B}(\mathbb{R}^n)\}$.
It suffices to show that $\{X^{-1}(\bm B):~\bm B\in\mathcal{B}(\mathbb{R}^n)\}$ is a $\sigma$-algebra to finish the proof, which is true since $\mathcal{B}(\mathbb{R}^n)=\sigma(\mathcal{U})$, with $\mathcal{U}$ being the topology of $X$.



\end{proof}






%\begin{remark}
%In other words, the $\sigma$-algebra generated by $X$ is the \emph{minimal $\sigma$-algebra} on $\Omega$ containing $X^{-1}(U)$, where $U\subseteq\mathbb{R}^n$ is any open set:
%\[
%\mathcal{H}_X = \{X^{-1}(B):~B\in\mathcal{B}(\mathbb{R}^n)\}.
%\]
%\end{remark}












